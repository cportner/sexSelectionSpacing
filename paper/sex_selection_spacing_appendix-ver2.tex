% Begun.: 2017-04-02
% Edited: 2017-04-02

\documentclass[12pt,letterpaper]{article}
% \documentclass[AER,draftmode]{AEA}

\usepackage{fontspec}
\setromanfont[Ligatures=TeX]{TeX Gyre Pagella}
\usepackage{unicode-math}
\setmathfont{TeX Gyre Pagella Math}
\usepackage[title]{appendix}
\usepackage[margin=1.0in]{geometry}
\usepackage[figuresleft]{rotating}
\usepackage[longnamesfirst]{natbib}
\usepackage{dcolumn}
\usepackage{booktabs}
\usepackage{multirow}
\usepackage[flushleft]{threeparttable}
\usepackage{setspace}
\usepackage[justification=centering]{caption}
\captionsetup[figure]{font=small}
\usepackage[font=scriptsize]{subfig}
\usepackage[xetex,colorlinks=true,linkcolor=black,citecolor=black,urlcolor=black]{hyperref}
\usepackage{adjustbox}
% \usepackage[nolists,tablesfirst]{endfloat}


% \bibpunct{(}{)}{;}{a}{}{,}
\newcommand{\mco}[1]{\multicolumn{1}{c}{#1}}
\newcommand{\mct}[1]{\multicolumn{2}{c}{#1}}
\newcommand{\X}{$\times$ }
\newcommand{\hs}{\hspace{15pt}}

% Attempt to squeeze more floats in
\renewcommand\floatpagefraction{.9}
\renewcommand\topfraction{.9}
\renewcommand\bottomfraction{.9}
\renewcommand\textfraction{.1}
\setcounter{totalnumber}{50}
\setcounter{topnumber}{50}
\setcounter{bottomnumber}{50}

% \renewcommand\subfigtopskip{0pt}
% \renewcommand\subfigcapskip{0pt}
%\renewcommand\subfigbottomskip{2pt}


\newcommand{\bkv}{\Bkk{5.2,0}{0,0}{d}}
\newcommand{\bkh}{\Bkk{5.2,0}{0,0}{dr}}
\newcommand{\bkl}{\Bkk{5.2,0}{5.2,0}{dl}}
\newcommand{\bkd}{\Bkk{5.2,0}{5.2,0}{dd}}
\newcommand{\bkr}{\Bkk{5.2,0}{5.2,0}{dr}}

%
% If tree is not showing remove divps in line 60 in the following file
% /Library/TeX/Root/texmf-dist/tex/latex/xyling/xyling.sty
%

% For estout tables
\def\sym#1{\ifmmode^{#1}\else\(^{#1}\)\fi}

%------------------------------------------------------------------------


\title{Sex Selective Abortions, Fertility, and Birth Spacing
% \protect\thanks{%
% I am grateful to Andrew Foster and Darryl Holman for discussions about the method.
% I owe thanks to Shelly Lundberg, Daniel Rees, David Ribar, Hendrik Wolff, 
% seminar participants at University of Copenhagen, University of Michigan, 
% University of Washington, University of \AA{}rhus, the Fourth 
% Annual Conference on Population, Reproductive Health, 
% and Economic Development, and the Economic Demography Workshop for helpful suggestions and comments.
% I would also like to thank Nalina Varanasi for research assistance.
% Support from the University of Washington Royalty Research Fund and the 
% Development Research Group of the World Bank is gratefully acknowledged.
% The views and findings expressed here are those of the author and
% should not be attributed to the World Bank or any of its member countries.
% Partial support for this research came from a Eunice Kennedy Shriver National
% Institute of Child Health and Human Development research infrastructure grant,
% 5R24HD042828, to the Center for Studies in Demography and Ecology at the
% University of Washington.
% Prior versions of this paper were presented under the title ``The 
% Determinants of Sex Selective Abortions.''
% }
}

\author{}

% \author{Claus C P\"ortner\\
%     Department of Economics\\
%     Albers School of Business and Economics\\
%     Seattle University, P.O. Box 222000\\
%     Seattle, WA 98122\\
%     \href{mailto:cportner@seattleu.edu}{\texttt{cportner@seattleu.edu}}\\
%     \href{http://www.portner.dk}{\texttt{www.portner.dk}}\\
%     \& \\
%     Center for Studies in Demography and Ecology \\
%     University of Washington\\ \vspace{2cm}
%     }

% \date{May 2010\\ \bigskip PRELIMINARY}
\date{December 2017}


%------------------------------------------------------------------------


\begin{document}

\graphicspath{{../figures/}}
\DeclareGraphicsExtensions{.eps,.jpg,.pdf,.mps,.png}
\doublespacing

\appendix

% CHANGING NUMBERING OF FIGURES AND TABLES FOR APPENDIX
\renewcommand\thefigure{\thesection.\arabic{figure}}    
% \setcounter{figure}{0}    
% \setcounter{postfig}{0}

\renewcommand\thetable{\thesection.\arabic{table}}    
\setcounter{table}{0}
% \setcounter{posttbl}{0}

\subsection{Sensitivity of Results to Recall Error}

% [referee 2 wants a discussion of how results are impacted by changes in number of
% women dropped because of recall error]


The last three sections in this Appendix show results for the three education groups for 
three samples: the preferred sample used in the paper (Figures \ref{fig:results_spell1_low}
through \ref{fig:results_spell4_high_rural}), 
a more restricted sample where only women married less than 19 years are 
included (Figures \ref{fig:results_limit_spell1_low} through \ref{fig:results_limit_spell3_high_rural}),
and the sample of women dropped because of potential recall error bias, i.e.\ all women 
dropped from the data to reach the preferred sample (Figures \ref{fig:results_dropped_spell1_low}
through \ref{fig:results_dropped_spell3_high_rural}).
For the ``dropped'' sample it is only possible to estimate sex ratios for the period
before sex-selective abortions became available.
The period covered is different from the other two samples because the births for 
the sample takes place over a much longer period; the sample covers spells that begin
between 1950 and 1984.

The different results for women with no education illustrates the potential problems 
with recall error.
Figures \ref{fig:results_spell1_low}, \ref{fig:results_limit_spell1_low}, and 
\ref{fig:results_dropped_spell1_low} show the different results for women with no
education across the three samples.
The preferred and the restricted samples show almost identical results with the main
difference that the restricted sample has wider confidence intervals because of the
smaller sample size.
The results from the dropped sample, on the other hand, show sex ratios that are 
statistically significantly above the natural sex ratio in many periods, which is
consistent with recall error.
Furthermore, the survival curves are substantially closer to being linear indicating
that observed births took place further away from the time of marriage as expected if 
the couple had a girl first but counted a subsequent son as their first-born.

The second spell results demonstrate the trade-off inherent in dealing with recall error.
While the results for women with no education are similar across the preferred and the
restricted samples in rural areas (Figures \ref{fig:results_spell2_low_rural} and
\ref{fig:results_limit_spell2_low_rural}), the results for urban women with no 
education in the restricted sample are much less stable and have substantially wider
confidence intervals than the preferred sample (Figures \ref{fig:results_spell2_low_urban}
and \ref{fig:results_limit_spell2_low_urban}).
This happens because the restricted sample is down to less than 900 urban women, compared
to close to 1,600 urban women in the preferred sample.

An interesting difference to this pattern is for the third spell for women with 8 or more
years of education.
This group shows no evidence on an unequal sex ratio for the ``dropped'' sample in 
either urban or rural areas (Figures \ref{fig:results_dropped_spell3_high_urban} and
\ref{fig:results_dropped_spell3_high_rural}).
However, for the preferred sample there are consistently more boys than girls for urban
women when the first two children are either two girls or one girl and one boy (Figure 
\ref{fig:results_spell3_high_urban}).
Furthermore, the sex ratios become even more unequal when restricting the sample to
women married less than 19 years (Figure \ref{fig:results_limit_spell3_high_urban}), 
although the sample sizes also become substantially smaller.
This is because spell periods are based on the starting year of the spell.
Hence, some of the pregnancies in the 1972--1984 period will have been exposed to the  
availability of prenatal sex determination.
For Figure \ref{fig:results_spell3_high_urban}(a) 25 percent of observed births 
occur in  1986 or after, meaning that conception took place in the 1985--1994 period, 
while for Figure \ref{fig:results_spell3_high_urban}(d) the number is 29 percent.
Hence, it is possible that the unequal sex ratios observed are, indeed, evidence of 
sex-selective abortions.

\clearpage
\newpage

\subsection{Comparing the Hazard and Simple Models}

As detailed in the main paper, one of the main advantages of the hazard model 
approach is that it can be used to predict what completed fertility, number of abortions, 
and sex ratios will be once women are done with childbearing.
This cannot be done with the simple model because it does not predict fertility
progression, does not take into account censoring, and cannot capture if parents
change their use of sex selection within a spell.
We can, however, still compare differences in predicted sex ratios for 
\emph{individual} parities.%
\footnote{
The simple model uses only observed births and the predicted sex ratio is 
therefore simply what we observe for the relevant uncensored sample.
}

For each parity, the simple model is
\begin{equation}
Y_i = \gamma + \alpha' \mathbf{Z_i} + \beta' \mathbf{X}_{i} + \epsilon_i,
\end{equation}
where $Y$ is a dummy variable that takes the value 1 if the child born is a boy and zero 
if it is a girl, and the vectors of explanatory variables, $\mathbf{Z}_{i}$ and 
$\mathbf{X}_{i}$, are the same as for the proposed method, except that 
$\mathbf{Z}_{i}$ obviously does not include the baseline hazard and that there are 
no time dependence.%
\footnote{
To ease presentation, the indicator for parity number is not shown.
}
I estimate this using a logit model, although a linear probability model leads to
similar results.
This model is estimated on all observed births for a given parity in the relevant 
period to ensure that the results are as close to the standard method of 
examining sex selection \citep[see, for example, ][]{retherford03b,jha06,abrevaya09}.


\begin{table}[htbp]
\begin{center}
\begin{footnotesize}
\begin{threeparttable}
\caption{Comparison of Hazard and Simple Models for  \protect\linebreak
Women with 8 or More Years of Education}
\label{tab:compare2}
\begin{tabular} {@{} l D{.}{.}{5.2} D{.}{.}{5.2} D{.}{.}{5.2} D{.}{.}{5.2}   @{}} \toprule
                                  &  \multicolumn{2}{c}{Urban}        & \multicolumn{2}{c}{Rural}         \\ \cmidrule(lr){2-3} \cmidrule(lr){4-5}
                                  & \mco{Simple}     &  \mco{Hazard}   &  \mco{Simple}    & \mco{Hazard}      \\   \midrule
                                  & \multicolumn{4}{c}{1985--1994 - Second Birth} \\ \cmidrule(l){2-5}
                                  & \multicolumn{4}{c}{1 girl} \\
Number of women\tnote{a}          &             & 4,869   &         & 3,105   \\
Births\tnote{b}                   &     3,848   & 3,926   & 2,417   & 2,795   \\
Percentage boys                   &      54.0   &  54.7   &  52.6   &  53.5   \\
Sex ratio (boys per 100 girls)    &     117.4   & 120.7   & 110.9   & 115.1   \\
\addlinespace 
                                  & \multicolumn{4}{c}{1985--1994 - Third Birth}  \\ \cmidrule(l){2-5}
%                                   & \multicolumn{4}{c}{1 boy and 1 girl}        \\
% Number of women\tnote{a}          &             & 3,460   &         & 1,987   \\
% Births\tnote{c}                   &     1,431   & 1,558   & 1,107   & 1,336   \\
% Percentage boys                   &      53.0   &  52.5   &  53.5   &  54.6   \\
% Sex ratio (boys per 100 girls)    &     112.9   & 110.4   & 115.0   & 120.2   \\
% \addlinespace
                                  & \multicolumn{4}{c}{2 girls}        \\
Number of women\tnote{a}          &             & 1,654   &         &   982   \\
Births\tnote{c}                   &     1,034   & 1,092   &   682   &   853   \\
Percentage boys                   &      60.8   &  63.7   &  54.7   &  57.4   \\
Sex ratio (boys per 100 girls)    &     155.3   & 175.3   & 120.7   & 134.6   \\
\addlinespace 
                                  & \multicolumn{4}{c}{1995--2006 - Second Birth} \\ \cmidrule(l){2-5}
                                  & \multicolumn{4}{c}{1 girl} \\                                                  
Number of women\tnote{a}          &             & 3,774   &         & 2,823   \\
Births\tnote{b}                   &     3,031   & 2,704   & 2,489   & 2,407   \\
Percentage boys                   &      58.0   &  58.6   &  56.0   &  56.9   \\
Sex ratio (boys per 100 girls)    &     138.1   & 141.3   & 127.1   & 132.2   \\
\addlinespace 
                                  & \multicolumn{4}{c}{1995--2006 - Third Birth}  \\ \cmidrule(l){2-5}
%                                   & \multicolumn{4}{c}{1 boy and 1 girl}        \\
% Number of women\tnote{a}          &             & 2,357   &         & 1,819   \\
% Births\tnote{c}                   &       830   &   775   &   853   &   998   \\
% Percentage boys                   &      53.9   &  52.9   &  55.5   &  56.5   \\
% Sex ratio (boys per 100 girls)    &     116.7   & 112.1   & 124.5   & 129.8   \\
% \addlinespace
                                  & \multicolumn{4}{c}{2 girls}        \\
Number of women\tnote{a}          &             & 1,000   &         &   863   \\
Births\tnote{c}                   &       695   &   600   &   690   &   661   \\
Percentage boys                   &      65.5   &  62.4   &  62.0   &  61.7   \\
Sex ratio (boys per 100 girls)    &     189.6   & 166.0   & 163.4   & 161.0   \\
\bottomrule
\end{tabular}                        
\begin{tablenotes} \tiny
\item \hspace*{-0.5em} \textbf{Note.} 
The simple models are estimated using logit on all births for a given parity that
occurred during the relevant time period.
The hazard models are estimated using all spells that began in the relevant time period.
For the second birth the hazard model covers the period from beginning of spell to 6 
years (24 quarters) after the birth of the first child.
For the third birth it covers the period from beginning of spell to 7.25 years 
(29 quarters) after the birth of the second child for the period 1985--1994 and from 
beginning of spell to 5.75 years (23 quarters) after the birth of the second child for 
the period 1995--2006.
\item[a] Number of women in period who began the spell with given sex composition of prior 
child/children in the relevant period.
\item[b] For Logit model the number of women who are observed to have a second birth
in the period. 
For hazard model the predicted number of second births that will occur between beginning 
of the spell and 6 years (24 quarters) after the birth of the first child.
% \item[c] Number of women in period who had one boy and one girl as their first two children.
\item[c] For Logit model the number of women who are observed to have a third birth
in the period. 
For hazard model the predicted number of third births that will occur between beginning of 
the spell and 7.25 years (29 quarters) after the birth of the second child for the
1985--1994 period and between beginning of the spell and 5.75 years (23 quarters) after 
the birth of the second child for the 1995--2006 period.
\end{tablenotes}
\end{threeparttable}
\end{footnotesize}
\end{center}
\end{table}

I focus on those cases where there is evidence of sex-selective abortions: the spells
from parity one to two and two to three for women with eight or more years of education.%
\footnote{
The results for the spell from parity three to four are available on request.
They are not presented because the number of educated women who end up with three births 
and the probability of having a fourth birth are small, which makes
comparing the simple and hazard models with any precision difficult.
}
Table \ref{tab:compare2} shows the number of women who enter the second and third spells
for the hazard model,
the number of births, observed in the case of the simple model and predicted in the case 
of the hazard model, predicted percent boys, and the associated predicted sex ratio.
It is possible for the number of births used for the simple model to be larger than the
hazard models' predicted number of births because the simple model is based
on all births that occurred in the period---independently of when the prior birth
occurred---whereas the hazard model uses only spells that began in the period. 




For the 1985--1994 period---which is when sex selection become available---the 
hazard model consistently predicts a higher sex ratio than the logit model.
Furthermore, some of these differences are substantial.
For urban women with 2 girls as their first two children, the hazard model
predicts a sex ratio of 175 boys per 100 girls, whereas the simple method
``only'' predicts 155 boys per 100 girls.
The difference is smaller in rural areas but still 14 boys per 100 girls
higher for the hazard model than for the simple model.
A similar, if less pronounced pattern, show up for the second birth,
where the hazard model predict 3 and 5 boys more per 100 girls for
urban and rural women.

For the latest period, 1995--2006, the hazard model still produces higher 
predicted sex ratios for second birth and the differences are of a 
similar magnitude to the 1985--1994 period.
The predicted sex ratios for the third birth are, however, 
higher for the simple model than for the hazard model.
This difference is substantially for urban women where the difference 
is 23 boys extra per 100 girls.  

A possible explanation for these differences is that the simple model 
fails to capture the move towards lower fertility and the subsequent 
increase in the use of sex-selective abortions at earlier parities that 
occurred over the periods.
The results also suggest that the take-up of sex selection occurred 
rapidly as the new methods became available and that this take-up
is not fully captured by the simple model.
More broadly, a possible interpretation of the differences is that
the simple model will tend to underestimate when sex-selective abortions
are initially spreading in use and overestimate when the spread of sex 
selection is slowing.
This is most clearly seen for the third births for urban women where
the hazard model show a \emph{decline} is use of sex-selective abortions,
consistent with the lower fertility as discussed in the main paper,
while the simple model show a substantial increase. 

\clearpage
\newpage

\section{Sex Ratios at First Birth}

\setcounter{figure}{0}
\setcounter{table}{0}


% [WHY DO EVERYBODY - COMPARISON WITH JHA ET AL. - ESTABLISH THE ACTUAL SEX RATIO - 
% SHOW THAT LENGTH OF 1ST SPELL GENERALLY SHORT AND RELATION TO FECUNDITY]
% [what do the graphs show - representative woman - later use sample -
% regression results ]

I present a separate analysis of first spell births for three reasons.
First, previous research claims that the largest number of missing girls
is for first order births \citep{jha06}.
Secondly, there are substantially more first births than subsequent births, allowing
for a precise estimation of the ``natural'' percentage boys born in India if there
are no sex selection.
Finally, the results provide an indication of whether first spell length is
a good indicator for fecundity.

Figures \ref{fig:results_spell1_low}, \ref{fig:results_spell1_med}, and 
\ref{fig:results_spell1_high} show the predicted
percentage boys born by quarter from marriage to first birth and the associated
survival functions for the lowest, middle, and highest education groups for representative 
women.
For the first spell the representative woman is 16 years old at the beginning of the 
spell for the no education group, 17 years old for the middle education group, and 
20 years old for the high education group.
Each column represents a time period with the top panel showing urban results and 
the bottom panel rural results.
The graphs also show the expected natural rate of boys, approximately 51.2 percent.%
\footnote{
For comparison, if 55 percent of children born in a given quarter were boys,
approximately 14 percent of the female fetuses were aborted.
Assume 105 boys per 100 girls born, the expected natural sex ratio.
With $b$ boys,  $b\frac{100}{105}$ girls should be born.
If $g$ girls are observed the number of abortions is therefore $b\frac{100}{105}-g$
and the percent aborted female fetuses is 
$\frac{b\frac{100}{105}-g}{b\frac{100}{105}}\times 100$.
With 55 percent boys we get $\frac{55\frac{100}{105}-45}{55\frac{100}{105}}\times 100 = 14.09$.
The corresponding numbers for 60 percent and 65 percent boys are approximately 30 percent and
43 percent of the female fetuses aborted.
}

The most interesting result is how close to the natural sex ratio the predicted percentage
boys is for each group and for each period.
As Figure \ref{fig:results_spell1_low}(d) shows, for rural women without education before 1985, 
who also represent the biggest group, the predicted sex ratios align almost perfectly with 
the expected sex ratio.
For the other groups there is more volatility in the predicted percentage boys, but
nowhere is it statistically significantly larger than 51.2 percent.%
\footnote{
The urban no education group for the 1972--1984 period show two quarters where the predicted
percentage boys is just statistically significantly higher than 51.2 percent, but this
is likely due to recall error not perfectly caught by the method above
and the periods around those two quarters are below the natural percentage boys.
}
Furthermore, for quarters with more substantial deviations from the natural
sex ratio, the predictions are generally based on few births.
In other words, it appears that the probability of having a boy is exactly the same in 
India as it is in other places.

For the group most likely to use sex selection, highly
educated, urban women in the 1995--2006 period, the predicted percentage
boys is also almost perfectly aligned with the expected percentage boys, as shown
in Figure \ref{fig:results_spell1_high}(c).
Hence, there is no evidence that Hindus in India use sex selection on first births.
This cast serious doubts on the data used by \cite{jha06} and their results, as
also discussed by \cite{george06} and \cite{bhat06}.
% \footnote{%
% \cite{jha06} found that in both urban and rural areas there were 54.4 percent boys born 
% among first-borns.
% This translate into 0.17 to 0.23 million females missing among first-borns per year.
% The recall error problem is also noted by \cite{george06} and \cite{bhat06}.
% }

For all education groups and for all periods more than ninety percent of women had their 
first child within 21 quarters of being married and the proportion is increasing in education.
Furthermore, 70 to 85 percent of women will had their first child within ten quarters 
(2.5 years ) of their marriage and the average time between marriage and first birth has 
become shorter over time.
The most likely explanation for the reductions in duration and the increase 
in the number of women who have their first child before 21 quarters is improvements
in health status.
This is also consistent with the differences between education groups where more
educated women are healthier and therefore more likely to conceive.
There are two implications of this.
First, it reinforces the need for estimating the models separately for different
education levels.
Secondly, it confirms that the first spell length can serve as a suitable
proxy for fecundity and that Hindu women in India have their first birth very soon
after marriage, even among highly educated, urban women.


% \subsection{High Education Group}
% 
% If lower fertility is the driving force behind sex selection,
% women with the most education should be the earliest adopters and use it
% for lower parity children than those with less education.
% The predicted percentage boys for the second, third and fourth spells for 
% women with eight or more years of education show that this is, indeed, the case.
% As for the first spell, the results are presented using a representative woman
% with the same characteristics as above, except that the ages at the beginning of
% the spell are 22, 24 and 25 years for the second, third and fourth
% spell and that the first spell length is set equal to 16 months.
% 
% Figures \ref{fig:results_spell2_high_urban} and \ref{fig:results_spell2_high_rural}
% show the second spell predicted percentage boys and the survival curves by spell length 
% for urban and rural women.%
% \footnote{
% Recall, zero quarters is equivalent to nine months after the first birth.
% }
% For both figures, the top panel shows the results if the first child was a girl and the
% bottom panel the results if the first child was a boy.
% Not surprisingly, there is no evidence of sex selection for the 1972--1984 period;
% the predicted percentage boys match up closely with the natural.
% There is some evidence that spacing is shorter after the birth of a girl than
% after a boy, but the differences between the survival curves in the beginning of
% the spells are not large.
% The largest difference at five quarters is for urban women and that 
% is only around five percentage points.
% This is consistent with more educated women being more aware of the
% potential negative effects of close spacing.
% 
% 
% % SPELL 2 - URBAN
% 
% \begin{center}
% [Figure 4 about here.]
% \end{center}
% 
% % \begin{figure}[htpb]
% % \centering
% % \caption*{First child a girl}
% % \setcounter{subfigure}{-2}
% % \subfloat[1972-1984 (N=2,791)]{
% % \begin{minipage}{0.25\textwidth}
% % \captionsetup[subfigure]{labelformat=empty,position=top,captionskip=-1pt,farskip=-0.5pt}
% % \subfloat[Prob.\ boy (\%)]{\includegraphics[width=\textwidth]{spell2_g1_high_urban_g_pc}}\\
% % \subfloat[Prob.\ no birth yet]{\includegraphics[width=\textwidth]{spell2_g1_high_urban_g_s}} 
% % \captionsetup[subfigure]{labelformat=parens}
% % \end{minipage}} 
% % \setcounter{subfigure}{-1}
% % \subfloat[1985--1994 (N=5,011)]{
% % \begin{minipage}{0.25\textwidth}
% % \captionsetup[subfigure]{labelformat=empty,position=top,captionskip=-1pt,farskip=-0.5pt}
% % \subfloat[Prob. boy (\%)]{\includegraphics[width=\textwidth]{spell2_g2_high_urban_g_pc}}\\
% % \subfloat[Prob. no birth yet]{\includegraphics[width=\textwidth]{spell2_g2_high_urban_g_s}}
% % \captionsetup[subfigure]{labelformat=parens}
% % \end{minipage}
% % }
% % \setcounter{subfigure}{0}
% % \subfloat[1995--2006 (N=3,827)]{
% % \begin{minipage}{0.25\textwidth}
% % \captionsetup[subfigure]{labelformat=empty,position=top,captionskip=-1pt,farskip=-0.5pt}
% % \subfloat[Prob. boy (\%)]{\includegraphics[width=\textwidth]{spell2_g3_high_urban_g_pc}}\\
% % \subfloat[Prob. no birth yet]{\includegraphics[width=\textwidth]{spell2_g3_high_urban_g_s}}
% % \captionsetup[subfigure]{labelformat=parens}
% % \end{minipage}
% % }
% % \caption*{First child a boy}
% % \setcounter{subfigure}{1}
% % \subfloat[1972-1984 (N=2,877)]{
% % \begin{minipage}{0.25\textwidth}
% % \captionsetup[subfigure]{labelformat=empty,position=top,captionskip=-1pt,farskip=-0.5pt}
% % \subfloat[Prob. boy (\%)]{\includegraphics[width=\textwidth]{spell2_g1_high_urban_b_pc}}\\
% % \subfloat[Prob. no birth yet]{\includegraphics[width=\textwidth]{spell2_g1_high_urban_b_s}} 
% % \captionsetup[subfigure]{labelformat=parens}
% % \end{minipage}
% % } 
% % \setcounter{subfigure}{2}
% % \subfloat[1985--1994 (N=5,421)]{
% % \begin{minipage}{0.25\textwidth}
% % \captionsetup[subfigure]{labelformat=empty,position=top,captionskip=-1pt,farskip=-0.5pt}
% % \subfloat[Prob. boy (\%)]{\includegraphics[width=\textwidth]{spell2_g2_high_urban_b_pc}}\\
% % \subfloat[Prob. no birth yet]{\includegraphics[width=\textwidth]{spell2_g2_high_urban_b_s}}
% % \captionsetup[subfigure]{labelformat=parens}
% % \end{minipage}
% % }
% % \setcounter{subfigure}{3}
% % \subfloat[1995--2006 (N=4,015)]{
% % \begin{minipage}{0.25\textwidth}
% % \captionsetup[subfigure]{labelformat=empty,position=top,captionskip=-1pt,farskip=-0.5pt}
% % \subfloat[Prob. boy (\%)]{\includegraphics[width=\textwidth]{spell2_g3_high_urban_b_pc}}\\
% % \subfloat[Prob. no birth yet]{\includegraphics[width=\textwidth]{spell2_g3_high_urban_b_s}}
% % \captionsetup[subfigure]{labelformat=parens}
% % \end{minipage}
% % }
% % \caption{Predicted probability of having a boy and probability of
% % no birth yet from nine months after first birth for urban 
% % women with 8 or more years of education by quarter. 
% % Predictions based on age 22 at first birth.
% % Left column shows results prior to sex selection available, middle column before
% % sex selection illegal and right column after sex selection illegal.
% % N indicates the number of women in the relevant group in the underlying samples.
% % }
% % \label{fig:results_spell2_high_urban}
% % \end{figure}
% 
% 
% 
% % SPELL 2 - RURAL
% 
% \begin{center}
% [Figure 5 about here.]
% \end{center}
% 
% 
% % \begin{figure}[htpb]
% % \centering
% % \caption*{First child a girl}
% % \setcounter{subfigure}{-2}
% % \subfloat[1972-1984 (N=1,183)]{
% % \begin{minipage}{0.25\textwidth}
% % \captionsetup[subfigure]{labelformat=empty,position=top,captionskip=-1pt,farskip=-0.5pt}
% % \subfloat[Prob.\ boy (\%)]{\includegraphics[width=\textwidth]{spell2_g1_high_rural_g_pc}}\\
% % \subfloat[Prob.\ no birth yet]{\includegraphics[width=\textwidth]{spell2_g1_high_rural_g_s}} 
% % \captionsetup[subfigure]{labelformat=parens}
% % \end{minipage}} 
% % \setcounter{subfigure}{-1}
% % \subfloat[1985--1994 (N=2,967)]{
% % \begin{minipage}{0.25\textwidth}
% % \captionsetup[subfigure]{labelformat=empty,position=top,captionskip=-1pt,farskip=-0.5pt}
% % \subfloat[Prob. boy (\%)]{\includegraphics[width=\textwidth]{spell2_g2_high_rural_g_pc}}\\
% % \subfloat[Prob. no birth yet]{\includegraphics[width=\textwidth]{spell2_g2_high_rural_g_s}}
% % \captionsetup[subfigure]{labelformat=parens}
% % \end{minipage}
% % }
% % \setcounter{subfigure}{0}
% % \subfloat[1995--2006 (N=2,740)]{
% % \begin{minipage}{0.25\textwidth}
% % \captionsetup[subfigure]{labelformat=empty,position=top,captionskip=-1pt,farskip=-0.5pt}
% % \subfloat[Prob. boy (\%)]{\includegraphics[width=\textwidth]{spell2_g3_high_rural_g_pc}}\\
% % \subfloat[Prob. no birth yet]{\includegraphics[width=\textwidth]{spell2_g3_high_rural_g_s}}
% % \captionsetup[subfigure]{labelformat=parens}
% % \end{minipage}
% % }
% % \caption*{First child a boy}
% % \setcounter{subfigure}{1}
% % \subfloat[1972-1984 (N=1,305)]{
% % \begin{minipage}{0.25\textwidth}
% % \captionsetup[subfigure]{labelformat=empty,position=top,captionskip=-1pt,farskip=-0.5pt}
% % \subfloat[Prob. boy (\%)]{\includegraphics[width=\textwidth]{spell2_g1_high_rural_b_pc}}\\
% % \subfloat[Prob. no birth yet]{\includegraphics[width=\textwidth]{spell2_g1_high_rural_b_s}} 
% % \captionsetup[subfigure]{labelformat=parens}
% % \end{minipage}
% % } 
% % \setcounter{subfigure}{2}
% % \subfloat[1985--1994 (N=3,213)]{
% % \begin{minipage}{0.25\textwidth}
% % \captionsetup[subfigure]{labelformat=empty,position=top,captionskip=-1pt,farskip=-0.5pt}
% % \subfloat[Prob. boy (\%)]{\includegraphics[width=\textwidth]{spell2_g2_high_rural_b_pc}}\\
% % \subfloat[Prob. no birth yet]{\includegraphics[width=\textwidth]{spell2_g2_high_rural_b_s}}
% % \captionsetup[subfigure]{labelformat=parens}
% % \end{minipage}
% % }
% % \setcounter{subfigure}{3}
% % \subfloat[1995--2006 (N=2,945)]{
% % \begin{minipage}{0.25\textwidth}
% % \captionsetup[subfigure]{labelformat=empty,position=top,captionskip=-1pt,farskip=-0.5pt}
% % \subfloat[Prob. boy (\%)]{\includegraphics[width=\textwidth]{spell2_g3_high_rural_b_pc}}\\
% % \subfloat[Prob. no birth yet]{\includegraphics[width=\textwidth]{spell2_g3_high_rural_b_s}}
% % \captionsetup[subfigure]{labelformat=parens}
% % \end{minipage}
% % }
% % \caption{Predicted probability of having a boy and probability of
% % no birth yet from nine months after first birth for rural
% % women with 8 or more years of education by quarter. 
% % Predictions based on age 22 at first birth.
% % Left column shows results prior to sex selection available, middle column before
% % sex selection illegal and right column after sex selection illegal.
% % N indicates the number of women in the relevant group in the underlying samples.
% % }
% % \label{fig:results_spell2_high_rural}
% % \end{figure}
% 
% The first substantive evidence of sex selection is for urban women with one girl
% in the 1985--1994 period (Figure \ref{fig:results_spell2_high_urban}b).
% The percentage boys begins at the natural level but then increases to almost 60 
% percent after which it drops slightly.
% The use of sex-selective abortions becomes even more apparent for the 1995--2006
% period, where the percentage boys born to highly educated women with one girl 
% begins at just below 60 percent, increases to above 65 percent, followed by a decline to
% just over 55 percent.
% These results are especially remarkable given that the numbers are for all of India, not
% just the states with traditionally strong son preferences.
% 
% % [explain shape of curves]
% Of particularly interest is how the predicted percentage boys changes with spell length.
% The development in the percentage boys by quarter for the 1995--2006 period
% (Figure \ref{fig:results_spell2_high_urban}c), 
% is consistent with some women changing their 
% decision to use sex selection after going through one or more abortions.
% The results are for the median level of fecundity, and it is therefore unlikely 
% that the decline is due to lower fecundity for women who give birth later in the spell.%
% \footnote{
% It is, of course, possible that women find it harder to conceive during the
% second spell, but the survival curve would be very close to the survival 
% curve for the first period if the female fetuses were carried to term.
% }
% The 1985--1994 period (Figure \ref{fig:results_spell2_high_urban}b)
% shows the same fall at the end of the observed spell, but it also
% begins at the natural level, consistent with increasing access and 
% acceptance of sex selection over the early period.
% Women who began their second spell early in the 1985--1994 period were less likely
% to have access than women who began their second spell later in the period.
% % Furthermore, women who did not conceive early after a girl but
% % either decided to wait or had trouble getting pregnant were also more likely
% % to have sex selection techniques available to them.
% The 1985--1994 results can be thought of as the pattern that will
% prevail as sex selection is introduced, and the 1995--2006 results show
% the pattern with a mature and readily available technology.
% That sex selection for the 1995--2006 period is used earlier in the 
% spell compared to later also shows that not taking account of timing of births
% and the potential censoring of birth spell can lead to an upward bias in the
% final sex ratio of second born children.
% 
% % [compare to changes in rural areas for shape]
% For rural areas there is no sign of sex-selective abortion during the 1985--1994 period,
% but for the 1995--2006 period the percentage boys born to highly educated women who have
% a girl as their first child increases gradually from 55 percent to over 60 percent until
% dropping off to the normal percentage boys (Figure \ref{fig:results_spell2_high_rural}c).
% Hence, there appears to be easy access to prenatal sex determination even
% in rural areas.
% The difference between urban and rural women for the 1985--1994
% period is more likely to be the result of differences in fertility than in access
% to prenatal sex determinantion as shown below.
% As more and more rural women have only two children the use of sex selection
% goes up.
% The implication is that rural areas lag behind urban areas but not by 
% much and that the expected pattern for the next round of 
% the NFHS for rural educated women should closely mirror what is found 
% for the 1995--2006 period for urban women.
% 
% Women who had a boy as their first child do not appear to be using sex-selective abortions.
% This holds for urban and rural areas and for all time periods.
% In addition, women with a boy are less likely to have a second birth within the
% 21 quarters.
% This is especially prevalent for urban women, although rural women show the same pattern.
% For urban women with one boy, more than thirty percent are not predicted to have
% their second child within the 21 quarters covered here.
% This means that six years have passed since the birth of their first child and given
% the flatness of the survival function it it unlikely that many of these women will
% ever have a second child.
% Even for rural women around twenty percent of women will likely not have a second child.
% 
% 
% [THIS NEEDS TO BE REWRITTEN TO INCORPORATE R3'S QUESTION ABOUT PANELS (A) AND (D) AND
% THE HIGH SEX RATIO FOR THE 1972-1984 PERIOD]
% 
% The results for the third spell are shown in Figures \ref{fig:results_spell3_high_urban} 
% and \ref{fig:results_spell3_high_rural} for the representative 
% urban and rural women.
% The top panel shows the result if the first two children were girls, the middle
% panel if the couple had one boy and one girl, and the bottom panel if the
% first two children were boys.
% Again there is no evidence of sex-selective abortions before 1985, but
% clearly there is more volatility in the predicted percentage boys due
% to the lower sample sizes.
% 
% 
% % SPELL 3 - URBAN
% 
% \begin{center}
% [Figure 6 about here.]
% \end{center}
% 
% % \begin{figure}[htpb]
% % \centering
% % \caption*{First two children girls}
% % \setcounter{subfigure}{-2}
% % \subfloat[1972-1984 (N=735)]{
% % \begin{minipage}{0.25\textwidth}
% % \captionsetup[subfigure]{labelformat=empty,position=top,captionskip=-1pt,farskip=-0.5pt}
% % \subfloat[Prob.\ boy (\%)]{\includegraphics[width=\textwidth]{spell3_g1_high_urban_gg_pc}}\\
% % \subfloat[Prob.\ no birth yet]{\includegraphics[width=\textwidth]{spell3_g1_high_urban_gg_s}} 
% % \captionsetup[subfigure]{labelformat=parens}
% % \end{minipage}} 
% % \setcounter{subfigure}{-1}
% % \subfloat[1985--1994 (N=1,681)]{
% % \begin{minipage}{0.25\textwidth}
% % \captionsetup[subfigure]{labelformat=empty,position=top,captionskip=-1pt,farskip=-0.5pt}
% % \subfloat[Prob. boy (\%)]{\includegraphics[width=\textwidth]{spell3_g2_high_urban_gg_pc}}\\
% % \subfloat[Prob. no birth yet]{\includegraphics[width=\textwidth]{spell3_g2_high_urban_gg_s}}
% % \captionsetup[subfigure]{labelformat=parens}
% % \end{minipage}
% % }
% % \setcounter{subfigure}{0}
% % \subfloat[1995--2006 (N=1,015)]{
% % \begin{minipage}{0.25\textwidth}
% % \captionsetup[subfigure]{labelformat=empty,position=top,captionskip=-1pt,farskip=-0.5pt}
% % \subfloat[Prob. boy (\%)]{\includegraphics[width=\textwidth]{spell3_g3_high_urban_gg_pc}}\\
% % \subfloat[Prob. no birth yet]{\includegraphics[width=\textwidth]{spell3_g3_high_urban_gg_s}}
% % \captionsetup[subfigure]{labelformat=parens}
% % \end{minipage}
% % }
% % \caption*{First two children one boy and one girl}
% % \setcounter{subfigure}{1}
% % \subfloat[1972-1984 (N=1,416)]{
% % \begin{minipage}{0.25\textwidth}
% % \captionsetup[subfigure]{labelformat=empty,position=top,captionskip=-1pt,farskip=-0.5pt}
% % \subfloat[Prob. boy (\%)]{\includegraphics[width=\textwidth]{spell3_g1_high_urban_bg_pc}}\\
% % \subfloat[Prob. no birth yet]{\includegraphics[width=\textwidth]{spell3_g1_high_urban_bg_s}} 
% % \captionsetup[subfigure]{labelformat=parens}
% % \end{minipage}
% % } 
% % \setcounter{subfigure}{2}
% % \subfloat[1985--1994 (N=3,506)]{
% % \begin{minipage}{0.25\textwidth}
% % \captionsetup[subfigure]{labelformat=empty,position=top,captionskip=-1pt,farskip=-0.5pt}
% % \subfloat[Prob. boy (\%)]{\includegraphics[width=\textwidth]{spell3_g2_high_urban_bg_pc}}\\
% % \subfloat[Prob. no birth yet]{\includegraphics[width=\textwidth]{spell3_g2_high_urban_bg_s}}
% % \captionsetup[subfigure]{labelformat=parens}
% % \end{minipage}
% % }
% % \setcounter{subfigure}{3}
% % \subfloat[1995--2006 (N=2,367)]{
% % \begin{minipage}{0.25\textwidth}
% % \captionsetup[subfigure]{labelformat=empty,position=top,captionskip=-1pt,farskip=-0.5pt}
% % \subfloat[Prob. boy (\%)]{\includegraphics[width=\textwidth]{spell3_g3_high_urban_bg_pc}}\\
% % \subfloat[Prob. no birth yet]{\includegraphics[width=\textwidth]{spell3_g3_high_urban_bg_s}}
% % \captionsetup[subfigure]{labelformat=parens}
% % \end{minipage}
% % }
% % \caption{Predicted probability of having a boy and probability of
% % no birth yet from nine months after second birth for urban 
% % women with 8 or more years of education by quarter. 
% % Predictions based on age 24 at second birth.
% % Left column shows results prior to sex selection available, middle column before
% % sex selection illegal and right column after sex selection illegal.
% % N indicates the number of women in the relevant group in the underlying samples.
% % }
% % \label{fig:results_spell3_high_urban}
% % \end{figure}
% % 
% % 
% % \begin{figure}[htpb]
% % \centering
% % \caption*{First two children boys}
% % \setcounter{subfigure}{4}
% % \subfloat[1972-1984 (N=789)]{
% % \begin{minipage}{0.25\textwidth}
% % \captionsetup[subfigure]{labelformat=empty,position=top,captionskip=-1pt,farskip=-0.5pt}
% % \subfloat[Prob. boy (\%)]{\includegraphics[width=\textwidth]{spell3_g1_high_urban_bb_pc}}\\
% % \subfloat[Prob. no birth yet]{\includegraphics[width=\textwidth]{spell3_g1_high_urban_bb_s}} 
% % \captionsetup[subfigure]{labelformat=parens}
% % \end{minipage}
% % } 
% % \setcounter{subfigure}{5}
% % \subfloat[1985--1994 (N=1,665)]{
% % \begin{minipage}{0.25\textwidth}
% % \captionsetup[subfigure]{labelformat=empty,position=top,captionskip=-1pt,farskip=-0.5pt}
% % \subfloat[Prob. boy (\%)]{\includegraphics[width=\textwidth]{spell3_g2_high_urban_bb_pc}}\\
% % \subfloat[Prob. no birth yet]{\includegraphics[width=\textwidth]{spell3_g2_high_urban_bb_s}}
% % \captionsetup[subfigure]{labelformat=parens}
% % \end{minipage}
% % }
% % \setcounter{subfigure}{6}
% % \subfloat[1995--2006 (N=1,086)]{
% % \begin{minipage}{0.25\textwidth}
% % \captionsetup[subfigure]{labelformat=empty,position=top,captionskip=-1pt,farskip=-0.5pt}
% % \subfloat[Prob. boy (\%)]{\includegraphics[width=\textwidth]{spell3_g3_high_urban_bb_pc}}\\
% % \subfloat[Prob. no birth yet]{\includegraphics[width=\textwidth]{spell3_g3_high_urban_bb_s}}
% % \captionsetup[subfigure]{labelformat=parens}
% % \end{minipage}
% % }
% % % \label{fig:results_spell3_high_urban}
% % \caption{(Continued) Predicted probability of having a boy and probability of
% % no birth yet from nine months after second birth for urban 
% % women with 8 or more years of education by quarter. 
% % Predictions based on age 24 at second birth.
% % Left column shows results prior to sex selection available, middle column before
% % sex selection illegal and right column after sex selection illegal.
% % N indicates the number of women in the relevant group in the underlying samples.
% % \ContinuedFloat
% % }
% % \addtocounter{figure}{1}
% % \end{figure}
% 
% % HAVE NO IDEA WHY THIS IS NEEDED BUT SOMETHING SEESM TO GO WRONG WITH THE COMBINATION
% % OF SUBFIGURE AND CONTINUEDFLOAT
% % 
% 
% % SPELL 3 - RURAL
% 
% \begin{center}
% [Figure 7 about here.]
% \end{center}
% 
% % \begin{figure}[htpb]
% % \centering
% % \caption*{First two children girls}
% % \setcounter{subfigure}{-2}
% % \subfloat[1972-1984 (N=312)]{
% % \begin{minipage}{0.25\textwidth}
% % \captionsetup[subfigure]{labelformat=empty,position=top,captionskip=-1pt,farskip=-0.5pt}
% % \subfloat[Prob.\ boy (\%)]{\includegraphics[width=\textwidth]{spell3_g1_high_rural_gg_pc}}\\
% % \subfloat[Prob.\ no birth yet]{\includegraphics[width=\textwidth]{spell3_g1_high_rural_gg_s}} 
% % \captionsetup[subfigure]{labelformat=parens}
% % \end{minipage}} 
% % \setcounter{subfigure}{-1}
% % \subfloat[1985--1994 (N=942)]{
% % \begin{minipage}{0.25\textwidth}
% % \captionsetup[subfigure]{labelformat=empty,position=top,captionskip=-1pt,farskip=-0.5pt}
% % \subfloat[Prob. boy (\%)]{\includegraphics[width=\textwidth]{spell3_g2_high_rural_gg_pc}}\\
% % \subfloat[Prob. no birth yet]{\includegraphics[width=\textwidth]{spell3_g2_high_rural_gg_s}}
% % \captionsetup[subfigure]{labelformat=parens}
% % \end{minipage}
% % }
% % \setcounter{subfigure}{0}
% % \subfloat[1995--2006 (N=842)]{
% % \begin{minipage}{0.25\textwidth}
% % \captionsetup[subfigure]{labelformat=empty,position=top,captionskip=-1pt,farskip=-0.5pt}
% % \subfloat[Prob. boy (\%)]{\includegraphics[width=\textwidth]{spell3_g3_high_rural_gg_pc}}\\
% % \subfloat[Prob. no birth yet]{\includegraphics[width=\textwidth]{spell3_g3_high_rural_gg_s}}
% % \captionsetup[subfigure]{labelformat=parens}
% % \end{minipage}
% % }
% % \caption*{First two children one boy and one girl}
% % \setcounter{subfigure}{1}
% % \subfloat[1972-1984 (N=666)]{
% % \begin{minipage}{0.25\textwidth}
% % \captionsetup[subfigure]{labelformat=empty,position=top,captionskip=-1pt,farskip=-0.5pt}
% % \subfloat[Prob. boy (\%)]{\includegraphics[width=\textwidth]{spell3_g1_high_rural_bg_pc}}\\
% % \subfloat[Prob. no birth yet]{\includegraphics[width=\textwidth]{spell3_g1_high_rural_bg_s}} 
% % \captionsetup[subfigure]{labelformat=parens}
% % \end{minipage}
% % } 
% % \setcounter{subfigure}{2}
% % \subfloat[1985--1994 (N=1,909)]{
% % \begin{minipage}{0.25\textwidth}
% % \captionsetup[subfigure]{labelformat=empty,position=top,captionskip=-1pt,farskip=-0.5pt}
% % \subfloat[Prob. boy (\%)]{\includegraphics[width=\textwidth]{spell3_g2_high_rural_bg_pc}}\\
% % \subfloat[Prob. no birth yet]{\includegraphics[width=\textwidth]{spell3_g2_high_rural_bg_s}}
% % \captionsetup[subfigure]{labelformat=parens}
% % \end{minipage}
% % }
% % \setcounter{subfigure}{3}
% % \subfloat[1995--2006 (N=1,763)]{
% % \begin{minipage}{0.25\textwidth}
% % \captionsetup[subfigure]{labelformat=empty,position=top,captionskip=-1pt,farskip=-0.5pt}
% % \subfloat[Prob. boy (\%)]{\includegraphics[width=\textwidth]{spell3_g3_high_rural_bg_pc}}\\
% % \subfloat[Prob. no birth yet]{\includegraphics[width=\textwidth]{spell3_g3_high_rural_bg_s}}
% % \captionsetup[subfigure]{labelformat=parens}
% % \end{minipage}
% % }
% % \caption{Predicted probability of having a boy and probability of
% % no birth yet from nine months after second birth for rural
% % women with 8 or more years of education by quarter. 
% % Predictions based on age 24 at second birth.
% % Left column shows results prior to sex selection available, middle column before
% % sex selection illegal and right column after sex selection illegal.
% % N indicates the number of women in the relevant group in the underlying samples.
% % }
% % \label{fig:results_spell3_high_rural}
% % \end{figure}
% % 
% % 
% % \begin{figure}[htpb]
% % % \TheContinuedFloat
% % \centering
% % \caption*{First two children boys}
% % \setcounter{subfigure}{4}
% % \subfloat[1972-1984 (N=339)]{
% % \begin{minipage}{0.25\textwidth}
% % \captionsetup[subfigure]{labelformat=empty,position=top,captionskip=-1pt,farskip=-0.5pt}
% % \subfloat[Prob. boy (\%)]{\includegraphics[width=\textwidth]{spell3_g1_high_rural_bb_pc}}\\
% % \subfloat[Prob. no birth yet]{\includegraphics[width=\textwidth]{spell3_g1_high_rural_bb_s}} 
% % \captionsetup[subfigure]{labelformat=parens}
% % \end{minipage}
% % } 
% % \setcounter{subfigure}{5}
% % \subfloat[1985--1994 (N=899)]{
% % \begin{minipage}{0.25\textwidth}
% % \captionsetup[subfigure]{labelformat=empty,position=top,captionskip=-1pt,farskip=-0.5pt}
% % \subfloat[Prob. boy (\%)]{\includegraphics[width=\textwidth]{spell3_g2_high_rural_bb_pc}}\\
% % \subfloat[Prob. no birth yet]{\includegraphics[width=\textwidth]{spell3_g2_high_rural_bb_s}}
% % \captionsetup[subfigure]{labelformat=parens}
% % \end{minipage}
% % }
% % \setcounter{subfigure}{6}
% % \subfloat[1995--2006 (N=763)]{
% % \begin{minipage}{0.25\textwidth}
% % \captionsetup[subfigure]{labelformat=empty,position=top,captionskip=-1pt,farskip=-0.5pt}
% % \subfloat[Prob. boy (\%)]{\includegraphics[width=\textwidth]{spell3_g3_high_rural_bb_pc}}\\
% % \subfloat[Prob. no birth yet]{\includegraphics[width=\textwidth]{spell3_g3_high_rural_bb_s}}
% % \captionsetup[subfigure]{labelformat=parens}
% % \end{minipage}
% % }
% % \caption{(Continued) Predicted probability of having a boy and probability of
% % no birth yet from nine months after second birth for rural
% % women with 8 or more years of education by quarter. 
% % Predictions based on age 24 at second birth.
% % Left column shows results prior to sex selection available, middle column before
% % sex selection illegal and right column after sex selection illegal.
% % N indicates the number of women in the relevant group in the underlying samples.
% % \ContinuedFloat
% % }
% % % \label{fig:results_spell3_high_rural}
% % \addtocounter{figure}{1}
% % \end{figure}
% 
% % HAVE NO IDEA WHY THIS IS NEEDED BUT SOMETHING SEESM TO GO WRONG WITH THE COMBINATION
% % OF SUBFIGURE AND CONTINUEDFLOAT
% % \addtocounter{figure}{1}
% 
% 
% 
% In urban areas there is clear evidence of sex-selective abortions among women
% with two girls for both 1985--1994 and 1995--2006.
% For the first of these periods, births early in the spell are close to the natural sex ratio,
% but births after the fifth quarter are significantly above the natural level
% at around 65 percent boys.%
% \footnote{
% The sex ratio increases substantially for births after the 15th quarter, but 
% given the low number of births it is not clear how much one can learn from this increase.
% }
% Interestingly, there does not appear to be an increase in the use of sex-selective
% abortions in the 1995--2006 period compared to the 1985--1994 period.
% This can be explained by the increase in the use of sex-selective
% abortions for second births combined with lower desired fertility.
% Fewer women have two girls and given the apparent ready access to prenatal
% sex determination those who do are more likely to do so by choice.%
% \footnote{
% For urban women, 25 percent of the sample in the first period had two girls, 
% while only 22.7 percent of the third period sample had two girls.
% }
% Even then, the percentage boys born is between 60 and 65 percent.
% This pattern of sex selection for women with two girls is mirrored by rural women.
% As expected, given that most educated women do not want more than two children,
% there is no evidence for declining use of sex selection toward the end of the spell.
% 
% Women with either two boys or one boy and one girl as 
% their first two children do not appear to use sex selection.
% Relatively few of these women go on to have a third child
% and the proportion that does declines substantially over time.
% This is best illustrated by the urban women in the 1995--2006 period (Figures 
% \ref{fig:results_spell3_high_urban}c, f, i).
% Close to sixty percent of the women with two girls will have a third child,
% while only around thirty percent of those with at least one boy will
% have a third child.
% For comparison, during the 1972--1984 period eighty percent of the urban women 
% with two girls had a third child and almost sixty percent of those
% with at least one boy had a third child.
% The numbers are higher for rural women, but the pattern is the same.
% 
% 
% Relatively few women with eight or more years of education have three children and 
% even fewer have four.
% This means that caution should be used in interpreting the results and in the interest
% of space the graphs are not presented here.%
% \footnote{
% Graphs and underlying estimation results are available on request.
% }
% As for the other spells there is no evidence of sex selection for the first period.
% Again, families without boys are the main users of sex-selective abortions.
% For urban women with three girls the predicted percentage boys is between 60 and 75 for both
% the 1985--1994 and 1995--2006 time periods.
% The numbers for rural women with three girls are in line with those for urban women.
% It is worth noting that there are only 211 and 155 observed births for
% urban women for the two time periods, and only 167 and 178 births for the rural women.
% % What appear to be the only evidence against the ``funeral pyre'' hypothesis
% % comes from urban women with two girls and one boy for whom the predicted percentage
% % boys is above 65 percent for all quarters, but this result is based on only 292 births
% % and does not show up for other urban women or for rural women.
% Urban women with with two girls and one boy show a predicted percentage boys above 65 
% percent for all quarters, but this result is based on only 292 births.
% Finally, it is interesting that the probability of having a fourth child is
% higher for women with three boys than it is for women with two boys and one girl.
% This might be evidence of a slight gender balance preference, although 
% there is absolutely no evidence that women are aborting boys and the number
% of observed births to both groups is very small.
% 
% 
% 
% % SPELL 2 - URBAN
% % \begin{figure}
% % \centering
% % \subfloat{\includegraphics[width=.25\textwidth]{want_spell2_high_r7_pc}}
% % \subfloat{\includegraphics[width=.25\textwidth]{want_spell2_high_r8_pc}} \\ \setcounter{subfigure}{0}
% % \subfloat[Rural (N=4,663)]{\includegraphics[width=.25\textwidth]{want_spell2_high_r7_s}}
% % \subfloat[Urban (N=7,940)]{\includegraphics[width=.25\textwidth]{want_spell2_high_r8_s}} \\
% % \subfloat{\includegraphics[width=.25\textwidth]{want_spell2_high_r3_pc}}
% % \subfloat{\includegraphics[width=.25\textwidth]{want_spell2_high_r4_pc}} \\ \setcounter{subfigure}{2}
% % \subfloat[Rural (N=1,044)]{\includegraphics[width=.25\textwidth]{want_spell2_high_r3_s}}
% % \subfloat[Urban (N=898)]{\includegraphics[width=.25\textwidth]{want_spell2_high_r4_s}}
% % \caption{Predicted percent boys born and ``survival'' functions by spacing after first birth
% % and by areas of residence for women with 8 or more years of education with the first child a girl.
% % Top Panel shows Wants 2 or Fewer Children and Bottom Panel 3 or More. \protect\linebreak
% % N indicates Number of Women who Began a Spell.}
% % \label{fig:results_spell2_high_want}
% % \end{figure}
% 
% 
% % \begin{figure}[htpb]
% % \centering
% % \subfloat{\includegraphics[width=.25\textwidth]{want_spell2_high_r7_pc}}
% % \subfloat{\includegraphics[width=.25\textwidth]{want_spell2_high_r3_pc}}
% % \subfloat{\includegraphics[width=.25\textwidth]{want_spell2_high_r8_pc}} 
% % \subfloat{\includegraphics[width=.25\textwidth]{want_spell2_high_r4_pc}} \\ \setcounter{subfigure}{0}
% % \subfloat[Rural (N=4,663)]{\includegraphics[width=.25\textwidth]{want_spell2_high_r7_s}}
% % \subfloat[Rural (N=1,044)]{\includegraphics[width=.25\textwidth]{want_spell2_high_r3_s}}
% % \subfloat[Urban (N=7,940)]{\includegraphics[width=.25\textwidth]{want_spell2_high_r8_s}} 
% % \subfloat[Urban (N=898)]{\includegraphics[width=.25\textwidth]{want_spell2_high_r4_s}}
% % \caption{Predicted percent boys born and survival functions by spacing after first birth
% % and by areas of residence for women with 8 or more years of education with first child a girl.
% % Left shows wants 2 or fewer children and right 3 or more. 
% % N indicates number of women who began a spell.}
% % \label{fig:results_spell2_high_want}
% % \end{figure}
% 
% \subsection{Middle and Low Education Groups}
% 
% For women with one to seven years of education, a representative woman is 
% used with the first spell length set to 16 months and the ages at the 
% beginning of the spell 19, 21 and 24 for the second, third and fourth spells, respectively.
% There are two factors that make it difficult to establish whether sex selection
% is used among the middle education group.
% First, there are fewer observations for this group compared with the high
% education group.
% That problem becomes especially acute when looking at the higher spells
% divided by sex composition of previous children.
% Secondly, even for those spells with more information there is substantially
% more noise in the 1972--1984 results than for the high education group.%
% \footnote{
% The results are available on request.
% }
% % The results are presented in Appendix A.
% 
% For the second spell there is no evidence of sex-selective abortions for
% either urban or rural women and the proportion of women who have a second
% birth remains high across all three time periods.
% The same is the case for the third spell, although there are signs of falling
% fertility in that the proportion of women who have a third child is decreasing,
% especially among those with one or two boys.%
% \footnote{
% There are relatively few urban women with two girls in this education group 
% and although there might be some evidence of sex-selective abortions the
% results are too noisy to conclude anything with any degree of confidence.
% }
% Even for the fourth spell, women with three girls do not show statistically
% significant use of sex-selective abortions and this holds for both urban 
% and rural women.
% For both urban and rural women there has been a substantial decline in the
% number of women who have a fourth child and again this decline is concentrated
% among women with one or more boys.
% 
% 
% For women with no education, the starting ages are 18, 21 and 23 for 
% the second, third and fourth spells, respectively.
% Although there are not a large number of women in urban areas without education
% there are substantial numbers in the rural areas.%
% \footnote{
% The results are available on request.
% }
% % The results are presented in Appendix A.
% For urban women without education there is little evidence of sex selection
% for any of the spells.%
% \footnote{
% Urban women with three girls do have a higher percentage boys, but only 208 are in this group.
% }
% Although the percentage boys for rural women with one girl for the second
% spell is a very precisely estimated flat line for the 1985--1994 period,
% the percentage boys for the third period is between 55 and 60 from the
% seventh to the thirteenth quarter and this percentage is statistically 
% significant.\footnote{
% These quarters account for more than twenty percent of the births or
% slightly more than 700 births.
% }
% Two things makes this result puzzling.
% First, there is little evidence of women stopping
% childbearing once they have a son.
% Even among rural women with two sons almost ninety percent go on to
% have a third child and eighty percent of women with three boys
% have a fourth child.
% Hence, fertility remains very high for this group and therefore also
% the probability of having at least one son without the use of sex-selective 
% abortions.
% Secondly, one would expect more use of sex selection for women with
% two girls in the third spell or three girls in the fourth spell, but
% for both the predicted percentage boys is close to the natural rate.
% 


\clearpage
\newpage


\section{Migration}

\setcounter{figure}{0}
\setcounter{table}{0}


\begin{table}[hp]
\centering
\footnotesize
\begin{threeparttable}
\caption{Migration by Beginning of Spell}
\label{tab:migrate}
\begin{tabular}{l  D{.}{.}{2,2} D{.}{.}{2,2} D{.}{.}{2,2} D{.}{.}{2,2} D{.}{.}{2,2} D{.}{.}{2,2} D{.}{.}{2,2} D{.}{.}{2,2} D{.}{.}{2,2} } \toprule
                & \multicolumn{9}{c}{Years of Education} \\ \cmidrule(lr){2-10}
                & \multicolumn{3}{c}{None}                   & \multicolumn{3}{c}{1--7 Years}              & \multicolumn{3}{c}{8 or More Years} \\ \cmidrule(lr){2-4} \cmidrule(lr){5-7} \cmidrule(lr){8-10}
                & \mco{1972--} & \mco{1985--} & \mco{1995--} & \mco{1972--} & \mco{1985--} & \mco{1995--}  & \mco{1972--} & \mco{1985--} & \mco{1995--} \\
                & \mco{1984}   & \mco{1994}   & \mco{2006}   & \mco{1984}   & \mco{1994}   & \mco{2006}    & \mco{1984}   & \mco{1994}   & \mco{2006} \\
\midrule                
\multicolumn{10}{l}{\textbf{After marriage (\%)}} \\
Did not move    &       51.20&       59.60&       64.04&       50.24&       58.38&       63.68&       42.16&       51.07&       61.07\\
Within same type&       39.76&       32.69&       27.99&       34.85&       29.94&       25.95&       41.38&       34.69&       27.35\\
Rural to urban  &        7.20&        5.91&        5.87&       10.91&        8.18&        7.33&       11.91&        9.93&        7.75\\
Urban to rural  &        1.85&        1.80&        2.10&        4.00&        3.49&        3.04&        4.55&        4.30&        3.83\\
\addlinespace 
\multicolumn{10}{l}{\textbf{After first birth (\%)}} \\
Did not move    &       80.66&       86.63&       88.78&       75.16&       82.93&       88.32&       66.60&       76.85&       86.33\\
Within same type&       14.06&        9.23&        7.02&       16.11&       10.95&        7.12&       24.16&       16.21&        9.44\\
Rural to urban  &        4.25&        3.28&        3.23&        6.39&        4.33&        3.36&        6.63&        4.97&        3.04\\
Urban to rural  &        1.03&        0.85&        0.97&        2.34&        1.78&        1.21&        2.61&        1.96&        1.18\\
\addlinespace 
\multicolumn{10}{l}{\textbf{After second birth (\%)}} \\
Did not move    &       80.66&       85.40&       85.77&       75.17&       81.42&       85.08&       66.64&       74.58&       82.48\\
Within same type&       14.08&       10.12&        8.91&       16.15&       11.93&        9.33&       24.08&       17.59&       11.84\\
Rural to urban  &        4.22&        3.56&        4.02&        6.37&        4.72&        4.07&        6.72&        5.62&        4.09\\
Urban to rural  &        1.04&        0.93&        1.30&        2.32&        1.94&        1.51&        2.56&        2.20&        1.58\\
\addlinespace 
\multicolumn{10}{l}{\textbf{After third birth (\%)}} \\
Did not move    &       80.76&       84.22&       84.47&       75.19&       81.01&       82.24&       68.55&       75.25&       82.54\\
Within same type&       13.98&       10.89&        9.98&       15.88&       12.13&       11.51&       21.98&       16.62&       10.82\\
Rural to urban  &        4.25&        3.91&        4.28&        6.57&        4.93&        4.74&        7.02&        5.89&        4.98\\
Urban to rural  &        1.01&        0.98&        1.26&        2.35&        1.93&        1.51&        2.46&        2.24&        1.66\\
\bottomrule
\end{tabular}
\begin{tablenotes} 
\scriptsize
\item \hspace*{-0.5em} \textbf{Note.}
``Within same type'' indicates either a move from a rural area to another rural area,
or a move from an urban area to another urban area.
\end{tablenotes}
\end{threeparttable}
\end{table}



% \begin{table}[hp]
% \centering
% \footnotesize
% \begin{threeparttable}
% \caption{Migration by beginning of spell for women with no education}
% \label{tab:migrate_low}
% \begin{tabular}{l  D{.}{.}{2,2} D{.}{.}{2,2} D{.}{.}{2,2} } \toprule
%                 & \mco{1972--1984} & \mco{1985--1994} & \mco{1995--2006} \\
%                 & \multicolumn{3}{c}{After marriage (\%)} \\
% Did not move    &       51.20&       59.60&       64.04\\
% Within same type&       39.76&       32.69&       27.99\\
% Rural to urban  &        7.20&        5.91&        5.87\\
% Urban to rural  &        1.85&        1.80&        2.10\\
%                 & \multicolumn{3}{c}{After first birth (\%)} \\
% Did not move&       80.66&       86.63&       88.78\\
% Within same type&       14.06&        9.23&        7.02\\
% Rural to urban&        4.25&        3.28&        3.23\\
% Urban to rural&        1.03&        0.85&        0.97\\
%                 & \multicolumn{3}{c}{After second birth (\%)} \\
% Did not move&       80.66&       85.40&       85.77\\
% Within same type&       14.08&       10.12&        8.91\\
% Rural to urban&        4.22&        3.56&        4.02\\
% Urban to rural&        1.04&        0.93&        1.30\\
%                 & \multicolumn{3}{c}{After third birth (\%)} \\
% Did not move&       80.76&       84.22&       84.47\\
% Within same type&       13.98&       10.89&        9.98\\
% Rural to urban&        4.25&        3.91&        4.28\\
% Urban to rural&        1.01&        0.98&        1.26\\
% \bottomrule
% \end{tabular}
% \begin{tablenotes} 
% \scriptsize
% \item \hspace*{-0.5em} \textbf{Note.}
% ``Within same type'' indicates either a move from a rural area to another rural area,
% or a move from an urban area to another urban area.
% \end{tablenotes}
% \end{threeparttable}
% \end{table}
% 
% 
% \begin{table}
% \centering
% \footnotesize
% \begin{threeparttable}
% \caption{Migration by beginning of spell for women with 1 through 7 years of education}
% \label{tab:migrate_med}
% \begin{tabular}{l  D{.}{.}{2,2} D{.}{.}{2,2} D{.}{.}{2,2} } \toprule
%                 & \mco{1972--1984} & \mco{1985--1994} & \mco{1995--2006} \\
%                 & \multicolumn{3}{c}{After marriage (\%)} \\
% Did not move    &       50.24&       58.38&       63.68\\
% Within same type&       34.85&       29.94&       25.95\\
% Rural to urban  &       10.91&        8.18&        7.33\\
% Urban to rural  &        4.00&        3.49&        3.04\\
%                 & \multicolumn{3}{c}{After first birth (\%)} \\
% Did not move    &       75.16&       82.93&       88.32\\
% Within same type&       16.11&       10.95&        7.12\\
% Rural to urban  &        6.39&        4.33&        3.36\\
% Urban to rural  &        2.34&        1.78&        1.21\\
%                 & \multicolumn{3}{c}{After second birth (\%)} \\
% Did not move    &       75.17&       81.42&       85.08\\
% Within same type&       16.15&       11.93&        9.33\\
% Rural to urban  &        6.37&        4.72&        4.07\\
% Urban to rural  &        2.32&        1.94&        1.51\\
%                 & \multicolumn{3}{c}{After third birth (\%)} \\
% Did not move    &       75.19&       81.01&       82.24\\
% Within same type&       15.88&       12.13&       11.51\\
% Rural to urban  &        6.57&        4.93&        4.74\\
% Urban to rural  &        2.35&        1.93&        1.51\\
% \bottomrule
% \end{tabular}
% \begin{tablenotes} 
% \scriptsize
% \item \hspace*{-0.5em} \textbf{Note.}
% ``Within same type'' indicates either a move from a rural area to another rural area,
% or a move from an urban area to another urban area.
% \end{tablenotes}
% \end{threeparttable}
% \end{table}
% 
% 
% \begin{table}
% \centering
% \footnotesize
% \begin{threeparttable}
% \caption{Migration by beginning of spell for women with 8 or more years of education}
% \label{tab:migrate_high}
% \begin{tabular}{l  D{.}{.}{2,2} D{.}{.}{2,2} D{.}{.}{2,2} } \toprule
%                 & \mco{1972--1984} & \mco{1985--1994} & \mco{1995--2006} \\
%                 & \multicolumn{3}{c}{After marriage (\%)} \\
% Did not move    &       42.16&       51.07&       61.07\\
% Within same type&       41.38&       34.69&       27.35\\
% Rural to urban  &       11.91&        9.93&        7.75\\
% Urban to rural  &        4.55&        4.30&        3.83\\
%                 & \multicolumn{3}{c}{After first birth (\%)} \\
% Did not move    &       66.60&       76.85&       86.33\\
% Within same type&       24.16&       16.21&        9.44\\
% Rural to urban  &        6.63&        4.97&        3.04\\
% Urban to rural  &        2.61&        1.96&        1.18\\
%                 & \multicolumn{3}{c}{After second birth (\%)} \\
% Did not move    &       66.64&       74.58&       82.48\\
% Within same type&       24.08&       17.59&       11.84\\
% Rural to urban  &        6.72&        5.62&        4.09\\
% Urban to rural  &        2.56&        2.20&        1.58\\
%                 & \multicolumn{3}{c}{After third birth (\%)} \\
% Did not move    &       68.55&       75.25&       82.54\\
% Within same type&       21.98&       16.62&       10.82\\
% Rural to urban  &        7.02&        5.89&        4.98\\
% Urban to rural  &        2.46&        2.24&        1.66\\
% \bottomrule
% \end{tabular}
% \begin{tablenotes} 
% \scriptsize
% \item \hspace*{-0.5em} \textbf{Note.}
% ``Within same type'' indicates either a move from a rural area to another rural area,
% or a move from an urban area to another urban area.
% \end{tablenotes}
% \end{threeparttable}
% \end{table}

\clearpage
\newpage

\section{All Graphs using Main Sample}

\setcounter{figure}{0}
\setcounter{table}{0}


% Figures \ref{fig:results_spell2_low_urban} to \ref{fig:results_spell4_low_rural_cont}
% show the results for the group of women with no education.
% The starting ages are 18, 21 and 23 for the second, third and fourth spells,
% respectively.

% The results for women with one to seven years of education are presented
% in Figures \ref{fig:results_spell2_med_urban} to \ref{fig:results_spell4_med_rural_cont}.
% Again a representative woman is used with the first spell length set to
% 16 months and the ages at the beginning of the spell 19, 21 and 24
% for the second, third and fourth spells, respectively.

% [FIRST SPELL]

\subsection{First Spell}

\input{../figures/appendix_spell1_low.tex}

\input{../figures/appendix_spell1_med.tex}

\input{../figures/appendix_spell1_high.tex}

\begin{figure}[htpb]
\centering
\caption*{No Education}
\subfloat[Urban]{\includegraphics[width=0.49\textwidth]{spell1_low_urban_pps}} 
\subfloat[Rural]{\includegraphics[width=0.49\textwidth]{spell1_low_rural_pps}} \\
\caption*{1-7 Years of Education}
\subfloat[Urban]{\includegraphics[width=0.49\textwidth]{spell1_med_urban_pps}} 
\subfloat[Rural]{\includegraphics[width=0.49\textwidth]{spell1_med_rural_pps}} \\
\caption*{8 or more Years of Education}
\subfloat[Urban]{\includegraphics[width=0.49\textwidth]{spell1_high_urban_pps}} 
\subfloat[Rural]{\includegraphics[width=0.49\textwidth]{spell1_high_rural_pps}} 
\caption{[Parity progression conditional survival]}
\label{fig:results_spell1_pps}
\end{figure}


\clearpage
\newpage

\subsection{Second Spell}


\input{../figures/appendix_spell2_low.tex}


% PARITY PROGRESSION SURVIVAL CURVES - Low

% Low education

\begin{figure}[htpb]
\centering
\caption*{Urban}
\setcounter{subfigure}{-1}
\subfloat[1972--1984]{
    \begin{minipage}{0.32\textwidth}
        \captionsetup[subfigure]{labelformat=empty,position=top,captionskip=-1pt,farskip=-0.5pt}
        \subfloat[Prob.\ no birth yet]{\includegraphics[width=\textwidth]{spell2_g1_low_urban_pps}} 
        \captionsetup[subfigure]{labelformat=parens}
    \end{minipage}
} 
\setcounter{subfigure}{-0}
\subfloat[1985--1994]{
    \begin{minipage}{0.32\textwidth}
        \captionsetup[subfigure]{labelformat=empty,position=top,captionskip=-1pt,farskip=-0.5pt}
        \subfloat[Prob. no birth yet]{\includegraphics[width=\textwidth]{spell2_g2_low_urban_pps}}
        \captionsetup[subfigure]{labelformat=parens}
    \end{minipage}
} \\
\setcounter{subfigure}{1}
\subfloat[1995--2005]{
    \begin{minipage}{0.32\textwidth}
        \captionsetup[subfigure]{labelformat=empty,position=top,captionskip=-1pt,farskip=-0.5pt}
        \subfloat[Prob. no birth yet]{\includegraphics[width=\textwidth]{spell2_g3_low_urban_pps}}
        \captionsetup[subfigure]{labelformat=parens}
    \end{minipage}
}
\setcounter{subfigure}{2}
\subfloat[2005--2016]{
    \begin{minipage}{0.32\textwidth}
        \captionsetup[subfigure]{labelformat=empty,position=top,captionskip=-1pt,farskip=-0.5pt}
        \subfloat[Prob. no birth yet]{\includegraphics[width=\textwidth]{spell2_g4_low_urban_pps}}
        \captionsetup[subfigure]{labelformat=parens}
    \end{minipage}
}
\caption*{Rural}
\setcounter{subfigure}{3}
\subfloat[1972--1984]{
    \begin{minipage}{0.32\textwidth}
        \captionsetup[subfigure]{labelformat=empty,position=top,captionskip=-1pt,farskip=-0.5pt}
        \subfloat[Prob. no birth yet]{\includegraphics[width=\textwidth]{spell2_g1_low_rural_pps}} 
        \captionsetup[subfigure]{labelformat=parens}
    \end{minipage}
} 
\setcounter{subfigure}{4}
\subfloat[1985--1994]{
    \begin{minipage}{0.32\textwidth}
        \captionsetup[subfigure]{labelformat=empty,position=top,captionskip=-1pt,farskip=-0.5pt}
        \subfloat[Prob. no birth yet]{\includegraphics[width=\textwidth]{spell2_g2_low_rural_pps}}
        \captionsetup[subfigure]{labelformat=parens}
    \end{minipage}
} \\
\setcounter{subfigure}{5}
\subfloat[1995--2004]{
    \begin{minipage}{0.32\textwidth}
        \captionsetup[subfigure]{labelformat=empty,position=top,captionskip=-1pt,farskip=-0.5pt}
        \subfloat[Prob. no birth yet]{\includegraphics[width=\textwidth]{spell2_g3_low_rural_pps}}
        \captionsetup[subfigure]{labelformat=parens}
    \end{minipage}
}
\setcounter{subfigure}{6}
\subfloat[2005--2016]{
    \begin{minipage}{0.32\textwidth}
        \captionsetup[subfigure]{labelformat=empty,position=top,captionskip=-1pt,farskip=-0.5pt}
        \subfloat[Prob. no birth yet]{\includegraphics[width=\textwidth]{spell2_g4_low_rural_pps}}
        \captionsetup[subfigure]{labelformat=parens}
    \end{minipage}
}
\caption{Survival curves conditional on parity progression
for women with no education by month beginning 9 months after prior birth.
Left column shows results prior to sex selection available, middle column before
sex selection illegal and right column after sex selection illegal.
}
\label{fig:results_spell2_low_pps}
\end{figure}




% SPELL 2 - URBAN - MEDIUM

\input{../figures/appendix_spell2_med.tex}

% PARITY PROGRESSION SURVIVAL - MEDIUM


% Low education

\begin{figure}[htpb]
\centering
\caption*{Urban}
\setcounter{subfigure}{-1}
\subfloat[1972--1984]{
    \begin{minipage}{0.32\textwidth}
        \captionsetup[subfigure]{labelformat=empty,position=top,captionskip=-1pt,farskip=-0.5pt}
        \subfloat[Prob.\ no birth yet]{\includegraphics[width=\textwidth]{spell2_g1_med_urban_pps}} 
        \captionsetup[subfigure]{labelformat=parens}
    \end{minipage}
} 
\setcounter{subfigure}{-0}
\subfloat[1985--1994]{
    \begin{minipage}{0.32\textwidth}
        \captionsetup[subfigure]{labelformat=empty,position=top,captionskip=-1pt,farskip=-0.5pt}
        \subfloat[Prob. no birth yet]{\includegraphics[width=\textwidth]{spell2_g2_med_urban_pps}}
        \captionsetup[subfigure]{labelformat=parens}
    \end{minipage}
} \\
\setcounter{subfigure}{1}
\subfloat[1995--2004]{
    \begin{minipage}{0.32\textwidth}
        \captionsetup[subfigure]{labelformat=empty,position=top,captionskip=-1pt,farskip=-0.5pt}
        \subfloat[Prob. no birth yet]{\includegraphics[width=\textwidth]{spell2_g3_med_urban_pps}}
        \captionsetup[subfigure]{labelformat=parens}
    \end{minipage}
}
\setcounter{subfigure}{2}
\subfloat[2005--2016]{
    \begin{minipage}{0.32\textwidth}
        \captionsetup[subfigure]{labelformat=empty,position=top,captionskip=-1pt,farskip=-0.5pt}
        \subfloat[Prob. no birth yet]{\includegraphics[width=\textwidth]{spell2_g4_med_urban_pps}}
        \captionsetup[subfigure]{labelformat=parens}
    \end{minipage}
}
\caption*{Rural}
\setcounter{subfigure}{3}
\subfloat[1972--1984]{
    \begin{minipage}{0.32\textwidth}
        \captionsetup[subfigure]{labelformat=empty,position=top,captionskip=-1pt,farskip=-0.5pt}
        \subfloat[Prob. no birth yet]{\includegraphics[width=\textwidth]{spell2_g1_med_rural_pps}} 
        \captionsetup[subfigure]{labelformat=parens}
    \end{minipage}
} 
\setcounter{subfigure}{4}
\subfloat[1985--1994]{
    \begin{minipage}{0.32\textwidth}
        \captionsetup[subfigure]{labelformat=empty,position=top,captionskip=-1pt,farskip=-0.5pt}
        \subfloat[Prob. no birth yet]{\includegraphics[width=\textwidth]{spell2_g2_med_rural_pps}}
        \captionsetup[subfigure]{labelformat=parens}
    \end{minipage}
} \\
\setcounter{subfigure}{5}
\subfloat[1995--2004]{
    \begin{minipage}{0.32\textwidth}
        \captionsetup[subfigure]{labelformat=empty,position=top,captionskip=-1pt,farskip=-0.5pt}
        \subfloat[Prob. no birth yet]{\includegraphics[width=\textwidth]{spell2_g3_med_rural_pps}}
        \captionsetup[subfigure]{labelformat=parens}
    \end{minipage}
}
\setcounter{subfigure}{6}
\subfloat[2005--2016]{
    \begin{minipage}{0.32\textwidth}
        \captionsetup[subfigure]{labelformat=empty,position=top,captionskip=-1pt,farskip=-0.5pt}
        \subfloat[Prob. no birth yet]{\includegraphics[width=\textwidth]{spell2_g4_med_rural_pps}}
        \captionsetup[subfigure]{labelformat=parens}
    \end{minipage}
}
\caption{Survival curves conditional on parity progression
for women with 1-7 years of education by month beginning 9 months after prior birth.
Left column shows results prior to sex selection available, middle column before
sex selection illegal and right column after sex selection illegal.
N indicates the number of women in the relevant group in the underlying samples.
}
\label{fig:results_spell2_med_pps}
\end{figure}



% High education

\input{../figures/appendix_spell2_high.tex}

% PARITY PROGRESSION SURVIVAL - High education

\begin{figure}[htpb]
\centering
\caption*{Urban}
\setcounter{subfigure}{-1}
\subfloat[1972--1984]{
    \begin{minipage}{0.31\textwidth}
        \captionsetup[subfigure]{labelformat=empty,position=top,captionskip=-1pt,farskip=-0.5pt}
        \subfloat[Prob.\ no birth yet]{\includegraphics[width=\textwidth]{spell2_g1_high_urban_pps}} 
        \captionsetup[subfigure]{labelformat=parens}
    \end{minipage}
} 
\setcounter{subfigure}{-0}
\subfloat[1985--1994]{
    \begin{minipage}{0.31\textwidth}
        \captionsetup[subfigure]{labelformat=empty,position=top,captionskip=-1pt,farskip=-0.5pt}
        \subfloat[Prob. no birth yet]{\includegraphics[width=\textwidth]{spell2_g2_high_urban_pps}}
        \captionsetup[subfigure]{labelformat=parens}
    \end{minipage}
} \\
\setcounter{subfigure}{1}
\subfloat[1995--2004]{
    \begin{minipage}{0.31\textwidth}
        \captionsetup[subfigure]{labelformat=empty,position=top,captionskip=-1pt,farskip=-0.5pt}
        \subfloat[Prob. no birth yet]{\includegraphics[width=\textwidth]{spell2_g3_high_urban_pps}}
        \captionsetup[subfigure]{labelformat=parens}
    \end{minipage}
}
\setcounter{subfigure}{2}
\subfloat[2005--2016]{
    \begin{minipage}{0.31\textwidth}
        \captionsetup[subfigure]{labelformat=empty,position=top,captionskip=-1pt,farskip=-0.5pt}
        \subfloat[Prob. no birth yet]{\includegraphics[width=\textwidth]{spell2_g4_high_urban_pps}}
        \captionsetup[subfigure]{labelformat=parens}
    \end{minipage}
}
\caption*{Rural}
\setcounter{subfigure}{3}
\subfloat[1972--1984]{
    \begin{minipage}{0.31\textwidth}
        \captionsetup[subfigure]{labelformat=empty,position=top,captionskip=-1pt,farskip=-0.5pt}
        \subfloat[Prob. no birth yet]{\includegraphics[width=\textwidth]{spell2_g1_high_rural_pps}} 
        \captionsetup[subfigure]{labelformat=parens}
    \end{minipage}
} 
\setcounter{subfigure}{4}
\subfloat[1985--1994]{
    \begin{minipage}{0.31\textwidth}
        \captionsetup[subfigure]{labelformat=empty,position=top,captionskip=-1pt,farskip=-0.5pt}
        \subfloat[Prob. no birth yet]{\includegraphics[width=\textwidth]{spell2_g2_high_rural_pps}}
        \captionsetup[subfigure]{labelformat=parens}
    \end{minipage}
} \\
\setcounter{subfigure}{5}
\subfloat[1995--2004]{
    \begin{minipage}{0.31\textwidth}
        \captionsetup[subfigure]{labelformat=empty,position=top,captionskip=-1pt,farskip=-0.5pt}
        \subfloat[Prob. no birth yet]{\includegraphics[width=\textwidth]{spell2_g3_high_rural_pps}}
        \captionsetup[subfigure]{labelformat=parens}
    \end{minipage}
}
\setcounter{subfigure}{6}
\subfloat[2005--2016]{
    \begin{minipage}{0.31\textwidth}
        \captionsetup[subfigure]{labelformat=empty,position=top,captionskip=-1pt,farskip=-0.5pt}
        \subfloat[Prob. no birth yet]{\includegraphics[width=\textwidth]{spell2_g4_high_rural_pps}}
        \captionsetup[subfigure]{labelformat=parens}
    \end{minipage}
}
\caption{Survival curves conditional on parity progression
for women with 8 or more years of education by month beginning 9 months after prior birth.
Left column shows results prior to sex selection available, middle column before
sex selection illegal and right column after sex selection illegal.
N indicates the number of women in the relevant group in the underlying samples.
}
\label{fig:results_spell2_high_pps}
\end{figure}


\clearpage
\newpage

\subsection{Third Spell}

% low education
\input{../figures/appendix_spell3_low.tex}

% PARITY PROGRESSION SURVIVAL - LOW

\begin{figure}[htpb]
\centering
\caption*{Urban}
\setcounter{subfigure}{-1}
\subfloat[1972--1984]{
    \begin{minipage}{0.31\textwidth}
        \captionsetup[subfigure]{labelformat=empty,position=top,captionskip=-1pt,farskip=-0.5pt}
        \subfloat[Prob.\ no birth yet]{\includegraphics[width=\textwidth]{spell3_g1_low_urban_pps}} 
        \captionsetup[subfigure]{labelformat=parens}
    \end{minipage}
} 
\setcounter{subfigure}{-0}
\subfloat[1985--1994]{
    \begin{minipage}{0.31\textwidth}
        \captionsetup[subfigure]{labelformat=empty,position=top,captionskip=-1pt,farskip=-0.5pt}
        \subfloat[Prob. no birth yet]{\includegraphics[width=\textwidth]{spell3_g2_low_urban_pps}}
        \captionsetup[subfigure]{labelformat=parens}
    \end{minipage}
} \\
\setcounter{subfigure}{1}
\subfloat[1995--2004]{
    \begin{minipage}{0.31\textwidth}
        \captionsetup[subfigure]{labelformat=empty,position=top,captionskip=-1pt,farskip=-0.5pt}
        \subfloat[Prob. no birth yet]{\includegraphics[width=\textwidth]{spell3_g3_low_urban_pps}}
        \captionsetup[subfigure]{labelformat=parens}
    \end{minipage}
}
\setcounter{subfigure}{2}
\subfloat[2005--2016]{
    \begin{minipage}{0.31\textwidth}
        \captionsetup[subfigure]{labelformat=empty,position=top,captionskip=-1pt,farskip=-0.5pt}
        \subfloat[Prob. no birth yet]{\includegraphics[width=\textwidth]{spell3_g4_low_urban_pps}}
        \captionsetup[subfigure]{labelformat=parens}
    \end{minipage}
}
\caption*{Rural}
\setcounter{subfigure}{3}
\subfloat[1972--1984]{
    \begin{minipage}{0.31\textwidth}
        \captionsetup[subfigure]{labelformat=empty,position=top,captionskip=-1pt,farskip=-0.5pt}
        \subfloat[Prob. no birth yet]{\includegraphics[width=\textwidth]{spell3_g1_low_rural_pps}} 
        \captionsetup[subfigure]{labelformat=parens}
    \end{minipage}
} 
\setcounter{subfigure}{4}
\subfloat[1985--1994]{
    \begin{minipage}{0.31\textwidth}
        \captionsetup[subfigure]{labelformat=empty,position=top,captionskip=-1pt,farskip=-0.5pt}
        \subfloat[Prob. no birth yet]{\includegraphics[width=\textwidth]{spell3_g2_low_rural_pps}}
        \captionsetup[subfigure]{labelformat=parens}
    \end{minipage}
} \\
\setcounter{subfigure}{5}
\subfloat[1995--2004]{
    \begin{minipage}{0.31\textwidth}
        \captionsetup[subfigure]{labelformat=empty,position=top,captionskip=-1pt,farskip=-0.5pt}
        \subfloat[Prob. no birth yet]{\includegraphics[width=\textwidth]{spell3_g3_low_rural_pps}}
        \captionsetup[subfigure]{labelformat=parens}
    \end{minipage}
}
\setcounter{subfigure}{6}
\subfloat[2005--2016]{
    \begin{minipage}{0.31\textwidth}
        \captionsetup[subfigure]{labelformat=empty,position=top,captionskip=-1pt,farskip=-0.5pt}
        \subfloat[Prob. no birth yet]{\includegraphics[width=\textwidth]{spell3_g4_low_rural_pps}}
        \captionsetup[subfigure]{labelformat=parens}
    \end{minipage}
}
\caption{Survival curves conditional on parity progression
for women with no education by month beginning 9 months after prior birth.
Left column shows results prior to sex selection available, middle column before
sex selection illegal and right column after sex selection illegal.
N indicates the number of women in the relevant group in the underlying samples.
}
\label{fig:results_spell3_low_pps}
\end{figure}




% Medium education

\input{../figures/appendix_spell3_med.tex}

% PARITY PROGRESSION SURVIVAL - MEDIUM

\begin{figure}[htpb]
\centering
\caption*{Urban}
\setcounter{subfigure}{-1}
\subfloat[1972--1984]{
    \begin{minipage}{0.31\textwidth}
        \captionsetup[subfigure]{labelformat=empty,position=top,captionskip=-1pt,farskip=-0.5pt}
        \subfloat[Prob.\ no birth yet]{\includegraphics[width=\textwidth]{spell3_g1_med_urban_pps}} 
        \captionsetup[subfigure]{labelformat=parens}
    \end{minipage}
} 
\setcounter{subfigure}{-0}
\subfloat[1985--1994]{
    \begin{minipage}{0.31\textwidth}
        \captionsetup[subfigure]{labelformat=empty,position=top,captionskip=-1pt,farskip=-0.5pt}
        \subfloat[Prob. no birth yet]{\includegraphics[width=\textwidth]{spell3_g2_med_urban_pps}}
        \captionsetup[subfigure]{labelformat=parens}
    \end{minipage}
} \\
\setcounter{subfigure}{1}
\subfloat[1995--2004]{
    \begin{minipage}{0.31\textwidth}
        \captionsetup[subfigure]{labelformat=empty,position=top,captionskip=-1pt,farskip=-0.5pt}
        \subfloat[Prob. no birth yet]{\includegraphics[width=\textwidth]{spell3_g3_med_urban_pps}}
        \captionsetup[subfigure]{labelformat=parens}
    \end{minipage}
}
\setcounter{subfigure}{2}
\subfloat[2005--2016]{
    \begin{minipage}{0.31\textwidth}
        \captionsetup[subfigure]{labelformat=empty,position=top,captionskip=-1pt,farskip=-0.5pt}
        \subfloat[Prob. no birth yet]{\includegraphics[width=\textwidth]{spell3_g4_med_urban_pps}}
        \captionsetup[subfigure]{labelformat=parens}
    \end{minipage}
}
\caption*{Rural}
\setcounter{subfigure}{3}
\subfloat[1972--1984]{
    \begin{minipage}{0.31\textwidth}
        \captionsetup[subfigure]{labelformat=empty,position=top,captionskip=-1pt,farskip=-0.5pt}
        \subfloat[Prob. no birth yet]{\includegraphics[width=\textwidth]{spell3_g1_med_rural_pps}} 
        \captionsetup[subfigure]{labelformat=parens}
    \end{minipage}
} 
\setcounter{subfigure}{4}
\subfloat[1985--1994]{
    \begin{minipage}{0.31\textwidth}
        \captionsetup[subfigure]{labelformat=empty,position=top,captionskip=-1pt,farskip=-0.5pt}
        \subfloat[Prob. no birth yet]{\includegraphics[width=\textwidth]{spell3_g2_med_rural_pps}}
        \captionsetup[subfigure]{labelformat=parens}
    \end{minipage}
} \\
\setcounter{subfigure}{5}
\subfloat[1995--2004]{
    \begin{minipage}{0.31\textwidth}
        \captionsetup[subfigure]{labelformat=empty,position=top,captionskip=-1pt,farskip=-0.5pt}
        \subfloat[Prob. no birth yet]{\includegraphics[width=\textwidth]{spell3_g3_med_rural_pps}}
        \captionsetup[subfigure]{labelformat=parens}
    \end{minipage}
}
\setcounter{subfigure}{6}
\subfloat[2005--2016]{
    \begin{minipage}{0.31\textwidth}
        \captionsetup[subfigure]{labelformat=empty,position=top,captionskip=-1pt,farskip=-0.5pt}
        \subfloat[Prob. no birth yet]{\includegraphics[width=\textwidth]{spell3_g4_med_rural_pps}}
        \captionsetup[subfigure]{labelformat=parens}
    \end{minipage}
}
\caption{Survival curves conditional on parity progression
for women with 1 to 7 years of education by month beginning 9 months after prior birth.
Left column shows results prior to sex selection available, middle column before
sex selection illegal and right column after sex selection illegal.
N indicates the number of women in the relevant group in the underlying samples.
}
\label{fig:results_spell3_med_pps}
\end{figure}




% High education

\input{../figures/appendix_spell3_high.tex}

% PARITY PROGRESSION SURVIVAL - high

\begin{figure}[htpb]
\centering
\caption*{Urban}
\setcounter{subfigure}{-1}
\subfloat[1972--1984]{
    \begin{minipage}{0.31\textwidth}
        \captionsetup[subfigure]{labelformat=empty,position=top,captionskip=-1pt,farskip=-0.5pt}
        \subfloat[Prob.\ no birth yet]{\includegraphics[width=\textwidth]{spell3_g1_high_urban_pps}} 
        \captionsetup[subfigure]{labelformat=parens}
    \end{minipage}
} 
\setcounter{subfigure}{-0}
\subfloat[1985--1994]{
    \begin{minipage}{0.31\textwidth}
        \captionsetup[subfigure]{labelformat=empty,position=top,captionskip=-1pt,farskip=-0.5pt}
        \subfloat[Prob. no birth yet]{\includegraphics[width=\textwidth]{spell3_g2_high_urban_pps}}
        \captionsetup[subfigure]{labelformat=parens}
    \end{minipage}
} \\
\setcounter{subfigure}{1}
\subfloat[1995--2004]{
    \begin{minipage}{0.31\textwidth}
        \captionsetup[subfigure]{labelformat=empty,position=top,captionskip=-1pt,farskip=-0.5pt}
        \subfloat[Prob. no birth yet]{\includegraphics[width=\textwidth]{spell3_g3_high_urban_pps}}
        \captionsetup[subfigure]{labelformat=parens}
    \end{minipage}
}
\setcounter{subfigure}{2}
\subfloat[2005--2016]{
    \begin{minipage}{0.31\textwidth}
        \captionsetup[subfigure]{labelformat=empty,position=top,captionskip=-1pt,farskip=-0.5pt}
        \subfloat[Prob. no birth yet]{\includegraphics[width=\textwidth]{spell3_g4_high_urban_pps}}
        \captionsetup[subfigure]{labelformat=parens}
    \end{minipage}
}
\caption*{Rural}
\setcounter{subfigure}{3}
\subfloat[1972--1984]{
    \begin{minipage}{0.31\textwidth}
        \captionsetup[subfigure]{labelformat=empty,position=top,captionskip=-1pt,farskip=-0.5pt}
        \subfloat[Prob. no birth yet]{\includegraphics[width=\textwidth]{spell3_g1_high_rural_pps}} 
        \captionsetup[subfigure]{labelformat=parens}
    \end{minipage}
} 
\setcounter{subfigure}{4}
\subfloat[1985--1994]{
    \begin{minipage}{0.31\textwidth}
        \captionsetup[subfigure]{labelformat=empty,position=top,captionskip=-1pt,farskip=-0.5pt}
        \subfloat[Prob. no birth yet]{\includegraphics[width=\textwidth]{spell3_g2_high_rural_pps}}
        \captionsetup[subfigure]{labelformat=parens}
    \end{minipage}
} \\
\setcounter{subfigure}{5}
\subfloat[1995--2004]{
    \begin{minipage}{0.31\textwidth}
        \captionsetup[subfigure]{labelformat=empty,position=top,captionskip=-1pt,farskip=-0.5pt}
        \subfloat[Prob. no birth yet]{\includegraphics[width=\textwidth]{spell3_g3_high_rural_pps}}
        \captionsetup[subfigure]{labelformat=parens}
    \end{minipage}
}
\setcounter{subfigure}{6}
\subfloat[2005--2016]{
    \begin{minipage}{0.31\textwidth}
        \captionsetup[subfigure]{labelformat=empty,position=top,captionskip=-1pt,farskip=-0.5pt}
        \subfloat[Prob. no birth yet]{\includegraphics[width=\textwidth]{spell3_g4_high_rural_pps}}
        \captionsetup[subfigure]{labelformat=parens}
    \end{minipage}
}
\caption{Survival curves conditional on parity progression
for women with 8 or more years of education by month beginning 9 months after prior birth.
Left column shows results prior to sex selection available, middle column before
sex selection illegal and right column after sex selection illegal.
N indicates the number of women in the relevant group in the underlying samples.
}
\label{fig:results_spell3_high_pps}
\end{figure}




\clearpage
\newpage

\subsection{Fourth Spell}

% low education

\input{../figures/appendix_spell4_low.tex}

% PARITY PROGRESSION SURVIVAL - LOW

\begin{figure}[htpb]
\centering
\caption*{Urban}
\setcounter{subfigure}{-1}
\subfloat[1972--1984]{
    \begin{minipage}{0.31\textwidth}
        \captionsetup[subfigure]{labelformat=empty,position=top,captionskip=-1pt,farskip=-0.5pt}
        \subfloat[Prob.\ no birth yet]{\includegraphics[width=\textwidth]{spell4_g1_low_urban_pps}} 
        \captionsetup[subfigure]{labelformat=parens}
    \end{minipage}
} 
\setcounter{subfigure}{-0}
\subfloat[1985--1994]{
    \begin{minipage}{0.31\textwidth}
        \captionsetup[subfigure]{labelformat=empty,position=top,captionskip=-1pt,farskip=-0.5pt}
        \subfloat[Prob. no birth yet]{\includegraphics[width=\textwidth]{spell4_g2_low_urban_pps}}
        \captionsetup[subfigure]{labelformat=parens}
    \end{minipage}
} \\
\setcounter{subfigure}{1}
\subfloat[1995--2004]{
    \begin{minipage}{0.31\textwidth}
        \captionsetup[subfigure]{labelformat=empty,position=top,captionskip=-1pt,farskip=-0.5pt}
        \subfloat[Prob. no birth yet]{\includegraphics[width=\textwidth]{spell4_g3_low_urban_pps}}
        \captionsetup[subfigure]{labelformat=parens}
    \end{minipage}
}
\setcounter{subfigure}{2}
\subfloat[2005--2016]{
    \begin{minipage}{0.31\textwidth}
        \captionsetup[subfigure]{labelformat=empty,position=top,captionskip=-1pt,farskip=-0.5pt}
        \subfloat[Prob. no birth yet]{\includegraphics[width=\textwidth]{spell4_g4_low_urban_pps}}
        \captionsetup[subfigure]{labelformat=parens}
    \end{minipage}
}
\caption*{Rural}
\setcounter{subfigure}{3}
\subfloat[1972--1984]{
    \begin{minipage}{0.31\textwidth}
        \captionsetup[subfigure]{labelformat=empty,position=top,captionskip=-1pt,farskip=-0.5pt}
        \subfloat[Prob. no birth yet]{\includegraphics[width=\textwidth]{spell4_g1_low_rural_pps}} 
        \captionsetup[subfigure]{labelformat=parens}
    \end{minipage}
} 
\setcounter{subfigure}{4}
\subfloat[1985--1994]{
    \begin{minipage}{0.31\textwidth}
        \captionsetup[subfigure]{labelformat=empty,position=top,captionskip=-1pt,farskip=-0.5pt}
        \subfloat[Prob. no birth yet]{\includegraphics[width=\textwidth]{spell4_g2_low_rural_pps}}
        \captionsetup[subfigure]{labelformat=parens}
    \end{minipage}
} \\
\setcounter{subfigure}{5}
\subfloat[1995--2004]{
    \begin{minipage}{0.31\textwidth}
        \captionsetup[subfigure]{labelformat=empty,position=top,captionskip=-1pt,farskip=-0.5pt}
        \subfloat[Prob. no birth yet]{\includegraphics[width=\textwidth]{spell4_g3_low_rural_pps}}
        \captionsetup[subfigure]{labelformat=parens}
    \end{minipage}
}
\setcounter{subfigure}{6}
\subfloat[2005--2016]{
    \begin{minipage}{0.31\textwidth}
        \captionsetup[subfigure]{labelformat=empty,position=top,captionskip=-1pt,farskip=-0.5pt}
        \subfloat[Prob. no birth yet]{\includegraphics[width=\textwidth]{spell4_g4_low_rural_pps}}
        \captionsetup[subfigure]{labelformat=parens}
    \end{minipage}
}
\caption{Survival curves conditional on parity progression
for women with no education by month beginning 9 months after prior birth.
Left column shows results prior to sex selection available, middle column before
sex selection illegal and right column after sex selection illegal.
N indicates the number of women in the relevant group in the underlying samples.
}
\label{fig:results_spell4_low_pps}
\end{figure}




% Medium education

\input{../figures/appendix_spell4_med.tex}

% PARITY PROGRESSION SURVIVAL - MEDIUM


\begin{figure}[htpb]
\centering
\caption*{Urban}
\setcounter{subfigure}{-1}
\subfloat[1972--1984]{
    \begin{minipage}{0.31\textwidth}
        \captionsetup[subfigure]{labelformat=empty,position=top,captionskip=-1pt,farskip=-0.5pt}
        \subfloat[Prob.\ no birth yet]{\includegraphics[width=\textwidth]{spell4_g1_med_urban_pps}} 
        \captionsetup[subfigure]{labelformat=parens}
    \end{minipage}
} 
\setcounter{subfigure}{-0}
\subfloat[1985--1994]{
    \begin{minipage}{0.31\textwidth}
        \captionsetup[subfigure]{labelformat=empty,position=top,captionskip=-1pt,farskip=-0.5pt}
        \subfloat[Prob. no birth yet]{\includegraphics[width=\textwidth]{spell4_g2_med_urban_pps}}
        \captionsetup[subfigure]{labelformat=parens}
    \end{minipage}
} \\
\setcounter{subfigure}{1}
\subfloat[1995--2004]{
    \begin{minipage}{0.31\textwidth}
        \captionsetup[subfigure]{labelformat=empty,position=top,captionskip=-1pt,farskip=-0.5pt}
        \subfloat[Prob. no birth yet]{\includegraphics[width=\textwidth]{spell4_g3_med_urban_pps}}
        \captionsetup[subfigure]{labelformat=parens}
    \end{minipage}
}
\setcounter{subfigure}{2}
\subfloat[2005--2016]{
    \begin{minipage}{0.31\textwidth}
        \captionsetup[subfigure]{labelformat=empty,position=top,captionskip=-1pt,farskip=-0.5pt}
        \subfloat[Prob. no birth yet]{\includegraphics[width=\textwidth]{spell4_g4_med_urban_pps}}
        \captionsetup[subfigure]{labelformat=parens}
    \end{minipage}
}
\caption*{Rural}
\setcounter{subfigure}{3}
\subfloat[1972--1984]{
    \begin{minipage}{0.31\textwidth}
        \captionsetup[subfigure]{labelformat=empty,position=top,captionskip=-1pt,farskip=-0.5pt}
        \subfloat[Prob. no birth yet]{\includegraphics[width=\textwidth]{spell4_g1_med_rural_pps}} 
        \captionsetup[subfigure]{labelformat=parens}
    \end{minipage}
} 
\setcounter{subfigure}{4}
\subfloat[1985--1994]{
    \begin{minipage}{0.31\textwidth}
        \captionsetup[subfigure]{labelformat=empty,position=top,captionskip=-1pt,farskip=-0.5pt}
        \subfloat[Prob. no birth yet]{\includegraphics[width=\textwidth]{spell4_g2_med_rural_pps}}
        \captionsetup[subfigure]{labelformat=parens}
    \end{minipage}
} \\
\setcounter{subfigure}{5}
\subfloat[1995--2004]{
    \begin{minipage}{0.31\textwidth}
        \captionsetup[subfigure]{labelformat=empty,position=top,captionskip=-1pt,farskip=-0.5pt}
        \subfloat[Prob. no birth yet]{\includegraphics[width=\textwidth]{spell4_g3_med_rural_pps}}
        \captionsetup[subfigure]{labelformat=parens}
    \end{minipage}
}
\setcounter{subfigure}{6}
\subfloat[2005--2016]{
    \begin{minipage}{0.31\textwidth}
        \captionsetup[subfigure]{labelformat=empty,position=top,captionskip=-1pt,farskip=-0.5pt}
        \subfloat[Prob. no birth yet]{\includegraphics[width=\textwidth]{spell4_g4_med_rural_pps}}
        \captionsetup[subfigure]{labelformat=parens}
    \end{minipage}
}
\caption{Survival curves conditional on parity progression
for women with 1 to 7 years of education by month beginning 9 months after prior birth.
Left column shows results prior to sex selection available, middle column before
sex selection illegal and right column after sex selection illegal.
N indicates the number of women in the relevant group in the underlying samples.
}
\label{fig:results_spell4_med_pps}
\end{figure}



% High education

\input{../figures/appendix_spell4_high.tex}

% PARITY PROGRESSION SURVIVAL - high

\begin{figure}[htpb]
\centering
\caption*{Urban}
\setcounter{subfigure}{-1}
\subfloat[1972--1984]{
    \begin{minipage}{0.31\textwidth}
        \captionsetup[subfigure]{labelformat=empty,position=top,captionskip=-1pt,farskip=-0.5pt}
        \subfloat[Prob.\ no birth yet]{\includegraphics[width=\textwidth]{spell4_g1_high_urban_pps}} 
        \captionsetup[subfigure]{labelformat=parens}
    \end{minipage}
} 
\setcounter{subfigure}{-0}
\subfloat[1985--1994]{
    \begin{minipage}{0.31\textwidth}
        \captionsetup[subfigure]{labelformat=empty,position=top,captionskip=-1pt,farskip=-0.5pt}
        \subfloat[Prob. no birth yet]{\includegraphics[width=\textwidth]{spell4_g2_high_urban_pps}}
        \captionsetup[subfigure]{labelformat=parens}
    \end{minipage}
} \\
\setcounter{subfigure}{1}
\subfloat[1995--2004]{
    \begin{minipage}{0.31\textwidth}
        \captionsetup[subfigure]{labelformat=empty,position=top,captionskip=-1pt,farskip=-0.5pt}
        \subfloat[Prob. no birth yet]{\includegraphics[width=\textwidth]{spell4_g3_high_urban_pps}}
        \captionsetup[subfigure]{labelformat=parens}
    \end{minipage}
}
\setcounter{subfigure}{2}
\subfloat[2005--2016]{
    \begin{minipage}{0.31\textwidth}
        \captionsetup[subfigure]{labelformat=empty,position=top,captionskip=-1pt,farskip=-0.5pt}
        \subfloat[Prob. no birth yet]{\includegraphics[width=\textwidth]{spell4_g4_high_urban_pps}}
        \captionsetup[subfigure]{labelformat=parens}
    \end{minipage}
}
\caption*{Rural}
\setcounter{subfigure}{3}
\subfloat[1972--1984]{
    \begin{minipage}{0.31\textwidth}
        \captionsetup[subfigure]{labelformat=empty,position=top,captionskip=-1pt,farskip=-0.5pt}
        \subfloat[Prob. no birth yet]{\includegraphics[width=\textwidth]{spell4_g1_high_rural_pps}} 
        \captionsetup[subfigure]{labelformat=parens}
    \end{minipage}
} 
\setcounter{subfigure}{4}
\subfloat[1985--1994]{
    \begin{minipage}{0.31\textwidth}
        \captionsetup[subfigure]{labelformat=empty,position=top,captionskip=-1pt,farskip=-0.5pt}
        \subfloat[Prob. no birth yet]{\includegraphics[width=\textwidth]{spell4_g2_high_rural_pps}}
        \captionsetup[subfigure]{labelformat=parens}
    \end{minipage}
} \\
\setcounter{subfigure}{5}
\subfloat[1995--2004]{
    \begin{minipage}{0.31\textwidth}
        \captionsetup[subfigure]{labelformat=empty,position=top,captionskip=-1pt,farskip=-0.5pt}
        \subfloat[Prob. no birth yet]{\includegraphics[width=\textwidth]{spell4_g3_high_rural_pps}}
        \captionsetup[subfigure]{labelformat=parens}
    \end{minipage}
}
\setcounter{subfigure}{6}
\subfloat[2005--2016]{
    \begin{minipage}{0.31\textwidth}
        \captionsetup[subfigure]{labelformat=empty,position=top,captionskip=-1pt,farskip=-0.5pt}
        \subfloat[Prob. no birth yet]{\includegraphics[width=\textwidth]{spell4_g4_high_rural_pps}}
        \captionsetup[subfigure]{labelformat=parens}
    \end{minipage}
}
\caption{Survival curves conditional on parity progression
for women with 8 or more years of education by month beginning 9 months after prior birth.
Left column shows results prior to sex selection available, middle column before
sex selection illegal and right column after sex selection illegal.
N indicates the number of women in the relevant group in the underlying samples.
}
\label{fig:results_spell4_high_pps}
\end{figure}



\clearpage

\bibliographystyle{aer}
\bibliography{sex_selection_spacing}

\addcontentsline{toc}{section}{References}



\end{document}



