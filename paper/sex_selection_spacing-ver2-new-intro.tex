% Revamped version for Demography focusing on method

\documentclass[12pt,letterpaper]{article}

\usepackage{amsmath}
\usepackage{fontspec}
\setromanfont[Ligatures=TeX]{TeX Gyre Pagella}
\usepackage{unicode-math}
\setmathfont{TeX Gyre Pagella Math}
\usepackage[title]{appendix}
\usepackage[margin=1.0in]{geometry}
\usepackage[figuresleft]{rotating}
\usepackage[longnamesfirst]{natbib}
\usepackage{dcolumn}
\usepackage{booktabs}
\usepackage{multirow}
\usepackage[flushleft]{threeparttable}
\usepackage{setspace}
\usepackage[justification=centering]{caption}
\usepackage[font=scriptsize]{subfig}
\usepackage[xetex,colorlinks=true,linkcolor=black,citecolor=black,urlcolor=black]{hyperref}
\usepackage{adjustbox}
\usepackage{xfrac}


% \bibpunct{(}{)}{;}{a}{}{,}
\newcommand{\mco}[1]{\multicolumn{1}{c}{#1}}
\newcommand{\mct}[1]{\multicolumn{2}{c}{#1}}
\newcommand{\X}{$\times$ }
\newcommand{\hs}{\hspace{15pt}}

% Attempt to squeeze more floats in
\renewcommand\floatpagefraction{.9}
\renewcommand\topfraction{.9}
\renewcommand\bottomfraction{.9}
\renewcommand\textfraction{.1}
\setcounter{totalnumber}{50}
\setcounter{topnumber}{50}
\setcounter{bottomnumber}{50}


%------------------------------------------------------------------------


\title{Birth Spacing in the Presence of Son Preference and Sex-Selective Abortions:
India's Experience over Four Decades%
\protect\thanks{%
I am grateful to Andrew Foster and Darryl Holman for discussions about the method.
I owe thanks to Shelly Lundberg, Daniel Rees, David Ribar, 
Hendrik Wolff, seminar participants at University of Copenhagen, University of Michigan, 
University of Washington, University of Aarhus, the Fourth 
Annual Conference on Population, Reproductive Health, 
and Economic Development, and the Economic Demography Workshop for helpful 
suggestions and comments.
I would also like to thank Nalina Varanasi for research assistance.
Support for development of the method from the University of Washington Royalty 
Research Fund and the Development Research Group of the World Bank is gratefully 
acknowledged.
The views and findings expressed here are those of the author and
should not be attributed to the World Bank or any of its member countries.
Partial support for this research came from a Eunice Kennedy Shriver National
Institute of Child Health and Human Development research infrastructure grant,
5R24HD042828, to the Center for Studies in Demography and Ecology at the
University of Washington.
}
}

\author{Claus C P\"ortner\\
    Department of Economics\\
    Albers School of Business and Economics\\
    Seattle University, P.O. Box 222000\\
    Seattle, WA 98122\\
    \href{mailto:claus@clausportner.com}{\texttt{claus@clausportner.com}}\\
    \href{http://www.clausportner.com}{\texttt{www.clausportner.com}}\\
    \& \\
    Center for Studies in Demography and Ecology \\
    University of Washington\\ \vspace{2cm}
    }

\date{November 2018}


%------------------------------------------------------------------------


\begin{document}
\graphicspath{{../figures/}}
\DeclareGraphicsExtensions{.eps,.jpg,.pdf,.mps,.png}

\setcounter{page}{-1}
\maketitle
\thispagestyle{empty}

% \setcounter{page}{0}


\newpage
\thispagestyle{empty}
\doublespacing

\begin{abstract}

% Demography abstract
\noindent 

Strong son preference is typically associated with shorter birth spacing
in the absence of sons, but access to sex selection has the potential to
reverse this pattern because each abortion extends spacing by six to
twelve months. 
I introduce a statistical method that simultaneously
accounts for how sex selection increases the spacing between 
births and the likelihood of a son. 
Using four rounds of India's National
Family and Health Surveys, I show that, except for first births,
the spacing between births increased substantially over the last four
decades, with the most substantial increases among women most 
likely to use sex selection.
Specifically, well-educated women with no boys now
exhibit significantly longer spacing and more male-biased sex ratios
than similar women with boys. 
Women with no education still follow the standard
pattern of short spacing when they have girls and little evidence of sex
selection, with medium-educated women showing mixed results. 
Finally,
sex ratios are more likely to decline within spells at lower parities,
where there is less pressure to ensure a son, and more likely to
increase or remain consistently high for higher-order spells, where the
pressure to provide a son is high.


\noindent JEL: J1, O12, I1
\noindent Keywords: India, prenatal sex determination, censoring, competing risk
\end{abstract}

\newpage


%------------------------------------------------------------------------

\section{Introduction\label{sec:intro}}

Parents' spacing of births has long served as a measure of son preference 
\citep{Leung1988}.
Before prenatal sex determination became available, the only recourse for 
parents who wanted a son---but did not yet have one---was to have the next 
birth sooner.
Son preference is therefore often associated with shorter spacing after the 
births of girls than boys 
\citep{Das1987,Rahman1993,Pong1994,Haughton1996,Arnold1997,Soest2012,Rossi2015}.
Shorter spacing is, in turn, associated with worse health outcomes for girls 
and mothers 
\citep{arnold98,Conde-Agudelo2000,Whitworth2002,Razzaque2005,Rutstein2005,Conde-Agudelo2006}.%
\footnote{
Parents are also more likely to cease childbearing after the birth of 
a son than after a daughter 
\citep{repetto72,ben-porath76b,Das1987,Arnold1997,arnold98,clark00,dreze01,filmer09,Basu2010,Altindag2016}.
}

However, the introduction of prenatal sex determination fundamentally changed the 
relationship between son preference and birth spacing.
Couples, who had shorter birth spacing before because of son preference,
now have access to prenatal sex determination and sex-selective abortions,
and each abortion increases birth spacing by six months to a year.
The increase consists of three parts.
First, starting from the time of the abortion, the uterus needs at 
least two menstrual cycles to recover;  otherwise, the likelihood 
of spontaneous abortion increases substantially \citep{zhou00b}.
The second part is the waiting time to conception, which is between 
one and six months \citep{Wang2003}.
Finally, sex determination tests are reliable only from three months 
of gestation onwards.

As a result, we now have a situation where we may observer \emph{longer} spacing
for families with daughters than for families with sons, precisely because
strong son preference leads them to use sex selection.
Working in the opposite direction, couples with strong son preference
will likely still try to conceive earlier in the absence of sons. 
They may even shorten the time until conception, knowing that they might
have to go through multiple pregnancies and abortions before they
conceive a son.
To further complicate matters, we can still observe shorter birth spacing 
after the births of daughters as a representation of son preference for 
families who---for one reason or another---do not use prenatal sex selection.

Spacing, by itself, can therefore no longer be used as a direct measure of 
son preference, but understanding birth spacing remains a critical undertaking.
First, 
birth spacing is still useful in understanding son preference, 
if combined with the likelihood of observing a boy or a girl.
Second, 
the duration between births may be an important factor in parents' 
decisions, either for preference or economic reasons.
It is, for example, possible that even parents with strong son preference 
may reverse their decision to use prenatal sex determination---and carry 
the next pregnancy to term whether male or female---as the duration from 
the previous birth becomes sufficiently long.

At a broader level, we know less about what determines spacing behavior in 
developing countries than in developed countries.%
\footnote{
Even the effects of access to contraceptive on spacing are uncertain.
On the one hand, access to contraceptives allows women to avoid too 
short spacing between birth, which would increase birth spacing.
On the other hand, increased reliability of access and effectiveness of
contraceptives can lead to shorter spacing between births 
\citep{Keyfitz1971,Heckman1976}.
With less reliable contraception parents choose a higher level 
of contraception, which results in longer spacing, to avoid having 
too many children by accident.
But, as contraception becomes more effective parents can more
easily avoid future births, allowing them to reduce the spacing 
between births without having to worry about overshooting their 
preferred number of children.
This idea may explain why better-educated women have
shorter spacing than less educated women in some instances
 \citep{Tulasidhar1993,Whitworth2002}.
These two counteracting effects may explain why finding statistically 
significant effects of contraception use on birth spacing is 
difficult \citep{Yeakey2009}.
}
With increasing numbers of women entering the labor force in developing
countries, understanding how couples make timing decisions will be necessary
for the design of suitable policies \citep{Portner2018}.
Furthermore, if spacing does affect health outcomes for mother and children, it is 
essential to understand what drives changes in spacing.
We know, for example, that health outcomes for girls appear to improve
in the presence of sex selection \citep{Lin2014,Hu2015}.
It is possible that these improvements are an unintended side-effect of
the longer spacing that arises from sex-selective abortions, rather than
because the smaller number of girls makes them more valued as is often
assumed.


In this paper, I introduce and apply a novel empirical method that directly incorporates 
the effects of sex-selective abortions on 
the duration between births
\emph{and} 
the likelihood of a son. 
The method can be used to analyze both situations with and without prenatal
sex selection.
My proposed method allows for the time since the previous 
birth to affect the decision on sex selection.
By examining under what circumstances sex selection decisions change with 
spacing, we can draw a more nuanced picture of the degree of son preference.

% [Why India?]
I apply the method to birth histories of Hindu women, using data from 
India's National Family and Health Surveys (NFHS), covering the period 
1972 to 2016. 
India is a particularly compelling case.
On the one hand, India has seen dramatic increases in the males-to-females ratio 
at birth over the last three decades as access to prenatal sex determination 
expanded 
\citep{das_gupta97,Sudha1999,Arnold2002,retherford03b,jha06}.%
\footnote{
India is not alone; both China and South Korea saw
significant changes in the sex ratio at birth over the same period 
\citep{Yi1993,park95}.
}
On the other hand, research suggests that son preference in 
India, when measured as ideally having more boys than girls, is decreasing 
over time and with higher education \citep{bhat03,pande07}.
% %
% \footnote{
% This measure of son preference is commonly used in the literature. 
% See, for example, \citet{clark00}, \citet{Jensen2009}, and \cite{Hu2015}.
% }

There have also been substantial changes in access and legality of
prenatal sex determination in India over the period covered.
Abortion has been legal in India since 1971 and still is.
The first reports of sex determination appeared around 
1982--83 \citep{Sudha1999,bhat06,Grover2006}.
The number of clinics quickly increased, and knowledge about sex selection 
became widespread after a senior government official's wife aborted a 
fetus that turned out to be male \citep[p.\ 598]{Sudha1999}.
In 1994, the Central Government passed the Prenatal Diagnostic Techniques 
(PNDT) Act, making determining and communicating the sex of a fetus illegal.%
\footnote{
Details about the act are at \href{http://pndt.gov.in/}{http://pndt.gov.in/}.
The number of convictions has been low.
It took until January 2008 for the first state, Haryana, to reach five convictions.
Hence, private clinics apparently operate with little risk of legal action 
\citep{Sudha1999}.
Furthermore, there is little evidence that bans like this significantly
affect sex ratios \citep{Das-Gupta2016}.
}
% Hence, the data make it possible to show how the spacing between births and
% sex ratios have changed with the introduction of prenatal sex determination,
% and whether the ban affected the relationship between birth spacing and sex 
% ratios.


% [Results]
There are four main results.
First, there has been a general increase in the length of spacing between births
over the four decades covered by the data.
The exception is for first births, where the median duration has either
remained the same or slightly declined, although this hides a
significant compression of the variation in spell length.
Second, the most substantial increase in spacing is for the women who
are most likely to use sex selection.
Specifically, among the best-educated women, those with no boys now has 
significantly longer spacing---and a more male-biased sex ratio---than 
similar women with boys.
Third, women with no education still follow the standard pattern of
short spacing when they have girls and little evidence of the use of sex
selection.
In other words, these women adhere to a strategy where they achieve a
son through higher fertility rather than the use of sex selection.
Finally, sex ratios are the most likely to decline within spells at
lower-order spells, where the pressure to provide a son is smaller, and
are more likely to increase or remain consistently high for higher-order
spells, where the pressure to ensure a son is high.



\clearpage

\onehalfspacing
\bibliographystyle{aer}
\bibliography{sex_selection_spacing}

\addcontentsline{toc}{section}{References}




\end{document}



