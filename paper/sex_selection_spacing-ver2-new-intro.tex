% Revamped version for Demography focusing on method

\documentclass[12pt,letterpaper]{article}

\usepackage{amsmath}
\usepackage{fontspec}
\setromanfont[Ligatures=TeX]{TeX Gyre Pagella}
\usepackage{unicode-math}
\setmathfont{TeX Gyre Pagella Math}
\usepackage[title]{appendix}
\usepackage[margin=1.0in]{geometry}
\usepackage[figuresleft]{rotating}
\usepackage[longnamesfirst]{natbib}
\usepackage{dcolumn}
\usepackage{booktabs}
\usepackage{multirow}
\usepackage[flushleft]{threeparttable}
\usepackage{setspace}
\usepackage[justification=centering]{caption}
\usepackage[font=scriptsize]{subfig}
\usepackage[xetex,colorlinks=true,linkcolor=black,citecolor=black,urlcolor=black]{hyperref}
\usepackage{adjustbox}
\usepackage{xfrac}


% \bibpunct{(}{)}{;}{a}{}{,}
\newcommand{\mco}[1]{\multicolumn{1}{c}{#1}}
\newcommand{\mct}[1]{\multicolumn{2}{c}{#1}}
\newcommand{\X}{$\times$ }
\newcommand{\hs}{\hspace{15pt}}

% Attempt to squeeze more floats in
\renewcommand\floatpagefraction{.9}
\renewcommand\topfraction{.9}
\renewcommand\bottomfraction{.9}
\renewcommand\textfraction{.1}
\setcounter{totalnumber}{50}
\setcounter{topnumber}{50}
\setcounter{bottomnumber}{50}


%------------------------------------------------------------------------


\title{Birth Spacing in the Presence of Son Preference and Sex-Selective Abortions:
India's Experience over Four Decades%
\protect\thanks{%
I am grateful to Andrew Foster and Darryl Holman for discussions about the method.
I owe thanks to Shelly Lundberg, Daniel Rees, David Ribar, 
Hendrik Wolff, seminar participants at University of Copenhagen, University of Michigan, 
University of Washington, University of Aarhus, the Fourth 
Annual Conference on Population, Reproductive Health, 
and Economic Development, and the Economic Demography Workshop for helpful 
suggestions and comments.
I would also like to thank Nalina Varanasi for research assistance.
Support for development of the method from the University of Washington Royalty 
Research Fund and the Development Research Group of the World Bank is gratefully 
acknowledged.
The views and findings expressed here are those of the author and
should not be attributed to the World Bank or any of its member countries.
Partial support for this research came from a Eunice Kennedy Shriver National
Institute of Child Health and Human Development research infrastructure grant,
5R24HD042828, to the Center for Studies in Demography and Ecology at the
University of Washington.
}
}

\author{Claus C P\"ortner\\
    Department of Economics\\
    Albers School of Business and Economics\\
    Seattle University, P.O. Box 222000\\
    Seattle, WA 98122\\
    \href{mailto:claus@clausportner.com}{\texttt{claus@clausportner.com}}\\
    \href{http://www.clausportner.com}{\texttt{www.clausportner.com}}\\
    \& \\
    Center for Studies in Demography and Ecology \\
    University of Washington\\ \vspace{2cm}
    }

\date{November 2018}


%------------------------------------------------------------------------


\begin{document}
\graphicspath{{../figures/}}
\DeclareGraphicsExtensions{.eps,.jpg,.pdf,.mps,.png}

\setcounter{page}{-1}
\maketitle
\thispagestyle{empty}

% \setcounter{page}{0}


\newpage
\thispagestyle{empty}
\doublespacing

\begin{abstract}

% Demography abstract
\noindent 

Strong son preference is typically associated with shorter birth spacing
in the absence of sons, but access to sex selection has the potential to
reverse this pattern because each abortion extends spacing by six to
twelve months. 
I introduce a statistical method that simultaneously
accounts for how sex selection increases the spacing between 
births and the likelihood of a son. 
Using four rounds of India's National
Family and Health Surveys, I show that, except for first births,
the spacing between births increased substantially over the last four
decades, with the most substantial increases among women most 
likely to use sex selection.
Specifically, well-educated women with no boys now
exhibit significantly longer spacing and more male-biased sex ratios
than similar women with boys. 
Women with no education still follow the standard
pattern of short spacing when they have girls and little evidence of sex
selection, with medium-educated women showing mixed results. 
Finally,
sex ratios are more likely to decline within spells at lower parities,
where there is less pressure to ensure a son, and more likely to
increase or remain consistently high for higher-order spells, where the
pressure to provide a son is high.


\noindent JEL: J1, O12, I1
\noindent Keywords: India, prenatal sex determination, censoring, competing risk
\end{abstract}

\newpage


%------------------------------------------------------------------------

\section{Introduction\label{sec:intro}}


Birth spacing has long served as a measure of son preference, with strong
son preference typically associated with shorter birth intervals the
fewer sons a family has \citep{ben-porath76b,Leung1988}.
However, the introduction of prenatal sex determination has the potential to
fundamentally change the relationship between son preference and birth spacing.
The stronger the son preference, the more likely couples are to resort to sex-selective 
abortions, and each abortion automatically increases the interval between births by six 
months to a year.%
\footnote{
The increase consists of three parts.
First, starting from the time of the abortion, the uterus needs at 
least two menstrual cycles to recover;  otherwise, the likelihood 
of spontaneous abortion increases substantially \citep{zhou00b}.
The second part is the waiting time to conception, which is between 
one and six months \citep{Wang2003}.
Finally, sex determination tests are reliable only from three months 
of gestation onwards.
}
As a result, we may now observe \emph{longer} average spacing for families with stronger 
son preference, precisely because it makes them more likely to use sex selection.
To further complicate matters, families with son preference who---for one reason or 
another---do not use prenatal sex selection may continue to have shorter birth spacing 
than those with a lower preference for sons.

In this paper, I examine how birth spacing in India has changed over time and across 
groups with the introduction of sex selection.
I introduce and apply an empirical method that directly incorporates the effects of 
sex-selective abortions on the duration between births 
\emph{and} 
the likelihood of a son. 
The method can be used to analyze both situations with and without prenatal
sex selection.
I apply the method to birth histories of Hindu women, using data from the four
India's National Family and Health Surveys (NFHS), covering the period 
1972 to 2016. 

% [Why India?]
India is a particularly compelling case.
India has long shown a strong male preference, especially in the northern states 
\citep{Kishor1993,murthi95,arnold98}.%
\footnote{
The proportion of couples who ideally wants more boys than girls does, however,
appear to be decreasing over time and with higher education \citep{bhat03,pande07}.
}
As a result, a significant proportion of Indian families follow a son preference based 
differential stopping behavior, where, for a given number of children, they are more 
likely to have an additional child if they have not yet reached their preferred number of 
sons \citep{repetto72,Das1987,Arnold1997,arnold98,clark00,Basu2010,Barcellos2014}.%
\footnote{
A smilar pattern is found in many other countries 
\citep[see, for example][]{larsen98,filmer09,Altindag2016}.
}
Furthermore, higher mortality risk for females than males at most ages has lead to an 
almost continuous increase in India's overall ratio of males to females over the last 
century \citep{dyson01,Navaneetham2011,Bongaarts2015}.
Over the last three decades, India has also seen dramatic increases in the 
males-to-females ratio at birth as access to prenatal sex determination 
expanded \citep{das_gupta97,Sudha1999,Arnold2002,retherford03b,jha06,Guilmoto2012}.
In addition, fertility has declined substantially, to the point where it is now
at, or even below, replacement in some areas 
\citep{Guilmoto2013,Dharmalingam2014,International-Institute-for-Population-Sciences-IIPS2017}.
This decline in fertility, combined with the strong preference for sons, has lead to an 
intensified pressure to use sex selection \citep{Guilmoto2009a,Bongaarts2013,Jayachandran2017}.%
\footnote{
In some instances, however, the fertility decline appears to have created a stronger 
aversion to daughters rather than a stronger preference for son \citet{DiamondSmith2008}.
}
 
 
% [Why spacing?] 
There are four main reasons for examining birth spacing and how it changes with 
access to sex selection.
First, it has been widely used as a measure of son preference.
In India the sex of the last-born child significantly impact the duration to next birth,
with the expected birth interval about three percent longer if the last-born was son than
if it was a daughter \citep{Bhalotra2008,Kumar2016}.
Similarly, ethic Indians in South Africa also show a longer duration after the birth of a 
son than after a daughter \citep{Gangadharan2003}.%
\footnote{
Ethic Indians in Malaysia do, however, show little evidence of son preference, although
the sample sizes are very small \citep{Pong1994}.
}
India is not alone in sex composition affecting subsequent birth spacing.
In both Bangladesh and Vietnam the more boys a family has the longer the expected duration 
to the next birth \citep{Haughton1995,Haughton1996,Rahman1993,Soest2018}.
Outside Asia the evidence is more mixed.
North Africa shows shorter spacing in the absense of sons, while a similar pattern does
not exists in Sub-Saharan Africa \citep{Rossi2015}.%
\footnote{
A possible exception is Senegal, where women with a higher risk of widowhood show
shorter birth spacing until they have secured a son \citep{Lambert2016}.
}

Second, birth spacing can have substantial impact on children's outcomes.
Most of the research finds a negative impact of short spacing on child health, although 
identifying the causal mechanisms have proven more difficult
\citep{Conde-Agudelo2006,Conde-Agudelo2012}.
Broadly speaking, there are three pathways that birth intervals can affect health.
First, maternal depletion, where the mother does not have sufficient
time to recover after a pregnancy.%
\footnote{
This depletion can take the form of nutrition, folate, or cervical insufficiency.
}
Second, disease transmission both from mother to child and between children close in age.
Third, sibling competition, where siblings close in age compete for scarce resources and
parental care.
Which mechanism dominates is especially of interest when spacing decisions are driven by 
son preference.
Both the maternal depletion and the transmission of disease explanations predominately
affect the subsequent child and the mother.
Only sibling competition has the potential to negatively affect the older child.

In India, there is evidence that both the prior and the subsequent child have higher
likelihood of dying the closer they are spaced, although the effect is not symmetrical
\citep{Whitworth2002,Bhargava2003,Rutstein2005,Maitra2008,Makepeace2008}.%
\footnote{
\cite{Bhalotra2008}, however, find that birth spacing only explain a limited amount
of neonatal mortality.
}
The increased mortality risk for prior children may come from shorter breastfeeding of 
girls to ensure that the mother can conceive again quicker, and with poor water
quality this shorter breastfeeding may result both in higher risk of dying and shorter
stature if the child survives \citep{Jayachandran2011,Jayachandran2017a}.
In other countries, focus has been mostly on the subsequent birth, but there is evidence 
in countries as diverse as Bangladesh, Brazil, and El Salvadore that
short intervals between birth has negative health effects
\citep{Curtis1993,Davanzo2008,Gribble2009,Saha2013}.

The potential effects of spacing are not restricted to early outcomes, although
there is less evidence on longer term effects, and most of what we have comes from
Western countries.
In the US, for example, longer spacing between births increases the older sibling's
test scores on the Peabody Individual Achievement Test, although there is no
statistically significant effect on the younger sibling \citep{Buckles2012}.
In line with this, close spacing appears to increase the risk of dropping out of
high school and lower the probability of attending a post-secondary school in both the
US and Sweden \citep{Powell1993,Pettersson-Lidbom2009}.
These results are, however, not uniformly supported, with recent results from Sweden 
showing no relationship between birth spacing, even if very short, and outcomes such as 
years of education completed, earnings, and unemployment \citep{Barclay2017}.

Third, the mother may also be affected by the length of the birth intervals.
The effect of spacing on mother's outcomes have received less attention than the
effect on children, and reviews of the literature show mixed results on mother's 
anthropometric status \citep{Dewey2007,Conde-Agudelo2012}.
For India, there is, however, evidence that women with first-born girls are both
more likely to have short birth spacing and more likely to have anemia, possibly as
the result of the short spacing resulting in maternal depletion \citep{Milazzo2018}.
In Western countries, having a longer interval between births appear to have positive
effects on long-term labor market outcomes, such as participation and 
income \citep{Gough2017,Karimi2014}.



[Other reasons for studying spacing]
[Education and socio-economic status and spacing]
Finally, we know less about what determines spacing behavior in developing countries than 
in developed countries, and with increasing numbers of women entering the labor force in 
developing countries, understanding how couples make timing decisions will be necessary
for the design of suitable policies \citep{Portner2018}.



Even the effects of access to contraceptive on spacing are uncertain.
On the one hand, access to contraceptives allows women to avoid too 
short spacing between birth, which would increase birth spacing.
On the other hand, increased reliability of access and effectiveness of
contraceptives can lead to shorter spacing between births 
\citep{Keyfitz1971,Heckman1976}.
With less reliable contraception parents choose a higher level 
of contraception, which results in longer spacing, to avoid having 
too many children by accident.
But, as contraception becomes more effective parents can more
easily avoid future births, allowing them to reduce the spacing 
between births without having to worry about overshooting their 
preferred number of children.
This idea may explain why better-educated women have
shorter spacing than less educated women in some instances
 \citep{Tulasidhar1993,Whitworth2002}.
These two counteracting effects may explain why finding statistically 
significant effects of contraception use on birth spacing is 
difficult \citep{Yeakey2009}.


[how is spacing achieved?]
\citep{Rahman1993}
\citep{Jayachandran2011}
\citep{Alam2018}


Despite the potential important impact of birth spacing on outcomes for both
mother and children, there has so far been no examination of what effect access to
sex selection has had on spacing.
We know that health outcomes for girls appear to improve in the presence of sex 
selection \citep{Lin2014,Hu2015}.
It is possible that these improvements are an unintended side-effect of the longer spacing 
that arises from sex-selective abortions, rather than because the smaller number of girls 
makes them more valued as is often assumed.
A first step 


Direct information on the use of sex selection is not available, but it is possible
to compare different periods as a proxy since there has been substantial changes in access 
and legality of prenatal sex determination in India.
Abortion has been legal in India since 1971 and still is.
The first reports of sex determination appeared around 
1982--83 \citep{Sudha1999,bhat06,Grover2006}.
The number of clinics quickly increased, and knowledge about sex selection 
became widespread after a senior government official's wife aborted a 
fetus that turned out to be male \citep[p.\ 598]{Sudha1999}.
In 1994, the Central Government passed the Prenatal Diagnostic Techniques 
(PNDT) Act, making determining and communicating the sex of a fetus illegal.%
\footnote{
Details about the act are at \href{http://pndt.gov.in/}{http://pndt.gov.in/}.
The number of convictions has been low.
It took until January 2008 for the first state, Haryana, to reach five convictions.
Hence, private clinics apparently operate with little risk of legal action 
\citep{Sudha1999}.
Furthermore, there is little evidence that bans like this significantly
affect sex ratios \citep{Das-Gupta2016}.
}
[TK something on split 1995 to 2016 into two groups?]


[What do we know about use of sex selection?]



[OLD VERSION FROM HERE]


Spacing, by itself, can therefore no longer be used as a direct measure of 
son preference, but understanding birth spacing remains a critical undertaking.
First, 
birth spacing is still useful in understanding son preference, 
if combined with the likelihood of observing a boy or a girl.
Second, 
the duration between births may be an important factor in parents' 
decisions, either for preference or economic reasons.
It is, for example, possible that even parents with strong son preference 
may reverse their decision to use prenatal sex determination---and carry 
the next pregnancy to term whether male or female---as the duration from 
the previous birth becomes sufficiently long.




% [Results]
There are four main results.
First, there has been a general increase in the length of spacing between births
over the four decades covered by the data.
The exception is for first births, where the median duration has either
remained the same or slightly declined, although this hides a
significant compression of the variation in spell length.
Second, the most substantial increase in spacing is for the women who
are most likely to use sex selection.
Specifically, among the best-educated women, those with no boys now has 
significantly longer spacing---and a more male-biased sex ratio---than 
similar women with boys.
Third, women with no education still follow the standard pattern of
short spacing when they have girls and little evidence of the use of sex
selection.
In other words, these women adhere to a strategy where they achieve a
son through higher fertility rather than the use of sex selection.
Finally, sex ratios are the most likely to decline within spells at
lower-order spells, where the pressure to provide a son is smaller, and
are more likely to increase or remain consistently high for higher-order
spells, where the pressure to ensure a son is high.



\clearpage

\onehalfspacing
\bibliographystyle{aer}
\bibliography{sex_selection_spacing}

\addcontentsline{toc}{section}{References}




\end{document}



