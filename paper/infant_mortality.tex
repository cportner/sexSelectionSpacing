\section{Infant Mortality}

The prior literature show that shorter birth spacing, especially very short spacing,
is associated with higher risk of infant mortality.
Hence, it is possible that the increasing use of sex-selective abortions could
directly reduce infant mortality by lengthening the duration from the prior birth.

There are a number of potential issues with understanding to what extent this
has happened.

First, the sex of the next child is no longer random, and boys, in the absence of
differential treatment, have higher infant mortality than girls.

Second, it may be that what matters is not the time between births, but rather 
between pregnancies, in which case multiple abortions followed quickly by new 
pregnancies would lead to shorter real spells of recovery.
The implication is that long spells might have a higher mortality risk since
they are now more likely with multiple abortions.

Third, as I show above the use of sex selection is more prevalent the higher the
education level of the mother.
Higher maternal education is, however, also associated with lower mortality risk
to begin with.
On top of this, with economic growth and better health care, infant mortality
may have fallen across all education groups and areas in the period covered.

I address three questions.

First, does the increased use of sex selection lead to higher mortality risk?

Second, has the negative effects of short spacing changed over time?

Third, how do the effects of short spacing vary by education level in India?


There are a number of problems with estimating the relationship between 
sex-selective abortions and infant mortality.

First, there are potential selection problem.
For example, women who face higher mortality risk of their children might forego 
the use of sex selection to ensure that they reach their desired number of 
children.
If these women also know to avoid very short birth spacing, the precense of 
these women in the sample would inflate the mortality of the "middle" birth
spacing, relative to the very short or very long birth spacing.

[our JDE paper]

An additional selection with the opposite effect is that women who have
difficulties conceiving or carrying a preganancy to term may also have higher 
mortality risk for their offspring, leading to a supriuos correlation
between very long birth spacing and mortality \citep{Kozuki2013}. 
If better health care and higher income means that more of these women are
now able to carry their pregnancy to term we can end up with more births
after long spells that have higher mortality risk but not because of the
increased use of sex selection.


Given the data and that the focus of this paper is mainly on the birth
spacing, a more in-depth treatment of these issue is the topic of future work.
Part of the problem here is also that the method used previously, such
as family fixed effects does not work well when the number of births is
very low as it is for higher education women who use sex 
selection, especially since the spacing from marriage to first birth
does not provide any information \citep{Kozuki2013,Molitoris2019}.

Second, the actual number of abortions are unobserved.
The results above do show, however, how the use of sex selection vary
by education and sex composition of prior children.
I, therefore, estimate infant mortality risk taking into account the
sex composition of the prior children, as well as the index child.

Starting with the sample used for estimating birth spacing, I select, by
parity, children born more than 12 months before the survey month.
The dependent variable is whether the child died within the first twelve
months of life (infant mortality).
As above, I restrict the analysis to parity two through four.

The main set of explanatory variable consists of dummies for birth spacing,
divided into less than 15 months from when conception could realistically
could happen (same definition as above) and based on the finding that 
spacing less than 24 months between birth is associated with higher
infant mortality \citep{Molitoris2019}, and then in 12 months intervals
until TK, where all births after and until 96 months are included.%
\footnote{
\citet{Molitoris2019} has used quartic functions, but the concern is
that with the relatively small number sample here the shape
of the function would be strongly influenced by a small number of
observations.
Using dummies provides a more flexible specification and the standard
error will directly reflect the higher degree of noise for the
spells periods with fewer observations.
}
In addition, I use dummies for sex of the index child (the child whose 
mortality is estimated) and the sex composition of the prior children.
Birth spacing dummies, sex of index child, and sex composition dummies
are all interacted.
The other explanatory variables are the same as above and estimations
are done separately by education level.

I estimate probability of infant mortality using a logit model.
Results are presented as the predicted probability of dying within the
first year at the possible combinations of index child sex, sex composition
of prior children, and the birth spacing dummied.
All other variables are set to their average values.
[TK ehh, did I also do urban/rural?]

Figures TK to TK show the estimated probabilities.
To ease legibility all confidence intervals are suppressed
(graphs with confidence intervals are available on request).




