Birth spacing has long served as a measure of son preference, with strong
son preference typically associated with shorter birth intervals the
fewer sons a family has \citep{ben-porath76b,Leung1988}.
However, the introduction of prenatal sex determination potentially  
changed the relationship between son preference and birth spacing fundamentally.
The stronger the son preference, the more likely couples are to resort to sex-selective 
abortions, and each abortion naturally increases the interval between births.
As a result, we may now observe \emph{longer} average spacing for families with stronger 
son preference, precisely because it makes them more likely to use sex selection.
To further complicate matters, families with a preference for son who---for one reason or 
another---do not use prenatal sex selection may continue to have shorter birth spacing 
than those with a lower preference for sons.

In this paper, I examine how birth spacing in India has changed over time and across 
groups with the introduction of sex selection.
I introduce and apply an empirical method that directly incorporates the effects of 
sex-selective abortions on the duration between births 
\emph{and} 
the likelihood of a son. 
The method can be used to analyze both situations with and without prenatal
sex selection.
I apply the method to birth histories of Hindu women, using data from the four
India's National Family and Health Surveys (NFHS), covering the period 
1972 to 2016. 

India is a particularly compelling case.
India has long shown a strong preference for boys, especially in the northern 
states \citep{Kishor1993,murthi95,arnold98}.%
\footnote{
The proportion of couples who ideally wants more boys than girls does, however,
appear to be decreasing over time and with higher education \citep{bhat03,pande07}.
}
As a result, mortality risk is higher for females than males, leading to an almost
continuous increase in India's overall ratio of males to females over the last century 
\citep{dyson01,Navaneetham2011,Bongaarts2015}.
India has also seen dramatic increases in the males-to-females ratio  
at birth over the last three decades as access to prenatal sex determination 
expanded \citep{das_gupta97,Sudha1999,Arnold2002,retherford03b,jha06,Guilmoto2012}.
In addition, fertility has declined substantially, to the point where it is now
at, or even below, replacement in some areas 
\citep{Guilmoto2013,Dharmalingam2014,International-Institute-for-Population-Sciences-IIPS2017}.
This decline in fertility, combined with the strong preference for sons, has lead to an 
intensified pressure to use sex selection \citet{Guilmoto2009a,Bongaarts2013,Jayachandran2017}.%
\footnote{
In some instances, however, the fertility decline appears to have created a stronger 
aversion to daughters rather than a stronger preference for son \citet{DiamondSmith2008}.
}
 
There is amble evidence that son preference has affected fertility decisions in India and 
elsewhere.
It is, for example, clear that a significant proportion of Indian families follow a
son preference based differential stopping behavior, where, for a given number of
children, they are more likely to have an additional child if they have not yet
reached their preferred number of sons 
\citep{repetto72,Das1987,Arnold1997,arnold98,clark00,Basu2010,Barcellos2014}.
A smilar pattern is found in many other countries 
\citep[see, for example][]{larsen98,filmer09,Altindag2016}.%
\footnote{
One potential effect of this behavior is that girls tend to end up in larger families, 
which may have negative impacts on survival and human capital investments, even in 
the absence of direct discrimination \citet{Jensen2003,Kugler2017}.
}

[emphasize that birth spacing is a superset of DSB?]
Of particular interest here is, however, the potential for differential \emph{spacing} 
behavior, 
both because it has been used as a measure for son preference and because birth spacing 
may directly impact health and other outcomes for both children and mother.
In India the sex of the last-born child significantly impact the duration to next birth,
with the expected birth interval about three percent longer if the last-born was son than
if it was a daughter \citep{Bhalotra2008,Kumar2016}.
Similarly, ethic Indians in South Africa also show a longer duration after the birth of a 
son than after a daughter \citep{Gangadharan2003}.%
\footnote{
Ethic Indians in Malaysia do, however, show little evidence of son preference, although
the sample sizes are very small \citep{Pong1994}.
}
India is not alone in sex composition affecting subsequent birth spacing.
In both Bangladesh and Vietnam the more boys a family has the longer the expected duration 
to the next birth \citep{Haughton1995,Haughton1996,Rahman1993,Soest2018}.
Outside Asia the evidence is more mixed.
North Africa shows shorter spacing in the absense of sons, while a similar pattern does
not exists in Sub-Saharan Africa \citep{Rossi2015}.%
\footnote{
A possible exception is Senegal, where women with a higher risk of widowhood show
shorter birth spacing until they have secured a son \citep{Lambert2016}.
}

There is an even more substantial literature addressing the potential effects of birth 
spacing on health outcomes, but, although most of the research finds a negative impact of 
short spacing on health, identifying the causal mechanisms have proven more difficult
\citep{Conde-Agudelo2006,Conde-Agudelo2012}.
Broadly speaking, there are three pathways that birth intervals can affect health.
First, maternal depletion, where the mother does not have sufficient
time to recover after a pregnancy.%
\footnote{
This depletion can take the form of nutrition, folate, or cervical insufficiency.
}
Second, disease transmission both from mother to child and between children close in age.
Third, sibling competition, where siblings close in age compete for scarce resources and
parental care.
Which mechanism dominates is especially of interest when spacing decisions are driven by 
son preference.
Both the maternal depletion and the transmission of disease explanations predominately
affect the subsequent child and the mother.
Only sibling competition has the potential to negatively affect the older child.

In India, there is evidence that both the prior and the subsequent child have higher
likelihood of dying the closer they are spaced, although the effect is not symmetrical
\citep{Whitworth2002,Bhargava2003,Maitra2008,Makepeace2008}.%
\footnote{
\cite{Bhalotra2008}, however, find that birth spacing only explain a limited amount
of neonatal mortality.
In other countries, focus has been mostly on the subsequent birth, but there is evidence 
in countries as diverse as Bangladesh, Brazil, and El Salvadore that
short intervals between birth has negative health effects
\citep{Curtis1993,Davanzo2008,Gribble2009,Saha2013}.
}
The increased mortality risk for prior children may come from shorter breastfeeding of 
girls to ensure that the mother can conceive again quicker, and with poor water
quality this shorter breastfeeding may result both in higher risk of dying and shorter
stature if the child survives \citep{Jayachandran2011,Jayachandran2017a}.





[how is spacing achieved?]
This longer spacing 
\citep{Rahman1993}
\citep{Jayachandran2011}


[Education and socio-economic status and spacing]





[spacing and outcomes - children]

[spacing and outcomes - mothers]


[Other reasons for studying spacing]


[Predicted changes in spacing with sex selection]

What is common in the literature reviewed above it that the potential effects of 
sex selection on spacing has not been incorporated.

[how much longer intervals]
 by six months to a year.
The increase consists of three parts.
First, starting from the time of the abortion, the uterus needs at 
least two menstrual cycles to recover;  otherwise, the likelihood 
of spontaneous abortion increases substantially \citep{zhou00b}.
The second part is the waiting time to conception, which is between 
one and six months \citep{Wang2003}.
Finally, sex determination tests are reliable only from three months 
of gestation onwards.

 
