Birth spacing has long served as a measure of son preference, with strong
son preference typically associated with shorter birth intervals the
fewer sons a family has \citep{ben-porath76b,Leung1988}.
However, the introduction of prenatal sex determination has the potential to 
change the relationship between son preference and birth spacing fundamentally.
Some of the couples who before had shorter birth spacing because of son preference,
might instead resort to sex-selective abortions, and each abortion naturally 
increases the interval between births.
As a result, we may now observe \emph{longer} average spacing for families with strong son 
preference, precisely because it leads them to use sex selection.
To further complicate matters, families who---for one reason or another---do 
not use prenatal sex selection but still have a preference for son may continue to have 
shorter birth spacing than those with a lower preference for sons.

In this paper, I examine how birth spacing in India has changed over time and across 
groups with the introduction of sex selection.
I introduce and apply an empirical method that directly incorporates the effects of 
sex-selective abortions on the duration between births 
\emph{and} 
the likelihood of a son. 
The method can be used to analyze both situations with and without prenatal
sex selection.
I apply the method to birth histories of Hindu women, using data from the four
India's National Family and Health Surveys (NFHS), covering the period 
1972 to 2016. 

India is a particularly compelling case.
India has long shown a strong preference for boys, especially in the northern 
states \citep{Kishor1993,murthi95,arnold98}.%
\footnote{
The proportion of couples who ideally wants more boys than girls does, however,
appear to be decreasing over time and with higher education \citep{bhat03,pande07}.
}
As a result, mortality risk is higher for females than males, leading to an almost
continuous increase in India's overall ratio of males to females over the last century 
\citep{dyson01,Navaneetham2011,Bongaarts2015}.
India has also seen dramatic increases in the males-to-females ratio  
at birth over the last three decades as access to prenatal sex determination 
expanded \citep{das_gupta97,Sudha1999,Arnold2002,retherford03b,jha06,Guilmoto2012}.
In addition, fertility has declined substantially, to the point where it is now
at, or even below, replacement in some areas 
\citep{Guilmoto2013,Dharmalingam2014,International-Institute-for-Population-Sciences-IIPS2017}.
This decline in fertility, combined with the strong preference for sons, has lead to an 
intensified pressure to use sex selection \citet{Guilmoto2009a,Bongaarts2013,Jayachandran2017}.%
\footnote{
In some instances, however, the fertility decline appears to have created a stronger 
aversion to daughters rather than a stronger preference for son \citet{DiamondSmith2008}.
}
 
 
 
 
 
 
 
 
 


 
 
 families are more likely to cease childbearing after the birth of 
a son than after a daughter \citep{Das1987,Arnold1997,arnold98,clark00,dreze01,Basu2010}.

Parents are also more likely to cease childbearing after the birth of 
a son than after a daughter 
\citep{repetto72,ben-porath76b,Das1987,Arnold1997,arnold98,clark00,dreze01,filmer09,Basu2010,Altindag2016}.
}






My proposed method allows for the time since the previous 
birth to affect the decision on sex selection.
By examining under what circumstances sex selection decisions change with 
spacing, we can draw a more nuanced picture of the degree of son preference.





[how much longer intervals]
 by six months to a year.
The increase consists of three parts.
First, starting from the time of the abortion, the uterus needs at 
least two menstrual cycles to recover;  otherwise, the likelihood 
of spontaneous abortion increases substantially \citep{zhou00b}.
The second part is the waiting time to conception, which is between 
one and six months \citep{Wang2003}.
Finally, sex determination tests are reliable only from three months 
of gestation onwards.











[son preference and spacing - evidence]

[spacing and outcomes - children]

[spacing and outcomes - mothers]

[falling fertility and introduction of sex selection]

[Predicted changes in spacing]

[Other reasons for studying spacing]





 
