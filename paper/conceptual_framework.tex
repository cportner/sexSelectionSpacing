Second, the paper lacks a conceptual framework to organize (and
motivate) the analysis. Following on the comments of one reviewer, one
suggestion would be to focus on the role of women’s education, with a
more thorough description of the context of changing education in India
over the last half century and clearly lay out a framework for the
mechanisms through which birth spacing is affected by educational
expansion.

\section{Birth Spacing and Education: Mechanisms and Findings}

This section provides a conceptual framework for understanding what determines the  spacing 
between births.
I first discuss the main determinants of spacing discussed in the prior research, and 
then turn to how son preference and sex selection fits into the decisions on spacing.
Since sex selection and education are closely linked, I pay special attention to how education affects spacing.

In the simplest possible dynamic model of fertility, the spacing decision is trivial.
Assume that parents derive benefits from the presence of children, rather than from 
the births themselves, that spacing of children by itself does impact the parental utility, 
and that the parents perfectly control when to have children. 
In this model, the optimal decision is then to have the children as early as possible 
\citep{Newman1988}.
The simplest model does not fit the observed spacing patterns, which suggests 
that one or more of the assumptions in the simple model are wrong.

First, parents do not entirely control the timing of births, but rather decide on the
levels of sexual activity and contraceptive effort, which, in turn, affect the 
likelihood of conceiving.

Couples make decisions on sexual activity and contraception, taking into account fecundity, determined by factors such as age and health of the mother, and their knowledge of and access to contraceptive technology.
There is evidence from ``natural fertility'' populations that higher fertility is associated with shorter spacing and that birth intervals increase 
with parity \citep{Henry1961,Leridon1977}.

Age and health of the mother are, however, unlikely to be significant factors in changes
in spacing over time or across education levels in India for the period covered, since
there is little evidence that women's health and nutritional status substantially affect 
spacing, outside of direct famine conditions \citep{Huffman1987,John1987,lindstrom99}.
The Maharashtra drought occurred in 1970--1973 and would, therefore, only impact the first
year or two of the data.
Furthermore, I only examine spacing up to parity fourth, and the age of the mother is,
therefore, unlikely to be a concern when it comes to fecundity, even though better -educated women are, on average, older when they marry. 

Despite a general belief that better access to and knowledge of contraception should lead
to longer average spacing, the empirical relationship between contraception use and birth 
spacing is ambiguous \citep{Tulasidhar1993,Whitworth2002,Bhalotra2008,Yeakey2009,Kim2010,Soest2018}.
This lack of evidence is consistent with the ambiguous theoretical effect of introducing 
modern contraceptives on birth spacing.
On the one hand, in the absence of modern contraceptive technologies, parents may try to 
avoid overshooting their desired number of children by spacing their children further apart.
Increased reliability of the access to and effectiveness of contraceptives then lead to 
shorter intervals between births because parents are more confident of effectively
stopping childbearing when they reach their preferred number of children and having
the children closer together provide higher utility because parents benefit from the
presence of children for longer as discussed above \citep{Keyfitz1971,Heckman1976}.
On the other hand, when contraceptives are effective, we would expect some birth 
intervals to be shorter than ideal, as discussed below, through simple random
variation.
With increased reliability of the  access to and effectiveness of contraceptives, parents 
would better be able to avoid too short spacing between birth and average spacing
should go up.

To further complicate the relationship between contraception and spacing, the ability to 
successfully use ``low efficacy'' contraceptive methods, such as the rhythm method, is 
increasing in education, while there is no difference across education levels 
when it comes to modern contraceptives such as pills or injectables \citep{Rosenzweig1989}.
Furthermore, even in the absence of modern contraceptives, families appear able to time or 
at least postpone births when they have substantial enough incentive 
\citep{Jayachandran2011,Alam2018}.


Second, parents may care about the spacing between birth because of the potential effects of spacing on the children.%
\footnote{
In principle, the mother could also be affected, but the results
so are mixed on mothers' 
anthropometric status, and there appears to be no effect on mortality 
\citep{Ronsmans1998,Menken2003,Dewey2007,Conde-Agudelo2012}.
}
One theory is that child ``quality'' might increases with spacing 
because the amount of parental attention increases with spacing
\citep{Zajonc1975,Zajonc1976,Razin1980}.
In this case, birth spacing should increase in family income, while the effect of maternal 
education is ambiguous since higher education, on the one hand, makes the mother's time spent 
out of the labor market more costly and on the other hand increases how productive the 
mother is in childrearing. 
For child quality measures such as IQ and education, there is, to my knowledge, no evidence 
from developing countries.%
\footnote{
The evidence from developed countries is mixed
\citep{Powell1993,Pettersson-Lidbom2009,Buckles2012,Barclay2017}.
} 

For one crucial aspect of child quality---health and mortality---shorter spacing 
does lead to worse outcomes, especially for very short intervals of 24 months or less,
although this effect decreases with maternal education
\citep{Whitworth2002,Conde-Agudelo2006,Conde-Agudelo2012,Molitoris2019}.%
\footnote{
It is also possible the underlying mortality risk affects birth 
spacing.
The predicted outcome then---ignoring the effect of shorter spacing on mortality risk---would be to shorten 
birth spacing to get to the ideal number of children earlier.
There is little empirical evidence for this effect \citep{Newman1983,Newman1988,Bhalotra2008}.
}
Hence, if parents realize short birth spacing increases their children's mortality risk, 
they likely prefer to space births further apart.
However, since women with secondary or tertiary education show relatively low mortality 
risk for their children, even with very short spacing is associated, this effect should
diminish with education level.


Third, having and rearing children involves both direct and time costs, and economic 
factors can, therefore, affect birth spacing \citep{Hotz1997,Schultz1997}.
I divide the economic factors into the effects of price and income shocks,
the expected income trajectory of the husband and wife,
and economies of scale considerations.

The expected effect of economies of scale in childrearing, which can arise, for example, because
childbearing and rearing are time-intensive, but taking care of two young children does not take 
twice as much time as taking care of one child, is shorter intervals between births
\citep{Vijverberg1982,Espenshade1984}.
To the extent that children require the mother to reduce her participation 
in productive activities, parents can lower the cost of children by shortening birth spacing, 
thereby reducing the amount of time the mother is away from productive activities.
In this case, women with higher education will both have lower fertility and shorter
spacing than women with lower education because of the higher opportunity cost of time 
\citep{Ross1974,Newman1981,Newman1984}.
If the mother mainly works at home, there is less incentive to space children for this 
reason and, similarly, higher household income, holding the mother's income constant, 
reduces the incentive to space births for this reason.

In the absence of perfect capital markets, theory predicts that the steeper the parents'
lifetime income profile, the longer the spacing between births, which suggests that 
with economic development we should see a general increase in spacing across education 
groups and that this increase in spacing should be stronger with increasing education levels
\citep{Heckman1976,Wolpin1984,Newman1988}.
The idea is that although parents would prefer to have their children as early as possible 
to enjoy them for longer, there is also a direct cost to children.
Early on income is low, which means that the marginal utility of consumption is higher, 
making parents less willing to give up consumption now to have a child, compare to later 
when they know that their income will be higher \citep{Newman1984,Happel1984}.


The income profile argument assumes that parents know their future income, but income and price shocks may also affect spacing. A temporary, higher cost of women's time
and other inputs into children together with lower income temporarily decreasing fertility 
and, thereby, increase the spacing between births \citep{Moffitt1984,Hotz1988,Portner2001,Alam2018}.
Increasing education likely reduces the risk of income shocks, but the effect on other types of shocks that might impact the spacing is unclear.


Finally, the discussion has so far assumed that children are identical---except for any 
potential differences that arise from spacing---but there is substantial 
evidence that the sex composition of previous children affects birth spacing, with the
absence of boys significantly decreasing the duration to next birth
\citep{Haughton1995,Haughton1996,Rahman1993,Bhalotra2008,Kumar2016,Soest2018}.%
\footnote{
There is also evidence of differential stopping behavior, with an additional child more 
likely the fewer sons a family has 
\citep{repetto72,Das1987,Arnold1997,arnold98,clark00,Basu2010,Barcellos2014}.
}

It is not, however, immediately apparent that we should expect shorter spacing in the absence 
of boys, if, as discussed above, short spacing lowers the survival probability of the next child.
There are two possibilities for why parents might still try to conceive sooner in the 
absence of boys.%
\footnote{
The fact that there are significant differences in spacing by sex composition indicates that
parents have---at least some---control over the timing of birth.
Hence, it is unlikely that the variations in spacing come about by random chance.
In Matlab, the effect of son preference on birth spacing is stronger in areas with better 
access to family planning \citep{Rahman1993}.
}
One possibility is that parents do not realize that short spacing is potentially
detrimental to the next child's health, resulting in high mortality for both boys and girls 
born after short spacing across all households whether or not they have a son preference.%
\footnote{
We may even see a higher likelihood of dying for sons than girls, given that boys 
have naturally higher mortality.
}

The other possibility is that parents realize that short spacing has adverse health effects, 
but their son-preference is so strong that they are willing to divert substantial resources 
to the next child's health if it is a boy \emph{and} that these investments can 
compensate for the adverse effects.
In this scenario, we should see substantial differences in mortality across sex for short spacing 
children in the absence of any other boys.
If a son is born next, his mortality risk should be substantially lower than expected for 
the spacing duration, and, because this argument relies on redistribution of resources, we should 
expect to see worse outcomes for all the other children (all girls) in the family.
If a girl is born next, there are two possible outcomes.
If parents treat all children/girls the same, we should see a slight worsening in health 
outcomes because fewer resources are available per child; 
if they do not, the last girl will have a much lower likelihood of surviving because the 
parents' incentive is to invest in their prior children rather than the new child.

As mentioned above, the adverse effects of shorter spacing are strongly mitigated by higher 
education, which, in theory, allows better-educated women
to have shorter spacing in the 
absence of a boy without substantial increasing mortality risk.
Differences in mortality outcomes across boys and girls are still possible,
but should not be as large as for less-educated women.

With the introduction of sex selection, parents can avoid the birth of girls, although at the cost of the procedure and having to wait longer until 
the next child if they abort the fetus.
The question is, who is likely to use sex selection, when, and why?

Theory suggests that the lower the desired fertility, the 
higher the likelihood of sex selection \citep{Portner2015b}.
Simple probabilities drive this prediction:
in the absence of sex selection, 
48.8 percent will not have a son when only having one child, 
23.8 percent for two children, 11.6 percent for three children, 
5.7 percent for four children, 2.8 percent for five children, 
and 1.4 percent for six children.
Furthermore, for a given level of desired fertility, the higher
the parity, the higher is the likelihood of using sex selection.

The higher the cost of prenatal sex determination and abortion,
the less use we should see.
Similarly, any potential adverse effects on the mother and 
offspring from repeated
abortions would also tend to reduce the use of sex selection---
provided that parents are aware of these costs.
Finally, the more abortions between births, the less the opportunity there is to take
advantage of any economies of scale in childrearing.

The use of sex selection in India increases with education and is higher in urban
areas than in rural
\citep{das_gupta97,retherford03b,jha06,Guilmoto2009a,Bongaarts2013,Portner2015b,
Jayachandran2017}.
Women with more education generally have lower desired fertility, either because of the
higher opportunity cost of time or because of other changes in the preferred number of
children.%
\footnote{
It is also possible that the effect is driven by a stronger son preference, although 
higher education women have shown declining stated son preference \citep{bhat03,pande07}.
}
Better-educated women also tend to live in households with higher 
income, which better enables them to access prenatal sex determination and lowers the relative
costs of using sex selection.

Given the increase in birth spacing with each abortion, parents may try to conceive even sooner to compensate for the potentially longer spacing and, thereby, still take advantage of economies of scale in childrearing.
By conceiving sooner, they are more likely to be able to take advantage of economies of
scale in childrearing, even if they have to go through one or more abortions.
Furthermore, if health outcomes for short spacing births has improved over time---as is 
likely---we should see an even stronger push toward shorter spacing for women using sex 
selection.

In summary, birth spacing is affected by female education through 
multiple pathways, with the main one of interest here the use of 
sex selection. But, to fully understand how sex selection affects 
spacing, we also need to understand the other changes that affect 
birth spacing over time.
The most likely factors are the changes in household income and 
wealth and the improvements in health as India has developed.
I, therefore, now look at how female education and labor force 
participation has changed over time in India.


