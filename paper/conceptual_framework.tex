Second, the paper lacks a conceptual framework to organize (and
motivate) the analysis. Following on the comments of one reviewer, one
suggestion would be to focus on the role of women’s education, with a
more thorough description of the context of changing education in India
over the last half century and clearly lay out a framework for the
mechanisms through which birth spacing is affected by educational
expansion.

\section{Birth Spacing and Education: Mechanisms and Findings}

This section provides a conceptual framework for understanding what determines spacing 
between births.
Since education is closely linked to the use of sex selection, I focus especially on the
the effects of education on spacing.
I first discuss generic spacing decisions frameworks, and then turn to how son preference
and sex selection fits into the decisions on spacing.

In the simplest possible dynamic model of fertility there are no real spacing decisions.
Assume that parents derive benefits from having their children around, rather than from 
the births themselves, that spacing of children by itself does impact parental utility, 
and that the parents perfectly control when to have children. 
In this set up, the optimal decision is then to have the children as early as possible 
\citep{Newman1988}.
The simplest model clearly does not fit the observed spacing patterns, but the question 
is why.

First, parents do not perfectly control the timing of births, but instead decide on their
levels of sexual activity and contraceptive effort, which, in turn, affect their likelihood 
of conceiving.

The decisions on sexual activity and contraceptive effort are constrained by biological 
factors, such as fecundity, age, and health of the mother, and the knowledge of and access
to contraceptive technology.
There is evidence from ``natural fertility'' populations that increasing number of 
children is associated with shorter spacing and that birth intervals increase 
with parity \citep{Henry1961,Leridon1977}.

Biological factors are unlikely to be an important factor in differences in spacing over 
time or across education levels in India for the period covered.
Outside of direct famine conditions, there is little evidence that women's health and 
nutritional status substantially affects spacing \citep{Huffman1987,John1987,lindstrom99}.
The Maharashtra drought occurred in 1970-1973 and, therefore, would only affect the first
year or two of the data.
Hence, although there are still differences in individual fecundity across women, it is
unlikely that we should see systematic differences across education groups and over time
in the data because of health status.
Furthermore, I only examine spacing up to parity fourth, and age of the mother is,
therefore, unlikely to be a concern when it comes to fecundity, when though better 
educated women are, on average, older when they marry. 

TK contraception

There is no clear relationship between contraception use and birth spacing in the 
literature, and better-educated women have shorter spacing than less educated women in 
some instances but not in others 
\citep{Tulasidhar1993,Whitworth2002,Bhalotra2008,Yeakey2009,Kim2010,Soest2018}.


Even the effects of access to contraceptive and declining fertility on spacing are unclear.
On the one hand, access to contraceptives allows women to avoid too short spacing between 
birth.
On the other hand, increased reliability of access and effectiveness of contraceptives can 
lead to shorter intervals between births if women used to have longer spacing to avoid
having too many children by accident \citep{Keyfitz1971,Heckman1976}.
%
\footnote{
In Matlab, the effect of son preference on birth spacing is stronger in areas with better 
access to family planning \citep{Rahman1993}.
Other evidence, however, points to families ability to time births, even in the absence
of modern contraceptives \citep{Jayachandran2011,Alam2018}.
}





Second, parents possibly care about the spacing between birth either in its own right or, 
more likely, because of the potential effects of spacing on the children and the mother.

It has, for example, been suggested that child ``quality'' might increases with spacing 
because the amount of parental attention increases with spacing
\citep{Zajonc1975,Zajonc1976,Razin1980}.
In this case, birth spacing should increase in family income, while the effect of maternal 
education is ambiguous since higher education on one hand makes the mother's time spent 
out of the labor market more costly and on the other hand increases how productive the 
mother is in childrearing. [TK Behrman et al. paper on mother's education in India]


Most of the research finds a negative impact of short spacing on child health, especially 
for very short intervals of 24 months or less 
\citep{Conde-Agudelo2006,Conde-Agudelo2012,Molitoris2019}.
There is evidence in countries as diverse as Bangladesh, Brazil, El Salvadore, India,
and Pakistan of worse health outcomes and a higher mortality risk the shorter the birth
interval
\citep{Cleland1984,Curtis1993,Whitworth2002,Bhargava2003,Rutstein2005,Bhalotra2008,Davanzo2008,Maitra2008,Makepeace2008,Gribble2009,Jayachandran2011,Saha2013,Jayachandran2017a,Ghosh2018}.
The effects of short spacing on child mortality are, however, strongly moderated by
maternal education, and even very short spacing has relatively low mortality for women
with secondary or tertiary education \citep{Molitoris2019}.

The mother may also be affected by the length of the birth intervals, although the
results on mother's anthropometric status are mixed \citep{Dewey2007,Conde-Agudelo2012}.
For India, there is, however, evidence that women with first-born girls are more likely to 
have anemia, possibly because short birth spacing can resulting in maternal depletion 
\citep{Milazzo2018}.











Hence, the determinants of the durations between births are an amalgam of biological 
factors, economic considerations, and individual preferences.






Education affects a number of factors related to birth spacing.
First, it affects the cost of time and the associated income profile over time.
Second, it affects the cost of using contraceptives (independently of the lower fertility
and changes in spacing). This could also include how good women are to use new 
contraceptive technology \citep{Kim2010}.
[I find this argument weird. \citep{Rosenzweig1989} shows that better educated women
are better at using more difficult ``technologies'' but those do not include most common
forms of modern contraceptives such as pills or injectables.]


First, if parents realize that having children close together increase the children's
mortality risk, they would prefer to space births further apart to lower mortality risk
(although possibly not very long since the mortality risk involves a trade-off between the
cost of births and the benefit of having a child).
Second, if there is a risk of ``overshooting'' the ideal number of children because of
imperfect fertility control and overshooting is sufficiently costly.
Parents would select a higher level of contraception than if they had perfect control and,
consequently, space children far apart on average [this is presumably
the \citep{Keyfitz1971,Heckman1976} arguments]
Third, if capital markets are not perfect and giving birth or rearing a child is costly
parents are forced to wait until they have enough income/wealth to have the next child
(this works through consumption; with low levels of consumption the cost of giving up
any will be higher so parents will want to wait until they have higher income and the
marginal cost of giving up consumption will be lower).
[would the opportunity cost of time work in the opposite direction? If women's 
opportunity cost of time goes up over time it would make sense to have the children
earlier]

Is there an expected mortality effect here as well?
If you expect a certain proportion of your children to die, it would probably make sense
to have your births earlier, everything else equal.
That way, you would get to the ideal number of children earlier.

Effect of son preference on birth spacing?
Do we have any theoretical results on this?

Biological/health effects on spacing that might change with education (or rather the
health improvements that come with higher education/income either from the women
herself or her husband)?
As better health leads to shorter post-partum amenorrhoea, we should expect shorter
spacing rather than longer with better health.
The evidence suggests that there is relatively little effect of health status on 
fecundity once nutrition is above famine or starvation levels \citep{Huffman1987,John1987,lindstrom99}


Breastfeeding would be part of this.
An healthier mother is more likely to successfully breastfeed and that would increase the
spacing between births.



Steeper income profile over time results in longer spacing in the absence of perfect
capital markets.
There are two effects of this theoretical results.
First, we should see an general increase in spacing across education groups because
development lead to a steeper income profile.
[This does require that economic development is associated with steeper income profiles;
not sure if anybody looked at this.
The lower fertility even for low education women does suggest that income is higher,
but does not, by itself, say anything about the age profile in income.]
Second, the increase in spacing should be stronger the higher the education level is.
In other words, we should see a general increase in spacing, and this increase should be
longer the higher the education level.






[Notes/lit review from paper]

Women with different education levels have different hazard profiles, although the size 
and direction of the effect vary across areas and time 
\citep{Tulasidhar1993,Whitworth2002,Bhalotra2008,Kim2010,Soest2018}.
Furthermore, the use of sex selection increases with education, either because of the 
associated lower fertility or because higher income enables them to access and
afford prenatal sex determination
\citep{das_gupta97,retherford03b,jha06,Guilmoto2009a,Bongaarts2013,Portner2015b,Jayachandran2017}.%
\footnote{
It is also possible that the effect is driven by a stronger son preference, although 
higher education women have shown declining son preference \citep{bhat03,pande07}.
}


The sex composition of previous children affects both the timing of births and the use of 
sex-selective abortions 
\citep{retherford03b,jha06,Bhalotra2008,abrevaya09,Portner2015b,Kumar2016,Soest2018}.
I capture sex composition with dummy variables for the
possible combinations for the specific spell, ignoring the ordering of births.
Finally, urban women have both lower fertility and, presumably, better access to
sex selection, both of which lead to higher use of sex-selective abortions than in 
rural areas \citep{retherford03b,jha06,Portner2015b}.


The longer average spacing and the relatively higher number of very long birth intervals 
seem to run counter to a general reduction in the length of women's reproductive spans 
found in earlier research \citep{Padmadas2004}.


Differential stopping behavior is widespread, with an additional child more likely the
fewer sons a family has 
\citep{repetto72,Das1987,Arnold1997,arnold98,clark00,Basu2010,Barcellos2014}.%
\footnote{
Other countries show a similar pattern
\citep[see, for example,][]{larsen98,filmer09,Altindag2016}.
}


For example, in Bangladesh, India, and Vietnam, having more boys, and particularly having 
a boy as the last-born child, significantly increase the duration to next birth,
\citep{Haughton1995,Haughton1996,Rahman1993,Bhalotra2008,Kumar2016,Soest2018}.%
\footnote{
Ethnic Indians in South Africa also show a longer duration after the birth of a 
son than after a daughter \citep{Gangadharan2003}.
}
Outside Asia, the evidence is more mixed, with North Africa showing shorter spacing in the 
absence of sons, while a similar pattern does not exist in Sub-Saharan Africa 
\citep{Rossi2015}.


Finally, we know less about what determines spacing behavior in developing countries than 
in developed countries, and with declining fertility and increasing numbers of women 
entering the labor force, understanding how couples make timing decisions is necessary 
for the design of suitable policies \citep{Portner2018}.%
\footnote{
It is clear that birth spacing does respond to policies in both developed and 
developing countries \citep{Pettersson-Lidbom2009,Todd2012,Meckel2015,Ghosh2018}.
}


\subsection{How Education Has Changed in India}

son preference and investment in female education (Behrman et al. paper)

labor force participation of women in India

returns to education in India (men and women)

[Notes/lit review paper from paper]

[Update with numbers from household module instead of individual recode using cohorts]

Finally, the population has become substantially better educated over time.
In the first period, almost 60\% of women had no education, and only twenty percent had
eight or more years, while in the last period the reverse was the case.
These changes underestimate the increase in education because many of the younger women 
with more education are not yet married and therefore not in the sample.%
\footnote{
Women with eight or more years of education accounted for 65.9\% of
unmarried women in NFHS-3 and 82.8\% of unmarried women in NFHS-4
\citep{International-Institute-for-Population-Sciences-IIPS2007,International-Institute-for-Population-Sciences-IIPS2017}.
}

% \citep[p 56]{International-Institute-for-Population-Sciences-IIPS2007}
% and \citet[p 61]{International-Institute-for-Population-Sciences-IIPS2017}
% for more information.

% Age at marriage increased over time across all three education groups.
% The most substantial increase was for women without any education, where
% the average went from below 16 years of age to 18.5.
% The smallest change is for the most educated women where the average age
% at marriage went up by less than a year---from 19.6 to 20.4---across the 
% four periods.

What factors have changed over time:
    - education levels / desired fertility
    - access to contraceptives
    - return to women's education
    - access to sex selection
    - breastfeeding patterns (?)
    - labor force participation (?)


