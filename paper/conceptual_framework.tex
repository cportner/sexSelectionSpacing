Second, the paper lacks a conceptual framework to organize (and
motivate) the analysis. Following on the comments of one reviewer, one
suggestion would be to focus on the role of women’s education, with a
more thorough description of the context of changing education in India
over the last half century and clearly lay out a framework for the
mechanisms through which birth spacing is affected by educational
expansion.

\section{Conceptual Framework}

% How education has changed in India

son preference and investment in education




Finally, the population has become substantially better educated over time.
In the first period, almost 60\% of women had no education, and only twenty percent had
eight or more years, while in the last period the reverse was the case.
These changes underestimate the increase in education because many of the younger women 
with more education are not yet married and therefore not in the sample.%
\footnote{
Women with eight or more years of education accounted for 65.9\% of
unmarried women in NFHS-3 and 82.8\% of unmarried women in NFHS-4
\citep{International-Institute-for-Population-Sciences-IIPS2007,International-Institute-for-Population-Sciences-IIPS2017}.
}

% \citep[p 56]{International-Institute-for-Population-Sciences-IIPS2007}
% and \citet[p 61]{International-Institute-for-Population-Sciences-IIPS2017}
% for more information.

% Age at marriage increased over time across all three education groups.
% The most substantial increase was for women without any education, where
% the average went from below 16 years of age to 18.5.
% The smallest change is for the most educated women where the average age
% at marriage went up by less than a year---from 19.6 to 20.4---across the 
% four periods.


% How does educational expansion affect birth spacing


Women with different education levels have different hazard profiles, although the size 
and direction of the effect vary across areas and time 
\citep{Tulasidhar1993,Whitworth2002,Bhalotra2008,Kim2010,Soest2018}.
Furthermore, the use of sex selection increases with education, either because of the 
associated lower fertility or because higher income enables them to access and
afford prenatal sex determination
\citep{das_gupta97,retherford03b,jha06,Guilmoto2009a,Bongaarts2013,Portner2015b,Jayachandran2017}.%
\footnote{
It is also possible that the effect is driven by a stronger son preference, although 
higher education women have shown declining son preference \citep{bhat03,pande07}.
}


The sex composition of previous children affects both the timing of births and the use of 
sex-selective abortions 
\citep{retherford03b,jha06,Bhalotra2008,abrevaya09,Portner2015b,Kumar2016,Soest2018}.
I capture sex composition with dummy variables for the
possible combinations for the specific spell, ignoring the ordering of births.
Finally, urban women have both lower fertility and, presumably, better access to
sex selection, both of which lead to higher use of sex-selective abortions than in 
rural areas \citep{retherford03b,jha06,Portner2015b}.


The longer average spacing and the relatively higher number of very long birth intervals 
seem to run counter to a general reduction in the length of women's reproductive spans 
found in earlier research \citep{Padmadas2004}.


Differential stopping behavior is widespread, with an additional child more likely the
fewer sons a family has 
\citep{repetto72,Das1987,Arnold1997,arnold98,clark00,Basu2010,Barcellos2014}.%
\footnote{
Other countries show a similar pattern
\citep[see, for example,][]{larsen98,filmer09,Altindag2016}.
}


For example, in Bangladesh, India, and Vietnam, having more boys, and particularly having 
a boy as the last-born child, significantly increase the duration to next birth,
\citep{Haughton1995,Haughton1996,Rahman1993,Bhalotra2008,Kumar2016,Soest2018}.%
\footnote{
Ethnic Indians in South Africa also show a longer duration after the birth of a 
son than after a daughter \citep{Gangadharan2003}.
}
Outside Asia, the evidence is more mixed, with North Africa showing shorter spacing in the 
absence of sons, while a similar pattern does not exist in Sub-Saharan Africa 
\citep{Rossi2015}.


Finally, we know less about what determines spacing behavior in developing countries than 
in developed countries, and with declining fertility and increasing numbers of women 
entering the labor force, understanding how couples make timing decisions is necessary 
for the design of suitable policies \citep{Portner2018}.%
\footnote{
It is clear that birth spacing does respond to policies in both developed and 
developing countries \citep{Pettersson-Lidbom2009,Todd2012,Meckel2015,Ghosh2018}.
}
Even the effects of access to contraceptive and declining fertility on spacing are unclear.
On the one hand, access to contraceptives allows women to avoid too short spacing between 
birth.
On the other hand, increased reliability of access and effectiveness of contraceptives can 
lead to shorter intervals between births if women used to have longer spacing to avoid
having too many children by accident \citep{Keyfitz1971,Heckman1976}.
These two counteracting effects may explain why better-educated women have shorter spacing 
than less educated women in some instances but not in others and why finding statistically 
significant effects of contraception use on birth spacing is difficult 
\citep{Tulasidhar1993,Whitworth2002,Bhalotra2008,Yeakey2009,Kim2010,Soest2018}.%
\footnote{
In Matlab, the effect of son preference on birth spacing is stronger in areas with better 
access to family planning \citep{Rahman1993}.
Other evidence, however, points to families ability to time births, even in the absence
of modern contraceptives \citep{Jayachandran2011,Alam2018}.
}

