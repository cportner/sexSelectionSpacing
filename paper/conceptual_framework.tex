Second, the paper lacks a conceptual framework to organize (and
motivate) the analysis. Following on the comments of one reviewer, one
suggestion would be to focus on the role of women’s education, with a
more thorough description of the context of changing education in India
over the last half century and clearly lay out a framework for the
mechanisms through which birth spacing is affected by educational
expansion.

\section{Birth Spacing and Education: Mechanisms and Findings}

This section provides a conceptual framework for understanding what determines  
birth spacing.
I first discuss the main determinants of spacing examined in the prior research, and 
then how son preference and sex selection fits into the spacing decisions.
Since sex selection and education are closely linked, I pay special attention to how 
education affects spacing.

In the simplest possible dynamic model of fertility, the spacing decision is trivial.
Assume that parents derive benefits from the presence of children, not their births, 
that spacing of children does impact parents' utility, 
and that the parents fully control when to have children. 
The optimal decision then is to have the children as early as possible 
\citep{Newman1988}.
The simplest model does not fit the observed spacing patterns, which suggests 
that one or more of the assumptions are wrong.

First, parents do not entirely control the timing of births, but rather decide on 
the levels of sexual activity and contraceptive effort, which, in turn, affect the 
likelihood of conceiving, while taking into account fecundity and their knowledge 
of and access to contraceptive technology.

Fecundity determinants, such as age and health of the mother, are, however, unlikely 
to be significant factors in how spacing changed over time or across education levels 
in India for the period covered.
Outside of direct famine conditions, women's health and nutritional status do not 
appear to affect spacing substantially \citep{Huffman1987,John1987,lindstrom99}.
The Maharashtra drought occurred in 1970--1973 and would, therefore, only impact the first
year or two of the data.
Furthermore, I only examine spacing up to parity fourth, and the age of the mother is,
therefore, unlikely to be a concern when it comes to fecundity, even though 
better-educated women are, on average, older when they marry. 

Despite a general belief that better access to and knowledge of contraception 
should lengthen average spacing, the empirical relationship between contraception 
use and birth spacing is ambiguous 
\citep{Tulasidhar1993,Whitworth2002,Bhalotra2008,Yeakey2009,Kim2010,Soest2018}.
This lack of evidence is consistent with the ambiguous theoretical effect of introducing 
modern contraceptives on birth spacing.
On the one hand, in the absence of high-efficacy contraceptive technologies, parents may 
try to avoid overshooting their desired number of children by spacing their children 
further apart than they would otherwise do.
Increased reliability of the access to and effectiveness of contraceptives then lead to 
shorter birth intervals because parents are more confident they can effectively
stop childbearing when they want \citep{Keyfitz1971,Heckman1976}.
On the other hand, through simple random variation, when contraceptives are ineffective, 
some birth intervals will be shorter than ideal, in the way discussed below.
With increased reliability of contraceptive efforts, parents can better avoid too short 
birth spacing, and average spacing should go up.

To further complicate the relationship between contraception and spacing, the ability 
to successfully use ``low efficacy'' contraceptive methods, such as the rhythm method, 
is increasing in education, while there is no difference across education levels 
when it comes to modern contraceptives such as pills or injectables \citep{Rosenzweig1989}.
Additionally, even in the absence of modern contraceptives, families appear able to 
time or at least postpone births when they have substantial enough incentive 
\citep{Jayachandran2011,Alam2018}.


Second, parents may care about the spacing between birth because of the 
potential effects of spacing on the children.%
\footnote{
In principle, the mother could also be affected, but the estimated impact on 
mothers' anthropometric status is mixed, and there appears to be no 
effect on mortality 
\citep{Ronsmans1998,Menken2003,Dewey2007,Conde-Agudelo2012}.
}
One theory is that child ``quality'' increases with spacing 
since parental attention per child increases
\citep{Zajonc1975,Zajonc1976,Razin1980}.
In this case, birth spacing should increase in family income, while the 
effect of maternal education is ambiguous since higher education, on the 
one hand, makes the mother's time spent out of the labor market more 
costly and on the other hand increases how productive the mother is 
in childrearing.
The evidence on spacing's effect on child quality measures such as IQ 
and education is, however, mixed for developed countries and non-existing 
for developing countries
\citep{Powell1993,Pettersson-Lidbom2009,Buckles2012,Barclay2017}.

For one crucial aspect of child quality---health and mortality---shorter spacing 
does lead to worse outcomes, especially for very short intervals of 24 months 
or less, although this effect decreases with maternal education
\citep{Whitworth2002,Conde-Agudelo2006,Conde-Agudelo2012,Molitoris2019}.%
\footnote{
The underlying mortality risk could also affect birth spacing, but
there is little empirical evidence for this 
\citep{Newman1983,Newman1988,Bhalotra2008}.
}
If parents realize short birth spacing increases their children's mortality risk, 
they likely prefer to space births further apart.
However, this effect should diminish with education level since women with 
secondary or tertiary education experience relatively low mortality risk for 
their children, even with very short spacing is associated.


Third, parents incur both direct and time costs when they have children, 
and economic considerations can, therefore, affect birth spacing 
\citep{Hotz1997,Schultz1997}.
I divide the economic factors into the effects of economies of scale and
the expected income trajectory of the husband and wife.

The expected effect of economies of scale in childrearing, which, for example, 
can arise because taking care of two young children does not take twice as much 
time as taking care of one child, is to shorten intervals between births
\citep{Vijverberg1982,Espenshade1984}.
To the extent that children require the mother to do less market work, parents 
can lower the cost of children by reducing birth spacing.
In this case, better-educated women will both have lower fertility and shorter
spacing than less-educated women because of their higher opportunity cost of time 
\citep{Ross1974,Newman1981,Newman1984}.
However, if the mother mainly works at home or as household income increases, there 
is less incentive to space children for this reason.

In the absence of perfect capital markets, theory predicts that the steeper 
the parents' lifetime income profile, the longer the spacing between births, 
which suggests that with economic development we should see a general increase 
in spacing across education groups and that this increase in spacing should be 
stronger with increasing education levels
\citep{Heckman1976,Wolpin1984,Newman1988}.
The idea is that although parents would prefer to have their children as early 
as possible to enjoy them for longer, there is also a direct cost to children.
Income is low early on, which means that the marginal utility of consumption is 
higher, making parents less willing to give up consumption to have a child, 
compare to later when they know that their income will be higher 
\citep{Newman1984,Happel1984}.

Finally, the discussion has so far assumed that children are identical---except 
for any differences that arise from spacing---but there is substantial evidence 
that the sex composition of previous children affects birth spacing, with the
absence of boys significantly decreasing the duration to next birth
\citep{Haughton1995,Haughton1996,Rahman1993,Bhalotra2008,Kumar2016,Soest2018}.
There is also evidence of differential stopping behavior, with an additional child 
more likely the fewer sons a family has 
\citep{repetto72,Das1987,Arnold1997,arnold98,clark00,Basu2010,Barcellos2014}.

If short spacing lowers the survival probability of the next child, as discussed above,
it is not, however, apparent that we should expect shorter spacing in the absence of boys.
There are two potential reasons why parents might still try to conceive sooner 
in the absence of boys.%
\footnote{
The significant differences in spacing by sex composition indicate that
parents control the timing of birth, at least to some extent.
In Matlab, the effect of son preference on birth spacing is stronger in 
areas with better access to family planning \citep{Rahman1993}.
}
One possibility is that parents do not realize that short spacing is potentially
detrimental to the next child's health, resulting in high mortality for both boys 
and girls born after short spacing across all households.

The other possibility is that parents realize that short spacing has adverse 
health effects, but their preference for sons is so strong that they are willing to 
divert substantial resources to the next child if it is a boy.
Assuming that these investments can compensate for the adverse effects,
we should see significant differences in mortality across sex for short-spacing 
children in the absence of boys.
If a son is born, his mortality risk should be substantially lower than expected 
for the spacing duration, and we should expect worse outcomes for the other 
children (all girls).
If a girl is born, there are two possible outcomes.
If parents treat all children/girls the same, we should see a slight worsening in 
health outcomes because fewer resources are available per child; 
if they do not, the last girl will have a much lower likelihood of surviving 
because the parents' incentive is to invest in their prior children rather than 
the new child.

With the adverse effects of shorter spacing strongly mitigated by higher education,
better-educated women can, in theory, have shorter spacing in the absence of a boy 
without substantial increasing mortality risk.
Differences in mortality outcomes across boys and girls are still possible,
but should not be as large as for less-educated women.

With the introduction of sex selection, parents can avoid the birth of girls, 
although at the cost of the procedure and having to wait longer until 
the next child if they abort the fetus.

Theory suggests that sex selection increases with lower desired fertility, and, for a 
given desired number of children, the higher the parity \citep{Portner2015b}.
The increased use of sex selection in India with education and in urban
areas is consistent with a lower desired fertility for both groups,
either because of the higher cost of children or because of other
changes in the preferred number of children
\citep{das_gupta97,retherford03b,jha06,Guilmoto2009a,Bongaarts2013,Portner2015b,
Jayachandran2017}.%
\footnote{
Stronger son preference could also contribute, but better-educated women 
have declining stated son preference over time \citep{bhat03,pande07}.
}

The higher the cost of prenatal sex determination and abortion, the less use we should 
see.
Similarly, any potential adverse effects on the mother and offspring 
from repeated abortions should also reduce the use of sex selection---provided that 
parents are aware of these costs.
Furthermore, the more abortions between births, the less the parents can take advantage 
of economies of scale.
Better-educated women tend to live in households with higher income, which better enables 
them to access prenatal sex determination and lowers the relative costs of using sex 
selection.

Given the increase in birth spacing with each abortion, parents may try to 
conceive even sooner to compensate for the potentially longer spacing and, 
thereby, still take advantage of economies of scale in childrearing.
Furthermore, if health outcomes for short spacing births have improved over 
time---as is likely---we should see an even stronger push toward shorter 
spacing for women using sex selection.

In summary, birth spacing is affected by female education through 
multiple pathways, with the main one of interest here the use of 
sex selection. 
But, to fully understand how sex selection affects spacing, we also 
need to understand the other changes that affect birth spacing over time.
The most likely factors are the changes in household income and 
wealth and the improvements in health as India has developed.
I, therefore, examine how female education and labor force 
participation has changed over time in India.


