Second, the paper lacks a conceptual framework to organize (and
motivate) the analysis. Following on the comments of one reviewer, one
suggestion would be to focus on the role of women’s education, with a
more thorough description of the context of changing education in India
over the last half century and clearly lay out a framework for the
mechanisms through which birth spacing is affected by educational
expansion.

\section{Birth Spacing and Education: Mechanisms and Findings}

This section provides a conceptual framework for understanding what determines spacing 
between births.
Since sex selection is closely linked to education, I focus especially on the
the effects of education on spacing.
I first discuss a generic spacing decisions frameworks, and then turn to how son 
preference and sex selection fits into the decisions on spacing.

In the simplest possible dynamic model of fertility the spacing decision is trivial.
Assume that parents derive benefits from having their children around, rather than from 
the births themselves, that spacing of children by itself does impact parental utility, 
and that the parents perfectly control when to have children. 
In this set up, the optimal decision is then to have the children as early as possible 
\citep{Newman1988}.
The simplest model clearly does not fit the observed spacing patterns, but the question 
is why.

First, parents do not perfectly control the timing of births, but instead decide on their
levels of sexual activity and contraceptive effort, which, in turn, affect their likelihood 
of conceiving.

The decisions on sexual activity and contraceptive effort are moderated by fecundity,
determined by biological factors such as age and health of the mother, and the knowledge of and 
access to contraceptive technology.
There is evidence from ``natural fertility'' populations that increasing number of 
children is associated with shorter spacing and that birth intervals increase 
with parity \citep{Henry1961,Leridon1977}.

Biological factors are, however, unlikely to be an important factor in differences in spacing over 
time or across education levels in India for the period covered.
Outside of direct famine conditions, there is little evidence that women's health and 
nutritional status substantially affects spacing \citep{Huffman1987,John1987,lindstrom99}.%
\footnote{
On one hand, with better health we should expect shorter post-partum amenorrhoea and a higher
likelihood of carrying a pregnancy to term, both of which lead to shorter spacing with better health.
On the other hand, an healthier mother is more likely to successfully breastfeed and that would 
increase the spacing between births, although if she is a strong producer of breastmilk the 
suckling relative to production is lower, which, in turn, lower the conceptraceptive effect
of breastfeeding.
}
The Maharashtra drought occurred in 1970-1973 and, therefore, would only affect the first
year or two of the data.
Hence, although there are still differences in individual fecundity across women, it is
unlikely that we should see systematic differences across education groups and over time
in the data because of health status.
Furthermore, I only examine spacing up to parity fourth, and age of the mother is,
therefore, unlikely to be a concern when it comes to fecundity, when though better 
educated women are, on average, older when they marry. 

Empirically, there is no clear relationship between contraception use and birth 
spacing or between education level and birth spacing
\citep{Tulasidhar1993,Whitworth2002,Bhalotra2008,Yeakey2009,Kim2010,Soest2018}.
The lack of evidence is consistent with the ambiguous theoretical effect of introducing 
modern contraceptives on birth spacing.
On one hand, when contraceptive technologies are less effective, parents may try to avoid 
overshooting their desired number of children by spacing their children further apart.
Increased reliability of access to and effectiveness of contraceptives can then lead to 
shorter intervals between births because parents are more certain of effectively
stopping childbearing when they reach their preferred number of children and having
the children closer together provide higher utility because parents get to enjoy the
children for longer as discussed above \citep{Keyfitz1971,Heckman1976}.
On the other hand, when contraceptives are less effective we would expect, through 
random variation, some birth intervals to be shorter than ideal for the reasons 
discussed below.
With increased reliability of access to and effectiveness of contraceptives, parents 
would better be able to avoid too short spacing between birth and average spacing
should go up.

To further complicate the relationship between contraception access and spacing,
the ability to successfully use ``low efficacy'' contraceptive methods, such as 
the rhythm method, is increasing in education, while there is no difference across education 
leves when it comes to modern contraceptives such as pills or injectables 
\citep{Rosenzweig1989}.
Furthermore, even in the absence of modern contraceptives, families appear able to time or 
at least postpone births when faced with strong enough incentives 
\citep{Jayachandran2011,Alam2018}.
[Does this need a tie-in to the paragraph above?]
[Can use "less effective" contraception and still have low fertility]
[Do parents move afway from certain contraceptive choices with education?]



Second, parents possibly care about the spacing between birth either in its own right or, 
more likely, because of the potential effects of spacing on the children and the mother.

It has, for example, been suggested that child ``quality'' might increases with spacing 
because the amount of parental attention increases with spacing
\citep{Zajonc1975,Zajonc1976,Razin1980}.
In this case, birth spacing should increase in family income, while the effect of maternal 
education is ambiguous since higher education on one hand makes the mother's time spent 
out of the labor market more costly and on the other hand increases how productive the 
mother is in childrearing. [TK Behrman et al. paper on mother's education in India]

One aspect of child ``quality'' where is is a strong effect of spacing is child health
and mortality, with most of the research finding that shorter spacing leads to worse
health and higher mortality for children, especially for very short intervals of 24 months 
or less \citep{Conde-Agudelo2006,Conde-Agudelo2012,Molitoris2019}.%
\footnote{
There is evidence in countries as diverse as Bangladesh, Brazil, El Salvadore, India,
and Pakistan of worse health outcomes and a higher mortality risk the shorter the birth
interval
\citep{Cleland1984,Curtis1993,Whitworth2002,Bhargava2003,Rutstein2005,Bhalotra2008,
Davanzo2008,Maitra2008,Makepeace2008,Gribble2009,Jayachandran2011,Saha2013,Jayachandran2017a,
Ghosh2018}.
This is different from the effect of the underlying mortality risk on birth spacing.
The expected effect of mortality risk---ignoring the effect of shorter spacing on mortality 
risk---is to shorten birth spacing to get to the ideal number of children earlier.
There is little empirical evidence for this type of effect 
\citep{Newman1983,Newman1988,Bhalotra2008}.
}
Hence, if parents realize that having children close together increase the children's
mortality risk, they would prefer to space births further apart to lower mortality risk
(although possibly not very long since the mortality risk involves a trade-off between the
cost of births and the benefit of having a child).
Furthermore, the effects of short spacing on child mortality are strongly moderated by
maternal education, and even very short spacing has relatively low mortality for women
with secondary or tertiary education \citep{Whitworth2002,Molitoris2019}.

The mother may also be affected by the length of the birth intervals, although the
results on mother's anthropometric status are mixed and there appear to be no
effect on maternal mortality \citep{Ronsmans1998,Menken2003,Dewey2007,Conde-Agudelo2012}.
For India, there is, however, evidence that women with first-born girls are more likely to 
have anemia, possibly because short birth spacing can resulting in maternal depletion 
\citep{Milazzo2018}.

Third, having and rearing children involves both direct and time costs, and economic factors 
can, therefore, affect birth spacing \citep{Hotz1997,Schultz1997}.
These economic factors can broadly be divided into 
the effects of price and income shocks,
the expected income trajectory of the husband and wife,
and economies of scale considerations.
There is a substantial amount of research on all these areas, although the majority is for
developed countries.

The expected effect of economies of scale in childrearing is to shorten the interval between 
births.
Childbearing and rearing is intensive in the mother's time, but taking care of two young
children is not twice as time consuming as taking care of one child 
\citep{Vijverberg1982,Espenshade1984}.
Hence, if childbearing and rearing requires the mother to reduce her participation in productive 
activities, parents can lower the cost of children by shorten birth spacing and, thereby, reduce
the amount of time the mother has to be away from productive activities.
In this set-up women with higher education will both have lower fertility and shorter
spacing than women with lower education because of the higher opportunity cost of time 
\citep{Ross1974,Newman1981,Newman1984}.
The extent to which this type of economies of scale is important depends on the extent to
which having children affects the mother's productive activities.
If she mainly works at home there likely is going to be less of an incentive to space 
children for this reason than if she is part of the labor force outside the home.
Similarly, higher household income, holding the mother's income contanst, reduces the incentive 
to space births for this reason.

In the absense of perfect capital markets, theory suggests that the steeper the parents' income 
profile over time, the longer the spacing between births 
\citep{Heckman1976,Wolpin1984,Newman1988}.
The idea is that although parents would prefer to have their children as early as possible to
enjoy them for longer there is also a direct cost to children.
At the early stage of life income is low, which means that the marginal utility of consumption 
is higher making parents less willing to give up consumption now to have a child, compare to 
later when they know that their income will be higher.
There is evidence that---at least for first births---there is, indeed, an effect of the income
profile over time \citep{Newman1984,Happel1984}.

There are two implications of this theoretical results.
First, we should see an general increase in spacing across education groups to the extent that
development lead to a steeper income profile.
[This does require that economic development is associated with steeper income profiles;
not sure if anybody looked at this.
The lower fertility even for low education women does suggest that income is higher,
but does not, by itself, say anything about the age profile in income.]
Second, the increase in spacing should be stronger with increasing education levels.
In other words, we should see a general increase in spacing, and this increase should be
larger the higher the education level.

The income profile argument assumes that parents know their future income, but 
shocks to income and prices may also affect spacing, with temporary higher cost of women's time
and other inputs into children together with lower income temporarily decreasing fertility 
and, thereby, increase spacing between births \citep{Moffitt1984,Hotz1988,Portner2001,Alam2018}.
It is likely that the risk of income shocks decrease with increasing education, although it
is unclear the extent to which exposure to other types of shocks that might impact the timing
of birth is affected by education.


Finally, the discussion has so far assumed that children are identical, except for any potential
differences that arise from differences in spacing, but there is substantial evidence that the 
sex composition of previous children affect birth spacing.
For example, in Bangladesh, India, and Vietnam, having more boys, and particularly having 
a boy as the last-born child, significantly increase the duration to next birth,
\citep{Haughton1995,Haughton1996,Rahman1993,Bhalotra2008,Kumar2016,Soest2018}.%
\footnote{
There is also evidence of differential stopping behavior, with an additional child more likely the
fewer sons a family has 
\citep{repetto72,Das1987,Arnold1997,arnold98,clark00,Basu2010,Barcellos2014}.%
Other countries show a similar pattern
\citep[see, for example,][]{larsen98,filmer09,Altindag2016}.
Outside Asia, the evidence is more mixed, with North Africa showing shorter spacing in the 
absence of sons, while a similar pattern does not exist in Sub-Saharan Africa 
\citep{Rossi2015}.
}

It is not immediately obvious that we should expect a result like this from a theoretical
standpoint.
If, as discussed above, there are potentially significant negative effects of short spacing
parents who choose short spacing in the absence of a boy lowers the survival probabilty of
the next child, including a boy if the next child is a boy.
There are two possibilities for why parents might still try to conceive sooner in the 
absence of boys.%
\footnote{
The fact that there are significant differences in spacing by sex composition indicates that
parents have---at least some---control over the timing of birth.
Hence, it is unlikely that the differences in spacing come about by random chance.
In Matlab, the effect of son preference on birth spacing is stronger in areas with better 
access to family planning \citep{Rahman1993}.
}
First, it is possible that parents do not, in fact, realize that short spacing is potentially
detrimental to the next child's health.
Second, parents realize that short spacing has negative health effects, but have a strong
enough son preference that they are willing to divert substantial resources to investing
in the next child's health if it is a boy and that these investments can compensate for 
the negative effects.
[Are there any other ones that might explain the pattern?
    - biological? this would have to work off the sex of only the last child, so that
    spacing after girls is naturally shorter.
    - parents care so much about having a son sooner that they are willing to accept a 
    higher mortality risk.
    - ??
    ]

[has anyone actual done the theory behind the short spacing?
It would seem that it requires fairly strong assumptions to make it attractive to
have short spacing.
Maybe check the Jensen paper on equal treatment]

[can I say anything about which one is true?]
It is worthwhile thinking a bit about under what circumstances it would be possible to
identify which of these explanations hold.
If the first one holds, we should see high mortality for both boys and girls born
after short spacing across all households whether or not they have a strong son preference.
In fact, it is possible that we see higher mortality for sons than girls given that boys 
have naturally higher mortality in the absense of any preference for a given sex.

If the second one holds, we should see substantial differences in mortality across
sex for short spacing children, with the boys having substantially lower mortality risk
than expected for the spacing length.

If a son is born next:
The boy should have substantially lower mortality risk than expected for the spacing
duration.
Because this argument relies on a redistribution of resources, we should expect to see
worse outcomes for all the other children in the family.
This means that the other children (all girls) should do substantially worse.

If a girl is born next:
there are a couple of possible outcomes.
If all children/girls are treated equally, we should see somewhat of a worsening in 
health outcomes because the available resources are now distributed across more children.
If children are not treated equally, we should see much worse outcomes for the last
girl because the parents have an incentive to keep investing in the children they have
already rather than the new child.

How does education play into this?
The negative effects of shorter spacing is strongly mitigated by higher education.
Hence, women with more education are in the position that they can better tolerate short 
spacing in the absence of a boy and that the potential negative effects on the next child 
should be much less pronounced.
Furthermore, the lower mortality risk should show up whether the next child is a boy 
or a girl.
It is clearly still possible that there would some difference in mortality outcomes 
across boys and girls because of a larger investment in the boy once born, but it should
not be as large.

With the introduction of sex selection parents can now avoid the birth of children with 
an undesired sex, although at the cost of the procedure and having to wait longer until 
the next child if the fetus is aborted.
The use of sex selection impacts the observed birth spacing since, as discussed 
above, each sex-selective abortions increases the interval between births by six 
to twelve months.
The question is who is likely to use sex selection, when, and why?

Theory suggests that the lower the desired fertility the more likely is the use of sex 
selection \citep{Portner2015b}.
This is driven by simple probabilities:
without use of sex selection, the probability of not having a son are 48.8 percent when
only having one child, 23.8 percent for two children, 11.6 percent for three children, 
5.7 percent for four children, 2.8 percent for five children, and 1.4 percent for six 
children.
Furthermore, given a desired level of fertility, the higher the parity, the higher the
likelihood of using sex selection.
Unless parents have a specific preference for a first-born boy, the costs associated with 
sex selection means that it is better to try for a boy without sex selection first and 
then use at later parities in the absence of a son as you get closer to the desired 
fertility.

The higher the cost of prenatal sex determination and the abortion, both in direct cost
and the possibility of criminal charges, the less use we should see.
Similarly, any potentially negative effects on the mother and offspring from repeated
abortions would also tend to reduce the use of sex selection---provided, obviously, that
parents are aware of these costs.
Finally, the more abortions between births, the less the opportunity there is to take
advantage of any economies of scale in childrearing.


The use of sex selection in India increases with education and is higher in urban
areas than in rural
\citep{das_gupta97,retherford03b,jha06,Guilmoto2009a,Bongaarts2013,Portner2015b,
Jayachandran2017}.
Women with more education generally have lower desired fertility, either because of the
higher opportunity cost of time or because of other changes in the preferred number of
children.%
\footnote{
It is also possible that the effect is driven by a stronger son preference, although 
higher education women have shown declining stated son preference \citep{bhat03,pande07}.
}
Women with more education also tend to live in households with higher income,
which better enables them to access prenatal sex determination and lowers the relative
costs of using sex selection (both direct and indirect, such the decline in 
economies of scale with repeated abortions).

Given that each abortion increases the spacing between births, it is possible that
parents may try to conceive even sooner than in the absence of sex selection.
By conceiving sooner, they would lower the expected cost of foregone economies of scale in 
childrearing.
This is especially relevant for better educated women, who, as discussed above, have
lower mortality risk for offspring born after short spacing.
If health outcomes for short spacing births has improved over time---as is likely---we
should see an even stronger push toward shorter spacing for women using sex selection.


[Need some kind of summary of spacing decisions and how they are likely to change
over time and with increased education]

As other countries like the US have developed, they have seen a progressively longer 
spacing between births as the proportion of births with very short intervals---less than 
24 months---have declined substantially \citep{Hotz1997}.
Most of the these countries do, however, not have the same degree of son preference that
we see in India.

What are the important factors that determine spacing?

- household income
- education
- labor force participation
- sex selection
- health (improvements, campaigns)


In summary, female education affects birth spacing through a variety of pathways.
The main pathway of interest here is the effect of sex selection on spacing.
However, to understand the effects of the increased use of sex selection over time it
is important to also understand the other changes that affect birth spacing.
The most likely factors are the changes in household income/wealth and the improvements
in health as India has developed.
I, therefore, now look at how female education and labor force participation has changed
over time in India.


