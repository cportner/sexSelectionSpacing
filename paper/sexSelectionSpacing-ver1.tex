% Revamped version for Demography focusing on method

% sex-selective Abortion
% Begun originally: 01/11/02
% Begun.: 2017-04-03
% Edited: 2017-04-03

\documentclass[12pt,letterpaper]{article}

\usepackage{mathpazo}
\usepackage[title]{appendix}
\usepackage[margin=1.0in]{geometry}
\usepackage[figuresleft]{rotating}
\usepackage[longnamesfirst]{natbib}
\usepackage{amssymb}
\usepackage{amsmath}
\usepackage{dcolumn}
\usepackage{booktabs}
\usepackage{multirow}
\usepackage[flushleft]{threeparttable}
\usepackage{setspace}
\usepackage{xyling}
\usepackage[justification=centering]{caption}
\usepackage[font=scriptsize]{subfig}
\usepackage[pdftex,colorlinks=true,linkcolor=black,citecolor=black,urlcolor=black]{hyperref}
\usepackage{adjustbox}
\usepackage{xfrac}


% \bibpunct{(}{)}{;}{a}{}{,}
\newcommand{\mco}[1]{\multicolumn{1}{c}{#1}}
\newcommand{\mct}[1]{\multicolumn{2}{c}{#1}}
\newcommand{\X}{$\times$ }
\newcommand{\hs}{\hspace{15pt}}

% Attempt to squeeze more floats in
\renewcommand\floatpagefraction{.9}
\renewcommand\topfraction{.9}
\renewcommand\bottomfraction{.9}
\renewcommand\textfraction{.1}
\setcounter{totalnumber}{50}
\setcounter{topnumber}{50}
\setcounter{bottomnumber}{50}


%------------------------------------------------------------------------

% Fertility, Birth Spacing and the Determinants of sex-selective Abortions
% Understanding Sex Selection: Fertility, Abortions and Birth Spacing
% The Determinants of sex-selective Abortions: A New Method/Approach
% Sex Selection: A New Method for Estimating Determinants
% Fertility, Birth Spacing and Abortions in India
% Fertility, Birth Spacing and Sex Selection
% Fertility, Birth Spacing and sex-selective Abortions
% Fertility, Spacing and Sex Selection
% Fertility, Sex Selection, and Spacing
% Fertility, Sex Selection, and Birth Spacing
% Fertility, Birth Spacing and Missing Girls
% Where have all the girls gone: An analysis of sex-selective abortions

\title{Son Preference, Sex-Selective Abortions, and Birth Spacing
\protect\thanks{%
I am grateful to Andrew Foster and Darryl Holman for discussions about the method.
I owe thanks to three anonymous referees, Shelly Lundberg, Daniel Rees, David Ribar, 
Hendrik Wolff, seminar participants at University of Copenhagen, University of Michigan, 
University of Washington, University of \AA{}rhus, the Fourth 
Annual Conference on Population, Reproductive Health, 
and Economic Development, and the Economic Demography Workshop for helpful suggestions and comments.
I would also like to thank Nalina Varanasi for research assistance.
Support from the University of Washington Royalty Research Fund and the 
Development Research Group of the World Bank is gratefully acknowledged.
The views and findings expressed here are those of the author and
should not be attributed to the World Bank or any of its member countries.
Partial support for this research came from a Eunice Kennedy Shriver National
Institute of Child Health and Human Development research infrastructure grant,
5R24HD042828, to the Center for Studies in Demography and Ecology at the
University of Washington.
% Prior versions of this paper were presented under the title ``The 
% Determinants of sex-selective Abortions.''
}
}

\author{}

\author{Claus C P\"ortner\\
    Department of Economics\\
    Albers School of Business and Economics\\
    Seattle University, P.O. Box 222000\\
    Seattle, WA 98122\\
    \href{mailto:cportner@seattleu.edu}{\texttt{cportner@seattleu.edu}}\\
    \href{http://www.clausportner.com}{\texttt{www.clausportner.com}}\\
    \& \\
    Center for Studies in Demography and Ecology \\
    University of Washington\\ \vspace{2cm}
    }

\date{April 2017}


%------------------------------------------------------------------------


\begin{document}
\graphicspath{{figures/}}
\DeclareGraphicsExtensions{.jpg,.pdf,.mps,.png}

\setcounter{page}{-1}
\maketitle
\thispagestyle{empty}

% \setcounter{page}{0}
\newpage

\doublespacing
% \onehalfspacing


% \thispagestyle{empty}
 
\maketitle
\thispagestyle{empty}


\newpage
\doublespacing


\begin{abstract}

% Demography abstract
\noindent This paper addresses the question:
how does birth spacing interact with the use of sex-selective abortions? 
I introduce a statistical method that incorporates how sex-selective abortions 
affect both the likelihood of a son and spacing between births.
Using India's National Family and Health Surveys,
I show that use of sex selection leads to longer spacing after a daughter is born.
Women with 8 or more years of education, both in urban and rural areas, are the main users 
of sex-selective abortions.
Women with less education do not appear to use sex selection.
Predicted lifetime fertility for high-education women declined more than 10\% 
between 1985--1994, when sex selection was legal, and 1995--2006, when sex selection was illegal.
Abortions per woman increased almost 20\% for urban women and 50\%
for rural women between the two periods, suggesting that making sex selection illegal has 
not reversed its use.
Finally, sex selection appears to be used to ensure one son rather than multiple sons.

\noindent JEL: J1, O12, I1
\noindent Keywords: India, prenatal sex determination, censoring, competing risk
\end{abstract}

\newpage


%------------------------------------------------------------------------

\section{Introduction\label{sec:intro}}

[ways to get at son preference:
- stated fertility preferences
- longer spacing after son
- differential stopping behavior
- biased sex ratio (overall before sex selection, and specifically at birth later)]

[why do these work less well after introduction of sex selection?]

[how can my method remedy those problems?]

[main results]

Understanding and measuring son preference has been an important research 
area for decades and is as salient as ever.
India, for example, has seen dramatic increases in the males-to-females ratio 
at birth over the last three decades as access to prenatal sex determination 
expanded.%
\footnote{
See \citet{das_gupta97}, \citet{Sudha1999},
\citet{Arnold2002}, \cite{retherford03b} and \citet{jha06}.
India is not alone; both China and South Korea saw
significant changes in the sex ratio at birth \citep{Yi1993,park95}.
}
Recent research suggests, however, that son preference in India, when measured 
as ideally having more boys than girls, is decreasing over time and with 
higher education \citep{bhat03,pande07}.%
\footnote{
This measure of son preference is commonly used in the literature. 
See, for example, \citet{clark00}, \citet{Jensen2009}, and \cite{Hu2015}.
}

In the absence of prenatal sex determination, parents' only choice if they 
want a son is to try for another pregnancy as soon as possible after
the birth of a girl. 
Hence, in the absence of sex-selective abortions, son preference  
leads to a shorter duration until the next birth if the previous 
birth was a daughter  
\citep[see, for example,][]{Das1987,Rahman1993,Pong1994,Haughton1996,Arnold1997}.
The resulting shorter spacing is thought to be associated with worse 
health outcomes for the girls 
\citep{arnold98,Whitworth2002,Rutstein2005,Conde-Agudelo2006}.
Parents are also more likely to cease childbearing after the birth of 
a son than after a daughter \citep{ben-porath76b,Das1987,Arnold1997,clark00}. 

With the introduction of sex-selective abortions, parents can now abort
a fetus of unwanted sex and the use of sex selection can therefore 
directly provide us a picture of the degree of son preference.
Our understanding of the use of sex selection is, however, constrained by 
lack of information;
there are no official data on sex selection, and the few surveys that ask 
about use of sex selection show signs of serious under-reporting \citep{goodkind96}.
With no direct information, researchers have relied on a simple method for 
establishing factors that affect sex selection: use the sex of children born as 
the dependent variable, and estimate the effects of variables on the probability 
of having a son.%
\footnote{
In the absence of any interventions, the probability of having a son
is approximately 0.512, which is independent of genetic factors \citep{ben-porath76b,jacobsen99}.
With fetus sex random, a statistically significant variable indicates an association with 
use of sex-selective abortions.
Examples of studies that have used this approach are \cite{retherford03b},
\cite{jha06}, and \cite{abrevaya09}. 
}
Based on this simple method, we know that families with no sons are more likely to
use sex selection the higher the parity;
that use of sex selection increases with socioeconomic status, especially education;
and that sex selection is more widespread in cities than in rural areas 
\citep{retherford03b,jha06,abrevaya09}.%
\footnote{
There is substantial disagreement on whether sex-selective abortion is used for the 
first birth \citep{retherford03b,jha06}.
}





Because of the potential health effects, most of the literature on spacing 
for countries like India naturally focus on whether spacing was too short
and used the shorter birth spacing after the birth of a girl than after the
birth of a boy as a measure of the degree of son preference.





I address an important questions that the prior literature has been unable
to examine because of the simple method's limitations:
how does birth spacing interact with use of sex-selective abortions?
I introduce a novel method that directly incorporates that sex-selective 
abortions affect both the likelihood of a son being born \emph{and} the duration between 
births.
I use the method to show how spacing between births play an important role in our 
understanding of sex selection decisions.


% [literature on birth spacing - why spacing is important]


What has not previously been appreciated is that the introduction of sex-selective 
abortions substantially changed the relationship between son preference and birth spacing.
The change happens because each abortion significantly increases the duration until the
next birth.
The increase in duration can be divided into three parts starting from the time of
the abortion.
First, the uterus needs at least two menstrual cycles to recover 
before conception, because a less than three months space 
between abortion and conception substantially increases the likelihood 
of a spontaneous abortion \citep{zhou00b}.
Secondly, the average expected time to conception is about 6 months.
Finally, sex determination tests are generally reliable only from 3 months of gestation onwards.
The result is that each abortion delays the next birth by approximately a year.%
\footnote{
There are three well-developed technologies for determining a fetus' sex: 
Chorionic villus sampling (CVS), amniocentesis, and ultrasound.
CVS can be applied after the shortest period
of gestation (8 to 12 weeks).
It is the most complicated, but also the most reliable, and an abortion can be done in 
the first trimester.
Amniocentesis can be performed after fourteen weeks, but requires three to four weeks
before the result is available, so an abortion cannot be performed until more than 
halfway through the second trimester.
Ultrasound has the advantages of being non-invasive and relatively cheap but 
is associated with higher risk of faulty sex determination if done early.
Generally, a fetus' sex can be determined in the third month of gestation if it is a
boy and the fourth month if it a girl.

The waiting time to conception does vary by woman, but
even if it is very short, say one  month, the minimum additional space between
births would be 6 months per abortion.
For a discussion of time to conception see \citep{Wang2003}
}

% [why this matters]
How spacing interacts with sex selection is important for two reasons.
First, 
spacing, by itself, can no longer be used to predict son preference. 
Because each abortions extends spacing by about a year, we now have a situation 
where the \emph{longest} spacing is likely observed for families with the 
strongest son preference.
To further complicate matters, 
we may still observe short spacing after the birth of daughter as a representation 
of son preference for families who either have less access to sex-selective 
abortions or are willing to have more children.
Spacing is still important and useful in understanding sex selection and son
preference, but only if combined with the outcome of that spacing---in other words 
the likelihood of observing a boy or a girl.

% using spacing itself to predict son preference becomes useless if
% families have low desired fertility and access to sex-selective abortions!!!!
Second, families may reverse their decision to use sex selection in-between births.
A reversal could happen if they do not want the space between children to be 
``too long'', or if there are concerns about possible infertility as a result of 
too many abortions in a row without a birth.
Hence, for a given parity, the sample of women for whom we do observe a birth 
may behave differently from the sample of women who have not yet had a birth.
This is a problem because births to women with shorter spacing are more likely 
to be in the survey for a given parity.
Hence, if decisions on sex selection change with duration since last birth,
the predicted sex ratio for a parity using the simple method will be a 
biased estimate of the final observed sex ratio for the parity when 
childbearing is complete.



% [contribution of paper]
[TK maybe not focus so much on the simple method but rather understanding son preference].
The main contribution of this paper is the introduction and application of a method that 
incorporates the decisions about birth spacing, and sex-selective abortions.
The empirical model is a competing risk, non-proportional hazard model with two exit 
states: either a boy or a girl is born.
This approach has TK major advantages over the simple method 

Secondly, by explicitly incorporating censoring of birth spacing, it addresses any 
potential bias from ignoring how sex selection decisions may change with duration from 
the previous birth.

None of these is possible using the simple method.

% [EXPLAIN RESULTS HERE]
% Main findings!
I apply the method to birth histories from India's National Family and Health Surveys 
for Hindu women covering the period 1972 to 2006. 
There are three main findings.

There is, however, still no evidence of sex selection being used on the first birth.

Secondly, sex selection appears to be used for securing one son, rather than a large 
number of sons.
There is only limited use of sex-selective abortions for better-educated
women with one or more sons.
The exception is rural women with one son and one daughter, presumably to compensate for 
the higher child mortality in rural areas.
These results are in line with the differential stopping behavior observed in many studies
before sex selection became available \citep{repetto72,arnold98,dreze01}.



The method can better predict ``final'' sex ratios than the simple model


\citep{Soest2012} for a recent example of spacing and son preference




\section{The Interaction between Son Preference and Birth Spacing\label{sec:model}}


With the introduction of sex-selective abortions, parents can now abort
a fetus of unwanted sex.
With an approximately 12 months additional duration per abortion,
this can result in substantially longer durations between births, and
``too long'' spacing may become a greater concern for parents than 
``too short'' spacing.%
\footnote{
Shorter spacing was likely less of a concern to begin with for
better educated mothers---who are also the most likely to use 
sex selection---since their children are substantially less likely 
to be negatively affected by short spacing than children of less 
educated mothers \citep{Whitworth2002}.
}
If the duration from last birth becomes long enough, parents may 
reverse their decision to use prenatal sex determination
and carry the next pregnancy to term whether male or female.

A number of factors make a reversal of the decision to use sex 
selection between one birth and the next more likely, 
everything else equal.
First, the more concerned parents are about possible infertility 
from repeated abortions or the closer they are to the end of their 
reproductive period, they more likely they are to decide that any 
child is better than not having an additional child at all.
Second, the higher the parents' discount factor is---that is, the
more the value present outcomes relative to future outcomes---the
more likely they are to reverse their decision to sex selection.
Third, economies of scale in childrearing---both in terms of time cost 
and direct cost---are easier to take advantage of if births 
are closer \citep[p 947]{Newman1984}.
Hence, the stronger these economies of scale are, the more likely
parents are to reverse their decision to use sex selection.
Finally, if women's wages increase with age, parents prefer
to have their children early and close together while 
the opportunity cost of children is lower \citep{Heckman1976}.%
\footnote{
The time needed to care for young children can also impact 
timing if skills depreciate when out of the labor market \citep{Happel1984}.
Whether depreciation leads to longer or shorter spacing depends
on the specification of the depreciation function.
See also the discussion in \citet[p 315]{Hotz1997}
and references therein.
}
The more pronounced the increase in wage is with age, the more 
attractive it will be to reverse the decision to use sex selection.
Although the latter two factor are most often associated with the literature
on shorter spacing in middle- and  high-income countries, 
the increasing education and wealth in India means that a growing 
portion of the population is likely to behave in a similar manner.%
\footnote{
The possible exception is for the first birth, where the pressure to show 
fecundity---which shortens the duration to the first birth---is potentially 
an important factor in India \citep{dyson83,Sethuraman2007,Dommaraju2009}.
}

For a given sex composition of prior children, there are two main drivers 
of sex selection: the degree of son preference and the cost of sex selection.
The more parents care about having a son as their next child, the more
likely they are to use sex selection and the less likely they are to
reverse the decision once started.
The result is an increased number of abortions, more spells where a
prenatal scan is used, and the spacing between births is longer.

Lowering the cost of sex selection bias the overall sex ratio further 
towards boys.
Correspondingly, both the number of abortions per woman and the
number of spells where a prenatal scan is used should also increase,
resulting in longer average spacing between births.
This increase in spacing is, however, not likely to be uniform across parities.
In addition to increasing the overall sex ratio, lower cost of prenatal 
scan should also lead to use of sex selection at earlier parities.
It is therefore possible to see either no change or a \emph{decrease} 
in average spacing for a later parity---although with a more than compensating 
increase in an earlier parity.
In other words, it is possible for individual parities to show a \emph{decline}
in sex selection at the same time as the overall use goes up.
This is an important caution against over-interpreting individual 
parity results.


Lowering the cost of prenatal sex determination generally leads to 
a smaller percentage of parents who change their use of sex selection.
But, the total number of spells with a change may still
go up because lower cost of prenatal sex determination also
increases the use of sex selection.



If ``too long'' spacing leads parents to reverse their decision to use
prenatal sex determination within a spell this has important implications 
for the empirical method [TK explain what they are instead!].






[TK find examples of the son preference spacing papers and look at what they did
for "theory"]


[TK this needs to move to later]
Finally, increased reliability of access and effectiveness of
contraceptive can lead to shorter spacing between births 
\citep{Keyfitz1971,Heckman1976}.
With less reliable contraception parents choose a higher level of 
contraception---resulting in longer spacing---to avoid having 
too many children by accident.
But, as contraception becomes more effective parents can more
easily avoid future births, allowing them to reduce the spacing 
between births without having to worry about overshooting their 
preferred number of children.
This idea may also help explain shorter spacing for better 
educated women than for less educated women, provided that   
knowledge and ability to use contraception differ
across education groups \citep{Tulasidhar1993,Whitworth2002}.



% What are the model highlights 
There are three main conclusions from the model.
The increased use of sex selection shows up both in the 
proportion of boys among births and longer spacing between 
births---and the higher the parity the longer the duration between 
births.
The other factors, such as the price of sex selection, 
discount factor, and son preference also affect the use
of sex selection, but to a much smaller degree.
Second, parents do reverse the use of sex selection, 
both when they use sex selection relatively infrequently and when 
they use it heavily.
Reversal are more likely the more impatient parents are or the
shorter time they have left in their reproductive horizon, which
means that the longer the duration has been since the last birth
the more likely they are to reverse their decision to use sex selection.
Finally, the model shows that son preference in the number of
children is not required for an unequal sex ratio.
As long as parents are only willing to abort female fetuses, declining
fertility will lead to increased use of sex selection to ensure 
a balanced number of sons and daughters.


% Empirical method
The model highlights two major limitations of the simple 
empirical method, which a better empirical approach needs to 
allow for.
First, increased use of sex selection leads to longer spacing which,
everything else equal, reduces the number of births available for 
estimation in the simple model and thereby lowers its precision
because fewer parents will make it to a given parity by the time of the survey.
Furthermore, the more intensively sex selection is used the larger 
the decline in sample size.
Second, by using only observed births the simple method cannot capture
when parents change their decision on the use of sex selection;
with a reversal the spell can still result in the birth of a girl.
This last point is especially important when trying to establish whether 
son preference has changed over time in a situation where the cost of prenatal 
sex determination has likely \emph{increased} over time because of the PNDT act.
As I show an increase the cost of prenatal sex determination does 
decrease the sex ratio and lower the number of abortions, but 
it also \emph{increases} the likelihood that parents change their initial
decision to use sex selection within a spell.


%------------------------------------------------------------------------------------


\section{Estimation Strategy\label{sec:strategy}}

To improve on the literature's current approaches to understanding 
son preference, a new empirical method must be able to capture the 
three main implications of the availability of sex-selective abortions
combined with son preference:
a higher probability that the next child is a son, 
longer spacing to next birth with use of sex selection, 
and that the use of prenatal sex determination may change between 
one observed birth and the next.%
\footnote{
In addition, it should be able to capture the shorter spacing between
births observed in the absence of use or availability of prenatal 
sex determination.
}
This means that we need an empirical model that can jointly estimate 
the association between covariates and both the sex of children born 
and the spacing from last birth. 
My proposed model is a discrete time, non-proportional competing risk 
hazard model with two exit states: either a boy or a girl is born.%
\footnote{
\cite{Merli2000} used a discrete hazard model to examine whether 
there were under-reporting of births in rural China, although they 
estimated separate waiting time regressions for boys and girls.
}

I divide a woman's reproductive life into spells that each covers 
the period between births (from marriage to first birth for the first spell).
For each woman, $i=1,\ldots,n$, the starting point for a spell is time $t=1$ and 
the spell continues until time $t_i$ when either a birth occurs or the spell 
is censored.%
\footnote{
The time of censoring is assumed independent of the hazard rate,
as is standard in the literature.
}
There are two exit states: birth of a boy, $j=1$, or birth of a girl, $j=2$, and 
$J_i$ is a random variable indicating which event took place.
The discrete time hazard rate $h_{ijt}$ is
\begin{equation}
 h_{ijt} = \Pr (T_i=t, J_i=j \mid T_i \geq t; \mathbf{Z}_{it},\mathbf{X}_{i} ),
\end{equation}
where $T_i$ is a discrete random variable that captures when woman $i$'s birth occurs.
To ease presentation the indicator for spell number is suppressed.
The vectors of explanatory variable $\mathbf{Z}_{it}$ and $\mathbf{X}_{i}$ include 
information about various individual, household, and community characteristics 
discussed below.

The hazard rate is specified as
\begin{equation}
 h_{ijt} = \frac{\exp(D_j(t) + \alpha_{jt}'\mathbf{Z}_{it} + \beta_j'\mathbf{X}_{i})} 
 {1 + \sum_{l=1}^2 \exp(D_j(t) + \alpha_{lt}'\mathbf{Z}_{it} + \beta_l'\mathbf{X}_{i})} \: \: \; \; \;  j = 1,2
 \label{eq:hazard}
\end{equation}
where $D_{j}(t)$ is the piece-wise linear baseline hazard for outcome $j$, captured
by dummies and the associated coefficients,
\begin{equation}
D_j(t) = \gamma_{j1} D_1 + \gamma_{j2} D_2 + \ldots + \gamma_{jT} D_T,
\end{equation}
where $D_m = 1$ if $t=m$ and zero otherwise.
This approach to modeling the baseline hazard is flexible and does not place 
overly strong restrictions on the baseline hazard.

[TK I am not very satisfied with this explanation of the problems with
proportionality and the associated bias; non-proportionality is required
to capture changes in use of sex selection]

In principle, specifying the model as a proportional hazard model, i.e.\ one 
where covariates simply shift the hazard rates up or down independent of 
spell length, is more efficient, but only provided that the proportionality 
assumption holds.
If the proportionality assumption does not hold, however, the result is
a potentially substantial bias in estimates.
The problem is that a proportional hazard model does not allow covariates
to have different effects at different times within a spell and therefore
cannot capture differences in the shape of the hazard functions between 
different groups. 
It is highly unlikely---even in the absence of prenatal sex determination---that 
the baseline hazards are the same across education levels, areas of residence, 
or sex composition of previous births.
Any bias from the proportionality assumption is likely exacerbated by the 
introduction of prenatal sex determination for two reasons.
First, one of the main points of this paper is that use of sex-selective 
abortions affects birth spacing, and use of sex selection differs across groups.
Second, as discussed above, the use of sex selection may vary within a spell,
depending on the length of the spell.

I therefore use a non-proportional model where the main explanatory variables 
and the interactions between them are interacted with the baseline hazards.
This is captured by the $Z$ set of explanatory variables
\begin{equation}
 \mathbf{Z}_{it} = D_j(t) \times (\mathbf{Z}_1 + Z_2 + \mathbf{Z}_1 \times Z_2),
\end{equation}
where $D_j(t)$ is the piece-wise linear baseline hazard and $\mathbf{Z}_1$ captures sex 
composition of previous children, if any, and $Z_2$ captures area of residence.
This allows the effects of the main explanatory variables on the probabilities 
of having a boy, a girl, or no birth to vary over time within a spell.
The use of a non-proportional specification, together with a flexible baseline hazard, 
also mitigates any potential effects of unobserved heterogeneity \citep{Dolton1995}.

The remaining explanatory variables, $\mathbf{X}$, enter proportionally.
To further minimize any potential bias from assuming proportionality, estimations 
are done separately for different levels of mothers' education and for different 
time periods.
The exact specifications and the individual variables are described below.

Equation (\ref{eq:hazard}) is equivalent to the logistic hazard model and has the same 
likelihood function as the multinomial logit model \citep{allison82,jenkins95}.
Hence, if the data are transformed so the unit of analysis is spell unit rather 
than the individual woman, the model can be estimated using a standard multinomial 
logit model.%
\footnote{
A potentially issue is that the multinomial model assumes that alternative 
exit states are stochastically independent,
also known as the Independence of Irrelevant Alternatives (IIA) assumption.
This assumption rules out any individual-specific unmeasured or 
unobservable factors that affect both the hazard of having a girl and the 
hazard of having a boy.
To address this issue the estimations include a proxy for fecundity
discussed in Section \ref{sec:data}.
In addition, the multivariate probit model can be used as an alternative
to the multinomial logit because the IIA is not imposed \citep{han90}.
The results are essentially identical between these two models and
available upon request.
}
In the reorganized data the outcome variable is zero if the
woman does not have a child in a given period, one if she gives birth to a son 
in that period, and two if she gives birth to a daughter in that period.

The main downside of my approach is that direct interpretation of the estimated 
coefficients for this model is challenging because of the competing risk setup.
First, coefficients show the change in hazards relative to the base outcome, 
no birth, rather than simply the hazard of an event.
Second, a positive coefficient does not necessarily imply that an increase in a
variable's value increases the probability of the associated event because the 
probability of another event may increase even more \citep{thomas96}.

It is, however, straightforward to calculate the predicted probabilities of 
having a boy and of having a girl for each $t$ within a spell, conditional on 
a set of explanatory variables and not having had a child before that period.
The predicted probability of having a boy in period $t$ for a given set of 
explanatory variable values, $\mathbf{Z}_k$ and $\mathbf{X}_k$, is
\begin{equation}
P(b_{t} | \mathbf{X}_{k}, \mathbf{Z}_{kt}, t ) 
=  
\frac{ \exp(D_j(t) + \alpha_{1t}' \mathbf{Z}_{kt} + \beta_1' \mathbf{X}_{k} )}
{1 + \sum_{l=1}^2 \exp(D_j(t) + \alpha_{lt} ' \mathbf{Z}_{kt} + \beta_l ' \mathbf{X}_{k})},
\label{eq:probability_boy}
\end{equation}
and the predicted probability of having a girl is
\begin{equation}
P(g_{t} | \mathbf{X}_{k}, \mathbf{Z}_{kt},t ) 
=  
\frac{ \exp(D_j(t) + \alpha_{2t}'\mathbf{Z}_{kt} + \beta_2'\mathbf{X}_{k} )}
{1 + \sum_{l=2}^2 \exp(D_j(t) + \alpha_{lt}'\mathbf{Z}_{kt} + \beta_l'\mathbf{X}_{k})}.
\label{eq:probability_girl}
\end{equation}
With these two probabilities is it easy to calculate, for each $t$, the estimated
percentage of children born that are boys, $\hat{Y}$, 
\begin{equation}
\hat{Y}_t 
= 
\frac{ P(b_{t} | \mathbf{X}_{k}, \mathbf{Z}_{kt},t )}
{ P(b_{t} | \mathbf{X}_{k}, \mathbf{Z}_{kt},t) + P(g_{t} | \mathbf{X}_{k}, \mathbf{Z}_{kt},t )} 
\times 100,
\label{eq:probability_son}
\end{equation}
together with the associated confidence interval for given values of explanatory 
variables.%
\footnote{
Within each period $1-P(b_{t})-P(g_{t})$ is the probability of not having a birth in 
period $t$.
}

For ease of exposition the procedure is presented here in two steps, but the actual
calculation of the percent boys is done in one step with the 95 percent confidence 
interval calculated using the Delta method.
Results are presented as the estimated percent boys born by length of birth spacing
using graphs.%
\footnote{
The parameter estimates are available on request.
}
For each graph, the extent to which the percent boys is statistically 
significantly above the natural sex ratio indicates the use of sex selection.


[TK  may need to combine with and without sons to show differences in spacing 
over time; maybe it should also have predicted average/median spacing with some 
confidence intervals to test]


The other important part of the model is the spacing between births.
Spacing is captured by the survival curve, which shows the probability of not 
having had a birth yet by spell duration.
The survival curve at time $t$ is 
\begin{equation}
S_{t} 
= 
\prod_{d=1}^t 
\left( 
	1- \left(P(b_{d} | \mathbf{X}_{k}, \mathbf{Z}_{kd}, d) 
	+ P(g_{d} | \mathbf{X}_{k}, \mathbf{Z}_{kd}, d) \right) 
\right),
 \label{eq:survival}
\end{equation}
or equivalently
\begin{equation}
S_{t} 
= 
\prod_{d=1}^t
\left(
\frac{ 1 }
{1 + \sum_{l=2}^2 \exp(D_j(t) + \alpha_{ld}'\mathbf{Z}_{kd} + \beta_l'\mathbf{X}_{k})}
\right).
\end{equation}

In the absence of sex selection, I expect most parities to show an 
``inverted s'' pattern, where there initially are relatively few births, 
followed by a substantial number of births over a 1 to 2 year period, and then
relatively few births thereafter.
As sex selection becomes more widely used the associated
longer spacing shows up by making the survival curve straighter,
indicating that some of the births that would originally have taken place
now take place later because of abortions.
The use of sex selection is clearly not the only factor that can change the shape of
the survival curves; factors such as the desired number of children and
use of contraceptives may also shape the shape.
To account for this it is best to compare survival curves for an individual
parity between women who are likely to use sex selection and women who are
not, for example because they already have one or more sons.

In addition to information about spacing, the survival curves also provide 
``weighting'' for the associated percentage boys born.
The steeper the survival curve, the more weight should be assigned to a given
spell period because it is based on more births, 
whereas a period with a flat survival curve should be given little weight because the 
percentage boys is based on few births.
Hence, although they cannot by themselves show sex selection, the survival
curves are a crucial complement to the estimated sex ratios by duration.


\section{Data\label{sec:data}}

The data come from the three rounds of the National Family Health Survey 
(NFHS-1, NFHS-2 and NFHS-3),
collected in 1992--93, 1998--99, and 2005--2006.%
\footnote{
A delay in the survey for Tripura means that NFHS-2 has a small number of observation 
collected in 2000.
}
The surveys are large: NFHS-1 covered 89,777 ever-married women 
aged 13--49 from 88,562 households,
NFHS-2 covered 90,303 ever-married women aged 15--49 from 92,486 households
and NFHS-3 covered 124,385 never-married and ever-married women aged 
15--49 from 109,041 households.

I exclude visitors to the household, as well as
women married more than once, divorced, or not living with their husband,
women with inconsistent information on age of marriage,
and those with missing information on education.
Women interviewed in NFHS-3 who were never married or where gauna had not
been performed were also dropped.
The same goes for women who had at least one multiple birth,
reported having a birth before age 12, had a birth before marriage, or
a duration between births less than 9 months.
Women who reported less than 9 months between marriage and first birth
remain in the sample unless they are dropped for another reason.%
\footnote{
Women who report less than 9 months between marriage and first birth are retained 
because between 10 and 20 percent fall into this category.
Although it is possible that some of these births are premature the high number of
women who report a birth less than 6 months after marriage indicates that conception
likely occurred before marriage in most cases.
}

Finally, I restrict the sample to Hindus,
who constitute about 80 percent of India's population.
If use of sex selection differ between Hindu and other religions, such 
as Sikhs, assuming that the baseline hazard is the same would lead to bias.
The other groups are each so small relative to Hindus that it is not
possible to estimate different baseline hazards for each group.
Furthermore, the groups are so different in terms of background and son preference
that combining them into one group would not make sense.

There are four advantages to using the NFHS.
First, surveys enumerators pay careful attention to spacing between births and
probe for ``missed'' births.
Second, no other surveys cover as long a period in the same amount of detail.
The three NFHS rounds allow me to show the development in spacing and 
sex ratio from before sex-selective abortions were available until 2006.
Third, NFHS has birth histories for a large number of women.%
\footnote{
The Special Fertility and Mortality Survey appears to cover a much large number of households
than the three rounds of the NFHS combined, but \citet{jha06} only use the births that 
took place in 1997 making their sample sizes by parity smaller than here.
Their sample consists of 133,738 births of which 38,177 were first
born, 36,241 second born, and 23,756 were third born.
The differences in results for first born children are discussed in the online 
Appendix.
}
Finally, even if probing for missing births may not completely eliminate recall error,   
the overlap in cohorts covered and the large sample size make it possible to establish 
where recall error remains a problem.

Recall error arise mainly from child mortality, when respondents are reluctant to
discuss deceased children.%
\footnote{
The online Appendix contains a more thorough discussion of recall error and how I address it.
} 
Systematic recall error, where the likelihood of reporting a deceased child depends on
the sex of the child, is especially problematic because it biases the sex ratios.
Probing catches many missed births, but systematic recall error is still a potentially 
substantial problem.
Three factors contribute to the problem here.
First, girls have significantly higher mortality risk than boys.
Second, son preference may increase the probability that boys are remembered relative to girls.
Finally, in NFHS-1 and NFHS-2 enumerators probed only for a missed birth if the
initial reported birth interval was four calendar years or more.
But, given short durations between births, especially after the birth of a girl,
that procedure is unlikely to pick up all missed children.

Observed sex ratios by cohort provide a straightforward way to determine 
whether recall error is a problem.
Because prenatal sex determination techniques did not become widely available until the 
mid-1980s, a higher than natural sex ratio for cohorts born before that time must be 
the result of systematic recall error.
As shown in the online Appendix, the observed sex ratio by parity becomes more male 
dominated the further back births took place.
In addition, births in the same cohort tend to be more male dominated the more recent the 
survey (births in the cohort took place longer ago relative to the survey).
Hence, there is evidence of recall error and the degree of recall error increases
with length of time between survey and cohort.

Using cohort year of birth to analyze recall error and decide which observations
to keep is, however, problematic because the year of birth for a given parity is affected 
by recall error; for example, a second born child listed as first born will be 
born later than the real first born child.
Year of marriage should, however, be unaffected by recall error.
Using year of marriage the basic recall error pattern remains with women married longer 
ago more likely to report that their child was a boy for a given parity.
Similarly, comparing women married in the same period across surveys shows
that women married longer ago are more likely to report having sons.

That recall error increases the longer ago somebody was married means
that duration of marriage is a better predictor of recall error than calendar year of 
marriage.
Figure \ref{fig:sexRatioMarriage} shows the observed sex ratio for children 
reported as first born as a function of duration of marriage combining all three surveys.%
\footnote{
The graph for second births shows a similar pattern.
The graphs for the second births and the individual survey rounds are available upon request.
}
The solid line is the sex ratio of children reported 
as first born by the number of years between the survey and marriage, 
the dashed lines indicate the 95 percent confidence interval 
and the horizontal line the natural sex ratio (approximately 0.512).
To ensure sufficient cell sizes the years are grouped in twos.

\begin{figure}[htp]%
\centering
\includegraphics[width=0.9\textwidth]{sex_ratio_marriage}
\caption{Ratio of boys in ``first'' births}
\label{fig:sexRatioMarriage}
\end{figure}


Figure \ref{fig:sexRatioMarriage} clearly illustrates the systematic recall error
problem.
The observed sex ratio is increasingly above the expected value the
longer ago the parents were married.
The increasingly unequal sex ratio with increasing marriage duration suggests that
a solution to the recall error problem is to drop women who were married ``too far'' from 
the survey year.
The main problem is establishing what the best cut-off point should be.
The observed sex ratio is consistently significantly higher than the natural sex ratio 
from around 24 years of marriage, so one possibility is simply to drop all women married 
more than 24 years at the time of the survey.
But, as the Appendix shows, there are differences across the three surveys and between 
parities.
I therefore use different cut-off points by survey round.
For NFHS-1 women married 22 years or more were dropped, with the corresponding cut-off 
points 23 years for NFHS-2  and 26 years for NFHS-3.
The final sample consists of 146,096 women, with 332,951 parity one through four births.%
\footnote{
The online Appendix presents the results for a more restrictive definition and for the
women dropped because of recall error concerns.
}


\subsection{Spell Definition\label{sec:spell_def}}

Spell duration is measured in quarters of a year, that is 3 month periods, hereafter
referred to as quarters.
The first spell begins at the month of marriage rather than 9 months after because many 
women report giving birth less than 9 months after they were married.
For women who began living with their husbands at too young an age to conceive the 
starting point should ideally be first ovulation, when she becomes ``at risk'' for a 
pregnancy, rather than month of marriage.
Unfortunately, information on age of menarche is only available in NFHS-1.
Instead, for women who began living with their husband before age 12, I set the 
the month they turned 12 years of age as staring point for the first spell.

The second and subsequent spells begin 9 months after the previous birth 
because that is the earliest we should expect to observe a new birth.
A few women report births that occurred less than 9 months 
after the previous birth; those women are dropped.

All spells continue until either a child is born or the spell is censored.
Censoring can happen for three reasons:
the survey takes place;
the woman is sterilized;
or the number of births observed becomes too sparse for the method to work.
The timing of censoring because of too few births vary slightly by spell but is generally 
21 quarters after the beginning of the spell.

I group spells into three time periods based on spell start date:
1972--1984, 1985--1994, and 1995--2006.
The first period covers the time before sex-selective abortions became widely available.
Abortion was legalized in 1971 and amniocentesis was introduced
in India in 1975, but the first newspaper reports on the availability of prenatal sex 
determination were not until 1982--83 \citep{Sudha1999,bhat06,Grover2006}.
The number of clinics quickly increased, and knowledge about sex selection became widespread
after a senior government official's wife aborted a fetus that turned out to be male \citep[p.\ 598]{Sudha1999}.
The second period covers the time from the widespread emergence of sex-selective abortions
until the prenatal Diagnostic Techniques (PNDT) act was passed in 1994.
The final period is from the PNDT act until the last available survey.
The PNDT act made it a criminal offence to reveal the sex of the fetus and was
followed by a campaign against the use of sex selection, although
enforcement appears to be relatively lax.

By dividing spells into these periods I can examine how the use of sex selection 
has changed across the three different regimes.
Note that the periods are based on the spells' beginning year, and some spells 
will therefore cover two periods.
A couple may, for example, be married in 1984, but not have their first child until 1986.
That couple's first spell will be in the 1972--1984 period, even though most of the 
spell actually falls in the 1985--1994 period.
Some children born from spells that began in the 1972--1984 period may therefore have been
conceive when prenatal sex determination techniques were available, which could result
in evidence of sex-selective abortions even for this period.
Similarly, a spell that began in the 1985--1994 period may have been partly or mostly
under the PDNT act.
The overall effect is to bias downwards any differences between the periods.

\subsection{Explanatory Variables}

The explanatory variables are divided into two groups.
The first group consists of variables expected to affect the shape of the hazard function: 
mother's education, sex composition of previous children, and area of residence.
Increasing the number of variables interacted with the baseline hazard lowers the risk 
of bias but requires more data to precisely estimate.
I chose these variables because the prior literature shows that they affect 
spacing choices and because the prior literature on sex selection indicates 
that these are correlated with sex selection.
The second group of variables are those expected to have an approximately 
proportional effect on the hazard: age of the mother at the beginning of the spell, 
length of her first spell (for second spell and above), whether the household owns 
land, and whether the household belongs to a scheduled tribe or caste.

Increasing education of mothers is strongly associated with lower fertility, with
the negative effect of higher opportunity cost on fertility more than outweighing the 
positive effect of higher income \citep{schultz97}.
Higher education should therefore be associated with higher use of sex selection.%
\footnote{
Fathers' education has two opposite predicted effects: the associated higher income
should increase fertility and therefore lower the pressure to use sex selection, but
the higher income also makes the use of sex selection cheaper.
In practice, fathers' education had little effect on the hazards and the use of 
sex-selective abortions and is not included.
}
I divide women into three education groups:
no education, 1 to 7 years of education, and 8 and more years of education.
The models are estimated separately for each education level.%
\footnote{
A potential concern here is reverse causation, where the sex of children
affect women's education.
Although no direct information is available on when women left school it is
possible to estimate school leaving age from the highest completed grade
and the usual starting age.
As described in the online Appendix, relatively few women could potentially 
have returned to school after the birth of their first child,
and only 56 could possibly have ended up in the wrong education group. 
Hence, there is little likelihood that reverse causation is a substantial
concern here.
}

As discussed, the sex composition of previous children affects both the timing
of births and the use of sex-selective abortions.
I capture sex composition of previous children with dummy variables for the
possible combinations for the specific spell, ignoring the ordering of births.
As an example, for the third spell three groups are used: Two boys,
one girl and one boy, and two girls.

The area of residence is a dummy variable for the household living in
an urban area.%
\footnote{
NFHS uses four categories for area of residence: Large city, small city, town
and countryside.
To reach a sufficient sample size urban areas are merged into one group.
}
The cost of children is higher in urban areas than in rural and access to prenatal
sex determination is easier, and both are expected to be associated with
greater use of sex-selective abortions in urban than in rural areas.
Because of concerns about selective migration I use where the household was living
at the end of each spell.%
\footnote{
As online Appendix Table D.1 shows there is little difference across
education levels in migration patterns within each period and patterns across periods
are also relatively stable.
}

The sex composition of children, area of residence, and the interactions between
these are all interacted with the piece-wise linear baseline hazard dummies.
In other words, the baseline hazards are assumed to be different depending on
where a woman lives and the sex composition of her previous children.
As an example, for the second spell a separate regression is
run for each education level and in each regression four different baseline hazards 
are included (first child a boy in rural area, first child a boy in urban area,
first child a girl in rural area, first child a girl in urban area).
Although this approach substantially increases the number of regressions and 
estimated parameters it reduces the potential problem of including other variables 
as proportional effects.

The remaining variables are expected to affect hazards proportionally.
Although fecundity cannot be observed directly a suitable
proxy is the duration from marriage until first birth.
Most Indian women do not use contraception before the first birth
and there is pressure to show that a newly married woman can conceive 
\citep{dyson83,Sethuraman2007,Dommaraju2009}.
This is confirmed by the very short spells between marriage and first birth,
even among the most educated.
Hence, a long spell between marriage and first birth is likely due to low fecundity.
For both this variable and the age of the mother at the beginning of the spell 
the squares are also included.
The remaining variables are dummies for household ownership of land and membership
of a scheduled caste or tribe.


\subsection{Descriptive Statistics}

Appendix Table \ref{tab:des_stat1} presents descriptive statistics for
the spells by education level and when the spell began.
There is a substantial number of censored observations.
As an example, for highly educated women who had their first child in the 1995--2006
period, almost half did not have their second child by the time of the survey.
Hence, although about 13,000 women began the second spell 
there are only about 7,000 births to these women.
Censoring becomes even more important for the third and fourth
spells, where around 70 percent of the observations are censored.
Generally, censoring increases with parity and time period.
This reflects a combination of factors: timing of the surveys
relative to the periods of interest, later beginning of childbearing, 
falling fertility, and the longer spells from sex-selective abortions.
The high number of censored observations underscore the importance of controlling for
censoring when examining the relationship between fertility and sex selection.

The descriptive statistics also provide a first indication of how the
sex ratio at birth changes over time and by spell.
For the first spell the sex ratio is very close to the natural
for all education groups and all three time periods.
As an example, among the highly educated group for the 1995--2006 period,
51.3 percent of the children born were boys.%
\footnote{
There still appears to be some recall error for the group of women without
education for the 1972--1984 period, where 52.3 percent of the children born were boys.
}
For the second spell, all but the highly educated group in the last two
periods have sex ratios in line with the natural sex ratio.
Women with 8 or more years of education have 53.1 and 54.3 percent
boys in the 1985--1994 and 1995--2006 periods, respectively.
This pattern repeats itself for the third spell, except the percentage
boys is higher for the high education group (55.3 and 55.9 for the last
two time periods).
Finally, for the fourth spell the high education group 
had 60 percent boys in the last period, i.e. after the PNDT act was introduced.
Note, however, that for the fourth spell the number of births is substantially
smaller and censoring even more important than for the other spells.

India's population has become progressively more urban.
For the first period, 32 percent of the women entering the first spell lived in urban areas.
This increases to 35 percent for the second period and to 42 percent for the final period.
The population is also substantially better educated.
Women with no education constituted almost 60 percent 
in the first period, but less than thirty percent in the last period.
Correspondingly, in the first period just over twenty percent had 8 or more 
years of education, but in the last period it was almost half.
Part of the increase in education is correlated with the increase in urbanization,
but the proportion of better-educated women has increased substantially
in the rural areas as well.
Among the high education group almost 70 percent lived in urban areas
during the first period but this had fallen to less than 60 percent
in the last period.

The increase in urbanization and education is likely to exert downward pressure
on fertility and the high censoring rates for the later periods are evidence of this.
The average number of children born to women by the time they turn 35 illustrate how
strong the decline in fertility has been.%
\footnote{
Figure \ref{fig:fertility} shows the full results.
}
Women born in the early 1940s had on average close to 5 children when 
they reached 35, but women born in the early 1970s had only just over three children.
The low number of children is especially remarkable because it combines
all education levels and all areas of residence.
% Hence, fertility in cities is likely substantially lower.


\section{Results\label{sec:results}}

One of the main questions here is how fertility affects sex selection use.
I start by showing the relationship between self-reported desired fertility and sex 
selection, which also illustrates how to interpret the results.
Although self-reported desired fertility is an imperfect proxy for actual
fertility behavior, the analysis of desired fertility indicates that there is 
a relationship between fertility decisions and the use of sex selection and motivates 
examining the relationship more closely.

Because of the desired fertility measure's potential shortcomings, I next examine the 
relationship between actual fertility and use of sex selection for better-educated women.
There are two steps in this process.
First, I show how fertility has declined for better-educated women with one son---a group 
unlikely to use sex selection, which means that sex selection will not affect the 
progression rates to next birth for these women.
Second, I show how, corresponding to the fertility decline, there is an increase in
sex-selective abortions for better-educated women without a son.
I next show how women with lower cost of children appear to follow a high fertility, low
use of sex selection strategy compared to the low fertility, high use of sex selection of
the better-educated women.
The results all show clear evidence of son preference in fertility and sex selection use, 
but exactly what form son preference takes is important, so I next discuss what we can
learn about son preference from the sex selection behavior.
Finally, I predict lifetime fertility, number of abortions, and final sex ratios for
the better-educated women.

Except for the predictions, results are presented using graphs of estimated percentage boys 
born by quarter, the associated 95 percent confidence interval, and the survival curve for 
a ``representative woman'' using the method detailed in Section \ref{sec:strategy}.
The ``representative woman'' characteristics are based on means of continuous 
explanatory variables and the majority category for categorical explanatory variables.
Common to all ``representative women'' is that they do not own land and belong to neither 
a scheduled caste nor a scheduled tribe.
The graphs also show the expected natural rate of boys, approximately 51.2 percent, for
comparison.
Graphs for all groups and spells, even if not discussed here, are available in 
the online Appendix.


\subsection{One Son or Many Sons?}

The extent to which people use sex selection for different sex compositions
of previous children is of interest for two reasons.
First and foremost, it can help us understand what form son preference takes in India.
This is important in itself, but is especially important as we try to predict 
what will happen to sex selection use in the future.
[need an explanation of this based on what I have in the introduction]
Second,
if families use sex selection to reach two or more sons, progression rates for families 
with one son may overestimate the decline in fertility.

Women with 8 or more years of education is the group the most likely to use sex 
selection if they have no son as shown above.
Hence, if sex selection is also used to achieve more than one son, we should be
most likely to find evidence for this group of women as well.
Figure \ref{fig:boys_latest} therefore shows sex ratios and survival curves for 
the 1995--2006 period for urban and rural women with 8 or more years of education, 
conditional on having at least one son.
The first column shows outcomes for the second spell if the first born is a son,
the second column shows third spell when the first two children were a son and 
a daughter, and the third column shows third spell when the first two children were sons.


\begin{figure}[htpb]
\centering
\caption*{Urban}
\setcounter{subfigure}{-2}
\subfloat[Spell 2 - 1 boy (N=3,969)]{
    \begin{minipage}{0.30\textwidth}
        \captionsetup[subfigure]{labelformat=empty,position=top,captionskip=-1pt,farskip=-0.5pt}
        \subfloat[Prob. boy (\%)]{\includegraphics[width=\textwidth]{spell2_g3_high_r2_pc}}\\
        \subfloat[Prob. no birth yet]{\includegraphics[width=\textwidth]{spell2_g3_high_r2_s}}
        \captionsetup[subfigure]{labelformat=parens}
    \end{minipage}
}
\setcounter{subfigure}{-1}
\subfloat[Spell 3 - 1 boy/1 girl (N=2,357)]{
    \begin{minipage}{0.30\textwidth}
        \captionsetup[subfigure]{labelformat=empty,position=top,captionskip=-1pt,farskip=-0.5pt}
        \subfloat[Prob. boy (\%)]{\includegraphics[width=\textwidth]{spell3_g3_high_r4_pc}}\\
        \subfloat[Prob. no birth yet]{\includegraphics[width=\textwidth]{spell3_g3_high_r4_s}}
        \captionsetup[subfigure]{labelformat=parens}
    \end{minipage}
}
\setcounter{subfigure}{0}
\subfloat[Spell 3 - 2 boys (N=1,098)]{
    \begin{minipage}{0.30\textwidth}
        \captionsetup[subfigure]{labelformat=empty,position=top,captionskip=-1pt,farskip=-0.5pt}
        \subfloat[Prob. boy (\%)]{\includegraphics[width=\textwidth]{spell3_g3_high_r2_pc}}\\
        \subfloat[Prob. no birth yet]{\includegraphics[width=\textwidth]{spell3_g3_high_r2_s}}
        \captionsetup[subfigure]{labelformat=parens}
    \end{minipage}
}
\caption*{Rural}
\setcounter{subfigure}{1}
\subfloat[Spell 2 - 1 boy (N=3,044)]{
    \begin{minipage}{0.30\textwidth}
        \captionsetup[subfigure]{labelformat=empty,position=top,captionskip=-1pt,farskip=-0.5pt}
        \subfloat[Prob. boy (\%)]{\includegraphics[width=\textwidth]{spell2_g3_high_r1_pc}}\\
        \subfloat[Prob. no birth yet]{\includegraphics[width=\textwidth]{spell2_g3_high_r1_s}}
        \captionsetup[subfigure]{labelformat=parens}
    \end{minipage}
}
\setcounter{subfigure}{2}
\subfloat[Spell 3 - 1 boy/1 girl (N=1,819)]{
    \begin{minipage}{0.30\textwidth}
        \captionsetup[subfigure]{labelformat=empty,position=top,captionskip=-1pt,farskip=-0.5pt}
        \subfloat[Prob. boy (\%)]{\includegraphics[width=\textwidth]{spell3_g3_high_r3_pc}}\\
        \subfloat[Prob. no birth yet]{\includegraphics[width=\textwidth]{spell3_g3_high_r3_s}}
        \captionsetup[subfigure]{labelformat=parens}
    \end{minipage}
}
\setcounter{subfigure}{3}
\subfloat[Spell 3 - 2 boys (N=790)]{
    \begin{minipage}{0.30\textwidth}
        \captionsetup[subfigure]{labelformat=empty,position=top,captionskip=-1pt,farskip=-0.5pt}
        \subfloat[Prob. boy (\%)]{\includegraphics[width=\textwidth]{spell3_g3_high_r1_pc}}\\
        \subfloat[Prob. no birth yet]{\includegraphics[width=\textwidth]{spell3_g3_high_r1_s}}
        \captionsetup[subfigure]{labelformat=parens}
    \end{minipage}
}
\caption{Predicted probability of having a boy and probability of
no birth by quarter (3 month period) for women with 8 or more years
of education for 1995--2006. 
N is number of women in the relevant group in the underlying samples.
}
\label{fig:boys_latest}
\end{figure}


For the second spell the predicted sex ratios closely follow the natural sex 
ratio for both urban and rural women.
Furthermore, the survival curves follow the expected pattern if no sex selection 
is taking place.
There is also no evidence of sex selection for the third spell for women with two sons;
the sex ratios follow the natural sex ratio for both urban and rural women.
Sample sizes are, however, less than \sfrac{1}{3} of the sample sizes for the second 
spell and the progression rate to a third birth is low at only 30\% for urban women and 
55\% for rural women, resulting in wide confidence intervals.
The survival curves are particularly flat for the first year, meaning that the apparent 
deviations from the natural sex ratio in the beginning of the spell are based on very few births.

% [Spell 3 - 1b/1g most interesting]

Even though there is no evidence of using sex selection to have an additional son when the 
previous children are all sons, it is possible that families want more sons than daughters 
and therefore use sex selection for the third spell when they already have one of each.
Urban women show little evidence of this being the case.
Although the sex ratio is above the natural from quarter 11 until the end of the spell,
only about \sfrac{1}{3} of the births occur in this interval.
Furthermore, the dip below the natural sex ratio prior to this point also accounts for
approximately \sfrac{1}{3} of the births.
Hence, these two parts cancel each other out.
With the first part being to the natural sex ratio the end result is that there
is little evidence for urban women using sex selection if they already have a son and
a daughter.

For rural women, however, the dip and the associated number of births is not
large enough to bring the overall sex ratio for the spell back to natural sex ratio.
The sex ratio for the entire spell is therefore likely more male dominated
than the natural rate, although it is only statistically significantly higher for two 
individual quarters of the spell.
There are two possible explanations for rural women appearing more likely than urban
women to use sex selection on the third spell if they already have a son and a daughter.
First, rural women could have a stronger son preference than urban women and want two sons 
rather than only one.
Second, mortality is higher in rural than urban areas and this higher mortality, either 
experienced or anticipated, make rural women use sex selection to secure a second son, 
essentially as an insurance.
Separating these two possible explanations is difficult, but the ``heir and a spare''
explanation appears  more likely than difference in preferences given that urban women use 
sex selection more than rural women for the second birth in the absence of a son.
An alternative explanation is that what we are observing is simply random variation.
This would explain why we get see differences in sex ratios between urban and rural
women with one son across the third and fourth spells as shown in the online Appendix.%
\footnote{
Urban women with a son and two daughters as their first 3 births show a sex ratio
that is significantly higher than the natural, but no evidence of sex
selection if they have two sons and a daughter as their first children.
Rural women show only a marginally higher sex ratio with a son and two daughters and no 
effect at all with two sons and a daughter.
These results are, however, based on very small samples.
There are 590 urban women with a son and two daughters, of which less than 40\% have a
fourth child, 344 urban women with two sons and one daughter, and only 20\% of those
have a fourth child.
For rural women the numbers are 564, with 40\% having a fourth child, and 367 with
approximately 30\% having a fourth birth.
This comes to approximately 300 urban births and only slightly more for rural across
the two sex compositions.
}

Son preference can manifest itself through other channels than sex selection.
Differential investments in education and health across boys and girls, resulting in
lower education and higher mortality of girls relatively to boys, could be thought 
of as evidence of son preference, although \cite{rosenzweig82} argue that both may 
represent parents' rational response to market opportunities rather than inherent 
preferences of parents.
Whatever way son preference presents itself, it appears that the dominant  
preference---as expressed through use of sex selection---is for one son, rather 
than multiple sons.


\subsection{Predicted Lifetime Number of Children and Abortions}

One of the main advantages of the method proposed is that we can predict what 
completed fertility, total number of abortions, and the sex ratios will be for 
women in the samples at the end of childbearing. 
This cannot be done with the simple model because it does not predict fertility
progression, does not take into account censoring, and cannot capture if parents
change their use of sex selection within a spell.

The predictions are important for three reasons.
First, they allows us to quantify whether overall use of sex selection is 
intensifying or easing over time by examining how the total number of abortions 
women will have over their childbearing, and the resulting final sex ratios, change.
This is a more accurate representation than what we get from sex ratios 
for individual parities.
As the theory section shows, it is possible to get no change---or even a 
decline---in use of sex selection for an individual parity, at the same time 
as the overall use of sex selection is increasing.
Second, completed fertility and sex ratios are what matters when examining the future 
impacts of sex selection in areas such as the chance of finding a spouse, savings 
behavior, and the potential problems associated with a surplus of bachelors 
\citep{lancaster02,Ding2009,Wei2009,Edlund2013}.
Finally, as I show in the online Appendix and discuss above, relying only on observed 
sex ratios of recorded births is problematic because these are based on selection and 
the observed sex ratio may be biased predictors of final sex ratio when people change 
their use of sex selection during a spell.

Table \ref{tab:predicted} shows predicted birth progression rates, number of abortions, 
sex ratios, and completed fertility for the three samples of women with 8 or more years 
of education:  
the 11,271 women who were married in the the 1972-1984 period,
the 19,072 women who were married in the 1985--1994 period,
and the 16,620 women who were married in the 1995--2006 period.%
\footnote{
These are the women used to estimate the results for the spell from marriage to first 
birth above.
For more information see Table \ref{tab:des_stat1}.
}
Predictions for each period assumes that behavior remains constant and that all spells 
a woman goes through are within the same period. 
Each woman is assumed to follow the fertility and sex selection behavior for 
each spell up to and including the fourth spell using the period's estimates and her 
characteristics. 
All results are for women predicted to have one or more births.%
\footnote{
This means that fertility predictions are larger than standard estimates, which include
women with no children.
Working in the opposite direction is that number of children is restricted to be at most 
four.
}

The predictions are done as follows.
Starting with the first spell I calculate each woman's probabilities of having a boy,
$P(b_t)$, a girl, $P(g_t)$, or no birth, $1-P(b_t)-P(g_t)$, in each quarter after marriage
based on the period results for 
equations (\ref{eq:probability_boy}) and (\ref{eq:probability_girl}) using her individual 
values for $\mathbf{Z}_{it}$ and $\mathbf{X}_{i}$.
She then is assigned one of these outcomes---boy, girl, or no birth---based on random 
draws from a uniform distribution.
The first quarter she has a draw that results in a birth she leaves the spell
and the quarter reached is used to calculate her starting age for the second spell.
If she has no draw that result in a birth by the end of the spell her childbearing 
stops and she does not enter the next spell.
If she does have birth, the process is repeated for the second through fourth spells.
Randomization is necessary because each spell's results, except for the first, depends on
the sex composition of previous children and the duration of the first spell. 
The results presented are averaged outcomes over all urban women and all rural women 
using 1,000 repetitions.


Birth progression rates are the percentage of women predicted to have a birth 
by the end of each spell.
Fertility is the average number of children born according to the randomization.
With the large number of replications this number is the same as using progression
rates at the end of the spell to calculate the average number of children.%
\footnote{
Given that all numbers are conditional on having at least one child, using progression
rates for, say, urban women from the 1985-1994 period leads to 
$1+1\times0.782+1\times0.782\times0.499+1\times0.782\times0.499\times0.448=2.3 $  
children.
}

For abortions I first calculate each quarter's probability of abortions using
$\frac{100}{105}P(b_t)-P(g_t)$,
where $\frac{100}{105}P(b_t)$ is the probability of having a girl that should have 
prevailed in the absence of sex selection (0.952 girls born per boy).
This probability is multiplied with the probability of not having had a birth yet at 
the beginning of the quarter.
Abortions for each spell are then found by summing over the entire spell.%
\footnote{
One way to see how this works is to imagine that we start the spell with 100 women.
After the first quarter we are left with $N_2 = (1-P(b_1)-P(g_1)) \times 100$, where
$N_2$ is the number of women who start the second quarter. 
After the second quarter we then have $N_3 = (1-P(b_2)-P(g_2)) \times N_2$, etc.

Say that 65.88 women of the original 100 still have not had a birth yet by the start
of quarter 6 and that the predicted probability of having a son in quarter 6 is 0.0931 and 
the probability of having a girl is 0.0566 (the probability of not having a birth in 
quarter 6 is 0.8503).
For this quarter the abortion probability is then 
$\frac{100}{105}\times 0.0931 - 0.0566 = 0.0321$.
In other words, the probability of having a girl should have been 
$\frac{100}{105}\times 0.0931 = 0.0887$, but only a probability of $0.0566$ was 
predicted.
Hence, in quarter 6 the predicted number of sons born is $ 0.0931 \times 65.88 = 6.13$, 
the predicted number of daughters born is $ 0.0566 \times 65.88 = 3.73$, and 
$0.0321 \times 65.88 = 2.11$ abortions must have taken place to reach these births.
}
To account for random variation, the abortion rate is the sum over both positive 
(more boys than the natural rate) and negative (more girls than the natural
rate) values of the abortion calculation within the spell.
Finally, the percentage of children born that are boys is based on the randomization of
births described above.%
\footnote{
This number is the predicted sex ratio at birth and differ from actual observed sex 
ratios when the children are older because of mortality.
On one hand, the anticipated or experienced death of a son increases the likelihood
of sex selection even if parents already had a son, which would make the later observed 
sex ratio lower than what is predicted here. 
On the other hand, mortality risk is higher for girls than for boy, which would make
the sex ratio more male dominated.
}

% Tables of predictions 
\begin{table}[htbp]
\begin{center}
\begin{scriptsize}
\begin{threeparttable}
\caption{Predicted Fertility Behavior, Sex Ratio, and \protect\linebreak
Abortions for Women with 8 or More Years of Education}
\label{tab:predicted}
\begin{tabular} {@{} l D{.}{.}{2.1} D{.}{.}{2.1} D{.}{.}{2.1} D{.}{.}{2.1}  D{.}{.}{2.1} D{.}{.}{2.1} D{.}{.}{2.1} D{.}{.}{2.1} D{.}{.}{2.1} D{.}{.}{2.1} D{.}{.}{2.1} D{.}{.}{2.1}  @{}} \toprule
                 & \multicolumn{4}{c}{1972-1984}                                                               & \multicolumn{4}{c}{1985-1994}                                                               & \multicolumn{4}{c}{1995-2006} \\ \cmidrule(lr){2-5} \cmidrule(lr){6-9} \cmidrule(lr){10-13}
                 & \mco{Births (\%)}          & \mct{Abortions per 100}                      & \mco{Boys}      & \mco{Births (\%)}          & \mct{Abortions per 100}                      & \mco{Boys}      & \mco{Births (\%)}          & \mct{Abortions per 100}                      & \mco{Boys}      \\
                 & \mco{/ fertility\tnote{a}} & \mco{births\tnote{b}} & \mco{women\tnote{c}} & \mco{(\%)}      & \mco{/ fertility\tnote{a}} & \mco{births\tnote{b}} & \mco{women\tnote{c}} & \mco{(\%)}      & \mco{/ fertility\tnote{a}} & \mco{births\tnote{b}} & \mco{women\tnote{c}} & \mco{(\%)}      \\
\midrule
\bf Urban        & \multicolumn{12}{c}{} \\                                                                    
Spell 2          &       85.9   &  2.0    &  1.8    &  52.3   &       78.2   &  4.7    &  3.7  &  53.7   &       68.0   &  8.0    &  5.4   &  55.3      \\
Spell 3          &       64.4   &  5.1    &  3.3    &  53.9   &       49.9   &  9.8    &  4.9  &  56.2   &       38.6   &  7.8    &  3.0   &  55.3      \\
Spell 4          &       57.0   &  2.8    &  1.6    &  52.6   &       44.8   &  2.6    &  1.1  &  52.6   &       33.7   &  21.8   &  7.3   &  62.3      \\
Overall\tnote{d} &       2.7    &  2.0    &  5.5    &  52.3   &       2.3    &  3.4    &  8.0  &  53.1   &       2.0    &  4.6    &  9.4   &  53.7      \\
\addlinespace 
\bf Rural        & \multicolumn{12}{c}{} \\                                                                                  
Spell 2          &       92.5   &  -1.4   &  -1.3   &  50.5   &       87.6   &  3.1    &  2.7  &  52.8   &       83.1   &  4.4    &  3.6   &  53.5      \\
Spell 3          &       80.9   &  2.6    &  2.1    &  52.5   &       70.6   &  6.5    &  4.6  &  54.5   &       59.6   &  10.6   &  6.3   &  56.7      \\
Spell 4          &       75.7   &  -3.1   &  -2.4   &  49.6   &       62.0   &  0.2    &  0.1  &  51.3   &       51.9   &  6.1    &  3.2   &  54.4      \\
Overall\tnote{d} &       3.2    &  -0.3   &  -1.1   &  51.2   &       2.9    &  2.4    &  6.8  &  52.8   &       2.6    &  4.1    &  10.5  &  53.4      \\
\bottomrule
\end{tabular}                        
\begin{tablenotes} \tiny
\item \hspace*{-0.5em} \textbf{Note.} Predictions are based on estimates of equations 
(\ref{eq:probability_boy}) and (\ref{eq:probability_girl}).
The samples consist of all women from the data with 8 or more years of education who
married during the indicated period using the estimation results for that periods 
and the women's characteristics at the start of their marriage. 
All numbers presented are for women predicted to have one or more births.
Birth/progression rates are based on the ratio of women predicted to have a birth 
by the end of each spell.
Abortion rates within each spell are based on the predicted number of abortions each
woman would have if she went through the entire spell.
To generate conditions such as sex composition and fecundity for second and higher spells 
each woman is randomly assigned outcomes based on the estimation results and
her characteristics as follows.
For each quarter, her probabilities of having a boy, a girl, or no birth is calculated.
She is assigned one of these outcomes based on a random draw from a uniform distribution.
The first quarter where she has a draw that results in a birth she leaves the spell
and the quarter reached is used to calculate her starting age for the subsequent spell.
If she has no draw that result in a birth within a spell her childbearing stops and
she does not enter the next spell.
Results are averaged over 1,000 repetitions.
\item[a] For individual spells the column shows the percent of women who started the
spell and ended up with a birth by the end of the spell period covered.
For the ``Overall'' row, the number is the expected number of children born per woman over 
the first four spells, conditional on having at least one birth.
\item[b] For individual spells the column shows the number of abortions per 100
births predicted to take place by the end of the spell.
For the ``Overall'' row, the number is the total number of abortion per 100 births 
predicted to take place over all four spells.
\item[c]  For individual spells the column shows the number of abortions 
predicted to take place by the end of the spell per 100 women who started the spell .
For the ``Overall'' row, the number is the total number of abortion 
predicted to take place over all four spells per 100 women who had at least one child.
\item[d] Covers first through fourth spells. 
\end{tablenotes}
\end{threeparttable}
\end{scriptsize}
\end{center}
\end{table}


The results show a substantial reduction in fertility over time.
Predicted completed fertility went from 2.7 for urban women married in  
the 1972--1984 period to just 2 for urban women married in the 1995--2006 period.
Hence, urban fertility is now below replacement, especially considering that
these predictions are conditional on having at least one child.
Rural women showed a similar decline in completed fertility, going from 3.2 to 2.6.
Corresponding to the decline in overall fertility, there were also substantial
reductions in the individual parity progression rates.
For example, urban women with 3 children went from having a 57\% chance of having a 
4th child for the 1972--1984 cohort to a 33\% chance for the 1995--2006 cohort---and 
they had a substantially lower probability of even making it to 3 children
in the first place.

Associated with the decline in completed fertility is an increase in sex selection.
For urban women the predicted number of abortions per 100 women by the end of
childbearing have gone up by over 17 percent between the last two cohorts (8.0 to 9.4) 
and the number of abortions per 100 births have gone up by over 35 percent (3.4 to 4.6).
For rural women, abortions per 100 women increased almost 55 percent (6.8 to 10.5), 
while abortions per 100 births increased over 70 percent (2.4 to 4.1).%
\footnote{
Unfortunately there are no solid official estimates to compare these numbers to.
The total number of abortions per year, not just those based on son preference, vary 
between the official count of 600,000--700,000 
and an estimate of 6.4 million from the 2002 Abortion Assessment Project-India.
The 6.4 million abortions per year may even be an underestimate \citep{Stillman2014}.
NFHS does ask about abortions in all three surveys, although the questions change 
substantially between surveys.
However, the PNDT act prohibits using prenatal diagnostic tests to determine a fetus' sex
and revealing a fetus' sex to parents.
Hence, respondents have a strong incentive not to disclose abortions motivated by son 
preference. 
In addition, abortion are considered a taboo subject and many women are unwilling or 
uncomfortable reporting their abortions to government sponsored enumerators 
\citep{Rossier2003,Stillman2014}.
}
The combined effects of the declines in fertility and increased use of sex selection
are that the percentage boys increases from 53.1 to 53.7 percent for urban
women, and from 52.8 to 53.4 percent for rural women.

Of particular note is the high number of abortions among rural women for the 
1985--1994 period.
Educated rural women obviously had few constraints on access to 
prenatal sex determination even when the techniques were relatively new.
The higher abortion rate per woman for rural women than urban women likely comes from
a combination of higher likelihood of having an additional child and changes in use of 
sex selection within spell as shown in the graphs above. 

The results from theoretical model showed that it is possible for the use of sex 
selection to fall for an individual parity, even as overall use of sex selection 
increases. 
This is the case here, where the use of sex selection fell for the third spell 
from 9.8 per 100 births to 7.8 for urban women between the last two periods.
As already discussed, total use of sex selection went up between the last two
periods, so the reduction for the third spell is not an indication of reduced 
use of sex selection overall, but simply the result of fewer women making it to a 
third birth and those who did were more likely to already have a son---for 
the second spell the number of abortions per 100 births for urban women 
went from 4.7 to 8.0, a 70 percent increase.%
\footnote{
The amount of sex selection in the fourth spell reinforces this point:
abortions per 100 births went up almost tenfold.
Only 26 percent of urban women with at least one child make it to the beginning 
of the fourth spell in the last period, and less than 10 of all women have a fourth birth, 
but for those who do make it to the fourth spell the pressure to ensure a son is large.
}
This underscores why focusing exclusively on individual parities can be misleading.

The results for rural women illustrate the two different strategies for achieving one
son: either have many children or have few children and use sex selection.
For the 1986--1994 period there is no use of sex selection on fourth birth, which fits
with 62 percent of women end up with three or more children and a full 62 percent of women with 
three children have a fourth birth (recall that there is a 94 percent chance of having at 
least one son if you have four children).
The use of sex selection on second and third spells indicates that one group of rural
women follows the high fertility route, whereas the other group follows the pattern of
urban women and combine restricting fertility with use of sex selection.

Another way to see the spread of access to sex selection over time is to compare
urban women in the first period with rural women in the last period.
At 2.7 fertility is already relatively low for urban women in the 1972--1984 period and 
there is some evidence of sex selection for these women.
As discussed in Section \ref{sec:spell_def}, the higher than natural sex ratio for
urban well-educated women this early is likely a combination of early access to prenatal
sex determination and that some of the spells end in the 1985--1994 period when there 
was more widespread access.
Rural women in the 1995--2006 period had a fertility only slightly lower than that
of urban women in the 1972--1984 periode, but the number of abortions both per birth and 
per woman were twice as high.


For comparison Appendix Tables \ref{tab:predicted_low} and \ref{tab:predicted_med}
show predicted fertility, number of abortions, and sex ratio for the two other
education groups.
As expected, fertility for women with no education is substantially higher
and the decline over time much smaller than for the high education group.
Predicted fertility fell from 3.6 to 3.3 for urban women and from 3.7 to 3.5 for rural 
women between the 1972--1984 and the 1995--2006 periods.%
\footnote{
A caveat is that my estimates did not cover parities higher than 4, which could
bias downward the change in predicted fertility.
}
Consistent with the discussion above, there is evidence of sex selection for neither
urban nor rural women, with the sex ratio for all predicted births in the last two periods 
close to normal.%
\footnote{
The higher than normal sex ratios for the 1972--1984 period are most likely
remnants of recall error not removed by the sample restriction. 
}
Random variation makes it appears that some individual parities had a higher
than normal sex ratio, which, if studied in isolation, could be taken as evidence
of use of sex selection, and reinforces why supplementing analysis of individual 
parities with this analysis of the overall pattern is important.

Fertility for women with between 1 and 7 years of education was slightly lower than the 
no education group in the 1972--1984 period, but the decline over time was larger.
For the 1995--2006 period predicted completed fertility was 2.8 for urban women and
3.2 for rural women.
There is only limited evidence of sex selection for urban women in this education group.
The exception is for rural women in the last period where the overall sex ratio is
52.1\% boys, although given the high fertility for this group and the small sample
size for this education group it is unclear whether this simply reflects random variation.


The predictions discussed here provide further support for the idea that there are 
two alternative strategies to ensure the birth of a son, either have many children 
or use sex-selective abortions.
The best educated group mostly follow the low fertility, high use of sex selection
strategy, whereas the two lower education groups still have relatively high fertility
and little to no use of sex selection.
It is, however, clear that fertility is declining over time for all groups.
With lower fertility in both urban and rural areas, we are likely to see 
further increases in the use of sex-selective abortions among those who already
use sex selection, and among the groups who have not previously use sex selection
we will see beginning use of sex-selective abortions.



\section{Conclusion\label{sec:conclusion}}


If women have preferences over both the number of children they have \emph{and} the sex 
composition of these children, they face a trade-off between the cost---both monetary 
and psychological---of sex selection and the cost of children.
This paper introduces a novel approach to understanding the relationship between
fertility and use of sex selection, and how sex selection interact with birth spacing.
The proposed method allows for joint estimation of fertility and sex-selective abortions 
using a non-proportional, competing risk hazard model.

Three results stand out.
First, lower fertility is an important factor in the decision to use sex-selective abortions.
% Women wanting more than two births did not use sex selection on the second birth, while
% those who want two or fewer use sex selection intensively.
In both urban and rural areas, only women with the lowest predicted fertility---those
with 8 or more years of education---use sex selection, and as fertility falls they use it 
for earlier parities.
Women with less education instead follow a high fertility strategy to ensure they have
a son and do not appear to use sex selection.
Previous research was unable to explain why sex selection in India only occurred 
among higher education women because it failed to tie the use of sex selection and the 
fertility decision together.

Second, the legal steps taken to combat sex-selective abortion have not been able
to reverse the practice.
The use of sex selection has been increasing over time and is higher now than before the
PNDT Act was passed.
The predicted number of sex-selective abortions per 100 women during their childbearing 
years is now around 10 for women with 8 or more years of education in both urban and
rural areas.
This is especially interesting given that the cost of prenatal sex determination 
has likely increased in response to the PNDT Act.
This also emphasizes why accounting for the possibility that parents may change
their decision to use sex selection within a spell is important. 
As the theory shows an increase in the cost of prenatal sex determination 
increases the likelihood of parents changing their decision to use sex selection.%
\footnote{
A feature of the new approach is that it allows for changes in the decision 
to use sex selection within a spell.
This is most clearly seen for better-educated urban women whose only child is a girl, 
for whom the use of sex selection declines as the second spell becomes longer.
}


Third,  there is little evidence that women with one or more sons use sex selection,
and their probability of another birth declines substantially once they have a son.
The main exception is for rural women with one son and one daughter, presumably as an 
insurance against mortality.
Hence, parents appear to have a preference for one son rather than multiple sons.

These results imply that the way we have been measuring son preference may not be useful
for understanding how people make decisions on use of sex selection.
Recent research suggests that son preference in India, when measured as ideally having 
more boys than girls, is decreasing over time and with higher education \citep{bhat03,pande07}.%
\footnote{
This measure of son preference is commonly used in the literature. 
See, for example, \citet{clark00}, \citet{Jensen2009}, and \cite{Hu2015}.
In NFHS, and Demographic and Health Surveys in general, son preference is measured as part 
of the general fertility questions.
The first question is how many children a woman would have if she could choose exactly how 
many.
This if followed by the sex preference question:
``How many of these children would you like to be boys, how many would you like to be girls 
and for how many would the sex not matter?''
}
Despite this, use of sex selection has increased and for exactly the group 
argued to show declining preference for sons.
The culprit may be that parents want \emph{one} son and will use sex selection 
intensively to get that son, but refrain from using sex selection to have more sons, 
and our standard measure of son preference cannot captured that.
A better approach may be to directly ask if there is a minimum number of each sex that 
respondents want, possibly in combination with ideal number questions.

This is of more than theoretical interest.
If son preference is expressed as a very strong desire for having at least one
son then reductions in fertility will lead to a further increase in the use of sex 
selection across all groups that hold that preference.
What is more, as shown in the theory section, it is possible to get increasingly
unequal sex ratio when fertility falls even if parents do not want more sons
than daughters as long as they only use abortion on female fetuses.
A deeper understanding of exactly what type of son preference is responsible 
for the increase use of sex selection is clearly an important question for
future data collection and research.

An often-repeated, but poorly substantiated, argument for why there is sex selection 
in India is the dowry system.
This explanation fits poorly with observed behavior.
If true, dowries increase the cost of having girls relative to boys because you have to 
pay to ensure your girls are married.
This implies that poorer households should be the most likely to use sex selection, and 
that the use of sex selection should increase the more girls they have.%
\footnote{
The exception would be if dowries increase more than proportionally with education level 
and income, but there does not appear to be evidence of this.
} 
Similarly, for all households there would be a financial incentive to achieve a higher 
number of sons than daughters.
As shown here, however, sex selection is predominantly used by better educated and 
therefore, on average, wealthier households and only to secure one son rather than multiple 
sons.
Finally, fertility's rapid decline after the introduction of prenatal sex 
determination, combined with low cost of sex selection, suggests that parents'
perceived costs of boys and girls are similar.%
\footnote{
See the discussion in Section \ref{sec:changes} for more detail.
}
Hence, banning dowries, as India has done, is unlikely to affect the use of sex 
selection. 

The results presented here lead to a number of important questions for future research.
One is whether marriage market sex ratios affect the use of sex selection.
Parents may care not only about the number and type of children they have but 
also whether they will have grandchildren.
If they care about grandchildren, then a boy could still be preferred over a girl, but a 
married girl would be preferred over an unmarried boy \citep{Bhaskar2011}.
The implication is that parents respond to changes in the expected sex ratio of their
children's marriage market, even given their preference for a son.
This hypothesis can be tested using measures of the observed distributions of
boys and girls and the method proposed here.

A second question for future research, which is especially important when designing 
policies aimed at reducing the use of sex selection, is who the agents are in the decisions 
to use sex selection.
Here, I have treated the household as a single unit, but it is possible, and indeed likely, 
that husband and wife have different preferences over both the factors that determine  
use of sex selection and sex selection itself.
Some of the popular press have assumed that the decision to use sex selection rests 
with the husband.
This, however, does not fit well with the general pattern that women, on average, prefer to 
have fewer children than men do and that declining fertility appear to be the main
driver of sex selection.
A better understanding of who holds what preferences in the family would be a good 
first step to disentangling who makes decisions about sex selection.

A third question is whether there are interactions between sex selection and child outcomes,
such a survival and education.
It has been argued that access to sex selection may benefit girls because those girls
who are born are more likely to be wanted by their parents.
For mortality it is, however, possible that improvements are the result of a purely 
mechanical effect.
As I have shown, increased use of sex selection lengthens the birth spacing after a girl 
and that by itself can improve her survival chances, even if she is no more wanted than before.
The method suggested here can predict the likelihood of a woman using prenatal sex 
determination for a given birth based on her characteristics.
This predicted probability can then be used in estimations of the determinants
of child health to directly test whether increasing use of sex selection is 
beneficial for the girls who are born.
It is unlikely, however, that a substantial mortality effect exists since sex 
selection is mainly used by well-educated women, who tend to have lower mortality, 
but there might be effects on education investments.

% [developments in the future]
In conclusion, the results provide strong clues to how sex selection use will change in the future.
Because lower fertility is responsible for the increase in sex-selective abortions, 
it is likely that we will see further increases in the practice as more families 
want fewer children, either because of urbanization or because of increases in female 
education.
With already low fertility for better-educated urban women, a substantial future increase 
in sex-selective abortions per woman is unlikely, 
but a higher proportion of women will belong to this group in the future.
For rural, better-educated women fertility is still falling.
To the extent that it falls to the same level as for urban
women, I expect a corresponding increase in sex selection.
Finally, and most importantly, we are beginning to see evidence of lower fertility 
for women with lower levels of education.
If women with less education hold the same preference for one son as the better-educated,
once the less-educated women's fertility begins to drop to three---and even two---we are 
likely to see a substantially increased use of sex selection in India.






\newpage
\onehalfspacing
\bibliographystyle{aer}
\bibliography{collection}

\addcontentsline{toc}{section}{References}



\clearpage
\newpage

\appendix
\section{Appendix}

% CHANGING NUMBERING OF FIGURES AND TABLES FOR APPENDIX
\renewcommand\thefigure{\thesection.\arabic{figure}}    
\setcounter{figure}{0}
\renewcommand\thetable{\thesection.\arabic{table}}    
\setcounter{table}{0}
  
% Descriptive statistics tables
\begin{table}[htp]
\begin{center}
\begin{scriptsize}
\begin{threeparttable}
% \caption{Descriptive Statistics by Education Level and Beginning of Spell for First and Second Spell}
\caption{Descriptive Statistics by Education Level and Beginning of Spell}
\label{tab:des_stat1}
\begin{tabular} {@{} c l D{.}{.}{1.3} D{.}{.}{1.3} D{.}{.}{1.3} D{.}{.}{1.3} D{.}{.}{1.3} D{.}{.}{1.3} D{.}{.}{1.3} D{.}{.}{1.3} D{.}{.}{1.3} @{}} \toprule
                    &                     & \multicolumn{3}{c}{No Education}                                & \multicolumn{3}{c}{1--7 Years of Education}                      & \multicolumn{3}{c}{8+ Years of Education} \\ \cmidrule(lr){3-5} \cmidrule(lr){6-8} \cmidrule(lr){9-11}
                    &                     & \mco{1972--1984}     & \mco{1985--1994}     & \mco{1995--2006}     & \mco{1972--1984}     & \mco{1985--1994}     & \mco{1995--2006}     & \mco{1972--1984}     & \mco{1985--1994}     & \mco{1995--2006}     \\ % \cmidrule(lr){3-5} \cmidrule(lr){6-8} \cmidrule(lr){9-11}
\midrule
\multirow{16}{*}{\rotatebox{90}{First Spell}}                   
                    &Boy born            &       0.467         &       0.438         &       0.363         &       0.476         &       0.447         &       0.368         &       0.483         &       0.453         &       0.377         \\
                    &                    &     (0.499)         &     (0.496)         &     (0.481)         &     (0.499)         &     (0.497)         &     (0.482)         &     (0.500)         &     (0.498)         &     (0.485)         \\
                    &Girl born           &       0.427         &       0.410         &       0.352         &       0.439         &       0.429         &       0.356         &       0.454         &       0.420         &       0.353         \\
                    &                    &     (0.495)         &     (0.492)         &     (0.478)         &     (0.496)         &     (0.495)         &     (0.479)         &     (0.498)         &     (0.494)         &     (0.478)         \\
                    &Censored            &       0.106         &       0.152         &       0.284         &       0.085         &       0.124         &       0.276         &       0.063         &       0.127         &       0.270         \\
                    &                    &     (0.308)         &     (0.359)         &     (0.451)         &     (0.279)         &     (0.330)         &     (0.447)         &     (0.243)         &     (0.333)         &     (0.444)         \\
                    &Urban               &       0.181         &       0.176         &       0.200         &       0.366         &       0.334         &       0.322         &       0.690         &       0.604         &       0.561         \\
                    &                    &     (0.385)         &     (0.381)         &     (0.400)         &     (0.482)         &     (0.472)         &     (0.467)         &     (0.463)         &     (0.489)         &     (0.496)         \\
                    &Age                 &      15.854         &      16.301         &      16.871         &      16.983         &      17.470         &      17.844         &      19.516         &      20.036         &      20.766         \\
                    &                    &     (2.482)         &     (2.643)         &     (2.858)         &     (2.733)         &     (3.041)         &     (3.221)         &     (3.331)         &     (3.614)         &     (3.916)         \\
                    &Owns land           &       0.589         &       0.578         &       0.540         &       0.500         &       0.497         &       0.482         &       0.321         &       0.387         &       0.414         \\
                    &                    &     (0.492)         &     (0.494)         &     (0.498)         &     (0.500)         &     (0.500)         &     (0.500)         &     (0.467)         &     (0.487)         &     (0.493)         \\
                    &Sched.\ caste/tribe &       0.357         &       0.395         &       0.441         &       0.170         &       0.227         &       0.318         &       0.071         &       0.119         &       0.175         \\
                    &                    &     (0.479)         &     (0.489)         &     (0.497)         &     (0.376)         &     (0.419)         &     (0.466)         &     (0.257)         &     (0.324)         &     (0.380)         \\
                    &Number of quarters  & \mco{  330,315         } & \mco{  257,757         } & \mco{   75,841         } & \mco{  108,463         } & \mco{  108,863         } & \mco{   54,704         } & \mco{  102,238         } & \mco{  154,582         } & \mco{  116,272         } \\
                    &Number of women     & \mco{   30,004         } & \mco{   27,669         } & \mco{    9,684         } & \mco{   11,162         } & \mco{   13,104         } & \mco{    7,510         } & \mco{   11,271         } & \mco{   19,072         } & \mco{   16,620         } \\
                    \addlinespace
% \cmidrule(lr){1-2} \cmidrule(lr){3-5} \cmidrule(lr){6-8} \cmidrule(lr){9-11}
\midrule
\multirow{22}{*}{\rotatebox{90}{Second Spell}}
                    & Boy born            &       0.490         &       0.444         &       0.359         &       0.477         &       0.437         &       0.338         &       0.457         &       0.399         &       0.285         \\
                    &                     &     (0.500)         &     (0.497)         &     (0.480)         &     (0.500)         &     (0.496)         &     (0.473)         &     (0.498)         &     (0.490)         &     (0.451)         \\
                    & Girl born           &       0.451         &       0.413         &       0.327         &       0.462         &       0.413         &       0.319         &       0.426         &       0.351         &       0.239         \\
                    &                     &     (0.498)         &     (0.492)         &     (0.469)         &     (0.499)         &     (0.492)         &     (0.466)         &     (0.495)         &     (0.477)         &     (0.427)         \\
                    & Censored            &       0.059         &       0.143         &       0.314         &       0.061         &       0.150         &       0.344         &       0.117         &       0.251         &       0.476         \\
                    &                     &     (0.236)         &     (0.350)         &     (0.464)         &     (0.239)         &     (0.357)         &     (0.475)         &     (0.321)         &     (0.433)         &     (0.499)         \\
                    & One boy             &       0.527         &       0.515         &       0.508         &       0.519         &       0.513         &       0.505         &       0.513         &       0.520         &       0.515         \\
                    &                     &     (0.499)         &     (0.500)         &     (0.500)         &     (0.500)         &     (0.500)         &     (0.500)         &     (0.500)         &     (0.500)         &     (0.500)         \\
                    & One girl            &       0.473         &       0.485         &       0.492         &       0.481         &       0.487         &       0.495         &       0.487         &       0.480         &       0.485         \\
                    &                     &     (0.499)         &     (0.500)         &     (0.500)         &     (0.500)         &     (0.500)         &     (0.500)         &     (0.500)         &     (0.500)         &     (0.500)         \\
                    & Urban               &       0.181         &       0.180         &       0.212         &       0.370         &       0.348         &       0.345         &       0.697         &       0.628         &       0.578         \\
                    &                     &     (0.385)         &     (0.384)         &     (0.409)         &     (0.483)         &     (0.476)         &     (0.475)         &     (0.459)         &     (0.483)         &     (0.494)         \\
                    & Age                 &      17.774         &      18.357         &      18.870         &      18.620         &      19.200         &      19.581         &      20.988         &      21.589         &      22.252         \\
                    &                     &     (2.743)         &     (3.054)         &     (3.284)         &     (2.864)         &     (3.206)         &     (3.373)         &     (3.371)         &     (3.621)         &     (3.901)         \\
                    & First spell length  &      25.630         &      29.145         &      29.133         &      21.924         &      24.513         &      24.787         &      20.608         &      22.154         &      22.217         \\
                    &                     &    (20.345)         &    (25.044)         &    (27.216)         &    (18.142)         &    (22.027)         &    (23.204)         &    (15.371)         &    (18.651)         &    (19.176)         \\
                    & Owns land           &       0.590         &       0.572         &       0.528         &       0.497         &       0.492         &       0.468         &       0.313         &       0.369         &       0.402         \\
                    &                     &     (0.492)         &     (0.495)         &     (0.499)         &     (0.500)         &     (0.500)         &     (0.499)         &     (0.464)         &     (0.483)         &     (0.490)         \\
                    & Sched.\ caste/tribe &       0.353         &       0.390         &       0.430         &       0.164         &       0.218         &       0.304         &       0.068         &       0.110         &       0.169         \\
                    &                     &     (0.478)         &     (0.488)         &     (0.495)         &     (0.371)         &     (0.413)         &     (0.460)         &     (0.251)         &     (0.313)         &     (0.374)         \\
                    & Number of quarters  & \mco{  176,227         } & \mco{  231,546         } & \mco{   68,455         } & \mco{   67,432         } & \mco{  103,656         } & \mco{   49,236         } & \mco{   77,416         } & \mco{  164,752         } & \mco{  119,864         } \\
                    & Number of women     & \mco{   21,171         } & \mco{   28,251         } & \mco{    9,428         } & \mco{    8,209         } & \mco{   12,449         } & \mco{    6,542         } & \mco{    8,204         } & \mco{   16,601         } & \mco{   13,610         } \\
\bottomrule
\end{tabular}
\begin{tablenotes} \tiny
% \item \hspace*{-7pt} \textbf{Note.}
\item \hspace*{-0.7em} \textbf{Note.}
Means without parentheses and standard deviation in parentheses.
Interactions between variables, baseline hazard dummies and squares not shown.
Quarters refer to number of 3 month periods observed.
\end{tablenotes}
\end{threeparttable}
\end{scriptsize}
\end{center}
\end{table}
% \end{sidewaystable}

\addtocounter{table}{-1}

% \begin{sidewaystable}
\begin{table}
\begin{center}
\begin{scriptsize}
\begin{threeparttable}
% \caption{Descriptive Statistics by Education Level and Beginning of Spell for Third and Fourth Spell}
\caption{(Continued) Descriptive Statistics by Education Level and Beginning of Spell}
% \label{tab:des_stat2}
\begin{tabular} {@{} c l D{.}{.}{1.3} D{.}{.}{1.3} D{.}{.}{1.3} D{.}{.}{1.3} D{.}{.}{1.3} D{.}{.}{1.3} D{.}{.}{1.3} D{.}{.}{1.3} D{.}{.}{1.3} @{}} \toprule
                    &                     & \multicolumn{3}{c}{No Education}                                & \multicolumn{3}{c}{1-7 Years of Education}                      & \multicolumn{3}{c}{8+ Years of Education} \\ \cmidrule(lr){3-5} \cmidrule(lr){6-8} \cmidrule(lr){9-11}
                    &                     & \mco{1972--1984}     & \mco{1985--1994}     & \mco{1995--2006}     & \mco{1972--1984}     & \mco{1985--1994}     & \mco{1995--2006}     & \mco{1972--1984}     & \mco{1985--1994}     & \mco{1995--2006}     \\ % \cmidrule(lr){3-5} \cmidrule(lr){6-8} \cmidrule(lr){9-11}
\midrule                    
\multirow{24}{*}{\rotatebox{90}{Third Spell}}
                    & Boy born            &       0.478         &       0.425         &       0.319         &       0.455         &       0.388         &       0.261         &       0.353         &       0.264         &       0.163         \\
                    &                     &     (0.500)         &     (0.494)         &     (0.466)         &     (0.498)         &     (0.487)         &     (0.439)         &     (0.478)         &     (0.441)         &     (0.369)         \\
                    & Girl born           &       0.442         &       0.393         &       0.306         &       0.423         &       0.357         &       0.239         &       0.313         &       0.214         &       0.128         \\
                    &                     &     (0.497)         &     (0.488)         &     (0.461)         &     (0.494)         &     (0.479)         &     (0.426)         &     (0.464)         &     (0.410)         &     (0.334)         \\
                    & Censored            &       0.080         &       0.182         &       0.376         &       0.121         &       0.255         &       0.500         &       0.334         &       0.521         &       0.709         \\
                    &                     &     (0.271)         &     (0.386)         &     (0.484)         &     (0.326)         &     (0.436)         &     (0.500)         &     (0.472)         &     (0.500)         &     (0.454)         \\
                    & Two boys            &       0.276         &       0.257         &       0.243         &       0.248         &       0.245         &       0.232         &       0.266         &       0.244         &       0.238         \\
                    &                     &     (0.447)         &     (0.437)         &     (0.429)         &     (0.432)         &     (0.430)         &     (0.422)         &     (0.442)         &     (0.429)         &     (0.426)         \\
                    & One boy, one girl   &       0.490         &       0.502         &       0.506         &       0.508         &       0.501         &       0.515         &       0.489         &       0.510         &       0.527         \\
                    &                     &     (0.500)         &     (0.500)         &     (0.500)         &     (0.500)         &     (0.500)         &     (0.500)         &     (0.500)         &     (0.500)         &     (0.499)         \\
                    & Two girls           &       0.234         &       0.241         &       0.252         &       0.243         &       0.254         &       0.253         &       0.245         &       0.247         &       0.235         \\
                    &                     &     (0.424)         &     (0.428)         &     (0.434)         &     (0.429)         &     (0.435)         &     (0.435)         &     (0.430)         &     (0.431)         &     (0.424)         \\
                    & Urban               &       0.179         &       0.182         &       0.208         &       0.378         &       0.354         &       0.343         &       0.693         &       0.646         &       0.568         \\
                    &                     &     (0.383)         &     (0.386)         &     (0.406)         &     (0.485)         &     (0.478)         &     (0.475)         &     (0.461)         &     (0.478)         &     (0.495)         \\
                    & Age                 &      19.978         &      20.693         &      21.278         &      20.794         &      21.453         &      21.998         &      22.963         &      23.825         &      24.548         \\
                    &                     &     (2.895)         &     (3.196)         &     (3.551)         &     (2.937)         &     (3.239)         &     (3.618)         &     (3.405)         &     (3.748)         &     (4.067)         \\
                    & First spell length  &      24.547         &      27.575         &      27.099         &      20.971         &      23.901         &      23.532         &      19.991         &      21.246         &      21.236         \\
                    &                     &    (18.832)         &    (22.779)         &    (24.364)         &    (16.875)         &    (20.378)         &    (21.155)         &    (14.646)         &    (16.835)         &    (17.363)         \\
                    & Owns land           &       0.602         &       0.575         &       0.538         &       0.503         &       0.503         &       0.490         &       0.320         &       0.369         &       0.421         \\
                    &                     &     (0.490)         &     (0.494)         &     (0.499)         &     (0.500)         &     (0.500)         &     (0.500)         &     (0.467)         &     (0.483)         &     (0.494)         \\
                    & Sched.\ caste/tribe &       0.344         &       0.392         &       0.432         &       0.159         &       0.218         &       0.298         &       0.067         &       0.107         &       0.169         \\
                    &                     &     (0.475)         &     (0.488)         &     (0.495)         &     (0.366)         &     (0.413)         &     (0.457)         &     (0.251)         &     (0.310)         &     (0.375)         \\
                    & Number of quarters  & \mco{  109,026         } & \mco{  206,988         } & \mco{   70,391         } & \mco{   43,434         } & \mco{   87,537         } & \mco{   41,400         } & \mco{   49,799         } & \mco{  126,733         } & \mco{   79,068         } \\
                    & Number of women     & \mco{   13,196         } & \mco{   25,135         } & \mco{    9,355         } & \mco{    4,960         } & \mco{    9,775         } & \mco{    5,034         } & \mco{    4,316         } & \mco{   10,688         } & \mco{    7,927         } \\
\addlinespace
% \cmidrule(lr){1-2} \cmidrule(lr){3-5} \cmidrule(lr){6-8} \cmidrule(lr){9-11}
\midrule
\multirow{26}{*}{\rotatebox{90}{Fourth Spell}}
                    & Boy born            &       0.463         &       0.384         &       0.291         &       0.403         &       0.347         &       0.233         &       0.310         &       0.227         &       0.170         \\
                    &                     &     (0.499)         &     (0.486)         &     (0.454)         &     (0.491)         &     (0.476)         &     (0.423)         &     (0.462)         &     (0.419)         &     (0.376)         \\
                    & Girl born           &       0.408         &       0.363         &       0.268         &       0.391         &       0.303         &       0.198         &       0.298         &       0.197         &       0.113         \\
                    &                     &     (0.491)         &     (0.481)         &     (0.443)         &     (0.488)         &     (0.460)         &     (0.399)         &     (0.458)         &     (0.397)         &     (0.316)         \\
                    & Censored            &       0.130         &       0.254         &       0.441         &       0.206         &       0.350         &       0.569         &       0.392         &       0.576         &       0.717         \\
                    &                     &     (0.336)         &     (0.435)         &     (0.497)         &     (0.405)         &     (0.477)         &     (0.495)         &     (0.488)         &     (0.494)         &     (0.450)         \\
                    & Three boys          &       0.134         &       0.124         &       0.111         &       0.107         &       0.105         &       0.102         &       0.106         &       0.101         &       0.082         \\
                    &                     &     (0.341)         &     (0.329)         &     (0.315)         &     (0.309)         &     (0.307)         &     (0.303)         &     (0.308)         &     (0.302)         &     (0.275)         \\
                    & Two boys, one girl  &       0.374         &       0.357         &       0.346         &       0.348         &       0.323         &       0.308         &       0.358         &       0.304         &       0.289         \\
                    &                     &     (0.484)         &     (0.479)         &     (0.476)         &     (0.476)         &     (0.468)         &     (0.462)         &     (0.480)         &     (0.460)         &     (0.453)         \\
                    & One boys, two girls &       0.362         &       0.389         &       0.400         &       0.400         &       0.413         &       0.423         &       0.393         &       0.439         &       0.469         \\
                    &                     &     (0.481)         &     (0.488)         &     (0.490)         &     (0.490)         &     (0.492)         &     (0.494)         &     (0.489)         &     (0.496)         &     (0.499)         \\
                    & Three girls         &       0.129         &       0.130         &       0.142         &       0.145         &       0.159         &       0.167         &       0.143         &       0.156         &       0.159         \\
                    &                     &     (0.335)         &     (0.337)         &     (0.350)         &     (0.353)         &     (0.365)         &     (0.373)         &     (0.350)         &     (0.363)         &     (0.366)         \\
                    & Urban               &       0.171         &       0.181         &       0.197         &       0.363         &       0.346         &       0.325         &       0.670         &       0.601         &       0.511         \\
                    &                     &     (0.377)         &     (0.385)         &     (0.398)         &     (0.481)         &     (0.476)         &     (0.469)         &     (0.470)         &     (0.490)         &     (0.500)         \\
                    & Age                 &      21.926         &      22.885         &      23.608         &      22.611         &      23.519         &      24.317         &      24.257         &      25.472         &      26.061         \\
                    &                     &     (3.019)         &     (3.332)         &     (3.711)         &     (2.894)         &     (3.364)         &     (3.759)         &     (3.171)         &     (3.702)         &     (4.143)         \\
                    & First spell length  &      23.281         &      26.609         &      26.460         &      20.495         &      23.069         &      24.215         &      19.106         &      20.826         &      21.423         \\
                    &                     &    (17.481)         &    (21.556)         &    (23.272)         &    (15.546)         &    (19.078)         &    (21.244)         &    (13.991)         &    (16.277)         &    (16.585)         \\
                    & Owns land           &       0.610         &       0.587         &       0.557         &       0.523         &       0.530         &       0.509         &       0.339         &       0.414         &       0.466         \\
                    &                     &     (0.488)         &     (0.492)         &     (0.497)         &     (0.500)         &     (0.499)         &     (0.500)         &     (0.474)         &     (0.493)         &     (0.499)         \\
                    & Sched.\ caste/tribe &       0.338         &       0.403         &       0.442         &       0.154         &       0.224         &       0.297         &       0.085         &       0.113         &       0.193         \\
                    &                     &     (0.473)         &     (0.491)         &     (0.497)         &     (0.361)         &     (0.417)         &     (0.457)         &     (0.280)         &     (0.316)         &     (0.395)         \\
                    & Number of quarters  & \mco{   54,446         } & \mco{  151,285         } & \mco{   56,756         } & \mco{   18,852         } & \mco{   49,108         } & \mco{   23,209         } & \mco{   14,287         } & \mco{   43,339         } & \mco{   23,586         } \\
                    & Number of women     & \mco{    6,832         } & \mco{   18,436         } & \mco{    7,637         } & \mco{    2,201         } & \mco{    5,564         } & \mco{    2,768         } & \mco{    1,347         } & \mco{    3,933         } & \mco{    2,459         } \\
\bottomrule
\end{tabular}
\begin{tablenotes} \tiny
% \item \hspace*{-7pt} \textbf{Note.}
\item \hspace*{-0.7em} \textbf{Note.}
Means without parentheses and standard deviation in parentheses.
Interactions between variables, baseline hazard dummies and squares not shown.
Quarters refer to number of 3 month periods observed.
\end{tablenotes}
\end{threeparttable}
\end{scriptsize}
\end{center}
\end{table}
% \end{sidewaystable}




\end{document}



