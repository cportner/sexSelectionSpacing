% Revamped version for Demography focusing on method

% sex-selective Abortion
% Begun originally: 01/11/02
% Begun.: 2017-04-03
% Edited: 2017-04-03

\documentclass[12pt,letterpaper]{article}

\usepackage{fontspec}
\setromanfont[Ligatures=TeX]{TeX Gyre Pagella}
\usepackage{unicode-math}
\setmathfont{TeX Gyre Pagella Math}
% \usepackage{mathpazo}
\usepackage[title]{appendix}
\usepackage[margin=1.0in]{geometry}
\usepackage[figuresleft]{rotating}
\usepackage[longnamesfirst]{natbib}
\usepackage{dcolumn}
\usepackage{booktabs}
\usepackage{multirow}
\usepackage[flushleft]{threeparttable}
\usepackage{setspace}
% \usepackage{xyling}
\usepackage[justification=centering]{caption}
\usepackage[font=scriptsize]{subfig}
\usepackage[xetex,colorlinks=true,linkcolor=black,citecolor=black,urlcolor=black]{hyperref}
\usepackage{adjustbox}
\usepackage{xfrac}


% \bibpunct{(}{)}{;}{a}{}{,}
\newcommand{\mco}[1]{\multicolumn{1}{c}{#1}}
\newcommand{\mct}[1]{\multicolumn{2}{c}{#1}}
\newcommand{\X}{$\times$ }
\newcommand{\hs}{\hspace{15pt}}

% Attempt to squeeze more floats in
\renewcommand\floatpagefraction{.9}
\renewcommand\topfraction{.9}
\renewcommand\bottomfraction{.9}
\renewcommand\textfraction{.1}
\setcounter{totalnumber}{50}
\setcounter{topnumber}{50}
\setcounter{bottomnumber}{50}


%------------------------------------------------------------------------

% Fertility, Birth Spacing and the Determinants of sex-selective Abortions
% Understanding Sex Selection: Fertility, Abortions and Birth Spacing
% The Determinants of sex-selective Abortions: A New Method/Approach
% Sex Selection: A New Method for Estimating Determinants
% Fertility, Birth Spacing and Abortions in India
% Fertility, Birth Spacing and Sex Selection
% Fertility, Birth Spacing and sex-selective Abortions
% Fertility, Spacing and Sex Selection
% Fertility, Sex Selection, and Spacing
% Fertility, Sex Selection, and Birth Spacing
% Fertility, Birth Spacing and Missing Girls
% Where have all the girls gone: An analysis of sex-selective abortions

\title{Son Preference, Birth Spacing, and Sex-Selective Abortions%
\protect\thanks{%
I am grateful to Andrew Foster and Darryl Holman for discussions about the method.
I owe thanks to Shelly Lundberg, Daniel Rees, David Ribar, 
Hendrik Wolff, seminar participants at University of Copenhagen, University of Michigan, 
University of Washington, University of Aarhus, the Fourth 
Annual Conference on Population, Reproductive Health, 
and Economic Development, and the Economic Demography Workshop for helpful suggestions and comments.
I would also like to thank Nalina Varanasi for research assistance.
Support from the University of Washington Royalty Research Fund and the 
Development Research Group of the World Bank is gratefully acknowledged.
The views and findings expressed here are those of the author and
should not be attributed to the World Bank or any of its member countries.
Partial support for this research came from a Eunice Kennedy Shriver National
Institute of Child Health and Human Development research infrastructure grant,
5R24HD042828, to the Center for Studies in Demography and Ecology at the
University of Washington.
% Prior versions of this paper were presented under the title ``The 
% Determinants of sex-selective Abortions.''
}
}

\author{}

\author{Claus C P\"ortner\\
    Department of Economics\\
    Albers School of Business and Economics\\
    Seattle University, P.O. Box 222000\\
    Seattle, WA 98122\\
    \href{mailto:claus@clausportner.com}{\texttt{claus@clausportner.com}}\\
    \href{http://www.clausportner.com}{\texttt{www.clausportner.com}}\\
    \& \\
    Center for Studies in Demography and Ecology \\
    University of Washington\\ \vspace{2cm}
    }

\date{December 2017\\
\bigskip
Preliminary}


%------------------------------------------------------------------------


\begin{document}
\graphicspath{{../figures/}}
\DeclareGraphicsExtensions{.eps,.jpg,.pdf,.mps,.png}

\setcounter{page}{-1}
\maketitle
\thispagestyle{empty}

% \setcounter{page}{0}


\newpage
\thispagestyle{empty}
\doublespacing

\begin{abstract}

% Demography abstract
\noindent 

Strong son preference has frequently been associated with shorter 
spacing between births after the birth of a girl.
The spread of access to prenatal sex determination and 
sex-selective abortions has the potential to reverse this pattern
because each abortion adds substantially to the duration between births.
I introduce a statistical method that simultaneously account for how sex 
selection increases the likelihood of a son \emph{and} the spacing between births.
Using India's National Family and Health Surveys,
I show that use of sex selection leads to \emph{longer} spacing after a daughter 
than after a son; a reversal of the pattern before the availability of sex selection.
For families that do not use sex selection, the duration to next birth is still 
shorter when no boys are present.

Women with 8 or more years of education, both in urban and rural areas, are 
the main users of sex-selective abortions, whereas women with less education 
do not appear to use sex selection.


\noindent JEL: J1, O12, I1
\noindent Keywords: India, prenatal sex determination, censoring, competing risk
\end{abstract}

\newpage


%------------------------------------------------------------------------

\section{Introduction\label{sec:intro}}

Spacing between births has long been used as a measure of son preference 
\citep{Leung1988}.
Before prenatal sex determination became available, the only recourse for 
parents who wanted a son---but did not yet have one---was to have the next 
birth sooner.
Son preference was therefore often associated with shorter spacing after the 
births of girls than boys, which, in turn, was associated with worse 
health outcomes for girls.%
\footnote{
See, for example, 
\citet{Das1987}, \citet{Rahman1993}, \citet{Pong1994}, \citet{Haughton1996},
\citet{Arnold1997}, and \citet{Soest2012} on shorter birth spacing after
girls
and 
\citet{arnold98}, \citet{Whitworth2002}, \citet{Rutstein2005},
and \citet{Conde-Agudelo2006} on the association between shorter birth
spacing and worse health outcomes for girls.
Parents were also more likely to cease childbearing after the birth of 
a son than after a daughter 
\citep{ben-porath76b,Das1987,Arnold1997,clark00,filmer09,Altindag2016}
}

What has not previously been appreciated is that the introduction 
of prenatal sex determination fundamentally changed the relationship 
between son preference and birth spacing.
Couples with stronger son preferences are more likely to use prenatal
sex determination and sex-selective abortions, but each sex-selective 
abortion substantially increases the duration from last birth. 
Each abortion increases spacing between births by approximately a year.
The increase consists of three parts.
First, starting from the time of the abortion, the uterus needs at 
least two menstrual cycles to recover;  otherwise, the likelihood 
of a spontaneous abortion increases substantially \citep{zhou00b}.
The second part is the waiting time to conception, which I set to 6 
months \citep{Wang2003}.
Finally, sex determination tests are generally reliable only from 3 months 
of gestation onwards.
These three parts add up to 12 months.%
\footnote{
The waiting time to conception varies by woman, but even if it is very 
short, say one  month, the additional space between births would still
be 6 months per abortion.
}

As a result, we now have a situation where families with the \emph{strongest} son 
preference may show the \emph{longest} spacing after the birth of a daughter,
because of their use of sex-selective abortions.
But, to further complicate matters, we may still observe short spacing 
after the births of daughters as a representation of son preference for 
families who---for one reason or another---do not use prenatal sex selection.

Spacing, by itself, can therefore no longer be used to measure son preference.
Understanding birth spacing remains, however, an important undertaking.
First, spacing is still important and useful in understanding son preference, 
if combined with the likelihood of observing a boy or a girl.
% Short spacing after a girl is indicative of son preference if observed
% births are close to the natural sex ratio, whereas long spacing is indicative
% of son preference if observed births are male biased.
Second, if spacing affect health outcomes for mother and/or children, it is 
important to understand what drives changes in spacing.
The better health outcomes for girls in the presence of sex selection may,
for example, be an unintended side-effect of the longer spacing that arise
from sex-selective abortions rather than be driven by a higher value of
girls resulting from lower supply \citep{Lin2014,Hu2015}.
Third, spacing may be an important factor in parents' decisions, either
for preference or economic reasons, such as labor force participation.
This is especially important here because even parents with strong
son preference may reverse their decision to use prenatal sex determination, 
and carry the next pregnancy to term whether male or female, as the duration 
from last birth becomes sufficiently sufficiently long.%
\footnote{
More generally, we know less about what determines spacing behavior in 
developing countries than in developed countries.
With increasing numbers of women entering the labor force in developing
countries, understanding how timing decisions are made will be important
for the design of suitable policies \citep{Portner2018}.
}

In this paper, I introduce and apply a novel method that directly incorporates 
the effects of sex-selective abortions on the likelihood of a son being born 
\emph{and} the duration between births.
The method can be used to analyse both situations with and without prenatal
sex selection.
Furthermore, my proposed method allow for the duration since the last 
birth to affect the decision on sex selection.
By examining whether sex selection decisions change with spacing, we can 
draw a more nuanced picture of the degree of son preference.

% [Why India?]
I apply the method to birth histories for Hindu women, using data from 
India's National Family and Health Surveys (NFHS), covering the period 
1972 to 2006. 
India is a particularly interesting case.
On one hand, India has seen dramatic increases in the males-to-females ratio 
at birth over the last three decades as access to prenatal sex determination 
expanded.%
\footnote{
See \citet{das_gupta97}, \citet{Sudha1999},
\citet{Arnold2002}, \cite{retherford03b} and \citet{jha06}.
India is clearly not alone; both China and South Korea saw
significant changes in the sex ratio at birth over the same period 
\citep{Yi1993,park95}.
}
On the other hand, recent research suggests that son preference in 
India, when measured as ideally having more boys than girls, is decreasing 
over time and with higher education \citep{bhat03,pande07}.%
\footnote{
This measure of son preference is commonly used in the literature. 
See, for example, \citet{clark00}, \citet{Jensen2009}, and \cite{Hu2015}.
}
In addition, there has been substantial changes in access and legality of
prenatal sex determination in India over the period covered.
Abortion has been legal in India since 1971, and still is, but the 
first reports of prenatal sex determination did
not appear until 1982--83 \citep{Sudha1999,bhat06,Grover2006}.
In 1994, the Central Government passed the Prenatal Diagnostic Techniques 
(PNDT) Act, making determining and communicating the sex of a fetus illegal.%
\footnote{
The Act is described in detail at \href{http://pndt.gov.in/}{http://pndt.gov.in/}.
The number of convictions has been low.
It took until January 2008 for the first state, Haryana, to reach 5 convictions.
Hence, private clinics apparently operate with little risk of legal action 
\citep{Sudha1999}.
Maharashtra was the first state to pass a similar law in 1988.
}
Hence, the data make it possible to show how spacing between births and
sex ratios have changed with the introduction of prenatal sex determination,
and whether the ban affected the relationship between birth spacing and sex 
ratios.

[Outline of findings here]


% [EXPLAIN RESULTS HERE]
% Main findings!
% There are three main findings.
% 
% There is, however, still no evidence of sex selection being used on the first birth.
% 
% Secondly, sex selection appears to be used for securing one son, rather than a large 
% number of sons.
% There is only limited use of sex-selective abortions for better-educated
% women with one or more sons.
% The exception is rural women with one son and one daughter, presumably to compensate for 
% the higher child mortality in rural areas.
% These results are in line with the differential stopping behavior observed in many studies
% before sex selection became available \citep{repetto72,arnold98,dreze01}.
% 



%------------------------------------------------------------------------------------


\section{Estimation Strategy\label{sec:strategy}}


[TK this needs to move to later]
Finally, increased reliability of access and effectiveness of
contraceptive can lead to shorter spacing between births 
\citep{Keyfitz1971,Heckman1976}.
With less reliable contraception parents choose a higher level 
of contraception---resulting in longer spacing---to avoid having 
too many children by accident.
But, as contraception becomes more effective parents can more
easily avoid future births, allowing them to reduce the spacing 
between births without having to worry about overshooting their 
preferred number of children.
This idea may also help explain shorter spacing for better 
educated women than for less educated women, provided that   
knowledge and ability to use contraception differ
across education groups \citep{Tulasidhar1993,Whitworth2002}.


To improve on the literature's current approaches to understanding 
son preference, a new empirical method must be able to capture the 
three main implications of the availability of sex-selective abortions
combined with son preference:
a higher probability that the next child is a son, 
longer spacing to next birth with use of sex selection, 
and that the use of prenatal sex determination may change between 
one observed birth and the next.%
\footnote{
In addition, it should be able to capture the shorter spacing between
births observed in the absence of use or availability of prenatal 
sex determination.
}
This means that we need an empirical model that can jointly estimate 
the association between covariates and both the sex of children born 
and the spacing from last birth. 
My proposed model is a discrete time, non-proportional competing risk 
hazard model with two exit states: either a boy or a girl is born.%
\footnote{
\cite{Merli2000} used a discrete hazard model to examine whether 
there were under-reporting of births in rural China, although they 
estimated separate waiting time regressions for boys and girls.
}

I divide a woman's reproductive life into spells that each covers 
the period between births (from marriage to first birth for the first spell).
For each woman, $i=1,\ldots,n$, the starting point for a spell is time $t=1$ and 
the spell continues until time $t_i$ when either a birth occurs or the spell 
is censored.%
\footnote{
The time of censoring is assumed independent of the hazard rate,
as is standard in the literature.
}
There are two exit states: birth of a boy, $j=1$, or birth of a girl, $j=2$, and 
$J_i$ is a random variable indicating which event took place.
The discrete time hazard rate $h_{ijt}$ is
\begin{equation}
 h_{ijt} = \Pr (T_i=t, J_i=j \mid T_i \geq t; \mathbf{Z}_{it},\mathbf{X}_{i} ),
\end{equation}
where $T_i$ is a discrete random variable that captures when woman $i$'s birth occurs.
To ease presentation the indicator for spell number is suppressed.
The vectors of explanatory variable $\mathbf{Z}_{it}$ and $\mathbf{X}_{i}$ include 
information about various individual, household, and community characteristics 
discussed below.

The hazard rate is specified as
\begin{equation}
 h_{ijt} = \frac{\exp(D_j(t) + \alpha_{jt}'\mathbf{Z}_{it} + \beta_j'\mathbf{X}_{i})} 
 {1 + \sum_{l=1}^2 \exp(D_j(t) + \alpha_{lt}'\mathbf{Z}_{it} + \beta_l'\mathbf{X}_{i})} \: \: \; \; \;  j = 1,2
 \label{eq:hazard}
\end{equation}
where $D_{j}(t)$ is the piece-wise linear baseline hazard for outcome $j$, captured
by dummies and the associated coefficients,
\begin{equation}
D_j(t) = \gamma_{j1} D_1 + \gamma_{j2} D_2 + \ldots + \gamma_{jT} D_T,
\end{equation}
where $D_m = 1$ if $t=m$ and zero otherwise.
This approach to modeling the baseline hazard is flexible and does not place 
overly strong restrictions on the baseline hazard.

[TK I am not very satisfied with this explanation of the problems with
proportionality and the associated bias; non-proportionality is required
to capture changes in use of sex selection]

In principle, specifying the model as a proportional hazard model, i.e.\ one 
where covariates simply shift the hazard rates up or down independent of 
spell length, is more efficient, but only provided that the proportionality 
assumption holds.
If the proportionality assumption does not hold, however, the result is
a potentially substantial bias in estimates.
The problem is that a proportional hazard model does not allow covariates
to have different effects at different times within a spell and therefore
cannot capture differences in the shape of the hazard functions between 
different groups. 
It is highly unlikely---even in the absence of prenatal sex determination---that 
the baseline hazards are the same across education levels, areas of residence, 
or sex composition of previous births.
Any bias from the proportionality assumption is likely exacerbated by the 
introduction of prenatal sex determination for two reasons.
First, one of the main points of this paper is that use of sex-selective 
abortions affects birth spacing, and use of sex selection differs across groups.
Second, as discussed above, the use of sex selection may vary within a spell,
depending on the length of the spell.

I therefore use a non-proportional model where the main explanatory variables 
and the interactions between them are interacted with the baseline hazards.
This is captured by the $Z$ set of explanatory variables
\begin{equation}
 \mathbf{Z}_{it} = D_j(t) \times (\mathbf{Z}_1 + Z_2 + \mathbf{Z}_1 \times Z_2),
\end{equation}
where $D_j(t)$ is the piece-wise linear baseline hazard and $\mathbf{Z}_1$ captures sex 
composition of previous children, if any, and $Z_2$ captures area of residence.
This allows the effects of the main explanatory variables on the probabilities 
of having a boy, a girl, or no birth to vary over time within a spell.
The use of a non-proportional specification, together with a flexible baseline hazard, 
also mitigates any potential effects of unobserved heterogeneity \citep{Dolton1995}.

The remaining explanatory variables, $\mathbf{X}$, enter proportionally.
To further minimize any potential bias from assuming proportionality, estimations 
are done separately for different levels of mothers' education and for different 
time periods.
The exact specifications and the individual variables are described below.

Equation (\ref{eq:hazard}) is equivalent to the logistic hazard model and has the same 
likelihood function as the multinomial logit model \citep{allison82,jenkins95}.
Hence, if the data are transformed so the unit of analysis is spell unit rather 
than the individual woman, the model can be estimated using a standard multinomial 
logit model.%
\footnote{
A potentially issue is that the multinomial model assumes that alternative 
exit states are stochastically independent,
also known as the Independence of Irrelevant Alternatives (IIA) assumption.
This assumption rules out any individual-specific unmeasured or 
unobservable factors that affect both the hazard of having a girl and the 
hazard of having a boy.
To address this issue the estimations include a proxy for fecundity
discussed in Section \ref{sec:data}.
In addition, the multivariate probit model can be used as an alternative
to the multinomial logit because the IIA is not imposed \citep{han90}.
The results are essentially identical between these two models and
available upon request.
}
In the reorganized data the outcome variable is zero if the
woman does not have a child in a given period, one if she gives birth to a son 
in that period, and two if she gives birth to a daughter in that period.

The main downside of this approach is that direct interpretation of the estimated 
coefficients for this model is challenging because of the competing risk setup.
First, coefficients show the change in hazards relative to the base outcome, 
no birth, rather than simply the hazard of an event.
Second, a positive coefficient does not necessarily imply that an increase in a
variable's value increases the probability of the associated event because the 
probability of another event may increase even more \citep{thomas96}.

It is, however, straightforward to calculate the predicted probabilities of 
having a boy and of having a girl for each $t$ within a spell, conditional on 
a set of explanatory variables and not having had a child before that period.
The predicted probability of having a boy in period $t$ for a given set of 
explanatory variable values, $\mathbf{Z}_k$ and $\mathbf{X}_k$, is
\begin{equation}
P(b_{t} | \mathbf{X}_{k}, \mathbf{Z}_{kt}, t ) 
=  
\frac{ \exp(D_j(t) + \alpha_{1t}' \mathbf{Z}_{kt} + \beta_1' \mathbf{X}_{k} )}
{1 + \sum_{l=1}^2 \exp(D_j(t) + \alpha_{lt} ' \mathbf{Z}_{kt} + \beta_l ' \mathbf{X}_{k})},
\label{eq:probability_boy}
\end{equation}
and the predicted probability of having a girl is
\begin{equation}
P(g_{t} | \mathbf{X}_{k}, \mathbf{Z}_{kt},t ) 
=  
\frac{ \exp(D_j(t) + \alpha_{2t}'\mathbf{Z}_{kt} + \beta_2'\mathbf{X}_{k} )}
{1 + \sum_{l=2}^2 \exp(D_j(t) + \alpha_{lt}'\mathbf{Z}_{kt} + \beta_l'\mathbf{X}_{k})}.
\label{eq:probability_girl}
\end{equation}
With these two probabilities is it easy to calculate, for each $t$, the estimated
percentage of children born that are boys, $\hat{Y}$, 
\begin{equation}
\hat{Y}_t 
= 
\frac{ P(b_{t} | \mathbf{X}_{k}, \mathbf{Z}_{kt},t )}
{ P(b_{t} | \mathbf{X}_{k}, \mathbf{Z}_{kt},t) + P(g_{t} | \mathbf{X}_{k}, \mathbf{Z}_{kt},t )} 
\times 100,
\label{eq:probability_son}
\end{equation}
together with the associated confidence interval for given values of explanatory 
variables.%
\footnote{
Within each period $1-P(b_{t})-P(g_{t})$ is the probability of not having a birth in 
period $t$.
}

For ease of exposition the procedure is presented here in two steps, but the actual
calculation of the percent boys is done in one step with the 95 percent confidence 
interval calculated using the Delta method.
Results are presented as the estimated percent boys born by length of birth spacing
using graphs.%
\footnote{
The parameter estimates are available on request.
}
For each graph, the extent to which the percent boys is statistically 
significantly above the natural sex ratio indicates the use of sex selection.


[TK  may need to combine with and without sons to show differences in spacing 
over time; maybe it should also have predicted average/median spacing with some 
confidence intervals to test]


The other important part of the model is the spacing between births.
Spacing is captured by the survival curve, which shows the probability of not 
having had a birth yet by spell duration.
The survival curve at time $t$ is 
\begin{equation}
S_{t} 
= 
\prod_{d=1}^t 
\left( 
	1- \left(P(b_{d} | \mathbf{X}_{k}, \mathbf{Z}_{kd}, d) 
	+ P(g_{d} | \mathbf{X}_{k}, \mathbf{Z}_{kd}, d) \right) 
\right),
 \label{eq:survival}
\end{equation}
or equivalently
\begin{equation}
S_{t} 
= 
\prod_{d=1}^t
\left(
\frac{ 1 }
{1 + \sum_{l=2}^2 \exp(D_j(t) + \alpha_{ld}'\mathbf{Z}_{kd} + \beta_l'\mathbf{X}_{k})}
\right).
\end{equation}

In the absence of sex selection, I expect most parities to show an 
``inverted s'' pattern, where there initially are relatively few births, 
followed by a substantial number of births over a 1 to 2 year period, and then
relatively few births thereafter.
As sex selection becomes more widely used the associated
longer spacing shows up by making the survival curve straighter,
indicating that some of the births that would originally have taken place
now take place later because of abortions.
The use of sex selection is clearly not the only factor that can change the shape of
the survival curves; factors such as the desired number of children and
use of contraceptives may also shape the shape.
To account for this it is best to compare survival curves for an individual
parity between women who are likely to use sex selection and women who are
not, for example because they already have one or more sons.

In addition to information about spacing, the survival curves also provide 
``weighting'' for the associated percentage boys born.
The steeper the survival curve, the more weight should be assigned to a given
spell period because it is based on more births, 
whereas a period with a flat survival curve should be given little weight because the 
percentage boys is based on few births.
Hence, although they cannot by themselves show sex selection, the survival
curves are a crucial complement to the estimated sex ratios by duration.


[The problem with comparing survival curves]

In principle, in order to understand how birth spacing changes in response
to the introduction of prenatal sex determination one could simply compare survival 
curves between different sex compositions of prior births over time.
If the idea that the availability of sex-selective abortion increases the
spacing more after the birth of girls than after the birth of boys is 
correct, we should see the two survival curves become closer to each other
and we might even be able to directly observe that the survival curve for prior
child being a girl is above the survival curve for prior child being a boy,
instead of substantially below as was generally the case before prenatal
sex determination become widely available.

There are, however, two (interrelated) problems.
First, the probability that a woman will progress to a next birth  
is not the same across different sex compositions of prior children.
Specifically, one of the ways that past research has tried to establish
son preference in fertility behavior is to compare parity progression
rates for different sex compositions [TK references - see footnote 1].
The idea was that if there is son preference a couple would be less likely
to progress to a next birth if they have one or more son already for a
given number of prior births.

The fact that the probability of having a next birth also depends on the
sex composition complicates the comparison of survival curves. 
A lower parity progression probability will naturally make the survival
curve flatter.
This means that the survival curve for a girl composition will be more
likely to be below the boy composition.
Even if the use of sex selection is substantial enough to move the
survival curve for girl composition to the right of the boy
composition the lower parity progression probability for boy 
composition means that the survival curves would eventually have to
cross.
This makes it difficult---if not impossible---to design a direct test
of how the survival curves compare.

The problem of different parity progression rates across different
sex composition is exacerbated by falling fertility over time.
This is because there are now more parities where they may be a 
substantial differences in parity progression rates and because
the difference in parity progression for a given parity will 
become larger if there is son preference.
As the parity progression probability decreases it becomes impossible
to capture the time until a certain survival point (like 0.5 of women
having had a child; the survival curve may never actually reach this
point).

In addition, it is possible that falling fertility also generically 
leads to longer spacing between births.
This means that we have to separate out which part of the longer
spacing comes from falling fertility (and therefore should also
affect boy composition) and which parts comes from increased use
of sex selection.

[We should expect sex selection to be mainly used for the last---or close to
last---birth that a couple expect to have because of the cost and risk
of the procedures.]


One solution to this problem is to condition on the parity progression
rates.
That is, instead of comparing say how long it takes for 50\% of women
who began the spell to have the next birth across boy and girl composition, 
which would work well if the next birth was nearly universal, we 
could calculate how long it takes for 50\% of those women who will 
eventually have a next child to have that child based on the estimated
survival curves.
This number can then be compared across sex compositions to 
understand if the spell length for girl composition increases
more than for boy composition with the introduction of sex selection.
Obviously, this should be calculated for different percentages
and the easiest way to interpret the result is to graph survival
conditional on parity progression occurring.
All compositions therefore begin at 100\% and end at 0\% because
we condition on having the next child.

Note, this is not the same as simply calculating the average 
\emph{observed} spell lengths among women who have a given parity
child in the survey because that number does not take into account
the censoring of some spells that will eventually lead to a birth.


\section{Data\label{sec:data}}

The data come from the three rounds of the National Family Health Survey 
(NFHS-1, NFHS-2 and NFHS-3),
collected in 1992--93, 1998--99, and 2005--2006.%
\footnote{
A delay in the survey for Tripura means that NFHS-2 has a small number of observation 
collected in 2000.
}
The surveys are large: NFHS-1 covered 89,777 ever-married women 
aged 13--49 from 88,562 households,
NFHS-2 covered 90,303 ever-married women aged 15--49 from 92,486 households
and NFHS-3 covered 124,385 never-married and ever-married women aged 
15--49 from 109,041 households.

I exclude visitors to the household, as well as
women married more than once, divorced, or not living with their husband,
women with inconsistent information on age of marriage,
and those with missing information on education.
Women interviewed in NFHS-3 who were never married or where gauna had not
been performed were also dropped.
The same goes for women who had at least one multiple birth,
reported having a birth before age 12, had a birth before marriage, or
a duration between births less than 9 months.
Women who reported less than 9 months between marriage and first birth
remain in the sample unless they are dropped for another reason.%
\footnote{
Women who report less than 9 months between marriage and first birth are retained 
because between 10 and 20 percent fall into this category.
Although it is possible that some of these births are premature the high number of
women who report a birth less than 6 months after marriage indicates that conception
likely occurred before marriage in most cases.
}

Finally, I restrict the sample to Hindus,
who constitute about 80 percent of India's population.
If use of sex selection differ between Hindu and other religions, such 
as Sikhs, assuming that the baseline hazard is the same would lead to bias.
The other groups are each so small relative to Hindus that it is not
possible to estimate different baseline hazards for each group.
Furthermore, the groups are so different in terms of background and son preference
that combining them into one group would not make sense.

There are four advantages to using the NFHS.
First, surveys enumerators pay careful attention to spacing between births and
probe for ``missed'' births.
Second, no other surveys cover as long a period in the same amount of detail.
The three NFHS rounds allow me to show the development in spacing and 
sex ratio from before sex-selective abortions were available until 2006.
Third, NFHS has birth histories for a large number of women.%
\footnote{
The Special Fertility and Mortality Survey appears to cover a much large number of households
than the three rounds of the NFHS combined, but \citet{jha06} only use the births that 
took place in 1997 making their sample sizes by parity smaller than here.
Their sample consists of 133,738 births of which 38,177 were first
born, 36,241 second born, and 23,756 were third born.
The differences in results for first born children are discussed in the online 
Appendix.
}
Finally, even if probing for missing births may not completely eliminate recall error,   
the overlap in cohorts covered and the large sample size make it possible to establish 
where recall error remains a problem.

Recall error arise mainly from child mortality, when respondents are reluctant to
discuss deceased children.%
\footnote{
The online Appendix contains a more thorough discussion of recall error and how I address it.
} 
Systematic recall error, where the likelihood of reporting a deceased child depends on
the sex of the child, is especially problematic because it biases the sex ratios.
Probing catches many missed births, but systematic recall error is still a potentially 
substantial problem.
Three factors contribute to the problem here.
First, girls have significantly higher mortality risk than boys.
Second, son preference may increase the probability that boys are remembered relative to girls.
Finally, in NFHS-1 and NFHS-2 enumerators probed only for a missed birth if the
initial reported birth interval was four calendar years or more.
But, given short durations between births, especially after the birth of a girl,
that procedure is unlikely to pick up all missed children.

Observed sex ratios by cohort provide a straightforward way to determine 
whether recall error is a problem.
Because prenatal sex determination techniques did not become widely available until the 
mid-1980s, a higher than natural sex ratio for cohorts born before that time must be 
the result of systematic recall error.
As shown in the online Appendix, the observed sex ratio by parity becomes more male 
dominated the further back births took place.
In addition, births in the same cohort tend to be more male dominated the more recent the 
survey (births in the cohort took place longer ago relative to the survey).
Hence, there is evidence of recall error and the degree of recall error increases
with length of time between survey and cohort.

Using cohort year of birth to analyze recall error and decide which observations
to keep is, however, problematic because the year of birth for a given parity is affected 
by recall error; for example, a second born child listed as first born will be 
born later than the real first born child.
Year of marriage should, however, be unaffected by recall error.
Using year of marriage the basic recall error pattern remains with women married longer 
ago more likely to report that their child was a boy for a given parity.
Similarly, comparing women married in the same period across surveys shows
that women married longer ago are more likely to report having sons.

That recall error increases the longer ago somebody was married means
that duration of marriage is a better predictor of recall error than calendar year of 
marriage.
Figure \ref{fig:sexRatioMarriage} shows the observed sex ratio for children 
reported as first born as a function of duration of marriage combining all three surveys.%
\footnote{
The graph for second births shows a similar pattern.
The graphs for the second births and the individual survey rounds are available upon request.
}
The solid line is the sex ratio of children reported 
as first born by the number of years between the survey and marriage, 
the dashed lines indicate the 95 percent confidence interval 
and the horizontal line the natural sex ratio (approximately 0.512).
To ensure sufficient cell sizes the years are grouped in twos.

% \begin{figure}[htp]%
% \centering
% \includegraphics[width=0.9\textwidth]{sex_ratio_marriage}
% \caption{Ratio of boys in ``first'' births}
% \label{fig:sexRatioMarriage}
% \end{figure}


Figure \ref{fig:sexRatioMarriage} clearly illustrates the systematic recall error
problem.
The observed sex ratio is increasingly above the expected value the
longer ago the parents were married.
The increasingly unequal sex ratio with increasing marriage duration suggests that
a solution to the recall error problem is to drop women who were married ``too far'' from 
the survey year.
The main problem is establishing what the best cut-off point should be.
The observed sex ratio is consistently significantly higher than the natural sex ratio 
from around 24 years of marriage, so one possibility is simply to drop all women married 
more than 24 years at the time of the survey.
But, as the Appendix shows, there are differences across the three surveys and between 
parities.
I therefore use different cut-off points by survey round.
For NFHS-1 women married 22 years or more were dropped, with the corresponding cut-off 
points 23 years for NFHS-2  and 26 years for NFHS-3.
The final sample consists of 146,096 women, with 332,951 parity one through four births.%
\footnote{
The online Appendix presents the results for a more restrictive definition and for the
women dropped because of recall error concerns.
}


\subsection{Spell Definition\label{sec:spell_def}}

Spell duration is measured in months.
The first spell begins at the month of marriage because many 
women report giving birth less than 9 months after they were married.
For women who began living with their husbands at too young an age to conceive, the 
starting point should ideally be first ovulation, when she becomes ``at risk'' for a 
pregnancy, rather than month of marriage.
Unfortunately, information on age of menarche is only available in NFHS-1.
Instead, for women who began living with their husband before age 12, I set the 
the month they turned 12 years of age as staring point for the first spell.

The second and subsequent spells begin 9 months after the previous birth 
because that is the earliest we should expect to observe a new birth.
A few women report births that occurred less than 9 months 
after the previous birth; those women are dropped.

All spells continue until either a child is born or the spell is censored.
Censoring can happen for three reasons:
the survey takes place;
the woman is sterilized;
or the number of births observed becomes too sparse for the method to work.
The timing of censoring because of too few births vary slightly by spell but 
is generally 5 years or longer after the beginning of the spell.

I group spells into three time periods based on spell start date:
1972--1984, 1985--1994, and 1995--2006.
The first period covers the time before sex-selective abortions became widely available.
Abortion was legalized in 1971 and amniocentesis was introduced
in India in 1975, but the first newspaper reports on the availability of prenatal sex 
determination were not until 1982--83 \citep{Sudha1999,bhat06,Grover2006}.
The number of clinics quickly increased, and knowledge about sex selection became widespread
after a senior government official's wife aborted a fetus that turned out to be male \citep[p.\ 598]{Sudha1999}.
The second period covers the time from the widespread emergence of sex-selective abortions
until the prenatal Diagnostic Techniques (PNDT) act was passed in 1994.
The final period is from the PNDT act until the last available survey.
The PNDT act made it a criminal offence to reveal the sex of the fetus and was
followed by a campaign against the use of sex selection, although
enforcement appears to be relatively lax.

By dividing spells into these periods I can examine how the use of sex selection 
has changed across the three different regimes.
Note that the periods are based on the spells' beginning year, and some spells 
will therefore cover two periods.
A couple may, for example, be married in 1984, but not have their first child until 1986.
That couple's first spell will be in the 1972--1984 period, even though most of the 
spell actually falls in the 1985--1994 period.
Some children born from spells that began in the 1972--1984 period may therefore have been
conceive when prenatal sex determination techniques were available, which could result
in evidence of sex-selective abortions even for this period.
Similarly, a spell that began in the 1985--1994 period may have been partly or mostly
under the PDNT act.
The overall effect is to bias downwards any differences between the periods.

\subsection{Explanatory Variables}

The explanatory variables are divided into two groups.
The first group consists of variables expected to affect the shape of the hazard function: 
mother's education, sex composition of previous children, and area of residence.
Increasing the number of variables interacted with the baseline hazard lowers the risk 
of bias but requires more data to precisely estimate.
I chose these variables because the prior literature shows that they affect 
spacing choices and because the prior literature on sex selection indicates 
that these are correlated with sex selection.
The second group of variables are those expected to have an approximately 
proportional effect on the hazard: age of the mother at the beginning of the spell, 
length of her first spell (for second spell and above), whether the household owns 
land, and whether the household belongs to a scheduled tribe or caste.

Increasing education of mothers is strongly associated with lower fertility, with
the negative effect of higher opportunity cost on fertility more than outweighing the 
positive effect of higher income \citep{schultz97}.
Higher education should therefore be associated with higher use of sex selection.%
\footnote{
Fathers' education has two opposite predicted effects: the associated higher income
should increase fertility and therefore lower the pressure to use sex selection, but
the higher income also makes the use of sex selection cheaper.
In practice, fathers' education had little effect on the hazards and the use of 
sex-selective abortions and is not included.
}
I divide women into three education groups:
no education, 1 to 7 years of education, and 8 and more years of education.
The models are estimated separately for each education level.%
\footnote{
A potential concern here is reverse causation, where the sex of children
affect women's education.
Although no direct information is available on when women left school it is
possible to estimate school leaving age from the highest completed grade
and the usual starting age.
As described in the online Appendix, relatively few women could potentially 
have returned to school after the birth of their first child,
and only 56 could possibly have ended up in the wrong education group. 
Hence, there is little likelihood that reverse causation is a substantial
concern here.
}

As discussed, the sex composition of previous children affects both the timing
of births and the use of sex-selective abortions.
I capture sex composition of previous children with dummy variables for the
possible combinations for the specific spell, ignoring the ordering of births.
As an example, for the third spell three groups are used: Two boys,
one girl and one boy, and two girls.

The area of residence is a dummy variable for the household living in
an urban area.%
\footnote{
NFHS uses four categories for area of residence: Large city, small city, town
and countryside.
To reach a sufficient sample size urban areas are merged into one group.
}
The cost of children is higher in urban areas than in rural and access to prenatal
sex determination is easier, and both are expected to be associated with
greater use of sex-selective abortions in urban than in rural areas.
Because of concerns about selective migration I use where the household was living
at the end of each spell.
% %
% \footnote{
% As online Appendix Table D.1 shows there is little difference across
% education levels in migration patterns within each period and patterns across periods
% are also relatively stable.
% }

The sex composition of children, area of residence, and the interactions between
these are all interacted with the piece-wise linear baseline hazard dummies.
In other words, the baseline hazards are assumed to be different depending on
where a woman lives and the sex composition of her previous children.
As an example, for the second spell a separate regression is
run for each education level and in each regression four different baseline hazards 
are included (first child a boy in rural area, first child a boy in urban area,
first child a girl in rural area, first child a girl in urban area).
Although this approach substantially increases the number of regressions and 
estimated parameters it reduces the potential problem of including other variables 
as proportional effects.

The remaining variables are expected to affect hazards proportionally.
Although fecundity cannot be observed directly a suitable
proxy is the duration from marriage until first birth.
Most Indian women do not use contraception before the first birth
and there is pressure to show that a newly married woman can conceive 
\citep{dyson83,Sethuraman2007,Dommaraju2009}.
This is confirmed by the very short spells between marriage and first birth,
even among the most educated.
Hence, a long spell between marriage and first birth is likely due to low fecundity.
For both this variable and the age of the mother at the beginning of the spell 
the squares are also included.
The remaining variables are dummies for household ownership of land and membership
of a scheduled caste or tribe.


\subsection{Descriptive Statistics}

Appendix Table \ref{tab:des_stat1} presents descriptive statistics for
the spells by education level and when the spell began.
There is a substantial number of censored observations.
As an example, for highly educated women who had their first child in the 1995--2006
period, almost half did not have their second child by the time of the survey.
Hence, although about 13,000 women began the second spell 
there are only about 7,000 births to these women.
Censoring becomes even more important for the third and fourth
spells, where around 70 percent of the observations are censored.
Generally, censoring increases with parity and time period.
This reflects a combination of factors: timing of the surveys
relative to the periods of interest, later beginning of childbearing, 
falling fertility, and the longer spells from sex-selective abortions.
The high number of censored observations underscore the importance of controlling for
censoring when examining the relationship between fertility and sex selection.

The descriptive statistics also provide a first indication of how the
sex ratio at birth changes over time and by spell.
For the first spell the sex ratio is very close to the natural
for all education groups and all three time periods.
As an example, among the highly educated group for the 1995--2006 period,
51.3 percent of the children born were boys.%
\footnote{
There still appears to be some recall error for the group of women without
education for the 1972--1984 period, where 52.3 percent of the children born were boys.
}
For the second spell, all but the highly educated group in the last two
periods have sex ratios in line with the natural sex ratio.
Women with 8 or more years of education have 53.1 and 54.3 percent
boys in the 1985--1994 and 1995--2006 periods, respectively.
This pattern repeats itself for the third spell, except the percentage
boys is higher for the high education group (55.3 and 55.9 for the last
two time periods).
Finally, for the fourth spell the high education group 
had 60 percent boys in the last period, i.e. after the PNDT act was introduced.
Note, however, that for the fourth spell the number of births is substantially
smaller and censoring even more important than for the other spells.

India's population has become progressively more urban.
For the first period, 32 percent of the women entering the first spell lived in urban areas.
This increases to 35 percent for the second period and to 42 percent for the final period.
The population is also substantially better educated.
Women with no education constituted almost 60 percent 
in the first period, but less than thirty percent in the last period.
Correspondingly, in the first period just over twenty percent had 8 or more 
years of education, but in the last period it was almost half.
Part of the increase in education is correlated with the increase in urbanization,
but the proportion of better-educated women has increased substantially
in the rural areas as well.
Among the high education group almost 70 percent lived in urban areas
during the first period but this had fallen to less than 60 percent
in the last period.

% The increase in urbanization and education is likely to exert downward pressure
% on fertility and the high censoring rates for the later periods are evidence of this.
% The average number of children born to women by the time they turn 35 illustrate how
% strong the decline in fertility has been.%
% \footnote{
% Figure \ref{fig:fertility} shows the full results.
% }
% Women born in the early 1940s had on average close to 5 children when 
% they reached 35, but women born in the early 1970s had only just over three children.
% The low number of children is especially remarkable because it combines
% all education levels and all areas of residence.
% % Hence, fertility in cities is likely substantially lower.


\section{Results\label{sec:results}}

[Difference in approach: use the sex composition of prior
children, rather the sex of the last child.
Problem is that the cell sizes become too small if we further split up
the higher spell cells.

Define cell]


The main question of interest here is how the duration between one
birth and the next is related to son preference.
The hypothese examined here is that in the absence of sex selection son 
preference leads to shorter spacing after the birth of a girl than after the 
birth of a boy, whereas son preference increases the spacing after the birth 
of a girl relative to after the birth of a boy when prenatal sex selection is 
used.
We would still expect parents to \emph{begin} trying to conceive earlier
after the birth of a girl than after a boy, but the abortions will automatically
increase the length of the space, and may even lead to longer spacing after the
birth of girls than after boys.

To provide a first look at how the hypothese holds up, Tables 
\ref{tab:median_sex_ratio_low},
\ref{tab:median_sex_ratio_med}, and
\ref{tab:median_sex_ratio_high}
show predicted median duration and sex ratios for the three different
education levels by spell, sex composition, and periods.
These predictions are based on the estimation results employing the method 
detailed in Section \ref{sec:strategy} and are calculated using the means of 
the continuous explanatory variables and the majority category for categorical 
explanatory variables.
The median duration is the number of months it is predicted to take for
half of the women who are predicted to have a birth by the end of spell length
covered.
For example, if 80\% of women with a given set of characteristics are predicted to
have a birth by the end of the end of spell, the median duration is the number of
months it would take for 40\% of women to have a birth.
The sex ratio is the percent of births that result in a son, based on all the 
births predicted to happen over the spell.
If the estimations are based on 100 or fewer births in a cell the predictions
for that cell are not presented.
Furthermore, those cells with 500 or fewer births are marked.


% Median duration and sex ratio tables
\input{../tables/median_sex_ratio_low.tex}

Table \ref{tab:median_sex_ratio_low} shows the predictions for women 
with no education.
Overall, spacing is shorter the fewer boys a couple has, but the
differences in median duration are never more than 3 months and most
are only one or two months
(the exceptions to this pattern are mostly for small cells).
Furthermore, although the median duration generally increases over 
time---with the exception of the first spell---there is little 
change in the relative pattern of durations across sex compositions
over time.
These results point to an absence of sex-selective abortions for 
this group of women, which is supported by predicted sex ratios
that mostly close to the natural sex ratio.%
\footnote{
I return to whether there are statistically significant 
differences below when examining sex ratios within spells.
}


\input{../tables/median_sex_ratio_med.tex}

For women with between 1 and 7 years of education it is harder
to discern a pattern, partly because a larger number of area,
sex composition, and periods combinations have relatively few 
births in them.
That said, the predicted median durations show less evidence 
of son preference than for women without education.
The exception to this are for the third spell for rural women
for the 1972-1984 and 1985-1994 periods where the median
durations are two and three months shorted with only girls
compared to only boys, although the duration is longer for
only girls in the 1995-2006 period.
As for women with no education, there is little evidence of
sex selection for women with 1 to 7 years of education,
consistent with the absence of changes in relative durations
across periods.





\input{../tables/median_sex_ratio_high.tex}



\clearpage

\subsection{[Specific spell/education/area combinations of special interest]}
[Only a subset of the planned analyses to demonstrate the method and interpretation]


Figure \ref{fig:results_spell2_high_urban} 
shows the predicted probabilities of having a boy and standard survival curves
for the three periods for urban women with 8 or more years of education.%
\footnote{
The corresponding graphs for rural women and for women with no education are
shown in Figures \ref{fig:results_spell2_high_rural}, \ref{fig:results_spell2_low_urban},
and \ref{fig:results_spell2_low_rural}
}
Two main results stand out.
First, women whose first child was a boy do not appear to use sex selection during
the second spell.
The predicted percent boys is consistently close to the natural level in all three
panels (\ref{fig:results_spell2_high_urban}(d)-(f)).
Second, as prenatal sex determination became available there is more and more
evidence of sex selection affecting the sex ratio for women with a girl as their
first child.
In panel (a), which covers the period before sex selection became generally
available, there is no significant evidence of a sex ratio biased towards males.
Panel (b) shows a more male biased sex ratio but only a few months show
a statistically significant difference from the natural rate and the shallow 
slope of the survival curve indicates that relatively few births took place
during these months.
Panel (c), however, shows clear evidence of the effect of access to prenatal
sex determination with a male biased sex ratio for more of the two first years.%
\footnote{
Recall that month 0 is equivalent to 9 months after the birth
of the first child.
}


% High education

\begin{figure}[htpb]
\centering
\caption*{First child a girl}
\setcounter{subfigure}{-2}
\subfloat[1972--1984 (N=2,689)]{
    \begin{minipage}{0.31\textwidth}
        \captionsetup[subfigure]{labelformat=empty,position=top,captionskip=-1pt,farskip=-0.5pt}
        \subfloat[Prob.\ boy (\%)]{\includegraphics[width=\textwidth]{spell2_g1_high_urban_g_pc}}\\
        \subfloat[Prob.\ no birth yet]{\includegraphics[width=\textwidth]{spell2_g1_high_urban_g_s}} 
        \captionsetup[subfigure]{labelformat=parens}
    \end{minipage}
} 
\setcounter{subfigure}{-1}
\subfloat[1985--1994 (N=4,869)]{
    \begin{minipage}{0.31\textwidth}
        \captionsetup[subfigure]{labelformat=empty,position=top,captionskip=-1pt,farskip=-0.5pt}
        \subfloat[Prob. boy (\%)]{\includegraphics[width=\textwidth]{spell2_g2_high_urban_g_pc}}\\
        \subfloat[Prob. no birth yet]{\includegraphics[width=\textwidth]{spell2_g2_high_urban_g_s}}
        \captionsetup[subfigure]{labelformat=parens}
    \end{minipage}
}
\setcounter{subfigure}{0}
\subfloat[1995--2006 (N=3,774)]{
    \begin{minipage}{0.31\textwidth}
        \captionsetup[subfigure]{labelformat=empty,position=top,captionskip=-1pt,farskip=-0.5pt}
        \subfloat[Prob. boy (\%)]{\includegraphics[width=\textwidth]{spell2_g3_high_urban_g_pc}}\\
        \subfloat[Prob. no birth yet]{\includegraphics[width=\textwidth]{spell2_g3_high_urban_g_s}}
        \captionsetup[subfigure]{labelformat=parens}
    \end{minipage}
}
\caption*{First child a boy}
\setcounter{subfigure}{1}
\subfloat[1972--1984 (N=2,806)]{
    \begin{minipage}{0.31\textwidth}
        \captionsetup[subfigure]{labelformat=empty,position=top,captionskip=-1pt,farskip=-0.5pt}
        \subfloat[Prob. boy (\%)]{\includegraphics[width=\textwidth]{spell2_g1_high_urban_b_pc}}\\
        \subfloat[Prob. no birth yet]{\includegraphics[width=\textwidth]{spell2_g1_high_urban_b_s}} 
        \captionsetup[subfigure]{labelformat=parens}
    \end{minipage}
} 
\setcounter{subfigure}{2}
\subfloat[1985--1994 (N=5,246)]{
    \begin{minipage}{0.31\textwidth}
        \captionsetup[subfigure]{labelformat=empty,position=top,captionskip=-1pt,farskip=-0.5pt}
        \subfloat[Prob. boy (\%)]{\includegraphics[width=\textwidth]{spell2_g2_high_urban_b_pc}}\\
        \subfloat[Prob. no birth yet]{\includegraphics[width=\textwidth]{spell2_g2_high_urban_b_s}}
        \captionsetup[subfigure]{labelformat=parens}
    \end{minipage}
}
\setcounter{subfigure}{3}
\subfloat[1995--2006 (N=3,969)]{
    \begin{minipage}{0.31\textwidth}
        \captionsetup[subfigure]{labelformat=empty,position=top,captionskip=-1pt,farskip=-0.5pt}
        \subfloat[Prob. boy (\%)]{\includegraphics[width=\textwidth]{spell2_g3_high_urban_b_pc}}\\
        \subfloat[Prob. no birth yet]{\includegraphics[width=\textwidth]{spell2_g3_high_urban_b_s}}
        \captionsetup[subfigure]{labelformat=parens}
    \end{minipage}
}
\caption{Predicted probability of having a boy and probability of
no birth yet from nine months after first birth for urban 
women with 8 or more years of education by month beginning 9 months after prior birth. 
Predictions based on age 22 at first birth.
Left column shows results prior to sex selection available, middle column before
sex selection illegal and right column after sex selection illegal.
N indicates the number of women in the relevant group in the underlying samples.
}
\label{fig:results_spell2_high_urban}
\end{figure}




% Survival curves conditional on progression 

Figures \ref{fig:results_spell2_low_pps} and \ref{fig:results_spell2_high_pps}
show survival curves for the second spell (from first birth) for women with
no education and women with 8 or more years of education.
To allow for comparison these survival curves show the ratio of women who
have not had a birth yet, given that they will eventually are predicted to
have a birth.
Women with no education show relatively similar survival curves whether their
first child was a girl or a boy, but to the extent that a pattern is present
it shows, as expected, shorter spacing a girl than a boy.
Rural women with 8 or more years of education show essentially the same 
pattern.
The lack of clear results is not surprising given that all three groups of
women are likely to have more than two children.

The effect of sex selection on spacing first show up for urban women with
8 or more years of education.
Panel (a) in Figure \ref{fig:results_spell2_high_pps} show shorter spacing
for the second spell when the first child is a girl than a boy, and this
pattern is also present in Panel (b).
Panel (c) on the other hand show first evidence of \emph{longer} spacing 
after the birth of a girl than after a boy, consistent with the male-biased
sex ratio in Panel (c) in Figure \ref{fig:results_spell2_high_urban}.


% Low education

\begin{figure}[htpb]
\centering
\caption*{Urban}
\setcounter{subfigure}{-1}
\subfloat[1972--1984]{
    \begin{minipage}{0.31\textwidth}
        \captionsetup[subfigure]{labelformat=empty,position=top,captionskip=-1pt,farskip=-0.5pt}
        \subfloat[Prob.\ no birth yet]{\includegraphics[width=\textwidth]{spell2_g1_low_urban_pps}} 
        \captionsetup[subfigure]{labelformat=parens}
    \end{minipage}
} 
\setcounter{subfigure}{-0}
\subfloat[1985--1994]{
    \begin{minipage}{0.31\textwidth}
        \captionsetup[subfigure]{labelformat=empty,position=top,captionskip=-1pt,farskip=-0.5pt}
        \subfloat[Prob. no birth yet]{\includegraphics[width=\textwidth]{spell2_g2_low_urban_pps}}
        \captionsetup[subfigure]{labelformat=parens}
    \end{minipage}
}
\setcounter{subfigure}{1}
\subfloat[1995--2006]{
    \begin{minipage}{0.31\textwidth}
        \captionsetup[subfigure]{labelformat=empty,position=top,captionskip=-1pt,farskip=-0.5pt}
        \subfloat[Prob. no birth yet]{\includegraphics[width=\textwidth]{spell2_g3_low_urban_pps}}
        \captionsetup[subfigure]{labelformat=parens}
    \end{minipage}
}
\caption*{Rural}
\setcounter{subfigure}{2}
\subfloat[1972--1984]{
    \begin{minipage}{0.31\textwidth}
        \captionsetup[subfigure]{labelformat=empty,position=top,captionskip=-1pt,farskip=-0.5pt}
        \subfloat[Prob. no birth yet]{\includegraphics[width=\textwidth]{spell2_g1_low_rural_pps}} 
        \captionsetup[subfigure]{labelformat=parens}
    \end{minipage}
} 
\setcounter{subfigure}{3}
\subfloat[1985--1994]{
    \begin{minipage}{0.31\textwidth}
        \captionsetup[subfigure]{labelformat=empty,position=top,captionskip=-1pt,farskip=-0.5pt}
        \subfloat[Prob. no birth yet]{\includegraphics[width=\textwidth]{spell2_g2_low_rural_pps}}
        \captionsetup[subfigure]{labelformat=parens}
    \end{minipage}
}
\setcounter{subfigure}{4}
\subfloat[1995--2006]{
    \begin{minipage}{0.31\textwidth}
        \captionsetup[subfigure]{labelformat=empty,position=top,captionskip=-1pt,farskip=-0.5pt}
        \subfloat[Prob. no birth yet]{\includegraphics[width=\textwidth]{spell2_g3_low_rural_pps}}
        \captionsetup[subfigure]{labelformat=parens}
    \end{minipage}
}
\caption{Survival curves conditional on parity progression
for women with no education by month beginning 9 months after prior birth.
Left column shows results prior to sex selection available, middle column before
sex selection illegal and right column after sex selection illegal.
N indicates the number of women in the relevant group in the underlying samples.
}
\label{fig:results_spell2_low_pps}
\end{figure}





% High education

\begin{figure}[htpb]
\centering
\caption*{Urban}
\setcounter{subfigure}{-1}
\subfloat[1972--1984]{
    \begin{minipage}{0.31\textwidth}
        \captionsetup[subfigure]{labelformat=empty,position=top,captionskip=-1pt,farskip=-0.5pt}
        \subfloat[Prob.\ no birth yet]{\includegraphics[width=\textwidth]{spell2_g1_high_urban_pps}} 
        \captionsetup[subfigure]{labelformat=parens}
    \end{minipage}
} 
\setcounter{subfigure}{-0}
\subfloat[1985--1994]{
    \begin{minipage}{0.31\textwidth}
        \captionsetup[subfigure]{labelformat=empty,position=top,captionskip=-1pt,farskip=-0.5pt}
        \subfloat[Prob. no birth yet]{\includegraphics[width=\textwidth]{spell2_g2_high_urban_pps}}
        \captionsetup[subfigure]{labelformat=parens}
    \end{minipage}
}
\setcounter{subfigure}{1}
\subfloat[1995--2006]{
    \begin{minipage}{0.31\textwidth}
        \captionsetup[subfigure]{labelformat=empty,position=top,captionskip=-1pt,farskip=-0.5pt}
        \subfloat[Prob. no birth yet]{\includegraphics[width=\textwidth]{spell2_g3_high_urban_pps}}
        \captionsetup[subfigure]{labelformat=parens}
    \end{minipage}
}
\caption*{Rural}
\setcounter{subfigure}{2}
\subfloat[1972--1984]{
    \begin{minipage}{0.31\textwidth}
        \captionsetup[subfigure]{labelformat=empty,position=top,captionskip=-1pt,farskip=-0.5pt}
        \subfloat[Prob. no birth yet]{\includegraphics[width=\textwidth]{spell2_g1_high_rural_pps}} 
        \captionsetup[subfigure]{labelformat=parens}
    \end{minipage}
} 
\setcounter{subfigure}{3}
\subfloat[1985--1994]{
    \begin{minipage}{0.31\textwidth}
        \captionsetup[subfigure]{labelformat=empty,position=top,captionskip=-1pt,farskip=-0.5pt}
        \subfloat[Prob. no birth yet]{\includegraphics[width=\textwidth]{spell2_g2_high_rural_pps}}
        \captionsetup[subfigure]{labelformat=parens}
    \end{minipage}
}
\setcounter{subfigure}{4}
\subfloat[1995--2006]{
    \begin{minipage}{0.31\textwidth}
        \captionsetup[subfigure]{labelformat=empty,position=top,captionskip=-1pt,farskip=-0.5pt}
        \subfloat[Prob. no birth yet]{\includegraphics[width=\textwidth]{spell2_g3_high_rural_pps}}
        \captionsetup[subfigure]{labelformat=parens}
    \end{minipage}
}
\caption{Survival curves conditional on parity progression
for women with 8 or more years of education by month beginning 9 months after prior birth.
Left column shows results prior to sex selection available, middle column before
sex selection illegal and right column after sex selection illegal.
N indicates the number of women in the relevant group in the underlying samples.
}
\label{fig:results_spell2_high_pps}
\end{figure}


% 3rd spell






% Low education

\begin{figure}[htpb]
\centering
\caption*{Urban}
\setcounter{subfigure}{-1}
\subfloat[1972--1984]{
    \begin{minipage}{0.31\textwidth}
        \captionsetup[subfigure]{labelformat=empty,position=top,captionskip=-1pt,farskip=-0.5pt}
        \subfloat[Prob.\ no birth yet]{\includegraphics[width=\textwidth]{spell3_g1_low_urban_pps}} 
        \captionsetup[subfigure]{labelformat=parens}
    \end{minipage}
} 
\setcounter{subfigure}{-0}
\subfloat[1985--1994]{
    \begin{minipage}{0.31\textwidth}
        \captionsetup[subfigure]{labelformat=empty,position=top,captionskip=-1pt,farskip=-0.5pt}
        \subfloat[Prob. no birth yet]{\includegraphics[width=\textwidth]{spell3_g2_low_urban_pps}}
        \captionsetup[subfigure]{labelformat=parens}
    \end{minipage}
}
\setcounter{subfigure}{1}
\subfloat[1995--2006]{
    \begin{minipage}{0.31\textwidth}
        \captionsetup[subfigure]{labelformat=empty,position=top,captionskip=-1pt,farskip=-0.5pt}
        \subfloat[Prob. no birth yet]{\includegraphics[width=\textwidth]{spell3_g3_low_urban_pps}}
        \captionsetup[subfigure]{labelformat=parens}
    \end{minipage}
}
\caption*{Rural}
\setcounter{subfigure}{2}
\subfloat[1972--1984]{
    \begin{minipage}{0.31\textwidth}
        \captionsetup[subfigure]{labelformat=empty,position=top,captionskip=-1pt,farskip=-0.5pt}
        \subfloat[Prob. no birth yet]{\includegraphics[width=\textwidth]{spell3_g1_low_rural_pps}} 
        \captionsetup[subfigure]{labelformat=parens}
    \end{minipage}
} 
\setcounter{subfigure}{3}
\subfloat[1985--1994]{
    \begin{minipage}{0.31\textwidth}
        \captionsetup[subfigure]{labelformat=empty,position=top,captionskip=-1pt,farskip=-0.5pt}
        \subfloat[Prob. no birth yet]{\includegraphics[width=\textwidth]{spell3_g2_low_rural_pps}}
        \captionsetup[subfigure]{labelformat=parens}
    \end{minipage}
}
\setcounter{subfigure}{4}
\subfloat[1995--2006]{
    \begin{minipage}{0.31\textwidth}
        \captionsetup[subfigure]{labelformat=empty,position=top,captionskip=-1pt,farskip=-0.5pt}
        \subfloat[Prob. no birth yet]{\includegraphics[width=\textwidth]{spell3_g3_low_rural_pps}}
        \captionsetup[subfigure]{labelformat=parens}
    \end{minipage}
}
\caption{Survival curves conditional on parity progression
for women with no education by month beginning 9 months after prior birth.
Left column shows results prior to sex selection available, middle column before
sex selection illegal and right column after sex selection illegal.
N indicates the number of women in the relevant group in the underlying samples.
}
\label{fig:results_spell3_low_pps}
\end{figure}





% High education

\begin{figure}[htpb]
\centering
\caption*{Urban}
\setcounter{subfigure}{-1}
\subfloat[1972--1984]{
    \begin{minipage}{0.31\textwidth}
        \captionsetup[subfigure]{labelformat=empty,position=top,captionskip=-1pt,farskip=-0.5pt}
        \subfloat[Prob.\ no birth yet]{\includegraphics[width=\textwidth]{spell3_g1_high_urban_pps}} 
        \captionsetup[subfigure]{labelformat=parens}
    \end{minipage}
} 
\setcounter{subfigure}{-0}
\subfloat[1985--1994]{
    \begin{minipage}{0.31\textwidth}
        \captionsetup[subfigure]{labelformat=empty,position=top,captionskip=-1pt,farskip=-0.5pt}
        \subfloat[Prob. no birth yet]{\includegraphics[width=\textwidth]{spell3_g2_high_urban_pps}}
        \captionsetup[subfigure]{labelformat=parens}
    \end{minipage}
}
\setcounter{subfigure}{1}
\subfloat[1995--2006]{
    \begin{minipage}{0.31\textwidth}
        \captionsetup[subfigure]{labelformat=empty,position=top,captionskip=-1pt,farskip=-0.5pt}
        \subfloat[Prob. no birth yet]{\includegraphics[width=\textwidth]{spell3_g3_high_urban_pps}}
        \captionsetup[subfigure]{labelformat=parens}
    \end{minipage}
}
\caption*{Rural}
\setcounter{subfigure}{2}
\subfloat[1972--1984]{
    \begin{minipage}{0.31\textwidth}
        \captionsetup[subfigure]{labelformat=empty,position=top,captionskip=-1pt,farskip=-0.5pt}
        \subfloat[Prob. no birth yet]{\includegraphics[width=\textwidth]{spell3_g1_high_rural_pps}} 
        \captionsetup[subfigure]{labelformat=parens}
    \end{minipage}
} 
\setcounter{subfigure}{3}
\subfloat[1985--1994]{
    \begin{minipage}{0.31\textwidth}
        \captionsetup[subfigure]{labelformat=empty,position=top,captionskip=-1pt,farskip=-0.5pt}
        \subfloat[Prob. no birth yet]{\includegraphics[width=\textwidth]{spell3_g2_high_rural_pps}}
        \captionsetup[subfigure]{labelformat=parens}
    \end{minipage}
}
\setcounter{subfigure}{4}
\subfloat[1995--2006]{
    \begin{minipage}{0.31\textwidth}
        \captionsetup[subfigure]{labelformat=empty,position=top,captionskip=-1pt,farskip=-0.5pt}
        \subfloat[Prob. no birth yet]{\includegraphics[width=\textwidth]{spell3_g3_high_rural_pps}}
        \captionsetup[subfigure]{labelformat=parens}
    \end{minipage}
}
\caption{Survival curves conditional on parity progression
for women with 8 or more years of education by month beginning 9 months after prior birth.
Left column shows results prior to sex selection available, middle column before
sex selection illegal and right column after sex selection illegal.
N indicates the number of women in the relevant group in the underlying samples.
}
\label{fig:results_spell3_high_pps}
\end{figure}

The effect of sex selection on spacing is even more pronounced in
Figure \ref{fig:results_spell3_high_pps}, which shows the conditional
survival curves for the third spell (from second birth) for women with
8 or more years of education.
Before prenatal sex determination became available there is little difference
in spacing pattern no matter the sex composition of the first two children.
This changes after the introduction as shown in Panel (b), where the spacing
is longer when the first two children are girls than either one boy/one girl
or two boys.
Even more substantial are the results from Panel (c), which shows much 
longer spacing when the first two children are girls.
These results are consistent with the biased sex ratios shown in 
Figure \ref{fig:results_spell3_high_urban}.



\section{Conclusion\label{sec:conclusion}}


[To be added]




\clearpage

\onehalfspacing
\bibliographystyle{aer}
\bibliography{sex_selection_spacing}

\addcontentsline{toc}{section}{References}



\clearpage
\newpage

\appendix
\section{Appendix}

% CHANGING NUMBERING OF FIGURES AND TABLES FOR APPENDIX
\renewcommand\thefigure{\thesection.\arabic{figure}}    
\setcounter{figure}{0}
\renewcommand\thetable{\thesection.\arabic{table}}    
\setcounter{table}{0}
  
% Descriptive statistics tables
\input{../tables/des_stat.tex}


\clearpage

\subsection{Second Spell}



% SPELL 2 - URBAN - LOW
\begin{figure}[h]
\centering
\caption*{First child a girl}
\setcounter{subfigure}{-2}
\subfloat[1972-1984 (N=1,596)]{
    \begin{minipage}{0.26\textwidth}
        \captionsetup[subfigure]{labelformat=empty,position=top,captionskip=-1pt,farskip=-0.5pt}
        \subfloat[Prob.\ boy (\%)]{\includegraphics[width=\textwidth]{spell2_g1_low_urban_g_pc}}\\
        \subfloat[Prob.\ no birth yet]{\includegraphics[width=\textwidth]{spell2_g1_low_urban_g_s}} 
        \captionsetup[subfigure]{labelformat=parens}
    \end{minipage}
} 
\setcounter{subfigure}{-1}
\subfloat[1985-1994 (N=2,232)]{
    \begin{minipage}{0.26\textwidth}
        \captionsetup[subfigure]{labelformat=empty,position=top,captionskip=-1pt,farskip=-0.5pt}
        \subfloat[Prob.\ boy (\%)]{\includegraphics[width=\textwidth]{spell2_g2_low_urban_g_pc}}\\
        \subfloat[Prob.\ no birth yet]{\includegraphics[width=\textwidth]{spell2_g2_low_urban_g_s}} 
        \captionsetup[subfigure]{labelformat=parens}
    \end{minipage}
} 
\setcounter{subfigure}{0}
\subfloat[1995-2006 (N=952)]{
    \begin{minipage}{0.26\textwidth}
        \captionsetup[subfigure]{labelformat=empty,position=top,captionskip=-1pt,farskip=-0.5pt}
        \subfloat[Prob.\ boy (\%)]{\includegraphics[width=\textwidth]{spell2_g3_low_urban_g_pc}}\\
        \subfloat[Prob.\ no birth yet]{\includegraphics[width=\textwidth]{spell2_g3_low_urban_g_s}} 
        \captionsetup[subfigure]{labelformat=parens}
    \end{minipage}
} 
\caption*{First child a boy}
\setcounter{subfigure}{1}
\subfloat[1972-1984 (N=1,754)]{
    \begin{minipage}{0.26\textwidth}
        \captionsetup[subfigure]{labelformat=empty,position=top,captionskip=-1pt,farskip=-0.5pt}
        \subfloat[Prob. boy (\%)]{\includegraphics[width=\textwidth]{spell2_g1_low_urban_b_pc}}\\
        \subfloat[Prob. no birth yet]{\includegraphics[width=\textwidth]{spell2_g1_low_urban_b_s}} 
        \captionsetup[subfigure]{labelformat=parens}
    \end{minipage}
} 
\setcounter{subfigure}{2}
\subfloat[1985-1994 (N=2,411)]{
    \begin{minipage}{0.26\textwidth}
        \captionsetup[subfigure]{labelformat=empty,position=top,captionskip=-1pt,farskip=-0.5pt}
        \subfloat[Prob. boy (\%)]{\includegraphics[width=\textwidth]{spell2_g2_low_urban_b_pc}}\\
        \subfloat[Prob. no birth yet]{\includegraphics[width=\textwidth]{spell2_g2_low_urban_b_s}}
        \captionsetup[subfigure]{labelformat=parens}
    \end{minipage}
}
\setcounter{subfigure}{3}
\subfloat[1995-2006 (N=930)]{
    \begin{minipage}{0.26\textwidth}
        \captionsetup[subfigure]{labelformat=empty,position=top,captionskip=-1pt,farskip=-0.5pt}
        \subfloat[Prob. boy (\%)]{\includegraphics[width=\textwidth]{spell2_g3_low_urban_b_pc}}\\
        \subfloat[Prob. no birth yet]{\includegraphics[width=\textwidth]{spell2_g3_low_urban_b_s}}
        \captionsetup[subfigure]{labelformat=parens}
    \end{minipage}
}
\caption{Predicted probability of having a boy and probability of
no birth yet from nine months after first birth for urban 
women with no education by month beginning 9 months after prior birth. 
Predictions based on age 18 at first birth.
Left column shows results prior to sex selection available, middle column before
sex selection illegal and right column after sex selection illegal.
N indicates the number of women in the relevant group in the underlying samples.
}
\label{fig:results_spell2_low_urban}
\end{figure}


% SPELL 2 - rural - LOW
\begin{figure}
\centering
\caption*{First child a girl}
\setcounter{subfigure}{-2}
\subfloat[1972-1984 (N=8,426)]{
    \begin{minipage}{0.31\textwidth}
        \captionsetup[subfigure]{labelformat=empty,position=top,captionskip=-1pt,farskip=-0.5pt}
        \subfloat[Prob.\ boy (\%)]{\includegraphics[width=\textwidth]{spell2_g1_low_rural_g_pc}}\\
        \subfloat[Prob.\ no birth yet]{\includegraphics[width=\textwidth]{spell2_g1_low_rural_g_s}} 
        \captionsetup[subfigure]{labelformat=parens}
    \end{minipage}
} 
\setcounter{subfigure}{-1}
\subfloat[1985-1994 (N=11,481)]{
    \begin{minipage}{0.31\textwidth}
        \captionsetup[subfigure]{labelformat=empty,position=top,captionskip=-1pt,farskip=-0.5pt}
        \subfloat[Prob.\ boy (\%)]{\includegraphics[width=\textwidth]{spell2_g2_low_rural_g_pc}}\\
        \subfloat[Prob.\ no birth yet]{\includegraphics[width=\textwidth]{spell2_g2_low_rural_g_s}} 
        \captionsetup[subfigure]{labelformat=parens}
    \end{minipage}
} 
\setcounter{subfigure}{0}
\subfloat[1995-2006 (N=3,686)]{
    \begin{minipage}{0.31\textwidth}
        \captionsetup[subfigure]{labelformat=empty,position=top,captionskip=-1pt,farskip=-0.5pt}
        \subfloat[Prob.\ boy (\%)]{\includegraphics[width=\textwidth]{spell2_g3_low_rural_g_pc}}\\
        \subfloat[Prob.\ no birth yet]{\includegraphics[width=\textwidth]{spell2_g3_low_rural_g_s}} 
        \captionsetup[subfigure]{labelformat=parens}
    \end{minipage}
} 
\caption*{First child a boy}
\setcounter{subfigure}{1}
\subfloat[1972-1984 (N=9,395)]{
    \begin{minipage}{0.31\textwidth}
        \captionsetup[subfigure]{labelformat=empty,position=top,captionskip=-1pt,farskip=-0.5pt}
        \subfloat[Prob. boy (\%)]{\includegraphics[width=\textwidth]{spell2_g1_low_rural_b_pc}}\\
        \subfloat[Prob. no birth yet]{\includegraphics[width=\textwidth]{spell2_g1_low_rural_b_s}} 
        \captionsetup[subfigure]{labelformat=parens}
    \end{minipage}
} 
\setcounter{subfigure}{2}
\subfloat[1985-1994 (N=12,127)]{
    \begin{minipage}{0.31\textwidth}
        \captionsetup[subfigure]{labelformat=empty,position=top,captionskip=-1pt,farskip=-0.5pt}
        \subfloat[Prob. boy (\%)]{\includegraphics[width=\textwidth]{spell2_g2_low_rural_b_pc}}\\
        \subfloat[Prob. no birth yet]{\includegraphics[width=\textwidth]{spell2_g2_low_rural_b_s}}
        \captionsetup[subfigure]{labelformat=parens}
    \end{minipage}
}
\setcounter{subfigure}{3}
\subfloat[1995-2006 (N=3,860)]{
    \begin{minipage}{0.31\textwidth}
        \captionsetup[subfigure]{labelformat=empty,position=top,captionskip=-1pt,farskip=-0.5pt}
        \subfloat[Prob. boy (\%)]{\includegraphics[width=\textwidth]{spell2_g3_low_rural_b_pc}}\\
        \subfloat[Prob. no birth yet]{\includegraphics[width=\textwidth]{spell2_g3_low_rural_b_s}}
        \captionsetup[subfigure]{labelformat=parens}
    \end{minipage}
}
\caption{Predicted probability of having a boy and probability of
no birth yet from nine months after first birth for rural 
women with no education by month beginning 9 months after prior birth. 
Predictions based on age 18 at first birth.
Left column shows results prior to sex selection available, middle column before
sex selection illegal and right column after sex selection illegal.
N indicates the number of women in the relevant group in the underlying samples.
}
\label{fig:results_spell2_low_rural}
\end{figure}







% \begin{figure}[htpb]
% \centering
% \caption*{First child a girl}
% \setcounter{subfigure}{-2}
% \subfloat[1972--1984 (N=2,689)]{
%     \begin{minipage}{0.31\textwidth}
%         \captionsetup[subfigure]{labelformat=empty,position=top,captionskip=-1pt,farskip=-0.5pt}
%         \subfloat[Prob.\ boy (\%)]{\includegraphics[width=\textwidth]{spell2_g1_high_urban_g_pc}}\\
%         \subfloat[Prob.\ no birth yet]{\includegraphics[width=\textwidth]{spell2_g1_high_urban_g_s}} 
%         \captionsetup[subfigure]{labelformat=parens}
%     \end{minipage}
% } 
% \setcounter{subfigure}{-1}
% \subfloat[1985--1994 (N=4,869)]{
%     \begin{minipage}{0.31\textwidth}
%         \captionsetup[subfigure]{labelformat=empty,position=top,captionskip=-1pt,farskip=-0.5pt}
%         \subfloat[Prob. boy (\%)]{\includegraphics[width=\textwidth]{spell2_g2_high_urban_g_pc}}\\
%         \subfloat[Prob. no birth yet]{\includegraphics[width=\textwidth]{spell2_g2_high_urban_g_s}}
%         \captionsetup[subfigure]{labelformat=parens}
%     \end{minipage}
% }
% \setcounter{subfigure}{0}
% \subfloat[1995--2006 (N=3,774)]{
%     \begin{minipage}{0.31\textwidth}
%         \captionsetup[subfigure]{labelformat=empty,position=top,captionskip=-1pt,farskip=-0.5pt}
%         \subfloat[Prob. boy (\%)]{\includegraphics[width=\textwidth]{spell2_g3_high_urban_g_pc}}\\
%         \subfloat[Prob. no birth yet]{\includegraphics[width=\textwidth]{spell2_g3_high_urban_g_s}}
%         \captionsetup[subfigure]{labelformat=parens}
%     \end{minipage}
% }
% \caption*{First child a boy}
% \setcounter{subfigure}{1}
% \subfloat[1972--1984 (N=2,806)]{
%     \begin{minipage}{0.31\textwidth}
%         \captionsetup[subfigure]{labelformat=empty,position=top,captionskip=-1pt,farskip=-0.5pt}
%         \subfloat[Prob. boy (\%)]{\includegraphics[width=\textwidth]{spell2_g1_high_urban_b_pc}}\\
%         \subfloat[Prob. no birth yet]{\includegraphics[width=\textwidth]{spell2_g1_high_urban_b_s}} 
%         \captionsetup[subfigure]{labelformat=parens}
%     \end{minipage}
% } 
% \setcounter{subfigure}{2}
% \subfloat[1985--1994 (N=5,246)]{
%     \begin{minipage}{0.31\textwidth}
%         \captionsetup[subfigure]{labelformat=empty,position=top,captionskip=-1pt,farskip=-0.5pt}
%         \subfloat[Prob. boy (\%)]{\includegraphics[width=\textwidth]{spell2_g2_high_urban_b_pc}}\\
%         \subfloat[Prob. no birth yet]{\includegraphics[width=\textwidth]{spell2_g2_high_urban_b_s}}
%         \captionsetup[subfigure]{labelformat=parens}
%     \end{minipage}
% }
% \setcounter{subfigure}{3}
% \subfloat[1995--2006 (N=3,969)]{
%     \begin{minipage}{0.31\textwidth}
%         \captionsetup[subfigure]{labelformat=empty,position=top,captionskip=-1pt,farskip=-0.5pt}
%         \subfloat[Prob. boy (\%)]{\includegraphics[width=\textwidth]{spell2_g3_high_urban_b_pc}}\\
%         \subfloat[Prob. no birth yet]{\includegraphics[width=\textwidth]{spell2_g3_high_urban_b_s}}
%         \captionsetup[subfigure]{labelformat=parens}
%     \end{minipage}
% }
% \caption{Predicted probability of having a boy and probability of
% no birth yet from nine months after first birth for urban 
% women with 8 or more years of education by month beginning 9 months after prior birth. 
% Predictions based on age 22 at first birth.
% Left column shows results prior to sex selection available, middle column before
% sex selection illegal and right column after sex selection illegal.
% N indicates the number of women in the relevant group in the underlying samples.
% }
% \label{fig:results_spell2_high_urban}
% \end{figure}


\begin{figure}[htpb]
\centering
\caption*{First child a girl}
\setcounter{subfigure}{-2}
\subfloat[1972--1984 (N=1,305)]{
    \begin{minipage}{0.31\textwidth}
        \captionsetup[subfigure]{labelformat=empty,position=top,captionskip=-1pt,farskip=-0.5pt}
        \subfloat[Prob.\ boy (\%)]{\includegraphics[width=\textwidth]{spell2_g1_high_rural_g_pc}}\\
        \subfloat[Prob.\ no birth yet]{\includegraphics[width=\textwidth]{spell2_g1_high_rural_g_s}} 
        \captionsetup[subfigure]{labelformat=parens}
    \end{minipage}
} 
\setcounter{subfigure}{-1}
\subfloat[1985--1994 (N=3,105)]{
    \begin{minipage}{0.31\textwidth}
        \captionsetup[subfigure]{labelformat=empty,position=top,captionskip=-1pt,farskip=-0.5pt}
        \subfloat[Prob. boy (\%)]{\includegraphics[width=\textwidth]{spell2_g2_high_rural_g_pc}}\\
        \subfloat[Prob. no birth yet]{\includegraphics[width=\textwidth]{spell2_g2_high_rural_g_s}}
        \captionsetup[subfigure]{labelformat=parens}
    \end{minipage}
}
\setcounter{subfigure}{0}
\subfloat[1995--2006 (N=2,823)]{
    \begin{minipage}{0.31\textwidth}
        \captionsetup[subfigure]{labelformat=empty,position=top,captionskip=-1pt,farskip=-0.5pt}
        \subfloat[Prob. boy (\%)]{\includegraphics[width=\textwidth]{spell2_g3_high_rural_g_pc}}\\
        \subfloat[Prob. no birth yet]{\includegraphics[width=\textwidth]{spell2_g3_high_rural_g_s}}
        \captionsetup[subfigure]{labelformat=parens}
    \end{minipage}
}
\caption*{First child a boy}
\setcounter{subfigure}{1}
\subfloat[1972--1984 (N=1,404)]{
    \begin{minipage}{0.31\textwidth}
        \captionsetup[subfigure]{labelformat=empty,position=top,captionskip=-1pt,farskip=-0.5pt}
        \subfloat[Prob. boy (\%)]{\includegraphics[width=\textwidth]{spell2_g1_high_rural_b_pc}}\\
        \subfloat[Prob. no birth yet]{\includegraphics[width=\textwidth]{spell2_g1_high_rural_b_s}} 
        \captionsetup[subfigure]{labelformat=parens}
    \end{minipage}
} 
\setcounter{subfigure}{2}
\subfloat[1985--1994 (N=3,381)]{
    \begin{minipage}{0.31\textwidth}
        \captionsetup[subfigure]{labelformat=empty,position=top,captionskip=-1pt,farskip=-0.5pt}
        \subfloat[Prob. boy (\%)]{\includegraphics[width=\textwidth]{spell2_g2_high_rural_b_pc}}\\
        \subfloat[Prob. no birth yet]{\includegraphics[width=\textwidth]{spell2_g2_high_rural_b_s}}
        \captionsetup[subfigure]{labelformat=parens}
    \end{minipage}
}
\setcounter{subfigure}{3}
\subfloat[1995--2006 (N=3,044)]{
    \begin{minipage}{0.31\textwidth}
        \captionsetup[subfigure]{labelformat=empty,position=top,captionskip=-1pt,farskip=-0.5pt}
        \subfloat[Prob. boy (\%)]{\includegraphics[width=\textwidth]{spell2_g3_high_rural_b_pc}}\\
        \subfloat[Prob. no birth yet]{\includegraphics[width=\textwidth]{spell2_g3_high_rural_b_s}}
        \captionsetup[subfigure]{labelformat=parens}
    \end{minipage}
}
\caption{Predicted probability of having a boy and probability of
no birth yet from nine months after first birth for rural
women with 8 or more years of education by month beginning 9 months after prior birth. 
Predictions based on age 22 at first birth.
Left column shows results prior to sex selection available, middle column before
sex selection illegal and right column after sex selection illegal.
N indicates the number of women in the relevant group in the underlying samples.
}
\label{fig:results_spell2_high_rural}
\end{figure}




\clearpage

\subsection{Third Spell}


% low education

\begin{figure}[htpb]
\centering
\caption*{First two children girls}
\setcounter{subfigure}{-2}
\subfloat[1972--1984 (N=493)]{
    \begin{minipage}{0.26\textwidth}
        \captionsetup[subfigure]{labelformat=empty,position=top,captionskip=-1pt,farskip=-0.5pt}
        \subfloat[Prob.\ boy (\%)]{\includegraphics[width=\textwidth]{spell3_g1_low_urban_gg_pc}}\\
        \subfloat[Prob.\ no birth yet]{\includegraphics[width=\textwidth]{spell3_g1_low_urban_gg_s}} 
        \captionsetup[subfigure]{labelformat=parens}
    \end{minipage}
} 
\setcounter{subfigure}{-1}
\subfloat[1985--1994 (N=996)]{
    \begin{minipage}{0.26\textwidth}
        \captionsetup[subfigure]{labelformat=empty,position=top,captionskip=-1pt,farskip=-0.5pt}
        \subfloat[Prob. boy (\%)]{\includegraphics[width=\textwidth]{spell3_g2_low_urban_gg_pc}}\\
        \subfloat[Prob. no birth yet]{\includegraphics[width=\textwidth]{spell3_g2_low_urban_gg_s}}
        \captionsetup[subfigure]{labelformat=parens}
    \end{minipage}
}
\setcounter{subfigure}{0}
\subfloat[1995--2006 (N=475)]{
    \begin{minipage}{0.26\textwidth}
        \captionsetup[subfigure]{labelformat=empty,position=top,captionskip=-1pt,farskip=-0.5pt}
        \subfloat[Prob. boy (\%)]{\includegraphics[width=\textwidth]{spell3_g3_low_urban_gg_pc}}\\
        \subfloat[Prob. no birth yet]{\includegraphics[width=\textwidth]{spell3_g3_low_urban_gg_s}}
        \captionsetup[subfigure]{labelformat=parens}
    \end{minipage}
}
\caption*{First two children one boy and one girl}
\setcounter{subfigure}{1}
\subfloat[1972--1984 (N=1,083)]{
    \begin{minipage}{0.26\textwidth}
        \captionsetup[subfigure]{labelformat=empty,position=top,captionskip=-1pt,farskip=-0.5pt}
        \subfloat[Prob. boy (\%)]{\includegraphics[width=\textwidth]{spell3_g1_low_urban_bg_pc}}\\
        \subfloat[Prob. no birth yet]{\includegraphics[width=\textwidth]{spell3_g1_low_urban_bg_s}} 
        \captionsetup[subfigure]{labelformat=parens}
    \end{minipage}
} 
\setcounter{subfigure}{2}
\subfloat[1985--1994 (N=2,141)]{
    \begin{minipage}{0.26\textwidth}
        \captionsetup[subfigure]{labelformat=empty,position=top,captionskip=-1pt,farskip=-0.5pt}
        \subfloat[Prob. boy (\%)]{\includegraphics[width=\textwidth]{spell3_g2_low_urban_bg_pc}}\\
        \subfloat[Prob. no birth yet]{\includegraphics[width=\textwidth]{spell3_g2_low_urban_bg_s}}
        \captionsetup[subfigure]{labelformat=parens}
    \end{minipage}
}
\setcounter{subfigure}{3}
\subfloat[1995--2006 (N=950)]{
    \begin{minipage}{0.26\textwidth}
        \captionsetup[subfigure]{labelformat=empty,position=top,captionskip=-1pt,farskip=-0.5pt}
        \subfloat[Prob. boy (\%)]{\includegraphics[width=\textwidth]{spell3_g3_low_urban_bg_pc}}\\
        \subfloat[Prob. no birth yet]{\includegraphics[width=\textwidth]{spell3_g3_low_urban_bg_s}}
        \captionsetup[subfigure]{labelformat=parens}
    \end{minipage}
}
\caption{Predicted probability of having a boy and probability of
no birth yet from nine months after second birth for urban 
women with no of education by month beginning 9 months after prior birth. 
Predictions based on age 21 at second birth.
Left column shows results prior to sex selection available, middle column before
sex selection illegal and right column after sex selection illegal.
N indicates the number of women in the relevant group in the underlying samples.
}
\label{fig:results_spell3_low_urban}
\end{figure}


\begin{figure}[htpb]
\centering
\ContinuedFloat
\caption*{First two children boys}
\setcounter{subfigure}{4}
\subfloat[1972--1984 (N=569)]{
    \begin{minipage}{0.31\textwidth}
        \captionsetup[subfigure]{labelformat=empty,position=top,captionskip=-1pt,farskip=-0.5pt}
        \subfloat[Prob. boy (\%)]{\includegraphics[width=\textwidth]{spell3_g1_low_urban_bb_pc}}\\
        \subfloat[Prob. no birth yet]{\includegraphics[width=\textwidth]{spell3_g1_low_urban_bb_s}} 
        \captionsetup[subfigure]{labelformat=parens}
    \end{minipage}
} 
\setcounter{subfigure}{5}
\subfloat[1985--1994 (N=1,124)]{
    \begin{minipage}{0.31\textwidth}
        \captionsetup[subfigure]{labelformat=empty,position=top,captionskip=-1pt,farskip=-0.5pt}
        \subfloat[Prob. boy (\%)]{\includegraphics[width=\textwidth]{spell3_g2_low_urban_bb_pc}}\\
        \subfloat[Prob. no birth yet]{\includegraphics[width=\textwidth]{spell3_g2_low_urban_bb_s}}
        \captionsetup[subfigure]{labelformat=parens}
    \end{minipage}
}
\setcounter{subfigure}{6}
\subfloat[1995--2006 (N=426)]{
    \begin{minipage}{0.31\textwidth}
        \captionsetup[subfigure]{labelformat=empty,position=top,captionskip=-1pt,farskip=-0.5pt}
        \subfloat[Prob. boy (\%)]{\includegraphics[width=\textwidth]{spell3_g3_low_urban_bb_pc}}\\
        \subfloat[Prob. no birth yet]{\includegraphics[width=\textwidth]{spell3_g3_low_urban_bb_s}}
        \captionsetup[subfigure]{labelformat=parens}
    \end{minipage}
}
% \label{fig:results_spell3_low_urban}
\caption{(Continued) Predicted probability of having a boy and probability of
no birth yet from nine months after second birth for urban 
women with no of education by month beginning 9 months after prior birth. 
Predictions based on age 21 at second birth.
Left column shows results prior to sex selection available, middle column before
sex selection illegal and right column after sex selection illegal.
N indicates the number of women in the relevant group in the underlying samples.
}
\end{figure}


\begin{figure}[htpb]
\centering
\caption*{First two children girls}
\setcounter{subfigure}{-2}
\subfloat[1972--1984 (N=2,599)]{
    \begin{minipage}{0.31\textwidth}
        \captionsetup[subfigure]{labelformat=empty,position=top,captionskip=-1pt,farskip=-0.5pt}
        \subfloat[Prob.\ boy (\%)]{\includegraphics[width=\textwidth]{spell3_g1_low_rural_gg_pc}}\\
        \subfloat[Prob.\ no birth yet]{\includegraphics[width=\textwidth]{spell3_g1_low_rural_gg_s}} 
        \captionsetup[subfigure]{labelformat=parens}
    \end{minipage}
} 
\setcounter{subfigure}{-1}
\subfloat[1985--1994 (N=5,062)]{
    \begin{minipage}{0.31\textwidth}
        \captionsetup[subfigure]{labelformat=empty,position=top,captionskip=-1pt,farskip=-0.5pt}
        \subfloat[Prob. boy (\%)]{\includegraphics[width=\textwidth]{spell3_g2_low_rural_gg_pc}}\\
        \subfloat[Prob. no birth yet]{\includegraphics[width=\textwidth]{spell3_g2_low_rural_gg_s}}
        \captionsetup[subfigure]{labelformat=parens}
    \end{minipage}
}
\setcounter{subfigure}{0}
\subfloat[1995--2006 (N=1,879)]{
    \begin{minipage}{0.31\textwidth}
        \captionsetup[subfigure]{labelformat=empty,position=top,captionskip=-1pt,farskip=-0.5pt}
        \subfloat[Prob. boy (\%)]{\includegraphics[width=\textwidth]{spell3_g3_low_rural_gg_pc}}\\
        \subfloat[Prob. no birth yet]{\includegraphics[width=\textwidth]{spell3_g3_low_rural_gg_s}}
        \captionsetup[subfigure]{labelformat=parens}
    \end{minipage}
}
\caption*{First two children one boy and one girl}
\setcounter{subfigure}{1}
\subfloat[1972--1984 (N=5,382)]{
    \begin{minipage}{0.31\textwidth}
        \captionsetup[subfigure]{labelformat=empty,position=top,captionskip=-1pt,farskip=-0.5pt}
        \subfloat[Prob. boy (\%)]{\includegraphics[width=\textwidth]{spell3_g1_low_rural_bg_pc}}\\
        \subfloat[Prob. no birth yet]{\includegraphics[width=\textwidth]{spell3_g1_low_rural_bg_s}} 
        \captionsetup[subfigure]{labelformat=parens}
    \end{minipage}
} 
\setcounter{subfigure}{2}
\subfloat[1985--1994 (N=10,473)]{
    \begin{minipage}{0.31\textwidth}
        \captionsetup[subfigure]{labelformat=empty,position=top,captionskip=-1pt,farskip=-0.5pt}
        \subfloat[Prob. boy (\%)]{\includegraphics[width=\textwidth]{spell3_g2_low_rural_bg_pc}}\\
        \subfloat[Prob. no birth yet]{\includegraphics[width=\textwidth]{spell3_g2_low_rural_bg_s}}
        \captionsetup[subfigure]{labelformat=parens}
    \end{minipage}
}
\setcounter{subfigure}{3}
\subfloat[1995--2006 (N=3,779)]{
    \begin{minipage}{0.31\textwidth}
        \captionsetup[subfigure]{labelformat=empty,position=top,captionskip=-1pt,farskip=-0.5pt}
        \subfloat[Prob. boy (\%)]{\includegraphics[width=\textwidth]{spell3_g3_low_rural_bg_pc}}\\
        \subfloat[Prob. no birth yet]{\includegraphics[width=\textwidth]{spell3_g3_low_rural_bg_s}}
        \captionsetup[subfigure]{labelformat=parens}
    \end{minipage}
}
\caption{Predicted probability of having a boy and probability of
no birth yet from nine months after second birth for rural
women with no of education by month beginning 9 months after prior birth. 
Predictions based on age 21 at second birth.
Left column shows results prior to sex selection available, middle column before
sex selection illegal and right column after sex selection illegal.
N indicates the number of women in the relevant group in the underlying samples.
}
\label{fig:results_spell3_low_rural}
\end{figure}


\begin{figure}[htpb]
\ContinuedFloat
\centering
\caption*{First two children boys}
\setcounter{subfigure}{4}
\subfloat[1972--1984 (N=3,070)]{
    \begin{minipage}{0.31\textwidth}
        \captionsetup[subfigure]{labelformat=empty,position=top,captionskip=-1pt,farskip=-0.5pt}
        \subfloat[Prob. boy (\%)]{\includegraphics[width=\textwidth]{spell3_g1_low_rural_bb_pc}}\\
        \subfloat[Prob. no birth yet]{\includegraphics[width=\textwidth]{spell3_g1_low_rural_bb_s}} 
        \captionsetup[subfigure]{labelformat=parens}
    \end{minipage}
} 
\setcounter{subfigure}{5}
\subfloat[1985--1994 (N=5,339)]{
    \begin{minipage}{0.31\textwidth}
        \captionsetup[subfigure]{labelformat=empty,position=top,captionskip=-1pt,farskip=-0.5pt}
        \subfloat[Prob. boy (\%)]{\includegraphics[width=\textwidth]{spell3_g2_low_rural_bb_pc}}\\
        \subfloat[Prob. no birth yet]{\includegraphics[width=\textwidth]{spell3_g2_low_rural_bb_s}}
        \captionsetup[subfigure]{labelformat=parens}
    \end{minipage}
}
\setcounter{subfigure}{6}
\subfloat[1995--2006 (N=1,846)]{
    \begin{minipage}{0.31\textwidth}
        \captionsetup[subfigure]{labelformat=empty,position=top,captionskip=-1pt,farskip=-0.5pt}
        \subfloat[Prob. boy (\%)]{\includegraphics[width=\textwidth]{spell3_g3_low_rural_bb_pc}}\\
        \subfloat[Prob. no birth yet]{\includegraphics[width=\textwidth]{spell3_g3_low_rural_bb_s}}
        \captionsetup[subfigure]{labelformat=parens}
    \end{minipage}
}
\caption{(Continued) Predicted probability of having a boy and probability of
no birth yet from nine months after second birth for rural
women with no of education by month beginning 9 months after prior birth. 
Predictions based on age 21 at second birth.
Left column shows results prior to sex selection available, middle column before
sex selection illegal and right column after sex selection illegal.
N indicates the number of women in the relevant group in the underlying samples.
}
% \label{fig:results_spell3_low_rural}
\end{figure}



% High education

\begin{figure}[htpb]
\centering
\caption*{First two children girls}
\setcounter{subfigure}{-2}
\subfloat[1972--1984 (N=709)]{
    \begin{minipage}{0.31\textwidth}
        \captionsetup[subfigure]{labelformat=empty,position=top,captionskip=-1pt,farskip=-0.5pt}
        \subfloat[Prob.\ boy (\%)]{\includegraphics[width=\textwidth]{spell3_g1_high_urban_gg_pc}}\\
        \subfloat[Prob.\ no birth yet]{\includegraphics[width=\textwidth]{spell3_g1_high_urban_gg_s}} 
        \captionsetup[subfigure]{labelformat=parens}
    \end{minipage}
} 
\setcounter{subfigure}{-1}
\subfloat[1985--1994 (N=1,654)]{
    \begin{minipage}{0.31\textwidth}
        \captionsetup[subfigure]{labelformat=empty,position=top,captionskip=-1pt,farskip=-0.5pt}
        \subfloat[Prob. boy (\%)]{\includegraphics[width=\textwidth]{spell3_g2_high_urban_gg_pc}}\\
        \subfloat[Prob. no birth yet]{\includegraphics[width=\textwidth]{spell3_g2_high_urban_gg_s}}
        \captionsetup[subfigure]{labelformat=parens}
    \end{minipage}
}
\setcounter{subfigure}{0}
\subfloat[1995--2006 (N=1,000)]{
    \begin{minipage}{0.31\textwidth}
        \captionsetup[subfigure]{labelformat=empty,position=top,captionskip=-1pt,farskip=-0.5pt}
        \subfloat[Prob. boy (\%)]{\includegraphics[width=\textwidth]{spell3_g3_high_urban_gg_pc}}\\
        \subfloat[Prob. no birth yet]{\includegraphics[width=\textwidth]{spell3_g3_high_urban_gg_s}}
        \captionsetup[subfigure]{labelformat=parens}
    \end{minipage}
}
\caption*{First two children one boy and one girl}
\setcounter{subfigure}{1}
\subfloat[1972--1984 (N=1,409)]{
    \begin{minipage}{0.31\textwidth}
        \captionsetup[subfigure]{labelformat=empty,position=top,captionskip=-1pt,farskip=-0.5pt}
        \subfloat[Prob. boy (\%)]{\includegraphics[width=\textwidth]{spell3_g1_high_urban_bg_pc}}\\
        \subfloat[Prob. no birth yet]{\includegraphics[width=\textwidth]{spell3_g1_high_urban_bg_s}} 
        \captionsetup[subfigure]{labelformat=parens}
    \end{minipage}
} 
\setcounter{subfigure}{2}
\subfloat[1985--1994 (N=3,460)]{
    \begin{minipage}{0.31\textwidth}
        \captionsetup[subfigure]{labelformat=empty,position=top,captionskip=-1pt,farskip=-0.5pt}
        \subfloat[Prob. boy (\%)]{\includegraphics[width=\textwidth]{spell3_g2_high_urban_bg_pc}}\\
        \subfloat[Prob. no birth yet]{\includegraphics[width=\textwidth]{spell3_g2_high_urban_bg_s}}
        \captionsetup[subfigure]{labelformat=parens}
    \end{minipage}
}
\setcounter{subfigure}{3}
\subfloat[1995--2006 (N=2,357)]{
    \begin{minipage}{0.31\textwidth}
        \captionsetup[subfigure]{labelformat=empty,position=top,captionskip=-1pt,farskip=-0.5pt}
        \subfloat[Prob. boy (\%)]{\includegraphics[width=\textwidth]{spell3_g3_high_urban_bg_pc}}\\
        \subfloat[Prob. no birth yet]{\includegraphics[width=\textwidth]{spell3_g3_high_urban_bg_s}}
        \captionsetup[subfigure]{labelformat=parens}
    \end{minipage}
}
\caption{Predicted probability of having a boy and probability of
no birth yet from nine months after second birth for urban 
women with 8 or more years of education by month beginning 9 months after prior birth. 
Predictions based on age 24 at second birth.
Left column shows results prior to sex selection available, middle column before
sex selection illegal and right column after sex selection illegal.
N indicates the number of women in the relevant group in the underlying samples.
}
\label{fig:results_spell3_high_urban}
\end{figure}


\begin{figure}[htpb]
\centering
\ContinuedFloat
\caption*{First two children boys}
\setcounter{subfigure}{4}
\subfloat[1972--1984 (N=780)]{
    \begin{minipage}{0.31\textwidth}
        \captionsetup[subfigure]{labelformat=empty,position=top,captionskip=-1pt,farskip=-0.5pt}
        \subfloat[Prob. boy (\%)]{\includegraphics[width=\textwidth]{spell3_g1_high_urban_bb_pc}}\\
        \subfloat[Prob. no birth yet]{\includegraphics[width=\textwidth]{spell3_g1_high_urban_bb_s}} 
        \captionsetup[subfigure]{labelformat=parens}
    \end{minipage}
} 
\setcounter{subfigure}{5}
\subfloat[1985--1994 (N=1,660)]{
    \begin{minipage}{0.31\textwidth}
        \captionsetup[subfigure]{labelformat=empty,position=top,captionskip=-1pt,farskip=-0.5pt}
        \subfloat[Prob. boy (\%)]{\includegraphics[width=\textwidth]{spell3_g2_high_urban_bb_pc}}\\
        \subfloat[Prob. no birth yet]{\includegraphics[width=\textwidth]{spell3_g2_high_urban_bb_s}}
        \captionsetup[subfigure]{labelformat=parens}
    \end{minipage}
}
\setcounter{subfigure}{6}
\subfloat[1995--2006 (N=1,098)]{
    \begin{minipage}{0.31\textwidth}
        \captionsetup[subfigure]{labelformat=empty,position=top,captionskip=-1pt,farskip=-0.5pt}
        \subfloat[Prob. boy (\%)]{\includegraphics[width=\textwidth]{spell3_g3_high_urban_bb_pc}}\\
        \subfloat[Prob. no birth yet]{\includegraphics[width=\textwidth]{spell3_g3_high_urban_bb_s}}
        \captionsetup[subfigure]{labelformat=parens}
    \end{minipage}
}
% \label{fig:results_spell3_high_urban}
\caption{(Continued) Predicted probability of having a boy and probability of
no birth yet from nine months after second birth for urban 
women with 8 or more years of education by month beginning 9 months after prior birth. 
Predictions based on age 24 at second birth.
Left column shows results prior to sex selection available, middle column before
sex selection illegal and right column after sex selection illegal.
N indicates the number of women in the relevant group in the underlying samples.
}
\end{figure}


\begin{figure}[htpb]
\centering
\caption*{First two children girls}
\setcounter{subfigure}{-2}
\subfloat[1972--1984 (N=348)]{
    \begin{minipage}{0.31\textwidth}
        \captionsetup[subfigure]{labelformat=empty,position=top,captionskip=-1pt,farskip=-0.5pt}
        \subfloat[Prob.\ boy (\%)]{\includegraphics[width=\textwidth]{spell3_g1_high_rural_gg_pc}}\\
        \subfloat[Prob.\ no birth yet]{\includegraphics[width=\textwidth]{spell3_g1_high_rural_gg_s}} 
        \captionsetup[subfigure]{labelformat=parens}
    \end{minipage}
} 
\setcounter{subfigure}{-1}
\subfloat[1985--1994 (N=982)]{
    \begin{minipage}{0.31\textwidth}
        \captionsetup[subfigure]{labelformat=empty,position=top,captionskip=-1pt,farskip=-0.5pt}
        \subfloat[Prob. boy (\%)]{\includegraphics[width=\textwidth]{spell3_g2_high_rural_gg_pc}}\\
        \subfloat[Prob. no birth yet]{\includegraphics[width=\textwidth]{spell3_g2_high_rural_gg_s}}
        \captionsetup[subfigure]{labelformat=parens}
    \end{minipage}
}
\setcounter{subfigure}{0}
\subfloat[1995--2006 (N=863)]{
    \begin{minipage}{0.31\textwidth}
        \captionsetup[subfigure]{labelformat=empty,position=top,captionskip=-1pt,farskip=-0.5pt}
        \subfloat[Prob. boy (\%)]{\includegraphics[width=\textwidth]{spell3_g3_high_rural_gg_pc}}\\
        \subfloat[Prob. no birth yet]{\includegraphics[width=\textwidth]{spell3_g3_high_rural_gg_s}}
        \captionsetup[subfigure]{labelformat=parens}
    \end{minipage}
}
\caption*{First two children one boy and one girl}
\setcounter{subfigure}{1}
\subfloat[1972--1984 (N=700)]{
    \begin{minipage}{0.31\textwidth}
        \captionsetup[subfigure]{labelformat=empty,position=top,captionskip=-1pt,farskip=-0.5pt}
        \subfloat[Prob. boy (\%)]{\includegraphics[width=\textwidth]{spell3_g1_high_rural_bg_pc}}\\
        \subfloat[Prob. no birth yet]{\includegraphics[width=\textwidth]{spell3_g1_high_rural_bg_s}} 
        \captionsetup[subfigure]{labelformat=parens}
    \end{minipage}
} 
\setcounter{subfigure}{2}
\subfloat[1985--1994 (N=1,987)]{
    \begin{minipage}{0.31\textwidth}
        \captionsetup[subfigure]{labelformat=empty,position=top,captionskip=-1pt,farskip=-0.5pt}
        \subfloat[Prob. boy (\%)]{\includegraphics[width=\textwidth]{spell3_g2_high_rural_bg_pc}}\\
        \subfloat[Prob. no birth yet]{\includegraphics[width=\textwidth]{spell3_g2_high_rural_bg_s}}
        \captionsetup[subfigure]{labelformat=parens}
    \end{minipage}
}
\setcounter{subfigure}{3}
\subfloat[1995--2006 (N=1,819)]{
    \begin{minipage}{0.31\textwidth}
        \captionsetup[subfigure]{labelformat=empty,position=top,captionskip=-1pt,farskip=-0.5pt}
        \subfloat[Prob. boy (\%)]{\includegraphics[width=\textwidth]{spell3_g3_high_rural_bg_pc}}\\
        \subfloat[Prob. no birth yet]{\includegraphics[width=\textwidth]{spell3_g3_high_rural_bg_s}}
        \captionsetup[subfigure]{labelformat=parens}
    \end{minipage}
}
\caption{Predicted probability of having a boy and probability of
no birth yet from nine months after second birth for rural
women with 8 or more years of education by month beginning 9 months after prior birth. 
Predictions based on age 24 at second birth.
Left column shows results prior to sex selection available, middle column before
sex selection illegal and right column after sex selection illegal.
N indicates the number of women in the relevant group in the underlying samples.
}
\label{fig:results_spell3_high_rural}
\end{figure}


\begin{figure}[htpb]
\ContinuedFloat
\centering
\caption*{First two children boys}
\setcounter{subfigure}{4}
\subfloat[1972--1984 (N=370)]{
    \begin{minipage}{0.31\textwidth}
        \captionsetup[subfigure]{labelformat=empty,position=top,captionskip=-1pt,farskip=-0.5pt}
        \subfloat[Prob. boy (\%)]{\includegraphics[width=\textwidth]{spell3_g1_high_rural_bb_pc}}\\
        \subfloat[Prob. no birth yet]{\includegraphics[width=\textwidth]{spell3_g1_high_rural_bb_s}} 
        \captionsetup[subfigure]{labelformat=parens}
    \end{minipage}
} 
\setcounter{subfigure}{5}
\subfloat[1985--1994 (N=945)]{
    \begin{minipage}{0.31\textwidth}
        \captionsetup[subfigure]{labelformat=empty,position=top,captionskip=-1pt,farskip=-0.5pt}
        \subfloat[Prob. boy (\%)]{\includegraphics[width=\textwidth]{spell3_g2_high_rural_bb_pc}}\\
        \subfloat[Prob. no birth yet]{\includegraphics[width=\textwidth]{spell3_g2_high_rural_bb_s}}
        \captionsetup[subfigure]{labelformat=parens}
    \end{minipage}
}
\setcounter{subfigure}{6}
\subfloat[1995--2006 (N=790)]{
    \begin{minipage}{0.31\textwidth}
        \captionsetup[subfigure]{labelformat=empty,position=top,captionskip=-1pt,farskip=-0.5pt}
        \subfloat[Prob. boy (\%)]{\includegraphics[width=\textwidth]{spell3_g3_high_rural_bb_pc}}\\
        \subfloat[Prob. no birth yet]{\includegraphics[width=\textwidth]{spell3_g3_high_rural_bb_s}}
        \captionsetup[subfigure]{labelformat=parens}
    \end{minipage}
}
\caption{(Continued) Predicted probability of having a boy and probability of
no birth yet from nine months after second birth for rural
women with 8 or more years of education by month beginning 9 months after prior birth. 
Predictions based on age 24 at second birth.
Left column shows results prior to sex selection available, middle column before
sex selection illegal and right column after sex selection illegal.
N indicates the number of women in the relevant group in the underlying samples.
}
% \label{fig:results_spell3_high_rural}
\end{figure}

\clearpage
\newpage



\end{document}



