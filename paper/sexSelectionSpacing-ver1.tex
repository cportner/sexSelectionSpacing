% Revamped version for Demography focusing on method

\documentclass[12pt,letterpaper]{article}

\usepackage{fontspec}
\setromanfont[Ligatures=TeX]{TeX Gyre Pagella}
\usepackage{unicode-math}
\setmathfont{TeX Gyre Pagella Math}
% \usepackage{mathpazo}
\usepackage[title]{appendix}
\usepackage[margin=1.0in]{geometry}
\usepackage[figuresleft]{rotating}
\usepackage[longnamesfirst]{natbib}
\usepackage{dcolumn}
\usepackage{booktabs}
\usepackage{multirow}
\usepackage[flushleft]{threeparttable}
\usepackage{setspace}
% \usepackage{xyling}
\usepackage[justification=centering]{caption}
\usepackage[font=scriptsize]{subfig}
\usepackage[xetex,colorlinks=true,linkcolor=black,citecolor=black,urlcolor=black]{hyperref}
\usepackage{adjustbox}
\usepackage{xfrac}


% \bibpunct{(}{)}{;}{a}{}{,}
\newcommand{\mco}[1]{\multicolumn{1}{c}{#1}}
\newcommand{\mct}[1]{\multicolumn{2}{c}{#1}}
\newcommand{\X}{$\times$ }
\newcommand{\hs}{\hspace{15pt}}

% Attempt to squeeze more floats in
\renewcommand\floatpagefraction{.9}
\renewcommand\topfraction{.9}
\renewcommand\bottomfraction{.9}
\renewcommand\textfraction{.1}
\setcounter{totalnumber}{50}
\setcounter{topnumber}{50}
\setcounter{bottomnumber}{50}


%------------------------------------------------------------------------


\title{Son Preference, Birth Spacing, and Sex-Selective Abortions%
\protect\thanks{%
I am grateful to Andrew Foster and Darryl Holman for discussions about the method.
I owe thanks to Shelly Lundberg, Daniel Rees, David Ribar, 
Hendrik Wolff, seminar participants at University of Copenhagen, University of Michigan, 
University of Washington, University of Aarhus, the Fourth 
Annual Conference on Population, Reproductive Health, 
and Economic Development, and the Economic Demography Workshop for helpful suggestions and comments.
I would also like to thank Nalina Varanasi for research assistance.
Support from the University of Washington Royalty Research Fund and the 
Development Research Group of the World Bank is gratefully acknowledged.
The views and findings expressed here are those of the author and
should not be attributed to the World Bank or any of its member countries.
Partial support for this research came from a Eunice Kennedy Shriver National
Institute of Child Health and Human Development research infrastructure grant,
5R24HD042828, to the Center for Studies in Demography and Ecology at the
University of Washington.
}
}

\author{Claus C P\"ortner\\
    Department of Economics\\
    Albers School of Business and Economics\\
    Seattle University, P.O. Box 222000\\
    Seattle, WA 98122\\
    \href{mailto:claus@clausportner.com}{\texttt{claus@clausportner.com}}\\
    \href{http://www.clausportner.com}{\texttt{www.clausportner.com}}\\
    \& \\
    Center for Studies in Demography and Ecology \\
    University of Washington\\ \vspace{2cm}
    }

\date{March 2018\\
\bigskip
Preliminary}


%------------------------------------------------------------------------


\begin{document}
\graphicspath{{../figures/}}
\DeclareGraphicsExtensions{.eps,.jpg,.pdf,.mps,.png}

\setcounter{page}{-1}
\maketitle
\thispagestyle{empty}

% \setcounter{page}{0}


\newpage
\thispagestyle{empty}
\doublespacing

\begin{abstract}

% Demography abstract
\noindent 

Strong son preference has frequently been associated with shorter 
spacing between births after the birth of a girl.
The spread of access to prenatal sex determination and 
sex-selective abortions has the potential to reverse this pattern
because each abortion adds substantially to the duration between births.
I introduce a statistical method that simultaneously account for how sex 
selection increases the likelihood of a son \emph{and} the spacing between births.
Using India's National Family and Health Surveys,
I show that use of sex selection leads to \emph{longer} spacing after a daughter 
than after a son; a reversal of the pattern before the availability of sex selection.
For families that do not use sex selection, the duration to next birth is still 
shorter when no boys are present.

Women with 8 or more years of education, both in urban and rural areas, are 
the main users of sex-selective abortions, whereas women with less education 
do not appear to use sex selection.


\noindent JEL: J1, O12, I1
\noindent Keywords: India, prenatal sex determination, censoring, competing risk
\end{abstract}

\newpage


%------------------------------------------------------------------------

\section{Introduction\label{sec:intro}}

Parents' spacing between births has long served as a measure of son preference 
\citep{Leung1988}.
Before prenatal sex determination became available, the only recourse for 
parents who wanted a son---but did not yet have one---was to have the next 
birth sooner.
Son preference is therefore often associated with shorter spacing after the 
births of girls than boys 
\citep{Das1987,Rahman1993,Pong1994,Haughton1996,Arnold1997,Soest2012}.
Shorter spacing is, in turn, associated with worse health outcomes for girls 
\citep{arnold98,Whitworth2002,Rutstein2005,Conde-Agudelo2006}.
\footnote{
Parents are also more likely to cease childbearing after the birth of 
a son than after a daughter 
\citep{ben-porath76b,Das1987,Arnold1997,clark00,filmer09,Altindag2016}.
}

The introduction of prenatal sex determination fundamentally changed the 
relationship between son preference and birth spacing.
Couples, who before had shorter birth spacing because of son preference,
now have access to prenatal sex determination and sex-selective abortions,
and each abortion increases birth spacing by approximately a year.
The increase consists of three parts.
First, starting from the time of the abortion, the uterus needs at 
least two menstrual cycles to recover;  otherwise, the likelihood 
of spontaneous abortion increases substantially \citep{zhou00b}.
The second part is the waiting time to conception, which I set to six 
months \citep{Wang2003}.%
\footnote{
The waiting time to conception varies by woman, but even if it is, say, one 
month, the additional space between births would still be six months per abortion.
}
Finally, sex determination tests are reliable only from 3 months 
of gestation onwards.

As a result, we now have a situation where families with the 
\emph{strongest} son preference may have \emph{longer} spacing after 
the birth of a daughter,
because of their use of sex-selective abortions.
To further complicate matters, we may still observe short spacing 
after the births of daughters as a representation of son preference for 
families who---for one reason or another---do not use prenatal sex selection.
Also, even if they have access to prenatal sex determination, couples 
with stronger son preference may still try to conceive faster after daughters, 
and may even shorten the time until they try to conceive, knowing that 
they may have to go through multiple pregnancies and sex-selective
abortions before they conceive a son.

Spacing, by itself, can therefore no longer be used as a direct measure of 
son preference.
However, understanding birth spacing remains a critical undertaking.
First, birth spacing is still useful in understanding son preference, 
if combined with the likelihood of observing a boy or a girl.
Second, if spacing affects health outcomes for mother and children, it is 
essential to understand what drives changes in spacing.
The better health outcomes for girls in the presence of sex selection may,
for example, be an unintended side-effect of the longer spacing that arises
from sex-selective abortions, rather than, as often assumed, be driven by a 
higher value placed on girls because of their lower number \citep{Lin2014,Hu2015}.
Third, the duration between births may be an important factor in parents' 
decisions, either for preference or economic reasons.
Whether duration plays a role in parents' decisions is especially important 
here, because even parents with strong son preference may reverse their 
decision to use prenatal sex determination, and carry the next pregnancy 
to term whether male or female, as the duration from the prior birth 
becomes sufficiently long.
Finally, we know less about what determines spacing behavior in 
developing countries than in developed countries.
With increasing numbers of women entering the labor force in developing
countries, understanding how timing decisions are made will be necessary
for the design of suitable policies \citep{Portner2018}.


In this paper, I introduce and apply a novel empirical method that directly incorporates 
the effects of sex-selective abortions on the likelihood of a son being born 
\emph{and} the duration between births.
The method can be used to analyze both situations with and without prenatal
sex selection.
Furthermore, my proposed method allows for the time since the previous 
birth to affect the decision on sex selection.
By examining whether sex selection decisions change with spacing, we can 
draw a more nuanced picture of the degree of son preference.

% [Why India?]
I apply the method to birth histories for Hindu women, using data from 
India's National Family and Health Surveys (NFHS), covering the period 
1972 to 2016. 
India is a particularly compelling case.
On the one hand, India has seen dramatic increases in the males-to-females ratio 
at birth over the last three decades as access to prenatal sex determination 
expanded 
\citep{das_gupta97,Sudha1999,Arnold2002,retherford03b,jha06}.%
\footnote{
India is not alone; both China and South Korea saw
significant changes in the sex ratio at birth over the same period 
\citep{Yi1993,park95}.
}
On the other hand, research suggests that son preference in 
India, when measured as ideally having more boys than girls, is decreasing 
over time and with higher education \citep{bhat03,pande07}.
% %
% \footnote{
% This measure of son preference is commonly used in the literature. 
% See, for example, \citet{clark00}, \citet{Jensen2009}, and \cite{Hu2015}.
% }
There have also been substantial changes in access and legality of
prenatal sex determination in India over the period covered.
Abortion has been legal in India since 1971 and still is, but the 
first reports of prenatal sex determination did
not appear until 1982--83 \citep{Sudha1999,bhat06,Grover2006}.
The number of clinics quickly increased, and knowledge about sex selection 
became widespread after a senior government official's wife aborted a 
fetus that turned out to be male \citep[p.\ 598]{Sudha1999}.
In 1994, the Central Government passed the Prenatal Diagnostic Techniques 
(PNDT) Act, making determining and communicating the sex of a fetus illegal.%
\footnote{
Details about the act are at \href{http://pndt.gov.in/}{http://pndt.gov.in/}.
The number of convictions has been low.
It took until January 2008 for the first state, Haryana, to reach five convictions.
Hence, private clinics apparently operate with little risk of legal action 
\citep{Sudha1999}.
Maharashtra was the first state to pass a similar law in 1988.
}
Hence, the data make it possible to show how the spacing between births and
sex ratios have changed with the introduction of prenatal sex determination,
and whether the ban affected the relationship between birth spacing and sex 
ratios.

[Outline of findings here]


% [EXPLAIN RESULTS HERE]
% Main findings!
% There are three main findings.
% 
% There is, however, still no evidence of sex selection being used on the first birth.
% 
% Secondly, sex selection appears to be used for securing one son, rather than a large 
% number of sons.
% There is only limited use of sex-selective abortions for better-educated
% women with one or more sons.
% The exception is rural women with one son and one daughter, presumably to compensate for 
% the higher child mortality in rural areas.
% These results are in line with the differential stopping behavior observed in many studies
% before sex selection became available \citep{repetto72,arnold98,dreze01}.
% 



%------------------------------------------------------------------------------------


\section{Estimation Strategy\label{sec:strategy}}

% Requirements for method and the intuition behind each:  
%  - Account for censoring (hazard model)
%  - Work both with and without sex selection (competing risk part)
%  - Allow for changes in use of sex selection within spell (flexible specification of baseline hazard ?)
%  - Capture that the shape of the hazard function differ across groups
%    (non-proportionality); needs a better description, maybe using example
%    of shorter spacing and/or use of sex selection. Differences in parity
%    progression likelihood. 

The unit of analysis is individual spells, the period from one birth, or marriage for
the first spell, to the next.
For some spells, the duration is censored, either because the survey
took place before the birth that would end the spell or because the couple
has finished childbearing.
The standard approach to address the censoring of spells, which I follow here, is 
to use a hazard model.
To capture the effects of sex selection, I extend the standard hazard model 
to allow for multiple exit states and a non-proportional hazard specification.

First, I use a competing risk framework with two possible exit states, either a boy
or a girl is born, because access to prenatal sex determination means that
the sex of children born is no longer necessarily a random event.
Prior research on birth spacing has used hazard models where the only 
possible exit state was the birth of a child. 
Having only one exit state works when there is no sex selection---making the 
sex of the child a random event---but not here because at least some couples use 
sex selection.%
\footnote{
\cite{Merli2000} used a discrete hazard model to examine whether 
there were under-reporting of births in rural China, although they 
estimated separate waiting time regressions for boys and girls.
}

Second, I use a non-proportional hazard specification because
sex composition and the use of sex selection are likely to affect 
the shape of the hazard functions across groups.
Proportional hazard models, where covariates have a multiplicative 
effect on the hazard rate, are more efficient than non-proportional 
models, but only provided that the proportionality assumption hold.
If the proportionality assumption does not hold, the estimates
are biased.
It is unlikely---even in the absence of prenatal sex determination---that 
the effect of, for example, the sex composition of previous births have 
the same effect throughout the entire spell.
Assuming that the effect of sex composition is the same throughout a spell
is especially problematic for higher-order spells where different
sex composition of previous births can lead to substantial difference 
both in the likelihood of progressing to the next birth and
how soon couples want their next child if they are going to have one.
The introduction of prenatal sex determination is likely to exacerbate 
any bias from the proportionality assumption.
First, sex-selective abortions affect birth spacing, and use
varies across groups.
Second, the use of sex selection may vary within a spell, which means 
that the effects of covariates vary within the spell as well.
The use of a non-proportional specification also mitigates any potential 
effects of unobserved heterogeneity when used in conjunction with a 
flexible baseline hazard \citep{Dolton1995}.

In summary, my proposed model is a discrete time, non-proportional, competing risk 
hazard model with two exit states: either a boy or a girl is born.
For each woman, $i=1,\ldots,n$, the starting point for a spell is time $t=1$ and 
the spell continues until time $t_i$, when either a birth occurs or the spell 
is censored.%
\footnote{
The time of censoring is assumed independent of the hazard rate,
as is standard in the literature.
}
There are two exit states: birth of a boy, $j=1$, or birth of a girl, $j=2$, and 
$J_i$ is a random variable indicating which event took place.
The discrete time hazard rate $h_{ijt}$ is
\begin{equation}
 h_{ijt} = \Pr (T_i=t, J_i=j \mid T_i \geq t; \mathbf{Z}_{it},\mathbf{X}_{i} ),
\end{equation}
where $T_i$ is a discrete random variable that captures when woman $i$'s birth occurs.
To ease presentation the indicator for spell number is suppressed.
The explanatory variable vectors, $\mathbf{Z}_{it}$ and $\mathbf{X}_{i}$, include 
individual, household, and community characteristics discussed below.

The hazard rate is specified as
\begin{equation}
 h_{ijt} = \frac{\exp(D_j(t) + \alpha_{jt}'\mathbf{Z}_{it} + \beta_j'\mathbf{X}_{i})} 
 {1 + \sum_{l=1}^2 \exp(D_j(t) + \alpha_{lt}'\mathbf{Z}_{it} + \beta_l'\mathbf{X}_{i})} \: \: \; \; \;  j = 1,2
 \label{eq:hazard}
\end{equation}
where $D_{j}(t)$ is the piece-wise linear baseline hazard for outcome $j$, captured
by dummies and the associated coefficients,
\begin{equation}
D_j(t) = \gamma_{j1} D_1 + \gamma_{j2} D_2 + \ldots + \gamma_{jT} D_T,
\end{equation}
where $D_m = 1$ if $t=m$ and zero otherwise.
This approach to modeling the baseline hazard is flexible and does not place 
overly strong restrictions on the baseline hazard.
The explanatory variables in $\mathbf{Z}$ and the interactions between them 
are the non-proportional part of the model, which means that they are
interacted with the baseline hazard:
\begin{equation}
 \mathbf{Z}_{it} = D_j(t) \times (\mathbf{Z}_1 + Z_2 + \mathbf{Z}_1 \times Z_2),
\end{equation}
where $D_j(t)$ is the piece-wise linear baseline hazard and $\mathbf{Z}_1$ captures sex 
composition of previous children, if any, and $Z_2$ captures area of residence.
This allows the effects of the main explanatory variables on the probabilities 
of having a boy, a girl, or no birth to vary over time within a spell.
The remaining explanatory variables, $\mathbf{X}$, enter proportionally,
but to further minimize any potential bias from assuming proportionality, estimations 
are done separately for different levels of mothers' education and for different 
time periods.
% The exact specifications and the individual variables are described below.

Equation (\ref{eq:hazard}) is equivalent to the logistic hazard model and has the same 
likelihood function as the multinomial logit model \citep{allison82,jenkins95}.
Hence, transforming the data, so each observation is a period---here equal
to three months---the model can be estimated using a standard multinomial logit model.
if the data are transformed so the unit of analysis is a time period, in 
this case 3 months periods, rather than the individual woman, the model can be 
estimated using a standard multinomial logit model.%
\footnote{
A potentially issue is that the multinomial model assumes that alternative 
exit states are stochastically independent,
also known as the Independence of Irrelevant Alternatives (IIA) assumption.
This assumption rules out any individual-specific unmeasured or 
unobservable factors that affect both the hazard of having a girl and the 
hazard of having a boy.
To address this issue the estimations include a proxy for fecundity
discussed in Section \ref{sec:data}.
In addition, the multivariate probit model can be used as an alternative
to the multinomial logit because the IIA is not imposed \citep{han90}.
The results are essentially identical between these two models and
available upon request.
}
In the reorganized data the outcome variable is 0 if the
woman does not have a child in a given period (the base outcome), 1 if 
she gives birth to a son in that period, and 2 if she gives birth to 
a daughter in that period.

The main downside of this set-up is that interpretation of the estimated 
coefficients is challenging;
the coefficients show the change in hazards relative to the base outcome, 
here no birth, rather than the hazard of an event.
Hence, a positive coefficient does not necessarily imply that the associated 
exit state becomes more likely as the explanatory variable increases because the 
probability of the other exit state may increase even more \citep{thomas96}.

It is, however, straightforward to calculate the predicted probabilities of 
having a boy and of having a girl for each $t$ within a spell, conditional on 
a set of explanatory variables and not having had a child before that period.
The predicted probability of having a boy in period $t$ for a given set of 
explanatory variable values, $\mathbf{Z}_k$ and $\mathbf{X}_k$, is
\begin{equation}
P(b_{t} | \mathbf{X}_{k}, \mathbf{Z}_{kt}, t ) 
=  
\frac{ \exp(D_j(t) + \alpha_{1t}' \mathbf{Z}_{kt} + \beta_1' \mathbf{X}_{k} )}
{1 + \sum_{l=1}^2 \exp(D_j(t) + \alpha_{lt} ' \mathbf{Z}_{kt} + \beta_l ' \mathbf{X}_{k})},
\label{eq:probability_boy}
\end{equation}
and the predicted probability of having a girl is
\begin{equation}
P(g_{t} | \mathbf{X}_{k}, \mathbf{Z}_{kt},t ) 
=  
\frac{ \exp(D_j(t) + \alpha_{2t}'\mathbf{Z}_{kt} + \beta_2'\mathbf{X}_{k} )}
{1 + \sum_{l=2}^2 \exp(D_j(t) + \alpha_{lt}'\mathbf{Z}_{kt} + \beta_l'\mathbf{X}_{k})}.
\label{eq:probability_girl}
\end{equation}
Within each period the probability of not having a birth in period $t$
is $1-P(b_{t})-P(g_{t})$.

The distribution of spacing is captured by the survival curve, which shows 
the probability of not having had a birth yet by spell duration, for
a given set of explanatory variables.
The survival curve at time $t$ is 
\begin{equation}
S_{t} 
= 
\prod_{d=1}^t 
\left( 
	1- \left(P(b_{d} | \mathbf{X}_{k}, \mathbf{Z}_{kd}, d) 
	+ P(g_{d} | \mathbf{X}_{k}, \mathbf{Z}_{kd}, d) \right) 
\right),
 \label{eq:survival}
\end{equation}
or equivalently
\begin{equation}
S_{t} 
= 
\prod_{d=1}^t
\left(
\frac{ 1 }
{1 + \sum_{l=2}^2 \exp(D_j(t) + \alpha_{ld}'\mathbf{Z}_{kd} + \beta_l'\mathbf{X}_{k})}
\right).
\end{equation}

An important issue is that the probability of ever having a next birth varies 
across groups.
For example, as the parity progression literature shows, couples with stronger 
son preference are less likely, for a given number of prior births, to ever have 
a next birth if they already have one or more sons.
Direct comparison of standard survival curves, therefore, tells us little 
about how the spread of sex selection affects birth spacing across, for
example, different sex compositions of prior children.

To overcome this problem, I condition on the likelihood of parity progression 
when examining birth spacing.
If, for example, women with a given set of characteristics have an 80\%
chance of ending a spell with a birth their median spell duration is when
their survival curve shows that 40\% are predicted to have exited the spell.%
\footnote{
It is important to note that this is not the same as simply calculating 
the median \emph{observed} spell lengths among women who have a given 
parity child in the survey because that number does not take into account
the censoring of spells that will eventually lead to a birth.
}
Median durations can then be averaged across women with different 
characteristics using the probabilities of ending a spell with a child
as weights.
How well this approach works depends on whether the spell length covered is 
sufficiently long that only very few women would be likely to give
birth later.
I discuss the choice of spell length below.
In addition to calculating conditional spell length for different 
percentages, I also present graphs of both standard survival curves and
survival curves conditional on parity progression (which therefore
begin at 100\% and end at 0\%).


Finally, with the predicted probabilities of having a boy and of having 
a girl for each $t$ within a spell, is it easy to calculate the estimated 
percentage of children born that are boys, $\hat{Y}$, at each $t$  
\begin{equation}
\hat{Y}_t 
= 
\frac{ P(b_{t} | \mathbf{X}_{k}, \mathbf{Z}_{kt},t )}
{ P(b_{t} | \mathbf{X}_{k}, \mathbf{Z}_{kt},t) + P(g_{t} | \mathbf{X}_{k}, \mathbf{Z}_{kt},t )} 
\times 100.
\label{eq:probability_son}
\end{equation}
Combining the percentage boys and the likelihood of exiting across all $t$ 
gives the predicted percent boys born over the entire spell.
In addition to the predicted percent boys over the entire spell, 
I also present graphs of how the percentage boys born vary across time
within a spell, together with the associated confidence interval for given 
values of explanatory variables calculated using the Delta method.


\section{Data\label{sec:data}}

The data come from the four rounds of the National Family Health Survey 
(NFHS-1, NFHS-2, NFHS-3, and NFHS-4),
collected in 1992--93, 1998--99, 2005--2006, and 2015--2016.%
\footnote{
A delay in the survey for Tripura means that NFHS-2 has some observation 
collected in 2000.
}
The surveys are large: NFHS-1 covered 89,777 ever-married women 
aged 13--49 from 88,562 households;
NFHS-2 covered 90,303 ever-married women aged 15--49 from 92,486 households;
NFHS-3 covered 124,385 never-married and ever-married women aged 
15--49 from 109,041 households;
and 
NFHS-4 covered 699,686 never-married and ever-married women aged
15--49 from 601,509 households surveyed.

I exclude visitors to the household, as well as
women who have been married more than once, divorced, or who are 
not living with their husband,
women with inconsistent age at marriage,
and those with missing information on education.
Women interviewed in NFHS-3 or NFHS-4 who were never married or where gauna 
had not been performed were also dropped.
The same goes for women who had at least one multiple births,
reported having a birth before age 12, had a birth before marriage, or
duration between births of less than nine months.
Women who reported less than nine months between marriage and first birth
remain in the sample---unless they are dropped for another reason---because 
10 to 20 percent fall into this category across the four surveys.
Although it is possible that some of these births are premature, the high number of
women who report a birth less than six months after their marriage indicates that for
a majority conception likely occurred before marriage.

Finally, I restrict the sample to Hindus,
who constitute about 80 percent of India's population.
If the use of sex selection differ between Hindu and other religions, such 
as Sikhs, assuming that the baseline hazard is the same would lead to bias.
The other groups are each so small relative to Hindus that it is not
possible to estimate different baseline hazards for each group.
Furthermore, the other groups are so different in background and son 
preference that combining them into one group would not make sense.

There are four advantages to using the NFHS.
First, surveys enumerators pay careful attention to spacing between births and
probe for ``missed'' births.
Second, no other surveys cover as extended a period in the same amount of detail.
The four NFHS rounds allow me to show how spacing and 
sex ratio changed from before sex-selective abortions were available until 2016.
Third, NFHS has birth histories for a large number of women.
% %
% \footnote{
% The Special Fertility and Mortality Survey appears to cover a much large number of households
% than the three rounds of the NFHS combined, but \citet{jha06} only use the births that 
% took place in 1997 making their sample sizes by parity smaller than here.
% Their sample consists of 133,738 births of which 38,177 were first-born, 
% 36,241 second-born, and 23,756 were third-born.
% }
Finally, even if probing for missing births does not eliminate recall error,   
the overlap in cohorts covered and the large sample size makes it possible 
to establish where recall error remains a problem.

Recall error arises mainly from child mortality when respondents are 
reluctant to discuss deceased children.
Systematic recall error, where the likelihood of reporting a deceased 
child depends on the sex of the child, is especially problematic because 
it biases both spacing and sex ratios.
Probing catches many missed births, but systematic recall error is 
still a potentially substantial problem.

Three factors contribute to the problem here.
First, girls have significantly higher mortality risk than boys.
Second, son preference may increase the probability that parents recall boys 
more readily than girls.
Finally, in NFHS-1 and NFHS-2 enumerators probed only for a missed birth if the
initially reported birth interval was four calendar years or more.
But, given short durations between births, especially after the birth of a girl,
that procedure is unlikely to pick up all missed children.

Recall error is heavily dependent on the length of marriage (TK).
I, therefore, drop women married 22 years or more 
for NFHS-1, with the corresponding cut-off points 23 years for NFHS-2, 
and 25 years for NFHS-3 and NFHS-4.
The final sample consists of 
  395,695 women, with   815,360 parity one through four births.
% [number] women, with [number] parity one through four births.


\subsection{Spell Definition\label{sec:spell_def}}

As mentioned above, 10 to 20 percent of women gave birth less than nine months
after they were married, and the first spell, therefore, begins at the 
month of marriage.
The exception is for women married very young, where I use the month they 
turned 12.
The second and subsequent spells begin nine months after the previous birth 
because that is the earliest we should expect to observe a new birth.
A few women report births that occurred less than nine months 
after the previous birth; I drop those women from the sample.

All spells continue until either a child is born or censoring occurs.
Censoring can happen for three reasons:
the survey takes place;
sterilization of the woman or her husband;
or too few births are observed for the method to work.
For spell 1 I set censoring at 72 months after marriage,
while for spells 2 and 3 censoring is at 72 months and for spell 4
57 months after a birth can occur. 

I group spells into four time periods based on spell start date:
1972--1984, 1985--1994, 1995--2004, and 2005--2016.
These years follow the main changes in availability and legality of 
prenatal sex determination discussed in the Introduction.
Some spells cover two periods because I classify periods based on the 
spell's beginning year.
A couple may, for example, be married in 1984, but not have their first 
child until 1986.
That couple's first spell will be in the 1972--1984 period, even though 
most of the interval falls in the 1985--1994 period.
Hence, prenatal sex determination was likely available when some
children from spells that began in the 1972--1984 period were conceived,
which could result in evidence of sex-selective abortions even for this
period.
Similarly, a spell that started in the 1985--1994 period may have been partly 
or mostly under the PDNT act.
The effect is a downward bias in the differences between the periods.

\subsection{Explanatory Variables}

I divide the explanatory variables into two groups.
The first group consists of variables expected to affect the shape of the hazard function
(the $\mathbf{Z}$ variables): 
mother's education, sex composition of previous children, and area of residence.
I chose these variables because the prior literature shows that they affect 
spacing choices and correlate with sex selection.
Increasing the number of variables interacted with the baseline hazard would
further lower the risk of bias but at the cost of requiring more data to 
precisely estimate.

I divide women into three groups based on education attainment: no
education, 1 to 7 years of education, and 8 and more years of education.
The models are estimated separately for each education level to reduce
the potential problem of including other variables as proportional.
Increasing education of mothers correlates closely with lower fertility 
\citep{schultz97}, and 
we should expect lower fertility to lead to more use of sex selection.%
\footnote{
Fathers' education has two opposite predicted effects: the associated higher income
should increase fertility and therefore lower the pressure to use sex selection, but
the higher income also makes the use of sex selection cheaper.
In practice, fathers' education had little effect on the hazards and the use of 
sex-selective abortions, and I, therefore, do not include it.
}

I capture sex composition with dummy variables for the
possible combinations for the specific spell, ignoring the ordering of births.
As an example, for the third spell, three groups are used: Two boys,
one girl and one boy, and two girls.
[TK this should be more specific - more boys less sex selection - maybe even use ordering]
As discussed, the sex composition of previous children affects both the timing
of births and the use of sex-selective abortions.

The area of residence is a dummy variable for the household living in
an urban area [TK check on this for NFHS-4].%
\footnote{
NFHS uses four categories for the area of residence: Large city, small city, town,
and countryside.
To reach a sufficient sample size urban areas are merged into one group.
}
The cost of children is higher in urban areas than in rural areas, and 
there is easier access to prenatal sex determination in urban areas.
Both factors are likely to lead to higher use of sex-selective abortions
in urban than in rural areas.

The second group of variables consists of those expected to have an 
approximately proportional effect on the hazard.
These include the length of the first spell 
(for the second and higher-order spells to capture fecundity), 
the age of the mother at the beginning of the spell, 
whether the household owns any land, 
and whether the household belongs to a scheduled tribe or caste.
We cannot observe fecundity directly, but the time from marriage until 
first birth is a suitable proxy because most Indian women do not use 
contraception before the first birth and there is pressure to show that 
a newly married woman can conceive \citep{dyson83,Sethuraman2007,Dommaraju2009}.
This is confirmed below by the very short spells between marriage and first birth,
even among the most educated.
Hence, a long spell between marriage and first birth is likely due to low fecundity.
For both duration from marriage to first birth and the age of the mother at the 
beginning of the spell, I also include their squares.
The remaining variables are dummies for household ownership of land and membership
of a scheduled caste or tribe.


\subsection{Descriptive Statistics}

Appendix Table \ref{tab:des_stat1} presents descriptive statistics for
the spells by education level and period they began.
There is a substantial number of censored observations.
As an example, for highly educated women who had their first child in the 2005--2016
period, more than 40\% did not have their second child by the time of the survey.
Censoring becomes even more important for the third and fourth
spells, with more than 70 percent of the observations censored.
The level of censoring also increases with parity and time period,
which reflects a combination of factors: timing of the surveys
relative to the periods of interest, later beginning of childbearing, 
falling fertility, and the increase in spell lengths from 
sex-selective abortions.

The share of urban women in the sample has fallen slightly over the
periods, even though India's population has become progressively more urbanized.
For the first and second periods, over 32\% of the women 
entering the first spell lived in urban areas, falling to 
30\% for the third period and 26\% in the last period.
The most likely explanation is that the age of marriage has increased
faster in urban areas than in rural areas.
Hence, there are relatively more urban women, but fewer show up in the
sample because they are not yet married.

The population has become substantially better educated over time.
Women with no education constituted almost 60 percent 
in the first period, but less than twenty percent in the last period.
Correspondingly, in the first period just over twenty percent had eight or more 
years of education, whereas in the last period more than 60\% did.
These changes are an underestimate of the increase in female
education overall because many of the younger women with more education
have not yet married than therefore are not in the sample.%
\footnote{
Women with eight or more years of education accounted for 65.9\% of
unmarried women in NFHS-3 and 82.8\% of unmarried women in NFHS-4.
See \citet[p 56]{International-Institute-for-Population-Sciences-IIPS2007}
and \citet[p 61]{International-Institute-for-Population-Sciences-IIPS2017}
for more information.
 }

Age at marriage increased over time across all three education groups.
The most substantial increase was for women without any education, where
the average went from below 16 years of age to 18.5.
The smallest change is for the most educated women where the average age
at marriage went up by less than a year---from 19.6 to 20.4---across the 
four periods.


\section{Results\label{sec:results}}

% [Hypotheses / questions]

The overall hypothesis is that spacing patterns significantly changed 
as prenatal sex determination became available.
To address this, I examine three questions in order of importance.

First, did the spacing after girls increase relative to the spacing
after boys?
The underlying hypothese is that in the absence of sex selection son 
preference leads to shorter spacing after the birth of a girl than after the 
birth of a boy, whereas son preference increases the spacing after the birth 
of a girl relative to after the birth of a boy when prenatal sex selection is 
used.
We would still expect parents to \emph{try} to conceive earlier after the birth 
of a girl than after a boy, but the abortions will increase the 
length of the space, and may even lead to longer spacing after the
birth of girls than after boys.
[TK something on general changes in spacing? Or hold off until discussing the
results]

Second, do the shapes of survival curves differ across sex compositions 
within a spell?
The shapes of the survival curves provides clues to whether parents are
still trying to conceive earlier after girls than boys.
I expect this to be the case if there is still a strong preference for
boys, and, as discussed above, this behavior may even become more pronounced
with access to sex selection.
At a more technical level, the shapes of the survival curve serve as an
indicator for the presence of non-proportionality.

Finally, do parents change their decision on sex selection within a spell?
There are two reasons why this is interesting.
If parents change their mind the standard method of observing sex ratios 
will not capture the full extent of the use of sex selection.
Furthermore, it tells us something about parents' preferences for ideal
spacing between births and how strong their son preference is.


% [How to read tables]
Tables \ref{tab:median_sex_ratio_low}, \ref{tab:median_sex_ratio_med}, and
\ref{tab:median_sex_ratio_high}
show predicted median duration and sex ratios by period, spell, and sex 
composition for the three different education levels separated by area of residence.%
\footnote{
Appendix Tables \ref{tab:p25_p75_low}, \ref{tab:p25_p75_med}, and 
\ref{tab:p25_p75_low} show 25th and 75th percentiles durations.
} 
The predictions are based on the estimation method described in 
Section \ref{sec:strategy}.
The standard errors for both measures are based on bootstrapping, where the 
original sample is resampled with replacement and the method is applied to 
the new sample.

The predicted median duration shown is the weighted average over the women in the 
sample's predicted median durations, using the individual probabilities of a birth 
as weights.
For each woman in a sample, I calculate her probability of ending the spell 
with a birth, and her contribution to median spacing is how long it will take
from the beginning of the spell to where she is predicted to reach half of that probability.
For example, if a woman has a 80\% probability of having a birth by the end of 
the spell, her median duration is the number of months it takes to reach
40\% probability of having exited with a birth (equivalent to the 60\% point
in a standard survival curve because that captures how many have not yet
exited).

For second through fourth spell I show whether durations for sex composition
other than only girls are statistically significantly different from the
duration with only girls.
The statistical significance is based on the bootstrapped differences in
durations across sex compositions.
For each new sample the difference in duration is calculated and these
differences are used to calculate the standard error and the level of
significance for the difference in the original sample.
 
The cleanest test is comparing durations after only boys with durations after
only girls, but the number of births to women with only sons becomes small 
in the later periods because of falling fertility.
Hence, it is possible to have substantial differences in spacing that are
not statistically significant because of low power, especially for the third 
and fourth spell.
It is, in principle, possible to estimate the model taking into account the 
ordering of children, but this would further lower the power of the test, by 
adding one extra group for the third spell and four extra groups for the fourth spell.

The predicted sex ratio is the weighted average over the women in the sample's 
individual predicted sex ratios, using the individual probabilities of a birth by 
the end of the spell as weights.
Each woman's predicted sex ratio is the weighted average of the predicted
percentage boys over the months in the spell calculated using equation 
(\ref{eq:probability_son}) and the probability of giving birth in each month 
as weights.%
\footnote{
Imagine a spell has two periods and that the estimated percentage boys for a
woman are 54\% and 66\% and that the likelihood of exiting the spell is 20\% and 40\%
(in other words, there is 40\% chance that she does not have a birth by the end of 
the spell).
This woman' percentage boys is then $\frac{54*0.2+66*0.4}{0.2+0.4} = 62$.
}
The predicted sex ratio captures the percent boys that will be born to women 
in the sample once child bearing for that spell is over.

Each predicted precent boys is tested against the natural percentage 
boys using the bootstrapped standard errors.
The natural sex ratio is approximately 105 boys to 100 girls or
51.2\% \citep{ben-porath76b,jacobsen99,Portner2015b}.
The predicted percentage boys may differ from the natural rate because of 
natural variation, any remaining recall error not corrected for, or 
sex selection. 


\input{../tables/bootstrap_duration_sex_ratio_low.tex}

Women with no education follow a pattern consistent with son
preference with shorter spacing when they have only girls compared 
to when they have one or more sons, and the differences are often statistically 
significant as shown in Table \ref{tab:median_sex_ratio_low}.
Most of the women without education live in rural areas,
while there are relatively few urban women with no education.
There are especially very few urban women to based the third
and fourth spell results on, and I therefore focus on rural women.

Predicted median duration has generally increased over time within
each sex composition.
The exception is for the first spell, where there is little
change, and possibly even a small decline.
This decline may be the result of a combination of lower recall
error, increasing age at marriage, and better health.

The differences in duration across sex compositions are more 
pronounced with higher-order spells.
For the second spell, the difference in median duration between
having a girl or a boy as the first-born are mostly between 0.5 and 1 
month, although it varies from a low of 0.2 months for rural women 
in 1995--2004 period to a high of 1.5 months
for urban women in the 1985--1994 period.
Focusing on rural women for the third spell, the difference is,
on average, slightly over 1 months between only girls and only boys 
and the differences are statistically significant in all periods.
For rural women the differences for the fourth spell are just below
2 months, with the exception of the 1972--1984 period, which is
also the only period where the difference is not statistically significant.

It is possible that remaining recall error is behind some of the
smaller differences in duration.
One example is the second spell in the 1995--2004 period, where there 
are only 0.2 months difference between duration after one girl and 
duration after one boy.
Similarly, the small difference for the fourth spell in the 1972--1984
period may be the result of recall error with a very high 
percentage boys for women with three girls, despite no availability
of sex selection.

There are no clear time trends in the differences in median duration
across sex compositions or the predicted sex ratios across the four periods.
Despite some sex ratios that are statistically significantly higher 
than the natural sex ratio there is therefore little consistent evidence 
for the use of sex selection for women without education.
There has, for example, been relatively little change in the length
of spacing for rural women with two girls in the third spell and 
the predicted sex ratio was higher in the 1985--1994 period
than in the 2005--2016 period.
The possible exception is the fourth spell, where both duration 
and sex ratio has increased for women with three girls, although
the \emph{difference} across sex compositions has remained almost
the same.


\input{../tables/bootstrap_duration_sex_ratio_med.tex}

For women with between 1 and 7 years of education it is harder
to discern a pattern, partly because a larger number of area,
sex composition, and periods combinations have relatively few 
births in them.
That said, the predicted median durations show less evidence 
of son preference than for women without education.
The exception to this are for the third spell for rural women
for the 1972-1984 and 1985-1994 periods where the median
durations are two and three months shorted with only girls
compared to only boys, although the duration is longer for
only girls in the 1995-2006 period.
As for women with no education, there is little evidence of
sex selection for women with 1 to 7 years of education,
consistent with the absence of changes in relative durations
across periods.





\input{../tables/bootstrap_duration_sex_ratio_high.tex}


\subsection{Differences in Survival Curves Across Sex Compositions}

[Moved from estimation strategy]
In the absence of sex selection, I expect most parities to follow
a pattern, where there initially are relatively few births, 
followed by a substantial number of births over a 1 to 2 year period, and then
relatively few births thereafter.
As sex selection becomes more widely used the associated
longer spacing shows up by making the survival curve straighter,
indicating that some of the births that would originally have taken place
now take place later because of abortions.
The use of sex selection is clearly not the only factor that can change the shape of
the survival curves; factors such as the desired number of children and
use of contraceptives may also shape the shape.
To account for this it is best to compare survival curves for an individual
parity between women who are likely to use sex selection and women who are
not, for example because they already have one or more sons.




\subsection{Does Sex Selection Change within Spells?}


\section{Discussion}

[TK not sure where this fits in]
In addition, it is possible that falling fertility also generically 
leads to longer spacing between births.
This means that we have to separate out which part of the longer
spacing comes from falling fertility (and therefore should also
affect boy composition) and which parts comes from increased use
of sex selection.

[We should expect sex selection to be mainly used for the last---or close to
last---birth that a couple expect to have because of the cost and risk
of the procedures.]


[TK not sure where to fit this into]
Finally, increased reliability of access and effectiveness of
contraceptive can lead to shorter spacing between births 
\citep{Keyfitz1971,Heckman1976}.
With less reliable contraception parents choose a higher level 
of contraception---resulting in longer spacing---to avoid having 
too many children by accident.
But, as contraception becomes more effective parents can more
easily avoid future births, allowing them to reduce the spacing 
between births without having to worry about overshooting their 
preferred number of children.
This idea may also help explain shorter spacing for better 
educated women than for less educated women, provided that   
knowledge and ability to use contraception differ
across education groups \citep{Tulasidhar1993,Whitworth2002}.





Figure \ref{fig:results_spell2_high_urban} 
shows the predicted probabilities of having a boy and standard survival curves
for the three periods for urban women with 8 or more years of education.%
\footnote{
The corresponding graphs for rural women and for women with no education are
shown in Figures \ref{fig:results_spell2_high_rural}, \ref{fig:results_spell2_low_urban},
and \ref{fig:results_spell2_low_rural}
}
Two main results stand out.
First, women whose first child was a boy do not appear to use sex selection during
the second spell.
The predicted percent boys is consistently close to the natural level in all three
panels (\ref{fig:results_spell2_high_urban}(d)-(f)).
Second, as prenatal sex determination became available there is more and more
evidence of sex selection affecting the sex ratio for women with a girl as their
first child.
In panel (a), which covers the period before sex selection became generally
available, there is no significant evidence of a sex ratio biased towards males.
Panel (b) shows a more male biased sex ratio but only a few months show
a statistically significant difference from the natural rate and the shallow 
slope of the survival curve indicates that relatively few births took place
during these months.
Panel (c), however, shows clear evidence of the effect of access to prenatal
sex determination with a male biased sex ratio for more of the two first years.%
\footnote{
Recall that month 0 is equivalent to 9 months after the birth
of the first child.
}


% High education

\begin{figure}[htpb]
\centering
\caption*{First child a girl}
\setcounter{subfigure}{-2}
\subfloat[1972--1984 (N=2,689)]{
    \begin{minipage}{0.31\textwidth}
        \captionsetup[subfigure]{labelformat=empty,position=top,captionskip=-1pt,farskip=-0.5pt}
        \subfloat[Prob.\ boy (\%)]{\includegraphics[width=\textwidth]{spell2_g1_high_urban_g_pc}}\\
        \subfloat[Prob.\ no birth yet]{\includegraphics[width=\textwidth]{spell2_g1_high_urban_g_s}} 
        \captionsetup[subfigure]{labelformat=parens}
    \end{minipage}
} 
\setcounter{subfigure}{-1}
\subfloat[1985--1994 (N=4,869)]{
    \begin{minipage}{0.31\textwidth}
        \captionsetup[subfigure]{labelformat=empty,position=top,captionskip=-1pt,farskip=-0.5pt}
        \subfloat[Prob. boy (\%)]{\includegraphics[width=\textwidth]{spell2_g2_high_urban_g_pc}}\\
        \subfloat[Prob. no birth yet]{\includegraphics[width=\textwidth]{spell2_g2_high_urban_g_s}}
        \captionsetup[subfigure]{labelformat=parens}
    \end{minipage}
}
\setcounter{subfigure}{0}
\subfloat[1995--2006 (N=3,774)]{
    \begin{minipage}{0.31\textwidth}
        \captionsetup[subfigure]{labelformat=empty,position=top,captionskip=-1pt,farskip=-0.5pt}
        \subfloat[Prob. boy (\%)]{\includegraphics[width=\textwidth]{spell2_g3_high_urban_g_pc}}\\
        \subfloat[Prob. no birth yet]{\includegraphics[width=\textwidth]{spell2_g3_high_urban_g_s}}
        \captionsetup[subfigure]{labelformat=parens}
    \end{minipage}
}
\caption*{First child a boy}
\setcounter{subfigure}{1}
\subfloat[1972--1984 (N=2,806)]{
    \begin{minipage}{0.31\textwidth}
        \captionsetup[subfigure]{labelformat=empty,position=top,captionskip=-1pt,farskip=-0.5pt}
        \subfloat[Prob. boy (\%)]{\includegraphics[width=\textwidth]{spell2_g1_high_urban_b_pc}}\\
        \subfloat[Prob. no birth yet]{\includegraphics[width=\textwidth]{spell2_g1_high_urban_b_s}} 
        \captionsetup[subfigure]{labelformat=parens}
    \end{minipage}
} 
\setcounter{subfigure}{2}
\subfloat[1985--1994 (N=5,246)]{
    \begin{minipage}{0.31\textwidth}
        \captionsetup[subfigure]{labelformat=empty,position=top,captionskip=-1pt,farskip=-0.5pt}
        \subfloat[Prob. boy (\%)]{\includegraphics[width=\textwidth]{spell2_g2_high_urban_b_pc}}\\
        \subfloat[Prob. no birth yet]{\includegraphics[width=\textwidth]{spell2_g2_high_urban_b_s}}
        \captionsetup[subfigure]{labelformat=parens}
    \end{minipage}
}
\setcounter{subfigure}{3}
\subfloat[1995--2006 (N=3,969)]{
    \begin{minipage}{0.31\textwidth}
        \captionsetup[subfigure]{labelformat=empty,position=top,captionskip=-1pt,farskip=-0.5pt}
        \subfloat[Prob. boy (\%)]{\includegraphics[width=\textwidth]{spell2_g3_high_urban_b_pc}}\\
        \subfloat[Prob. no birth yet]{\includegraphics[width=\textwidth]{spell2_g3_high_urban_b_s}}
        \captionsetup[subfigure]{labelformat=parens}
    \end{minipage}
}
\caption{Predicted probability of having a boy and probability of
no birth yet from nine months after first birth for urban 
women with 8 or more years of education by month beginning 9 months after prior birth. 
Predictions based on age 22 at first birth.
Left column shows results prior to sex selection available, middle column before
sex selection illegal and right column after sex selection illegal.
N indicates the number of women in the relevant group in the underlying samples.
}
\label{fig:results_spell2_high_urban}
\end{figure}




% Survival curves conditional on progression 

Figures \ref{fig:results_spell2_low_pps} and \ref{fig:results_spell2_high_pps}
show survival curves for the second spell (from first birth) for women with
no education and women with 8 or more years of education.
To allow for comparison these survival curves show the ratio of women who
have not had a birth yet, given that they will eventually are predicted to
have a birth.
Women with no education show relatively similar survival curves whether their
first child was a girl or a boy, but to the extent that a pattern is present
it shows, as expected, shorter spacing a girl than a boy.
Rural women with 8 or more years of education show essentially the same 
pattern.
The lack of clear results is not surprising given that all three groups of
women are likely to have more than two children.

The effect of sex selection on spacing first show up for urban women with
8 or more years of education.
Panel (a) in Figure \ref{fig:results_spell2_high_pps} show shorter spacing
for the second spell when the first child is a girl than a boy, and this
pattern is also present in Panel (b).
Panel (c) on the other hand show first evidence of \emph{longer} spacing 
after the birth of a girl than after a boy, consistent with the male-biased
sex ratio in Panel (c) in Figure \ref{fig:results_spell2_high_urban}.


% Low education

\begin{figure}[htpb]
\centering
\caption*{Urban}
\setcounter{subfigure}{-1}
\subfloat[1972--1984]{
    \begin{minipage}{0.32\textwidth}
        \captionsetup[subfigure]{labelformat=empty,position=top,captionskip=-1pt,farskip=-0.5pt}
        \subfloat[Prob.\ no birth yet]{\includegraphics[width=\textwidth]{spell2_g1_low_urban_pps}} 
        \captionsetup[subfigure]{labelformat=parens}
    \end{minipage}
} 
\setcounter{subfigure}{-0}
\subfloat[1985--1994]{
    \begin{minipage}{0.32\textwidth}
        \captionsetup[subfigure]{labelformat=empty,position=top,captionskip=-1pt,farskip=-0.5pt}
        \subfloat[Prob. no birth yet]{\includegraphics[width=\textwidth]{spell2_g2_low_urban_pps}}
        \captionsetup[subfigure]{labelformat=parens}
    \end{minipage}
} \\
\setcounter{subfigure}{1}
\subfloat[1995--2006]{
    \begin{minipage}{0.32\textwidth}
        \captionsetup[subfigure]{labelformat=empty,position=top,captionskip=-1pt,farskip=-0.5pt}
        \subfloat[Prob. no birth yet]{\includegraphics[width=\textwidth]{spell2_g3_low_urban_pps}}
        \captionsetup[subfigure]{labelformat=parens}
    \end{minipage}
}
\setcounter{subfigure}{2}
\subfloat[1995--2006]{
    \begin{minipage}{0.32\textwidth}
        \captionsetup[subfigure]{labelformat=empty,position=top,captionskip=-1pt,farskip=-0.5pt}
        \subfloat[Prob. no birth yet]{\includegraphics[width=\textwidth]{spell2_g4_low_urban_pps}}
        \captionsetup[subfigure]{labelformat=parens}
    \end{minipage}
}
\caption*{Rural}
\setcounter{subfigure}{3}
\subfloat[1972--1984]{
    \begin{minipage}{0.32\textwidth}
        \captionsetup[subfigure]{labelformat=empty,position=top,captionskip=-1pt,farskip=-0.5pt}
        \subfloat[Prob. no birth yet]{\includegraphics[width=\textwidth]{spell2_g1_low_rural_pps}} 
        \captionsetup[subfigure]{labelformat=parens}
    \end{minipage}
} 
\setcounter{subfigure}{4}
\subfloat[1985--1994]{
    \begin{minipage}{0.32\textwidth}
        \captionsetup[subfigure]{labelformat=empty,position=top,captionskip=-1pt,farskip=-0.5pt}
        \subfloat[Prob. no birth yet]{\includegraphics[width=\textwidth]{spell2_g2_low_rural_pps}}
        \captionsetup[subfigure]{labelformat=parens}
    \end{minipage}
} \\
\setcounter{subfigure}{5}
\subfloat[1995--2006]{
    \begin{minipage}{0.32\textwidth}
        \captionsetup[subfigure]{labelformat=empty,position=top,captionskip=-1pt,farskip=-0.5pt}
        \subfloat[Prob. no birth yet]{\includegraphics[width=\textwidth]{spell2_g3_low_rural_pps}}
        \captionsetup[subfigure]{labelformat=parens}
    \end{minipage}
}
\setcounter{subfigure}{6}
\subfloat[1995--2006]{
    \begin{minipage}{0.32\textwidth}
        \captionsetup[subfigure]{labelformat=empty,position=top,captionskip=-1pt,farskip=-0.5pt}
        \subfloat[Prob. no birth yet]{\includegraphics[width=\textwidth]{spell2_g4_low_rural_pps}}
        \captionsetup[subfigure]{labelformat=parens}
    \end{minipage}
}
\caption{Survival curves conditional on parity progression
for women with no education by month beginning 9 months after prior birth.
Left column shows results prior to sex selection available, middle column before
sex selection illegal and right column after sex selection illegal.
}
\label{fig:results_spell2_low_pps}
\end{figure}





% High education

\begin{figure}[htpb]
\centering
\caption*{Urban}
\setcounter{subfigure}{-1}
\subfloat[1972--1984]{
    \begin{minipage}{0.31\textwidth}
        \captionsetup[subfigure]{labelformat=empty,position=top,captionskip=-1pt,farskip=-0.5pt}
        \subfloat[Prob.\ no birth yet]{\includegraphics[width=\textwidth]{spell2_g1_high_urban_pps}} 
        \captionsetup[subfigure]{labelformat=parens}
    \end{minipage}
} 
\setcounter{subfigure}{-0}
\subfloat[1985--1994]{
    \begin{minipage}{0.31\textwidth}
        \captionsetup[subfigure]{labelformat=empty,position=top,captionskip=-1pt,farskip=-0.5pt}
        \subfloat[Prob. no birth yet]{\includegraphics[width=\textwidth]{spell2_g2_high_urban_pps}}
        \captionsetup[subfigure]{labelformat=parens}
    \end{minipage}
} \\
\setcounter{subfigure}{1}
\subfloat[1995--2004]{
    \begin{minipage}{0.31\textwidth}
        \captionsetup[subfigure]{labelformat=empty,position=top,captionskip=-1pt,farskip=-0.5pt}
        \subfloat[Prob. no birth yet]{\includegraphics[width=\textwidth]{spell2_g3_high_urban_pps}}
        \captionsetup[subfigure]{labelformat=parens}
    \end{minipage}
}
\setcounter{subfigure}{2}
\subfloat[2005--2016]{
    \begin{minipage}{0.31\textwidth}
        \captionsetup[subfigure]{labelformat=empty,position=top,captionskip=-1pt,farskip=-0.5pt}
        \subfloat[Prob. no birth yet]{\includegraphics[width=\textwidth]{spell2_g4_high_urban_pps}}
        \captionsetup[subfigure]{labelformat=parens}
    \end{minipage}
}
\caption*{Rural}
\setcounter{subfigure}{3}
\subfloat[1972--1984]{
    \begin{minipage}{0.31\textwidth}
        \captionsetup[subfigure]{labelformat=empty,position=top,captionskip=-1pt,farskip=-0.5pt}
        \subfloat[Prob. no birth yet]{\includegraphics[width=\textwidth]{spell2_g1_high_rural_pps}} 
        \captionsetup[subfigure]{labelformat=parens}
    \end{minipage}
} 
\setcounter{subfigure}{4}
\subfloat[1985--1994]{
    \begin{minipage}{0.31\textwidth}
        \captionsetup[subfigure]{labelformat=empty,position=top,captionskip=-1pt,farskip=-0.5pt}
        \subfloat[Prob. no birth yet]{\includegraphics[width=\textwidth]{spell2_g2_high_rural_pps}}
        \captionsetup[subfigure]{labelformat=parens}
    \end{minipage}
} \\
\setcounter{subfigure}{5}
\subfloat[1995--2004]{
    \begin{minipage}{0.31\textwidth}
        \captionsetup[subfigure]{labelformat=empty,position=top,captionskip=-1pt,farskip=-0.5pt}
        \subfloat[Prob. no birth yet]{\includegraphics[width=\textwidth]{spell2_g3_high_rural_pps}}
        \captionsetup[subfigure]{labelformat=parens}
    \end{minipage}
}
\setcounter{subfigure}{6}
\subfloat[2005--2016]{
    \begin{minipage}{0.31\textwidth}
        \captionsetup[subfigure]{labelformat=empty,position=top,captionskip=-1pt,farskip=-0.5pt}
        \subfloat[Prob. no birth yet]{\includegraphics[width=\textwidth]{spell2_g4_high_rural_pps}}
        \captionsetup[subfigure]{labelformat=parens}
    \end{minipage}
}
\caption{Survival curves conditional on parity progression
for women with 8 or more years of education by month beginning 9 months after prior birth.
Left column shows results prior to sex selection available, middle column before
sex selection illegal and right column after sex selection illegal.
N indicates the number of women in the relevant group in the underlying samples.
}
\label{fig:results_spell2_high_pps}
\end{figure}


% 3rd spell






% Low education

\begin{figure}[htpb]
\centering
\caption*{Urban}
\setcounter{subfigure}{-1}
\subfloat[1972--1984]{
    \begin{minipage}{0.31\textwidth}
        \captionsetup[subfigure]{labelformat=empty,position=top,captionskip=-1pt,farskip=-0.5pt}
        \subfloat[Prob.\ no birth yet]{\includegraphics[width=\textwidth]{spell3_g1_low_urban_pps}} 
        \captionsetup[subfigure]{labelformat=parens}
    \end{minipage}
} 
\setcounter{subfigure}{-0}
\subfloat[1985--1994]{
    \begin{minipage}{0.31\textwidth}
        \captionsetup[subfigure]{labelformat=empty,position=top,captionskip=-1pt,farskip=-0.5pt}
        \subfloat[Prob. no birth yet]{\includegraphics[width=\textwidth]{spell3_g2_low_urban_pps}}
        \captionsetup[subfigure]{labelformat=parens}
    \end{minipage}
} \\
\setcounter{subfigure}{1}
\subfloat[1995--2004]{
    \begin{minipage}{0.31\textwidth}
        \captionsetup[subfigure]{labelformat=empty,position=top,captionskip=-1pt,farskip=-0.5pt}
        \subfloat[Prob. no birth yet]{\includegraphics[width=\textwidth]{spell3_g3_low_urban_pps}}
        \captionsetup[subfigure]{labelformat=parens}
    \end{minipage}
}
\setcounter{subfigure}{2}
\subfloat[2005--2016]{
    \begin{minipage}{0.31\textwidth}
        \captionsetup[subfigure]{labelformat=empty,position=top,captionskip=-1pt,farskip=-0.5pt}
        \subfloat[Prob. no birth yet]{\includegraphics[width=\textwidth]{spell3_g4_low_urban_pps}}
        \captionsetup[subfigure]{labelformat=parens}
    \end{minipage}
}
\caption*{Rural}
\setcounter{subfigure}{3}
\subfloat[1972--1984]{
    \begin{minipage}{0.31\textwidth}
        \captionsetup[subfigure]{labelformat=empty,position=top,captionskip=-1pt,farskip=-0.5pt}
        \subfloat[Prob. no birth yet]{\includegraphics[width=\textwidth]{spell3_g1_low_rural_pps}} 
        \captionsetup[subfigure]{labelformat=parens}
    \end{minipage}
} 
\setcounter{subfigure}{4}
\subfloat[1985--1994]{
    \begin{minipage}{0.31\textwidth}
        \captionsetup[subfigure]{labelformat=empty,position=top,captionskip=-1pt,farskip=-0.5pt}
        \subfloat[Prob. no birth yet]{\includegraphics[width=\textwidth]{spell3_g2_low_rural_pps}}
        \captionsetup[subfigure]{labelformat=parens}
    \end{minipage}
} \\
\setcounter{subfigure}{5}
\subfloat[1995--2004]{
    \begin{minipage}{0.31\textwidth}
        \captionsetup[subfigure]{labelformat=empty,position=top,captionskip=-1pt,farskip=-0.5pt}
        \subfloat[Prob. no birth yet]{\includegraphics[width=\textwidth]{spell3_g3_low_rural_pps}}
        \captionsetup[subfigure]{labelformat=parens}
    \end{minipage}
}
\setcounter{subfigure}{6}
\subfloat[2005--2016]{
    \begin{minipage}{0.31\textwidth}
        \captionsetup[subfigure]{labelformat=empty,position=top,captionskip=-1pt,farskip=-0.5pt}
        \subfloat[Prob. no birth yet]{\includegraphics[width=\textwidth]{spell4_g3_low_rural_pps}}
        \captionsetup[subfigure]{labelformat=parens}
    \end{minipage}
}
\caption{Survival curves conditional on parity progression
for women with no education by month beginning 9 months after prior birth.
Left column shows results prior to sex selection available, middle column before
sex selection illegal and right column after sex selection illegal.
N indicates the number of women in the relevant group in the underlying samples.
}
\label{fig:results_spell3_low_pps}
\end{figure}





% High education

\begin{figure}[htpb]
\centering
\caption*{Urban}
\setcounter{subfigure}{-1}
\subfloat[1972--1984]{
    \begin{minipage}{0.31\textwidth}
        \captionsetup[subfigure]{labelformat=empty,position=top,captionskip=-1pt,farskip=-0.5pt}
        \subfloat[Prob.\ no birth yet]{\includegraphics[width=\textwidth]{spell3_g1_high_urban_pps}} 
        \captionsetup[subfigure]{labelformat=parens}
    \end{minipage}
} 
\setcounter{subfigure}{-0}
\subfloat[1985--1994]{
    \begin{minipage}{0.31\textwidth}
        \captionsetup[subfigure]{labelformat=empty,position=top,captionskip=-1pt,farskip=-0.5pt}
        \subfloat[Prob. no birth yet]{\includegraphics[width=\textwidth]{spell3_g2_high_urban_pps}}
        \captionsetup[subfigure]{labelformat=parens}
    \end{minipage}
} \\
\setcounter{subfigure}{1}
\subfloat[1995--2004]{
    \begin{minipage}{0.31\textwidth}
        \captionsetup[subfigure]{labelformat=empty,position=top,captionskip=-1pt,farskip=-0.5pt}
        \subfloat[Prob. no birth yet]{\includegraphics[width=\textwidth]{spell3_g3_high_urban_pps}}
        \captionsetup[subfigure]{labelformat=parens}
    \end{minipage}
}
\setcounter{subfigure}{2}
\subfloat[2005--2016]{
    \begin{minipage}{0.31\textwidth}
        \captionsetup[subfigure]{labelformat=empty,position=top,captionskip=-1pt,farskip=-0.5pt}
        \subfloat[Prob. no birth yet]{\includegraphics[width=\textwidth]{spell3_g4_high_urban_pps}}
        \captionsetup[subfigure]{labelformat=parens}
    \end{minipage}
}
\caption*{Rural}
\setcounter{subfigure}{3}
\subfloat[1972--1984]{
    \begin{minipage}{0.31\textwidth}
        \captionsetup[subfigure]{labelformat=empty,position=top,captionskip=-1pt,farskip=-0.5pt}
        \subfloat[Prob. no birth yet]{\includegraphics[width=\textwidth]{spell3_g1_high_rural_pps}} 
        \captionsetup[subfigure]{labelformat=parens}
    \end{minipage}
} 
\setcounter{subfigure}{4}
\subfloat[1985--1994]{
    \begin{minipage}{0.31\textwidth}
        \captionsetup[subfigure]{labelformat=empty,position=top,captionskip=-1pt,farskip=-0.5pt}
        \subfloat[Prob. no birth yet]{\includegraphics[width=\textwidth]{spell3_g2_high_rural_pps}}
        \captionsetup[subfigure]{labelformat=parens}
    \end{minipage}
} \\
\setcounter{subfigure}{5}
\subfloat[1995--2004]{
    \begin{minipage}{0.31\textwidth}
        \captionsetup[subfigure]{labelformat=empty,position=top,captionskip=-1pt,farskip=-0.5pt}
        \subfloat[Prob. no birth yet]{\includegraphics[width=\textwidth]{spell3_g3_high_rural_pps}}
        \captionsetup[subfigure]{labelformat=parens}
    \end{minipage}
}
\setcounter{subfigure}{6}
\subfloat[2005--2016]{
    \begin{minipage}{0.31\textwidth}
        \captionsetup[subfigure]{labelformat=empty,position=top,captionskip=-1pt,farskip=-0.5pt}
        \subfloat[Prob. no birth yet]{\includegraphics[width=\textwidth]{spell3_g4_high_rural_pps}}
        \captionsetup[subfigure]{labelformat=parens}
    \end{minipage}
}
\caption{Survival curves conditional on parity progression
for women with 8 or more years of education by month beginning 9 months after prior birth.
Left column shows results prior to sex selection available, middle column before
sex selection illegal and right column after sex selection illegal.
N indicates the number of women in the relevant group in the underlying samples.
}
\label{fig:results_spell3_high_pps}
\end{figure}

The effect of sex selection on spacing is even more pronounced in
Figure \ref{fig:results_spell3_high_pps}, which shows the conditional
survival curves for the third spell (from second birth) for women with
8 or more years of education.
Before prenatal sex determination became available there is little difference
in spacing pattern no matter the sex composition of the first two children.
This changes after the introduction as shown in Panel (b), where the spacing
is longer when the first two children are girls than either one boy/one girl
or two boys.
Even more substantial are the results from Panel (c), which shows much 
longer spacing when the first two children are girls.
These results are consistent with the biased sex ratios shown in 
Figure \ref{fig:results_spell3_high_urban}.



\section{Conclusion\label{sec:conclusion}}


[To be added]




\clearpage

\onehalfspacing
\bibliographystyle{aer}
\bibliography{sex_selection_spacing}

\addcontentsline{toc}{section}{References}



\clearpage
\newpage

\appendix
\section{Appendix}

% CHANGING NUMBERING OF FIGURES AND TABLES FOR APPENDIX
\renewcommand\thefigure{\thesection.\arabic{figure}}    
\setcounter{figure}{0}
\renewcommand\thetable{\thesection.\arabic{table}}    
\setcounter{table}{0}
  
% Descriptive statistics tables
\input{../tables/des_stat.tex}


\clearpage

\subsection{Additional Duration Results}

\input{../tables/bootstrap_duration_p25_p75_low.tex}

\input{../tables/bootstrap_duration_p25_p75_med.tex}

\input{../tables/bootstrap_duration_p25_p75_high.tex}

\clearpage

\subsection{Second Spell}



% SPELL 2 - URBAN - LOW
\begin{figure}[h]
\centering
\caption*{First child a girl}
\setcounter{subfigure}{-2}
\subfloat[1972-1984 (N=1,596)]{
    \begin{minipage}{0.26\textwidth}
        \captionsetup[subfigure]{labelformat=empty,position=top,captionskip=-1pt,farskip=-0.5pt}
        \subfloat[Prob.\ boy (\%)]{\includegraphics[width=\textwidth]{spell2_g1_low_urban_g_pc}}\\
        \subfloat[Prob.\ no birth yet]{\includegraphics[width=\textwidth]{spell2_g1_low_urban_g_s}} 
        \captionsetup[subfigure]{labelformat=parens}
    \end{minipage}
} 
\setcounter{subfigure}{-1}
\subfloat[1985-1994 (N=2,232)]{
    \begin{minipage}{0.26\textwidth}
        \captionsetup[subfigure]{labelformat=empty,position=top,captionskip=-1pt,farskip=-0.5pt}
        \subfloat[Prob.\ boy (\%)]{\includegraphics[width=\textwidth]{spell2_g2_low_urban_g_pc}}\\
        \subfloat[Prob.\ no birth yet]{\includegraphics[width=\textwidth]{spell2_g2_low_urban_g_s}} 
        \captionsetup[subfigure]{labelformat=parens}
    \end{minipage}
} 
\setcounter{subfigure}{0}
\subfloat[1995-2006 (N=952)]{
    \begin{minipage}{0.26\textwidth}
        \captionsetup[subfigure]{labelformat=empty,position=top,captionskip=-1pt,farskip=-0.5pt}
        \subfloat[Prob.\ boy (\%)]{\includegraphics[width=\textwidth]{spell2_g3_low_urban_g_pc}}\\
        \subfloat[Prob.\ no birth yet]{\includegraphics[width=\textwidth]{spell2_g3_low_urban_g_s}} 
        \captionsetup[subfigure]{labelformat=parens}
    \end{minipage}
} 
\caption*{First child a boy}
\setcounter{subfigure}{1}
\subfloat[1972-1984 (N=1,754)]{
    \begin{minipage}{0.26\textwidth}
        \captionsetup[subfigure]{labelformat=empty,position=top,captionskip=-1pt,farskip=-0.5pt}
        \subfloat[Prob. boy (\%)]{\includegraphics[width=\textwidth]{spell2_g1_low_urban_b_pc}}\\
        \subfloat[Prob. no birth yet]{\includegraphics[width=\textwidth]{spell2_g1_low_urban_b_s}} 
        \captionsetup[subfigure]{labelformat=parens}
    \end{minipage}
} 
\setcounter{subfigure}{2}
\subfloat[1985-1994 (N=2,411)]{
    \begin{minipage}{0.26\textwidth}
        \captionsetup[subfigure]{labelformat=empty,position=top,captionskip=-1pt,farskip=-0.5pt}
        \subfloat[Prob. boy (\%)]{\includegraphics[width=\textwidth]{spell2_g2_low_urban_b_pc}}\\
        \subfloat[Prob. no birth yet]{\includegraphics[width=\textwidth]{spell2_g2_low_urban_b_s}}
        \captionsetup[subfigure]{labelformat=parens}
    \end{minipage}
}
\setcounter{subfigure}{3}
\subfloat[1995-2006 (N=930)]{
    \begin{minipage}{0.26\textwidth}
        \captionsetup[subfigure]{labelformat=empty,position=top,captionskip=-1pt,farskip=-0.5pt}
        \subfloat[Prob. boy (\%)]{\includegraphics[width=\textwidth]{spell2_g3_low_urban_b_pc}}\\
        \subfloat[Prob. no birth yet]{\includegraphics[width=\textwidth]{spell2_g3_low_urban_b_s}}
        \captionsetup[subfigure]{labelformat=parens}
    \end{minipage}
}
\caption{Predicted probability of having a boy and probability of
no birth yet from nine months after first birth for urban 
women with no education by month beginning 9 months after prior birth. 
Predictions based on age 18 at first birth.
Left column shows results prior to sex selection available, middle column before
sex selection illegal and right column after sex selection illegal.
N indicates the number of women in the relevant group in the underlying samples.
}
\label{fig:results_spell2_low_urban}
\end{figure}


% SPELL 2 - rural - LOW
\begin{figure}
\centering
\caption*{First child a girl}
\setcounter{subfigure}{-2}
\subfloat[1972-1984 (N=8,426)]{
    \begin{minipage}{0.31\textwidth}
        \captionsetup[subfigure]{labelformat=empty,position=top,captionskip=-1pt,farskip=-0.5pt}
        \subfloat[Prob.\ boy (\%)]{\includegraphics[width=\textwidth]{spell2_g1_low_rural_g_pc}}\\
        \subfloat[Prob.\ no birth yet]{\includegraphics[width=\textwidth]{spell2_g1_low_rural_g_s}} 
        \captionsetup[subfigure]{labelformat=parens}
    \end{minipage}
} 
\setcounter{subfigure}{-1}
\subfloat[1985-1994 (N=11,481)]{
    \begin{minipage}{0.31\textwidth}
        \captionsetup[subfigure]{labelformat=empty,position=top,captionskip=-1pt,farskip=-0.5pt}
        \subfloat[Prob.\ boy (\%)]{\includegraphics[width=\textwidth]{spell2_g2_low_rural_g_pc}}\\
        \subfloat[Prob.\ no birth yet]{\includegraphics[width=\textwidth]{spell2_g2_low_rural_g_s}} 
        \captionsetup[subfigure]{labelformat=parens}
    \end{minipage}
} 
\setcounter{subfigure}{0}
\subfloat[1995-2006 (N=3,686)]{
    \begin{minipage}{0.31\textwidth}
        \captionsetup[subfigure]{labelformat=empty,position=top,captionskip=-1pt,farskip=-0.5pt}
        \subfloat[Prob.\ boy (\%)]{\includegraphics[width=\textwidth]{spell2_g3_low_rural_g_pc}}\\
        \subfloat[Prob.\ no birth yet]{\includegraphics[width=\textwidth]{spell2_g3_low_rural_g_s}} 
        \captionsetup[subfigure]{labelformat=parens}
    \end{minipage}
} 
\caption*{First child a boy}
\setcounter{subfigure}{1}
\subfloat[1972-1984 (N=9,395)]{
    \begin{minipage}{0.31\textwidth}
        \captionsetup[subfigure]{labelformat=empty,position=top,captionskip=-1pt,farskip=-0.5pt}
        \subfloat[Prob. boy (\%)]{\includegraphics[width=\textwidth]{spell2_g1_low_rural_b_pc}}\\
        \subfloat[Prob. no birth yet]{\includegraphics[width=\textwidth]{spell2_g1_low_rural_b_s}} 
        \captionsetup[subfigure]{labelformat=parens}
    \end{minipage}
} 
\setcounter{subfigure}{2}
\subfloat[1985-1994 (N=12,127)]{
    \begin{minipage}{0.31\textwidth}
        \captionsetup[subfigure]{labelformat=empty,position=top,captionskip=-1pt,farskip=-0.5pt}
        \subfloat[Prob. boy (\%)]{\includegraphics[width=\textwidth]{spell2_g2_low_rural_b_pc}}\\
        \subfloat[Prob. no birth yet]{\includegraphics[width=\textwidth]{spell2_g2_low_rural_b_s}}
        \captionsetup[subfigure]{labelformat=parens}
    \end{minipage}
}
\setcounter{subfigure}{3}
\subfloat[1995-2006 (N=3,860)]{
    \begin{minipage}{0.31\textwidth}
        \captionsetup[subfigure]{labelformat=empty,position=top,captionskip=-1pt,farskip=-0.5pt}
        \subfloat[Prob. boy (\%)]{\includegraphics[width=\textwidth]{spell2_g3_low_rural_b_pc}}\\
        \subfloat[Prob. no birth yet]{\includegraphics[width=\textwidth]{spell2_g3_low_rural_b_s}}
        \captionsetup[subfigure]{labelformat=parens}
    \end{minipage}
}
\caption{Predicted probability of having a boy and probability of
no birth yet from nine months after first birth for rural 
women with no education by month beginning 9 months after prior birth. 
Predictions based on age 18 at first birth.
Left column shows results prior to sex selection available, middle column before
sex selection illegal and right column after sex selection illegal.
N indicates the number of women in the relevant group in the underlying samples.
}
\label{fig:results_spell2_low_rural}
\end{figure}







% \begin{figure}[htpb]
% \centering
% \caption*{First child a girl}
% \setcounter{subfigure}{-2}
% \subfloat[1972--1984 (N=2,689)]{
%     \begin{minipage}{0.31\textwidth}
%         \captionsetup[subfigure]{labelformat=empty,position=top,captionskip=-1pt,farskip=-0.5pt}
%         \subfloat[Prob.\ boy (\%)]{\includegraphics[width=\textwidth]{spell2_g1_high_urban_g_pc}}\\
%         \subfloat[Prob.\ no birth yet]{\includegraphics[width=\textwidth]{spell2_g1_high_urban_g_s}} 
%         \captionsetup[subfigure]{labelformat=parens}
%     \end{minipage}
% } 
% \setcounter{subfigure}{-1}
% \subfloat[1985--1994 (N=4,869)]{
%     \begin{minipage}{0.31\textwidth}
%         \captionsetup[subfigure]{labelformat=empty,position=top,captionskip=-1pt,farskip=-0.5pt}
%         \subfloat[Prob. boy (\%)]{\includegraphics[width=\textwidth]{spell2_g2_high_urban_g_pc}}\\
%         \subfloat[Prob. no birth yet]{\includegraphics[width=\textwidth]{spell2_g2_high_urban_g_s}}
%         \captionsetup[subfigure]{labelformat=parens}
%     \end{minipage}
% }
% \setcounter{subfigure}{0}
% \subfloat[1995--2006 (N=3,774)]{
%     \begin{minipage}{0.31\textwidth}
%         \captionsetup[subfigure]{labelformat=empty,position=top,captionskip=-1pt,farskip=-0.5pt}
%         \subfloat[Prob. boy (\%)]{\includegraphics[width=\textwidth]{spell2_g3_high_urban_g_pc}}\\
%         \subfloat[Prob. no birth yet]{\includegraphics[width=\textwidth]{spell2_g3_high_urban_g_s}}
%         \captionsetup[subfigure]{labelformat=parens}
%     \end{minipage}
% }
% \caption*{First child a boy}
% \setcounter{subfigure}{1}
% \subfloat[1972--1984 (N=2,806)]{
%     \begin{minipage}{0.31\textwidth}
%         \captionsetup[subfigure]{labelformat=empty,position=top,captionskip=-1pt,farskip=-0.5pt}
%         \subfloat[Prob. boy (\%)]{\includegraphics[width=\textwidth]{spell2_g1_high_urban_b_pc}}\\
%         \subfloat[Prob. no birth yet]{\includegraphics[width=\textwidth]{spell2_g1_high_urban_b_s}} 
%         \captionsetup[subfigure]{labelformat=parens}
%     \end{minipage}
% } 
% \setcounter{subfigure}{2}
% \subfloat[1985--1994 (N=5,246)]{
%     \begin{minipage}{0.31\textwidth}
%         \captionsetup[subfigure]{labelformat=empty,position=top,captionskip=-1pt,farskip=-0.5pt}
%         \subfloat[Prob. boy (\%)]{\includegraphics[width=\textwidth]{spell2_g2_high_urban_b_pc}}\\
%         \subfloat[Prob. no birth yet]{\includegraphics[width=\textwidth]{spell2_g2_high_urban_b_s}}
%         \captionsetup[subfigure]{labelformat=parens}
%     \end{minipage}
% }
% \setcounter{subfigure}{3}
% \subfloat[1995--2006 (N=3,969)]{
%     \begin{minipage}{0.31\textwidth}
%         \captionsetup[subfigure]{labelformat=empty,position=top,captionskip=-1pt,farskip=-0.5pt}
%         \subfloat[Prob. boy (\%)]{\includegraphics[width=\textwidth]{spell2_g3_high_urban_b_pc}}\\
%         \subfloat[Prob. no birth yet]{\includegraphics[width=\textwidth]{spell2_g3_high_urban_b_s}}
%         \captionsetup[subfigure]{labelformat=parens}
%     \end{minipage}
% }
% \caption{Predicted probability of having a boy and probability of
% no birth yet from nine months after first birth for urban 
% women with 8 or more years of education by month beginning 9 months after prior birth. 
% Predictions based on age 22 at first birth.
% Left column shows results prior to sex selection available, middle column before
% sex selection illegal and right column after sex selection illegal.
% N indicates the number of women in the relevant group in the underlying samples.
% }
% \label{fig:results_spell2_high_urban}
% \end{figure}


\begin{figure}[htpb]
\centering
\caption*{First child a girl}
\setcounter{subfigure}{-2}
\subfloat[1972--1984 (N=1,305)]{
    \begin{minipage}{0.31\textwidth}
        \captionsetup[subfigure]{labelformat=empty,position=top,captionskip=-1pt,farskip=-0.5pt}
        \subfloat[Prob.\ boy (\%)]{\includegraphics[width=\textwidth]{spell2_g1_high_rural_g_pc}}\\
        \subfloat[Prob.\ no birth yet]{\includegraphics[width=\textwidth]{spell2_g1_high_rural_g_s}} 
        \captionsetup[subfigure]{labelformat=parens}
    \end{minipage}
} 
\setcounter{subfigure}{-1}
\subfloat[1985--1994 (N=3,105)]{
    \begin{minipage}{0.31\textwidth}
        \captionsetup[subfigure]{labelformat=empty,position=top,captionskip=-1pt,farskip=-0.5pt}
        \subfloat[Prob. boy (\%)]{\includegraphics[width=\textwidth]{spell2_g2_high_rural_g_pc}}\\
        \subfloat[Prob. no birth yet]{\includegraphics[width=\textwidth]{spell2_g2_high_rural_g_s}}
        \captionsetup[subfigure]{labelformat=parens}
    \end{minipage}
}
\setcounter{subfigure}{0}
\subfloat[1995--2006 (N=2,823)]{
    \begin{minipage}{0.31\textwidth}
        \captionsetup[subfigure]{labelformat=empty,position=top,captionskip=-1pt,farskip=-0.5pt}
        \subfloat[Prob. boy (\%)]{\includegraphics[width=\textwidth]{spell2_g3_high_rural_g_pc}}\\
        \subfloat[Prob. no birth yet]{\includegraphics[width=\textwidth]{spell2_g3_high_rural_g_s}}
        \captionsetup[subfigure]{labelformat=parens}
    \end{minipage}
}
\caption*{First child a boy}
\setcounter{subfigure}{1}
\subfloat[1972--1984 (N=1,404)]{
    \begin{minipage}{0.31\textwidth}
        \captionsetup[subfigure]{labelformat=empty,position=top,captionskip=-1pt,farskip=-0.5pt}
        \subfloat[Prob. boy (\%)]{\includegraphics[width=\textwidth]{spell2_g1_high_rural_b_pc}}\\
        \subfloat[Prob. no birth yet]{\includegraphics[width=\textwidth]{spell2_g1_high_rural_b_s}} 
        \captionsetup[subfigure]{labelformat=parens}
    \end{minipage}
} 
\setcounter{subfigure}{2}
\subfloat[1985--1994 (N=3,381)]{
    \begin{minipage}{0.31\textwidth}
        \captionsetup[subfigure]{labelformat=empty,position=top,captionskip=-1pt,farskip=-0.5pt}
        \subfloat[Prob. boy (\%)]{\includegraphics[width=\textwidth]{spell2_g2_high_rural_b_pc}}\\
        \subfloat[Prob. no birth yet]{\includegraphics[width=\textwidth]{spell2_g2_high_rural_b_s}}
        \captionsetup[subfigure]{labelformat=parens}
    \end{minipage}
}
\setcounter{subfigure}{3}
\subfloat[1995--2006 (N=3,044)]{
    \begin{minipage}{0.31\textwidth}
        \captionsetup[subfigure]{labelformat=empty,position=top,captionskip=-1pt,farskip=-0.5pt}
        \subfloat[Prob. boy (\%)]{\includegraphics[width=\textwidth]{spell2_g3_high_rural_b_pc}}\\
        \subfloat[Prob. no birth yet]{\includegraphics[width=\textwidth]{spell2_g3_high_rural_b_s}}
        \captionsetup[subfigure]{labelformat=parens}
    \end{minipage}
}
\caption{Predicted probability of having a boy and probability of
no birth yet from nine months after first birth for rural
women with 8 or more years of education by month beginning 9 months after prior birth. 
Predictions based on age 22 at first birth.
Left column shows results prior to sex selection available, middle column before
sex selection illegal and right column after sex selection illegal.
N indicates the number of women in the relevant group in the underlying samples.
}
\label{fig:results_spell2_high_rural}
\end{figure}




\clearpage

\subsection{Third Spell}


% low education

\begin{figure}[htpb]
\centering
\caption*{First two children girls}
\setcounter{subfigure}{-2}
\subfloat[1972--1984 (N=493)]{
    \begin{minipage}{0.26\textwidth}
        \captionsetup[subfigure]{labelformat=empty,position=top,captionskip=-1pt,farskip=-0.5pt}
        \subfloat[Prob.\ boy (\%)]{\includegraphics[width=\textwidth]{spell3_g1_low_urban_gg_pc}}\\
        \subfloat[Prob.\ no birth yet]{\includegraphics[width=\textwidth]{spell3_g1_low_urban_gg_s}} 
        \captionsetup[subfigure]{labelformat=parens}
    \end{minipage}
} 
\setcounter{subfigure}{-1}
\subfloat[1985--1994 (N=996)]{
    \begin{minipage}{0.26\textwidth}
        \captionsetup[subfigure]{labelformat=empty,position=top,captionskip=-1pt,farskip=-0.5pt}
        \subfloat[Prob. boy (\%)]{\includegraphics[width=\textwidth]{spell3_g2_low_urban_gg_pc}}\\
        \subfloat[Prob. no birth yet]{\includegraphics[width=\textwidth]{spell3_g2_low_urban_gg_s}}
        \captionsetup[subfigure]{labelformat=parens}
    \end{minipage}
}
\setcounter{subfigure}{0}
\subfloat[1995--2006 (N=475)]{
    \begin{minipage}{0.26\textwidth}
        \captionsetup[subfigure]{labelformat=empty,position=top,captionskip=-1pt,farskip=-0.5pt}
        \subfloat[Prob. boy (\%)]{\includegraphics[width=\textwidth]{spell3_g3_low_urban_gg_pc}}\\
        \subfloat[Prob. no birth yet]{\includegraphics[width=\textwidth]{spell3_g3_low_urban_gg_s}}
        \captionsetup[subfigure]{labelformat=parens}
    \end{minipage}
}
\caption*{First two children one boy and one girl}
\setcounter{subfigure}{1}
\subfloat[1972--1984 (N=1,083)]{
    \begin{minipage}{0.26\textwidth}
        \captionsetup[subfigure]{labelformat=empty,position=top,captionskip=-1pt,farskip=-0.5pt}
        \subfloat[Prob. boy (\%)]{\includegraphics[width=\textwidth]{spell3_g1_low_urban_bg_pc}}\\
        \subfloat[Prob. no birth yet]{\includegraphics[width=\textwidth]{spell3_g1_low_urban_bg_s}} 
        \captionsetup[subfigure]{labelformat=parens}
    \end{minipage}
} 
\setcounter{subfigure}{2}
\subfloat[1985--1994 (N=2,141)]{
    \begin{minipage}{0.26\textwidth}
        \captionsetup[subfigure]{labelformat=empty,position=top,captionskip=-1pt,farskip=-0.5pt}
        \subfloat[Prob. boy (\%)]{\includegraphics[width=\textwidth]{spell3_g2_low_urban_bg_pc}}\\
        \subfloat[Prob. no birth yet]{\includegraphics[width=\textwidth]{spell3_g2_low_urban_bg_s}}
        \captionsetup[subfigure]{labelformat=parens}
    \end{minipage}
}
\setcounter{subfigure}{3}
\subfloat[1995--2006 (N=950)]{
    \begin{minipage}{0.26\textwidth}
        \captionsetup[subfigure]{labelformat=empty,position=top,captionskip=-1pt,farskip=-0.5pt}
        \subfloat[Prob. boy (\%)]{\includegraphics[width=\textwidth]{spell3_g3_low_urban_bg_pc}}\\
        \subfloat[Prob. no birth yet]{\includegraphics[width=\textwidth]{spell3_g3_low_urban_bg_s}}
        \captionsetup[subfigure]{labelformat=parens}
    \end{minipage}
}
\caption{Predicted probability of having a boy and probability of
no birth yet from nine months after second birth for urban 
women with no of education by month beginning 9 months after prior birth. 
Predictions based on age 21 at second birth.
Left column shows results prior to sex selection available, middle column before
sex selection illegal and right column after sex selection illegal.
N indicates the number of women in the relevant group in the underlying samples.
}
\label{fig:results_spell3_low_urban}
\end{figure}


\begin{figure}[htpb]
\centering
\ContinuedFloat
\caption*{First two children boys}
\setcounter{subfigure}{4}
\subfloat[1972--1984 (N=569)]{
    \begin{minipage}{0.31\textwidth}
        \captionsetup[subfigure]{labelformat=empty,position=top,captionskip=-1pt,farskip=-0.5pt}
        \subfloat[Prob. boy (\%)]{\includegraphics[width=\textwidth]{spell3_g1_low_urban_bb_pc}}\\
        \subfloat[Prob. no birth yet]{\includegraphics[width=\textwidth]{spell3_g1_low_urban_bb_s}} 
        \captionsetup[subfigure]{labelformat=parens}
    \end{minipage}
} 
\setcounter{subfigure}{5}
\subfloat[1985--1994 (N=1,124)]{
    \begin{minipage}{0.31\textwidth}
        \captionsetup[subfigure]{labelformat=empty,position=top,captionskip=-1pt,farskip=-0.5pt}
        \subfloat[Prob. boy (\%)]{\includegraphics[width=\textwidth]{spell3_g2_low_urban_bb_pc}}\\
        \subfloat[Prob. no birth yet]{\includegraphics[width=\textwidth]{spell3_g2_low_urban_bb_s}}
        \captionsetup[subfigure]{labelformat=parens}
    \end{minipage}
}
\setcounter{subfigure}{6}
\subfloat[1995--2006 (N=426)]{
    \begin{minipage}{0.31\textwidth}
        \captionsetup[subfigure]{labelformat=empty,position=top,captionskip=-1pt,farskip=-0.5pt}
        \subfloat[Prob. boy (\%)]{\includegraphics[width=\textwidth]{spell3_g3_low_urban_bb_pc}}\\
        \subfloat[Prob. no birth yet]{\includegraphics[width=\textwidth]{spell3_g3_low_urban_bb_s}}
        \captionsetup[subfigure]{labelformat=parens}
    \end{minipage}
}
% \label{fig:results_spell3_low_urban}
\caption{(Continued) Predicted probability of having a boy and probability of
no birth yet from nine months after second birth for urban 
women with no of education by month beginning 9 months after prior birth. 
Predictions based on age 21 at second birth.
Left column shows results prior to sex selection available, middle column before
sex selection illegal and right column after sex selection illegal.
N indicates the number of women in the relevant group in the underlying samples.
}
\end{figure}


\begin{figure}[htpb]
\centering
\caption*{First two children girls}
\setcounter{subfigure}{-2}
\subfloat[1972--1984 (N=2,599)]{
    \begin{minipage}{0.31\textwidth}
        \captionsetup[subfigure]{labelformat=empty,position=top,captionskip=-1pt,farskip=-0.5pt}
        \subfloat[Prob.\ boy (\%)]{\includegraphics[width=\textwidth]{spell3_g1_low_rural_gg_pc}}\\
        \subfloat[Prob.\ no birth yet]{\includegraphics[width=\textwidth]{spell3_g1_low_rural_gg_s}} 
        \captionsetup[subfigure]{labelformat=parens}
    \end{minipage}
} 
\setcounter{subfigure}{-1}
\subfloat[1985--1994 (N=5,062)]{
    \begin{minipage}{0.31\textwidth}
        \captionsetup[subfigure]{labelformat=empty,position=top,captionskip=-1pt,farskip=-0.5pt}
        \subfloat[Prob. boy (\%)]{\includegraphics[width=\textwidth]{spell3_g2_low_rural_gg_pc}}\\
        \subfloat[Prob. no birth yet]{\includegraphics[width=\textwidth]{spell3_g2_low_rural_gg_s}}
        \captionsetup[subfigure]{labelformat=parens}
    \end{minipage}
}
\setcounter{subfigure}{0}
\subfloat[1995--2006 (N=1,879)]{
    \begin{minipage}{0.31\textwidth}
        \captionsetup[subfigure]{labelformat=empty,position=top,captionskip=-1pt,farskip=-0.5pt}
        \subfloat[Prob. boy (\%)]{\includegraphics[width=\textwidth]{spell3_g3_low_rural_gg_pc}}\\
        \subfloat[Prob. no birth yet]{\includegraphics[width=\textwidth]{spell3_g3_low_rural_gg_s}}
        \captionsetup[subfigure]{labelformat=parens}
    \end{minipage}
}
\caption*{First two children one boy and one girl}
\setcounter{subfigure}{1}
\subfloat[1972--1984 (N=5,382)]{
    \begin{minipage}{0.31\textwidth}
        \captionsetup[subfigure]{labelformat=empty,position=top,captionskip=-1pt,farskip=-0.5pt}
        \subfloat[Prob. boy (\%)]{\includegraphics[width=\textwidth]{spell3_g1_low_rural_bg_pc}}\\
        \subfloat[Prob. no birth yet]{\includegraphics[width=\textwidth]{spell3_g1_low_rural_bg_s}} 
        \captionsetup[subfigure]{labelformat=parens}
    \end{minipage}
} 
\setcounter{subfigure}{2}
\subfloat[1985--1994 (N=10,473)]{
    \begin{minipage}{0.31\textwidth}
        \captionsetup[subfigure]{labelformat=empty,position=top,captionskip=-1pt,farskip=-0.5pt}
        \subfloat[Prob. boy (\%)]{\includegraphics[width=\textwidth]{spell3_g2_low_rural_bg_pc}}\\
        \subfloat[Prob. no birth yet]{\includegraphics[width=\textwidth]{spell3_g2_low_rural_bg_s}}
        \captionsetup[subfigure]{labelformat=parens}
    \end{minipage}
}
\setcounter{subfigure}{3}
\subfloat[1995--2006 (N=3,779)]{
    \begin{minipage}{0.31\textwidth}
        \captionsetup[subfigure]{labelformat=empty,position=top,captionskip=-1pt,farskip=-0.5pt}
        \subfloat[Prob. boy (\%)]{\includegraphics[width=\textwidth]{spell3_g3_low_rural_bg_pc}}\\
        \subfloat[Prob. no birth yet]{\includegraphics[width=\textwidth]{spell3_g3_low_rural_bg_s}}
        \captionsetup[subfigure]{labelformat=parens}
    \end{minipage}
}
\caption{Predicted probability of having a boy and probability of
no birth yet from nine months after second birth for rural
women with no of education by month beginning 9 months after prior birth. 
Predictions based on age 21 at second birth.
Left column shows results prior to sex selection available, middle column before
sex selection illegal and right column after sex selection illegal.
N indicates the number of women in the relevant group in the underlying samples.
}
\label{fig:results_spell3_low_rural}
\end{figure}


\begin{figure}[htpb]
\ContinuedFloat
\centering
\caption*{First two children boys}
\setcounter{subfigure}{4}
\subfloat[1972--1984 (N=3,070)]{
    \begin{minipage}{0.31\textwidth}
        \captionsetup[subfigure]{labelformat=empty,position=top,captionskip=-1pt,farskip=-0.5pt}
        \subfloat[Prob. boy (\%)]{\includegraphics[width=\textwidth]{spell3_g1_low_rural_bb_pc}}\\
        \subfloat[Prob. no birth yet]{\includegraphics[width=\textwidth]{spell3_g1_low_rural_bb_s}} 
        \captionsetup[subfigure]{labelformat=parens}
    \end{minipage}
} 
\setcounter{subfigure}{5}
\subfloat[1985--1994 (N=5,339)]{
    \begin{minipage}{0.31\textwidth}
        \captionsetup[subfigure]{labelformat=empty,position=top,captionskip=-1pt,farskip=-0.5pt}
        \subfloat[Prob. boy (\%)]{\includegraphics[width=\textwidth]{spell3_g2_low_rural_bb_pc}}\\
        \subfloat[Prob. no birth yet]{\includegraphics[width=\textwidth]{spell3_g2_low_rural_bb_s}}
        \captionsetup[subfigure]{labelformat=parens}
    \end{minipage}
}
\setcounter{subfigure}{6}
\subfloat[1995--2006 (N=1,846)]{
    \begin{minipage}{0.31\textwidth}
        \captionsetup[subfigure]{labelformat=empty,position=top,captionskip=-1pt,farskip=-0.5pt}
        \subfloat[Prob. boy (\%)]{\includegraphics[width=\textwidth]{spell3_g3_low_rural_bb_pc}}\\
        \subfloat[Prob. no birth yet]{\includegraphics[width=\textwidth]{spell3_g3_low_rural_bb_s}}
        \captionsetup[subfigure]{labelformat=parens}
    \end{minipage}
}
\caption{(Continued) Predicted probability of having a boy and probability of
no birth yet from nine months after second birth for rural
women with no of education by month beginning 9 months after prior birth. 
Predictions based on age 21 at second birth.
Left column shows results prior to sex selection available, middle column before
sex selection illegal and right column after sex selection illegal.
N indicates the number of women in the relevant group in the underlying samples.
}
% \label{fig:results_spell3_low_rural}
\end{figure}



% High education

\begin{figure}[htpb]
\centering
\caption*{First two children girls}
\setcounter{subfigure}{-2}
\subfloat[1972--1984 (N=709)]{
    \begin{minipage}{0.31\textwidth}
        \captionsetup[subfigure]{labelformat=empty,position=top,captionskip=-1pt,farskip=-0.5pt}
        \subfloat[Prob.\ boy (\%)]{\includegraphics[width=\textwidth]{spell3_g1_high_urban_gg_pc}}\\
        \subfloat[Prob.\ no birth yet]{\includegraphics[width=\textwidth]{spell3_g1_high_urban_gg_s}} 
        \captionsetup[subfigure]{labelformat=parens}
    \end{minipage}
} 
\setcounter{subfigure}{-1}
\subfloat[1985--1994 (N=1,654)]{
    \begin{minipage}{0.31\textwidth}
        \captionsetup[subfigure]{labelformat=empty,position=top,captionskip=-1pt,farskip=-0.5pt}
        \subfloat[Prob. boy (\%)]{\includegraphics[width=\textwidth]{spell3_g2_high_urban_gg_pc}}\\
        \subfloat[Prob. no birth yet]{\includegraphics[width=\textwidth]{spell3_g2_high_urban_gg_s}}
        \captionsetup[subfigure]{labelformat=parens}
    \end{minipage}
}
\setcounter{subfigure}{0}
\subfloat[1995--2006 (N=1,000)]{
    \begin{minipage}{0.31\textwidth}
        \captionsetup[subfigure]{labelformat=empty,position=top,captionskip=-1pt,farskip=-0.5pt}
        \subfloat[Prob. boy (\%)]{\includegraphics[width=\textwidth]{spell3_g3_high_urban_gg_pc}}\\
        \subfloat[Prob. no birth yet]{\includegraphics[width=\textwidth]{spell3_g3_high_urban_gg_s}}
        \captionsetup[subfigure]{labelformat=parens}
    \end{minipage}
}
\caption*{First two children one boy and one girl}
\setcounter{subfigure}{1}
\subfloat[1972--1984 (N=1,409)]{
    \begin{minipage}{0.31\textwidth}
        \captionsetup[subfigure]{labelformat=empty,position=top,captionskip=-1pt,farskip=-0.5pt}
        \subfloat[Prob. boy (\%)]{\includegraphics[width=\textwidth]{spell3_g1_high_urban_bg_pc}}\\
        \subfloat[Prob. no birth yet]{\includegraphics[width=\textwidth]{spell3_g1_high_urban_bg_s}} 
        \captionsetup[subfigure]{labelformat=parens}
    \end{minipage}
} 
\setcounter{subfigure}{2}
\subfloat[1985--1994 (N=3,460)]{
    \begin{minipage}{0.31\textwidth}
        \captionsetup[subfigure]{labelformat=empty,position=top,captionskip=-1pt,farskip=-0.5pt}
        \subfloat[Prob. boy (\%)]{\includegraphics[width=\textwidth]{spell3_g2_high_urban_bg_pc}}\\
        \subfloat[Prob. no birth yet]{\includegraphics[width=\textwidth]{spell3_g2_high_urban_bg_s}}
        \captionsetup[subfigure]{labelformat=parens}
    \end{minipage}
}
\setcounter{subfigure}{3}
\subfloat[1995--2006 (N=2,357)]{
    \begin{minipage}{0.31\textwidth}
        \captionsetup[subfigure]{labelformat=empty,position=top,captionskip=-1pt,farskip=-0.5pt}
        \subfloat[Prob. boy (\%)]{\includegraphics[width=\textwidth]{spell3_g3_high_urban_bg_pc}}\\
        \subfloat[Prob. no birth yet]{\includegraphics[width=\textwidth]{spell3_g3_high_urban_bg_s}}
        \captionsetup[subfigure]{labelformat=parens}
    \end{minipage}
}
\caption{Predicted probability of having a boy and probability of
no birth yet from nine months after second birth for urban 
women with 8 or more years of education by month beginning 9 months after prior birth. 
Predictions based on age 24 at second birth.
Left column shows results prior to sex selection available, middle column before
sex selection illegal and right column after sex selection illegal.
N indicates the number of women in the relevant group in the underlying samples.
}
\label{fig:results_spell3_high_urban}
\end{figure}


\begin{figure}[htpb]
\centering
\ContinuedFloat
\caption*{First two children boys}
\setcounter{subfigure}{4}
\subfloat[1972--1984 (N=780)]{
    \begin{minipage}{0.31\textwidth}
        \captionsetup[subfigure]{labelformat=empty,position=top,captionskip=-1pt,farskip=-0.5pt}
        \subfloat[Prob. boy (\%)]{\includegraphics[width=\textwidth]{spell3_g1_high_urban_bb_pc}}\\
        \subfloat[Prob. no birth yet]{\includegraphics[width=\textwidth]{spell3_g1_high_urban_bb_s}} 
        \captionsetup[subfigure]{labelformat=parens}
    \end{minipage}
} 
\setcounter{subfigure}{5}
\subfloat[1985--1994 (N=1,660)]{
    \begin{minipage}{0.31\textwidth}
        \captionsetup[subfigure]{labelformat=empty,position=top,captionskip=-1pt,farskip=-0.5pt}
        \subfloat[Prob. boy (\%)]{\includegraphics[width=\textwidth]{spell3_g2_high_urban_bb_pc}}\\
        \subfloat[Prob. no birth yet]{\includegraphics[width=\textwidth]{spell3_g2_high_urban_bb_s}}
        \captionsetup[subfigure]{labelformat=parens}
    \end{minipage}
}
\setcounter{subfigure}{6}
\subfloat[1995--2006 (N=1,098)]{
    \begin{minipage}{0.31\textwidth}
        \captionsetup[subfigure]{labelformat=empty,position=top,captionskip=-1pt,farskip=-0.5pt}
        \subfloat[Prob. boy (\%)]{\includegraphics[width=\textwidth]{spell3_g3_high_urban_bb_pc}}\\
        \subfloat[Prob. no birth yet]{\includegraphics[width=\textwidth]{spell3_g3_high_urban_bb_s}}
        \captionsetup[subfigure]{labelformat=parens}
    \end{minipage}
}
% \label{fig:results_spell3_high_urban}
\caption{(Continued) Predicted probability of having a boy and probability of
no birth yet from nine months after second birth for urban 
women with 8 or more years of education by month beginning 9 months after prior birth. 
Predictions based on age 24 at second birth.
Left column shows results prior to sex selection available, middle column before
sex selection illegal and right column after sex selection illegal.
N indicates the number of women in the relevant group in the underlying samples.
}
\end{figure}


\begin{figure}[htpb]
\centering
\caption*{First two children girls}
\setcounter{subfigure}{-2}
\subfloat[1972--1984 (N=348)]{
    \begin{minipage}{0.31\textwidth}
        \captionsetup[subfigure]{labelformat=empty,position=top,captionskip=-1pt,farskip=-0.5pt}
        \subfloat[Prob.\ boy (\%)]{\includegraphics[width=\textwidth]{spell3_g1_high_rural_gg_pc}}\\
        \subfloat[Prob.\ no birth yet]{\includegraphics[width=\textwidth]{spell3_g1_high_rural_gg_s}} 
        \captionsetup[subfigure]{labelformat=parens}
    \end{minipage}
} 
\setcounter{subfigure}{-1}
\subfloat[1985--1994 (N=982)]{
    \begin{minipage}{0.31\textwidth}
        \captionsetup[subfigure]{labelformat=empty,position=top,captionskip=-1pt,farskip=-0.5pt}
        \subfloat[Prob. boy (\%)]{\includegraphics[width=\textwidth]{spell3_g2_high_rural_gg_pc}}\\
        \subfloat[Prob. no birth yet]{\includegraphics[width=\textwidth]{spell3_g2_high_rural_gg_s}}
        \captionsetup[subfigure]{labelformat=parens}
    \end{minipage}
}
\setcounter{subfigure}{0}
\subfloat[1995--2006 (N=863)]{
    \begin{minipage}{0.31\textwidth}
        \captionsetup[subfigure]{labelformat=empty,position=top,captionskip=-1pt,farskip=-0.5pt}
        \subfloat[Prob. boy (\%)]{\includegraphics[width=\textwidth]{spell3_g3_high_rural_gg_pc}}\\
        \subfloat[Prob. no birth yet]{\includegraphics[width=\textwidth]{spell3_g3_high_rural_gg_s}}
        \captionsetup[subfigure]{labelformat=parens}
    \end{minipage}
}
\caption*{First two children one boy and one girl}
\setcounter{subfigure}{1}
\subfloat[1972--1984 (N=700)]{
    \begin{minipage}{0.31\textwidth}
        \captionsetup[subfigure]{labelformat=empty,position=top,captionskip=-1pt,farskip=-0.5pt}
        \subfloat[Prob. boy (\%)]{\includegraphics[width=\textwidth]{spell3_g1_high_rural_bg_pc}}\\
        \subfloat[Prob. no birth yet]{\includegraphics[width=\textwidth]{spell3_g1_high_rural_bg_s}} 
        \captionsetup[subfigure]{labelformat=parens}
    \end{minipage}
} 
\setcounter{subfigure}{2}
\subfloat[1985--1994 (N=1,987)]{
    \begin{minipage}{0.31\textwidth}
        \captionsetup[subfigure]{labelformat=empty,position=top,captionskip=-1pt,farskip=-0.5pt}
        \subfloat[Prob. boy (\%)]{\includegraphics[width=\textwidth]{spell3_g2_high_rural_bg_pc}}\\
        \subfloat[Prob. no birth yet]{\includegraphics[width=\textwidth]{spell3_g2_high_rural_bg_s}}
        \captionsetup[subfigure]{labelformat=parens}
    \end{minipage}
}
\setcounter{subfigure}{3}
\subfloat[1995--2006 (N=1,819)]{
    \begin{minipage}{0.31\textwidth}
        \captionsetup[subfigure]{labelformat=empty,position=top,captionskip=-1pt,farskip=-0.5pt}
        \subfloat[Prob. boy (\%)]{\includegraphics[width=\textwidth]{spell3_g3_high_rural_bg_pc}}\\
        \subfloat[Prob. no birth yet]{\includegraphics[width=\textwidth]{spell3_g3_high_rural_bg_s}}
        \captionsetup[subfigure]{labelformat=parens}
    \end{minipage}
}
\caption{Predicted probability of having a boy and probability of
no birth yet from nine months after second birth for rural
women with 8 or more years of education by month beginning 9 months after prior birth. 
Predictions based on age 24 at second birth.
Left column shows results prior to sex selection available, middle column before
sex selection illegal and right column after sex selection illegal.
N indicates the number of women in the relevant group in the underlying samples.
}
\label{fig:results_spell3_high_rural}
\end{figure}


\begin{figure}[htpb]
\ContinuedFloat
\centering
\caption*{First two children boys}
\setcounter{subfigure}{4}
\subfloat[1972--1984 (N=370)]{
    \begin{minipage}{0.31\textwidth}
        \captionsetup[subfigure]{labelformat=empty,position=top,captionskip=-1pt,farskip=-0.5pt}
        \subfloat[Prob. boy (\%)]{\includegraphics[width=\textwidth]{spell3_g1_high_rural_bb_pc}}\\
        \subfloat[Prob. no birth yet]{\includegraphics[width=\textwidth]{spell3_g1_high_rural_bb_s}} 
        \captionsetup[subfigure]{labelformat=parens}
    \end{minipage}
} 
\setcounter{subfigure}{5}
\subfloat[1985--1994 (N=945)]{
    \begin{minipage}{0.31\textwidth}
        \captionsetup[subfigure]{labelformat=empty,position=top,captionskip=-1pt,farskip=-0.5pt}
        \subfloat[Prob. boy (\%)]{\includegraphics[width=\textwidth]{spell3_g2_high_rural_bb_pc}}\\
        \subfloat[Prob. no birth yet]{\includegraphics[width=\textwidth]{spell3_g2_high_rural_bb_s}}
        \captionsetup[subfigure]{labelformat=parens}
    \end{minipage}
}
\setcounter{subfigure}{6}
\subfloat[1995--2006 (N=790)]{
    \begin{minipage}{0.31\textwidth}
        \captionsetup[subfigure]{labelformat=empty,position=top,captionskip=-1pt,farskip=-0.5pt}
        \subfloat[Prob. boy (\%)]{\includegraphics[width=\textwidth]{spell3_g3_high_rural_bb_pc}}\\
        \subfloat[Prob. no birth yet]{\includegraphics[width=\textwidth]{spell3_g3_high_rural_bb_s}}
        \captionsetup[subfigure]{labelformat=parens}
    \end{minipage}
}
\caption{(Continued) Predicted probability of having a boy and probability of
no birth yet from nine months after second birth for rural
women with 8 or more years of education by month beginning 9 months after prior birth. 
Predictions based on age 24 at second birth.
Left column shows results prior to sex selection available, middle column before
sex selection illegal and right column after sex selection illegal.
N indicates the number of women in the relevant group in the underlying samples.
}
% \label{fig:results_spell3_high_rural}
\end{figure}

\clearpage
\newpage



\end{document}



