% Revamped version for Demography focusing on method

% sex-selective Abortion
% Begun originally: 01/11/02
% Begun.: 2017-04-03
% Edited: 2017-04-03

\documentclass[12pt,letterpaper]{article}

\usepackage{fontspec}
\setromanfont[Ligatures=TeX]{TeX Gyre Pagella}
\usepackage{unicode-math}
\setmathfont{TeX Gyre Pagella Math}
% \usepackage{mathpazo}
\usepackage[title]{appendix}
\usepackage[margin=1.0in]{geometry}
\usepackage[figuresleft]{rotating}
\usepackage[longnamesfirst]{natbib}
\usepackage{dcolumn}
\usepackage{booktabs}
\usepackage{multirow}
\usepackage[flushleft]{threeparttable}
\usepackage{setspace}
\usepackage{xyling}
\usepackage[justification=centering]{caption}
\usepackage[font=scriptsize]{subfig}
\usepackage[xetex,colorlinks=true,linkcolor=black,citecolor=black,urlcolor=black]{hyperref}
\usepackage{adjustbox}
\usepackage{xfrac}


% \bibpunct{(}{)}{;}{a}{}{,}
\newcommand{\mco}[1]{\multicolumn{1}{c}{#1}}
\newcommand{\mct}[1]{\multicolumn{2}{c}{#1}}
\newcommand{\X}{$\times$ }
\newcommand{\hs}{\hspace{15pt}}

% Attempt to squeeze more floats in
\renewcommand\floatpagefraction{.9}
\renewcommand\topfraction{.9}
\renewcommand\bottomfraction{.9}
\renewcommand\textfraction{.1}
\setcounter{totalnumber}{50}
\setcounter{topnumber}{50}
\setcounter{bottomnumber}{50}


%------------------------------------------------------------------------

% Fertility, Birth Spacing and the Determinants of sex-selective Abortions
% Understanding Sex Selection: Fertility, Abortions and Birth Spacing
% The Determinants of sex-selective Abortions: A New Method/Approach
% Sex Selection: A New Method for Estimating Determinants
% Fertility, Birth Spacing and Abortions in India
% Fertility, Birth Spacing and Sex Selection
% Fertility, Birth Spacing and sex-selective Abortions
% Fertility, Spacing and Sex Selection
% Fertility, Sex Selection, and Spacing
% Fertility, Sex Selection, and Birth Spacing
% Fertility, Birth Spacing and Missing Girls
% Where have all the girls gone: An analysis of sex-selective abortions

\title{Son Preference, Birth Spacing, and Sex-Selective Abortions%
\protect\thanks{%
I am grateful to Andrew Foster and Darryl Holman for discussions about the method.
I owe thanks to three anonymous referees, Shelly Lundberg, Daniel Rees, David Ribar, 
Hendrik Wolff, seminar participants at University of Copenhagen, University of Michigan, 
University of Washington, University of Aarhus, the Fourth 
Annual Conference on Population, Reproductive Health, 
and Economic Development, and the Economic Demography Workshop for helpful suggestions and comments.
I would also like to thank Nalina Varanasi for research assistance.
Support from the University of Washington Royalty Research Fund and the 
Development Research Group of the World Bank is gratefully acknowledged.
The views and findings expressed here are those of the author and
should not be attributed to the World Bank or any of its member countries.
Partial support for this research came from a Eunice Kennedy Shriver National
Institute of Child Health and Human Development research infrastructure grant,
5R24HD042828, to the Center for Studies in Demography and Ecology at the
University of Washington.
% Prior versions of this paper were presented under the title ``The 
% Determinants of sex-selective Abortions.''
}
}

\author{}

\author{Claus C P\"ortner\\
    Department of Economics\\
    Albers School of Business and Economics\\
    Seattle University, P.O. Box 222000\\
    Seattle, WA 98122\\
    \href{mailto:cportner@seattleu.edu}{\texttt{cportner@seattleu.edu}}\\
    \href{http://www.clausportner.com}{\texttt{www.clausportner.com}}\\
    \& \\
    Center for Studies in Demography and Ecology \\
    University of Washington\\ \vspace{2cm}
    }

\date{April 2017}


%------------------------------------------------------------------------


\begin{document}
\graphicspath{{../figures/}}
\DeclareGraphicsExtensions{.eps,.jpg,.pdf,.mps,.png}

\setcounter{page}{-1}
\maketitle
\thispagestyle{empty}

% \setcounter{page}{0}


\newpage
\thispagestyle{empty}
\doublespacing

\begin{abstract}

% Demography abstract
\noindent This paper examines how birth spacing interact with the use of 
sex-selective abortions. 
I introduce a statistical method that incorporates that sex-selective abortions 
affect both the likelihood of a son and spacing between births.
Using India's National Family and Health Surveys,
I show that use of sex selection leads to \emph{longer} spacing after a daughter 
is born.
Women with 8 or more years of education, both in urban and rural areas, are 
the main users of sex-selective abortions, whereas women with less education 
do not appear to use sex selection.
Predicted lifetime fertility for high-education women declined more than 10\% 
between 1985--1994, when sex selection was legal, and 1995--2006, when sex selection 
was illegal.
Abortions per woman increased almost 20\% for urban women and 50\%
for rural women between the two periods, suggesting that making sex 
selection illegal has not reversed its use.
Finally, sex selection appears to be used to ensure one son rather 
than multiple sons.

\noindent JEL: J1, O12, I1
\noindent Keywords: India, prenatal sex determination, censoring, competing risk
\end{abstract}

\newpage


%------------------------------------------------------------------------

\section{Introduction\label{sec:intro}}

Spacing between births has long been used as a measure of son preference 
\citep{Leung1988}.
Before prenatal sex determination became available, the only recourse for 
parents who wanted a son---but did not yet have one---was to have the next 
birth sooner.
Son preference was therefore often associated with shorter average spacing 
after the births of girls than boys, which, in turn, was associated with worse 
health outcomes for girls.%
\footnote{
See, for example, 
\citet{Das1987}, \citet{Rahman1993}, \citet{Pong1994}, \citet{Haughton1996},
\citet{Arnold1997}, and \citet{Soest2012} on shorter birth spacing after
girls
and 
\citet{arnold98}, \citet{Whitworth2002}, \citet{Rutstein2005},
and \citet{Conde-Agudelo2006} on the association between shorter birth
spacing and worse health outcomes for girls.
Parents were also more likely to cease childbearing after the birth of 
a son than after a daughter 
\citep{ben-porath76b,Das1987,Arnold1997,clark00,filmer09,Altindag2016}
}

What has not previously been appreciated is that the introduction 
of prenatal sex determination fundamentally changed the relationship between 
son preference and birth spacing.
This is because each sex-selective abortion increases the duration between 
observed births by approximately a year.%
\footnote{
This increase consists of three parts starting from the time of the 
abortion.
First, the uterus needs at least two menstrual cycles to recover 
before conception, because a less than three months space 
between abortion and conception substantially increases the likelihood 
of a spontaneous abortion \citep{zhou00b}.
Second, an average of 6 months to next conception \citep{Wang2003}.
Finally, sex determination tests are generally reliable only from 3 months 
of gestation onwards.
These three parts add up to 12 months.
The waiting time to conception does vary by woman, but even if it is very 
short, say one  month, the additional space between births would still
be 6 months per abortion.
}
As a result, we now have a situation where families with the \emph{strongest} son 
preference likely show the \emph{longest} spacing after the birth of a daughter.
But, to further complicate matters, we may still observe short spacing 
after the birth of daughter as a representation of son preference for 
families who---for one reason or another---do not use prenatal sex selection.

Spacing, by itself, can therefore no longer be used to measure son preference. 
Spacing is, however, still important and useful in understanding son preference, 
but only if combined with the likelihood of observing a boy or a girl.
Short spacing after a girl is indicative of son preference if observed
births are close to the natural sex ratio, whereas long spacing is indicative
of son preference if observed births are male biased.

In this paper, I introduce and apply a novel method that directly incorporates 
the effects of sex-selective abortions on the likelihood of a son being born 
\emph{and} the duration between births.
The method can be used to analyse both situations with and without prenatal
sex selection.
Furthermore, my proposed method allow for the duration since the last 
birth to affect the decision on sex selection.
For example, as the duration from last birth becomes long enough, parents 
may reverse their decision to use prenatal sex determination,
and carry the next pregnancy to term whether male or female.
By examining whether sex selection decisions change with spacing, we can 
draw a more nuanced picture of the degree of son preference.


% [Why India?]
I apply the method to birth histories for Hindu women, using data from 
India's National Family and Health Surveys (NFHS), covering the period 
1972 to 2006. 
India is a particularly interesting case.
On one hand, India has seen dramatic increases in the males-to-females ratio 
at birth over the last three decades as access to prenatal sex determination 
expanded.%
\footnote{
See \citet{das_gupta97}, \citet{Sudha1999},
\citet{Arnold2002}, \cite{retherford03b} and \citet{jha06}.
India is clearly not alone; both China and South Korea saw
significant changes in the sex ratio at birth over the same period 
\citep{Yi1993,park95}.
}
On the other hand, recent research suggests that son preference in 
India, when measured as ideally having more boys than girls, is decreasing 
over time and with higher education \citep{bhat03,pande07}.%
\footnote{
This measure of son preference is commonly used in the literature. 
See, for example, \citet{clark00}, \citet{Jensen2009}, and \cite{Hu2015}.
}
In addition, there has been substantial changes in access and legality of
prenatal sex determination in India over the period covered.
Abortion has been legal in India since 1971 and still is, but the 
first reports of on the availability of prenatal sex determination did
not appear until 1982--83 \citep{Sudha1999,bhat06,Grover2006}.
In 1994, the Central Government passed the Prenatal Diagnostic Techniques 
(PNDT) Act, making determining and communicating the sex of a fetus illegal.%
\footnote{
The Act is described in detail at \href{http://pndt.gov.in/}{http://pndt.gov.in/}.
The number of convictions has been low.
It took until January 2008 for the first state, Haryana, to reach 5 convictions.
Hence, private clinics apparently operate with little risk of legal action 
\citep{Sudha1999}.
Maharashtra was the first state to pass a similar law in 1988.
}
Hence, the data make it possible to show how spacing between births and
sex ratios have changed with the introduction of prenatal sex determination,
and whether the ban affected the relationship between birth spacing and sex 
ratios.




% [EXPLAIN RESULTS HERE]
% Main findings!
There are three main findings.

There is, however, still no evidence of sex selection being used on the first birth.

Secondly, sex selection appears to be used for securing one son, rather than a large 
number of sons.
There is only limited use of sex-selective abortions for better-educated
women with one or more sons.
The exception is rural women with one son and one daughter, presumably to compensate for 
the higher child mortality in rural areas.
These results are in line with the differential stopping behavior observed in many studies
before sex selection became available \citep{repetto72,arnold98,dreze01}.




%------------------------------------------------------------------------------------


\section{Estimation Strategy\label{sec:strategy}}


[TK this needs to move to later]
Finally, increased reliability of access and effectiveness of
contraceptive can lead to shorter spacing between births 
\citep{Keyfitz1971,Heckman1976}.
With less reliable contraception parents choose a higher level 
of contraception---resulting in longer spacing---to avoid having 
too many children by accident.
But, as contraception becomes more effective parents can more
easily avoid future births, allowing them to reduce the spacing 
between births without having to worry about overshooting their 
preferred number of children.
This idea may also help explain shorter spacing for better 
educated women than for less educated women, provided that   
knowledge and ability to use contraception differ
across education groups \citep{Tulasidhar1993,Whitworth2002}.


To improve on the literature's current approaches to understanding 
son preference, a new empirical method must be able to capture the 
three main implications of the availability of sex-selective abortions
combined with son preference:
a higher probability that the next child is a son, 
longer spacing to next birth with use of sex selection, 
and that the use of prenatal sex determination may change between 
one observed birth and the next.%
\footnote{
In addition, it should be able to capture the shorter spacing between
births observed in the absence of use or availability of prenatal 
sex determination.
}
This means that we need an empirical model that can jointly estimate 
the association between covariates and both the sex of children born 
and the spacing from last birth. 
My proposed model is a discrete time, non-proportional competing risk 
hazard model with two exit states: either a boy or a girl is born.%
\footnote{
\cite{Merli2000} used a discrete hazard model to examine whether 
there were under-reporting of births in rural China, although they 
estimated separate waiting time regressions for boys and girls.
}

I divide a woman's reproductive life into spells that each covers 
the period between births (from marriage to first birth for the first spell).
For each woman, $i=1,\ldots,n$, the starting point for a spell is time $t=1$ and 
the spell continues until time $t_i$ when either a birth occurs or the spell 
is censored.%
\footnote{
The time of censoring is assumed independent of the hazard rate,
as is standard in the literature.
}
There are two exit states: birth of a boy, $j=1$, or birth of a girl, $j=2$, and 
$J_i$ is a random variable indicating which event took place.
The discrete time hazard rate $h_{ijt}$ is
\begin{equation}
 h_{ijt} = \Pr (T_i=t, J_i=j \mid T_i \geq t; \mathbf{Z}_{it},\mathbf{X}_{i} ),
\end{equation}
where $T_i$ is a discrete random variable that captures when woman $i$'s birth occurs.
To ease presentation the indicator for spell number is suppressed.
The vectors of explanatory variable $\mathbf{Z}_{it}$ and $\mathbf{X}_{i}$ include 
information about various individual, household, and community characteristics 
discussed below.

The hazard rate is specified as
\begin{equation}
 h_{ijt} = \frac{\exp(D_j(t) + \alpha_{jt}'\mathbf{Z}_{it} + \beta_j'\mathbf{X}_{i})} 
 {1 + \sum_{l=1}^2 \exp(D_j(t) + \alpha_{lt}'\mathbf{Z}_{it} + \beta_l'\mathbf{X}_{i})} \: \: \; \; \;  j = 1,2
 \label{eq:hazard}
\end{equation}
where $D_{j}(t)$ is the piece-wise linear baseline hazard for outcome $j$, captured
by dummies and the associated coefficients,
\begin{equation}
D_j(t) = \gamma_{j1} D_1 + \gamma_{j2} D_2 + \ldots + \gamma_{jT} D_T,
\end{equation}
where $D_m = 1$ if $t=m$ and zero otherwise.
This approach to modeling the baseline hazard is flexible and does not place 
overly strong restrictions on the baseline hazard.

[TK I am not very satisfied with this explanation of the problems with
proportionality and the associated bias; non-proportionality is required
to capture changes in use of sex selection]

In principle, specifying the model as a proportional hazard model, i.e.\ one 
where covariates simply shift the hazard rates up or down independent of 
spell length, is more efficient, but only provided that the proportionality 
assumption holds.
If the proportionality assumption does not hold, however, the result is
a potentially substantial bias in estimates.
The problem is that a proportional hazard model does not allow covariates
to have different effects at different times within a spell and therefore
cannot capture differences in the shape of the hazard functions between 
different groups. 
It is highly unlikely---even in the absence of prenatal sex determination---that 
the baseline hazards are the same across education levels, areas of residence, 
or sex composition of previous births.
Any bias from the proportionality assumption is likely exacerbated by the 
introduction of prenatal sex determination for two reasons.
First, one of the main points of this paper is that use of sex-selective 
abortions affects birth spacing, and use of sex selection differs across groups.
Second, as discussed above, the use of sex selection may vary within a spell,
depending on the length of the spell.

I therefore use a non-proportional model where the main explanatory variables 
and the interactions between them are interacted with the baseline hazards.
This is captured by the $Z$ set of explanatory variables
\begin{equation}
 \mathbf{Z}_{it} = D_j(t) \times (\mathbf{Z}_1 + Z_2 + \mathbf{Z}_1 \times Z_2),
\end{equation}
where $D_j(t)$ is the piece-wise linear baseline hazard and $\mathbf{Z}_1$ captures sex 
composition of previous children, if any, and $Z_2$ captures area of residence.
This allows the effects of the main explanatory variables on the probabilities 
of having a boy, a girl, or no birth to vary over time within a spell.
The use of a non-proportional specification, together with a flexible baseline hazard, 
also mitigates any potential effects of unobserved heterogeneity \citep{Dolton1995}.

The remaining explanatory variables, $\mathbf{X}$, enter proportionally.
To further minimize any potential bias from assuming proportionality, estimations 
are done separately for different levels of mothers' education and for different 
time periods.
The exact specifications and the individual variables are described below.

Equation (\ref{eq:hazard}) is equivalent to the logistic hazard model and has the same 
likelihood function as the multinomial logit model \citep{allison82,jenkins95}.
Hence, if the data are transformed so the unit of analysis is spell unit rather 
than the individual woman, the model can be estimated using a standard multinomial 
logit model.%
\footnote{
A potentially issue is that the multinomial model assumes that alternative 
exit states are stochastically independent,
also known as the Independence of Irrelevant Alternatives (IIA) assumption.
This assumption rules out any individual-specific unmeasured or 
unobservable factors that affect both the hazard of having a girl and the 
hazard of having a boy.
To address this issue the estimations include a proxy for fecundity
discussed in Section \ref{sec:data}.
In addition, the multivariate probit model can be used as an alternative
to the multinomial logit because the IIA is not imposed \citep{han90}.
The results are essentially identical between these two models and
available upon request.
}
In the reorganized data the outcome variable is zero if the
woman does not have a child in a given period, one if she gives birth to a son 
in that period, and two if she gives birth to a daughter in that period.

The main downside of my approach is that direct interpretation of the estimated 
coefficients for this model is challenging because of the competing risk setup.
First, coefficients show the change in hazards relative to the base outcome, 
no birth, rather than simply the hazard of an event.
Second, a positive coefficient does not necessarily imply that an increase in a
variable's value increases the probability of the associated event because the 
probability of another event may increase even more \citep{thomas96}.

It is, however, straightforward to calculate the predicted probabilities of 
having a boy and of having a girl for each $t$ within a spell, conditional on 
a set of explanatory variables and not having had a child before that period.
The predicted probability of having a boy in period $t$ for a given set of 
explanatory variable values, $\mathbf{Z}_k$ and $\mathbf{X}_k$, is
\begin{equation}
P(b_{t} | \mathbf{X}_{k}, \mathbf{Z}_{kt}, t ) 
=  
\frac{ \exp(D_j(t) + \alpha_{1t}' \mathbf{Z}_{kt} + \beta_1' \mathbf{X}_{k} )}
{1 + \sum_{l=1}^2 \exp(D_j(t) + \alpha_{lt} ' \mathbf{Z}_{kt} + \beta_l ' \mathbf{X}_{k})},
\label{eq:probability_boy}
\end{equation}
and the predicted probability of having a girl is
\begin{equation}
P(g_{t} | \mathbf{X}_{k}, \mathbf{Z}_{kt},t ) 
=  
\frac{ \exp(D_j(t) + \alpha_{2t}'\mathbf{Z}_{kt} + \beta_2'\mathbf{X}_{k} )}
{1 + \sum_{l=2}^2 \exp(D_j(t) + \alpha_{lt}'\mathbf{Z}_{kt} + \beta_l'\mathbf{X}_{k})}.
\label{eq:probability_girl}
\end{equation}
With these two probabilities is it easy to calculate, for each $t$, the estimated
percentage of children born that are boys, $\hat{Y}$, 
\begin{equation}
\hat{Y}_t 
= 
\frac{ P(b_{t} | \mathbf{X}_{k}, \mathbf{Z}_{kt},t )}
{ P(b_{t} | \mathbf{X}_{k}, \mathbf{Z}_{kt},t) + P(g_{t} | \mathbf{X}_{k}, \mathbf{Z}_{kt},t )} 
\times 100,
\label{eq:probability_son}
\end{equation}
together with the associated confidence interval for given values of explanatory 
variables.%
\footnote{
Within each period $1-P(b_{t})-P(g_{t})$ is the probability of not having a birth in 
period $t$.
}

For ease of exposition the procedure is presented here in two steps, but the actual
calculation of the percent boys is done in one step with the 95 percent confidence 
interval calculated using the Delta method.
Results are presented as the estimated percent boys born by length of birth spacing
using graphs.%
\footnote{
The parameter estimates are available on request.
}
For each graph, the extent to which the percent boys is statistically 
significantly above the natural sex ratio indicates the use of sex selection.


[TK  may need to combine with and without sons to show differences in spacing 
over time; maybe it should also have predicted average/median spacing with some 
confidence intervals to test]


The other important part of the model is the spacing between births.
Spacing is captured by the survival curve, which shows the probability of not 
having had a birth yet by spell duration.
The survival curve at time $t$ is 
\begin{equation}
S_{t} 
= 
\prod_{d=1}^t 
\left( 
	1- \left(P(b_{d} | \mathbf{X}_{k}, \mathbf{Z}_{kd}, d) 
	+ P(g_{d} | \mathbf{X}_{k}, \mathbf{Z}_{kd}, d) \right) 
\right),
 \label{eq:survival}
\end{equation}
or equivalently
\begin{equation}
S_{t} 
= 
\prod_{d=1}^t
\left(
\frac{ 1 }
{1 + \sum_{l=2}^2 \exp(D_j(t) + \alpha_{ld}'\mathbf{Z}_{kd} + \beta_l'\mathbf{X}_{k})}
\right).
\end{equation}

In the absence of sex selection, I expect most parities to show an 
``inverted s'' pattern, where there initially are relatively few births, 
followed by a substantial number of births over a 1 to 2 year period, and then
relatively few births thereafter.
As sex selection becomes more widely used the associated
longer spacing shows up by making the survival curve straighter,
indicating that some of the births that would originally have taken place
now take place later because of abortions.
The use of sex selection is clearly not the only factor that can change the shape of
the survival curves; factors such as the desired number of children and
use of contraceptives may also shape the shape.
To account for this it is best to compare survival curves for an individual
parity between women who are likely to use sex selection and women who are
not, for example because they already have one or more sons.

In addition to information about spacing, the survival curves also provide 
``weighting'' for the associated percentage boys born.
The steeper the survival curve, the more weight should be assigned to a given
spell period because it is based on more births, 
whereas a period with a flat survival curve should be given little weight because the 
percentage boys is based on few births.
Hence, although they cannot by themselves show sex selection, the survival
curves are a crucial complement to the estimated sex ratios by duration.


\section{Data\label{sec:data}}

The data come from the three rounds of the National Family Health Survey 
(NFHS-1, NFHS-2 and NFHS-3),
collected in 1992--93, 1998--99, and 2005--2006.%
\footnote{
A delay in the survey for Tripura means that NFHS-2 has a small number of observation 
collected in 2000.
}
The surveys are large: NFHS-1 covered 89,777 ever-married women 
aged 13--49 from 88,562 households,
NFHS-2 covered 90,303 ever-married women aged 15--49 from 92,486 households
and NFHS-3 covered 124,385 never-married and ever-married women aged 
15--49 from 109,041 households.

I exclude visitors to the household, as well as
women married more than once, divorced, or not living with their husband,
women with inconsistent information on age of marriage,
and those with missing information on education.
Women interviewed in NFHS-3 who were never married or where gauna had not
been performed were also dropped.
The same goes for women who had at least one multiple birth,
reported having a birth before age 12, had a birth before marriage, or
a duration between births less than 9 months.
Women who reported less than 9 months between marriage and first birth
remain in the sample unless they are dropped for another reason.%
\footnote{
Women who report less than 9 months between marriage and first birth are retained 
because between 10 and 20 percent fall into this category.
Although it is possible that some of these births are premature the high number of
women who report a birth less than 6 months after marriage indicates that conception
likely occurred before marriage in most cases.
}

Finally, I restrict the sample to Hindus,
who constitute about 80 percent of India's population.
If use of sex selection differ between Hindu and other religions, such 
as Sikhs, assuming that the baseline hazard is the same would lead to bias.
The other groups are each so small relative to Hindus that it is not
possible to estimate different baseline hazards for each group.
Furthermore, the groups are so different in terms of background and son preference
that combining them into one group would not make sense.

There are four advantages to using the NFHS.
First, surveys enumerators pay careful attention to spacing between births and
probe for ``missed'' births.
Second, no other surveys cover as long a period in the same amount of detail.
The three NFHS rounds allow me to show the development in spacing and 
sex ratio from before sex-selective abortions were available until 2006.
Third, NFHS has birth histories for a large number of women.%
\footnote{
The Special Fertility and Mortality Survey appears to cover a much large number of households
than the three rounds of the NFHS combined, but \citet{jha06} only use the births that 
took place in 1997 making their sample sizes by parity smaller than here.
Their sample consists of 133,738 births of which 38,177 were first
born, 36,241 second born, and 23,756 were third born.
The differences in results for first born children are discussed in the online 
Appendix.
}
Finally, even if probing for missing births may not completely eliminate recall error,   
the overlap in cohorts covered and the large sample size make it possible to establish 
where recall error remains a problem.

Recall error arise mainly from child mortality, when respondents are reluctant to
discuss deceased children.%
\footnote{
The online Appendix contains a more thorough discussion of recall error and how I address it.
} 
Systematic recall error, where the likelihood of reporting a deceased child depends on
the sex of the child, is especially problematic because it biases the sex ratios.
Probing catches many missed births, but systematic recall error is still a potentially 
substantial problem.
Three factors contribute to the problem here.
First, girls have significantly higher mortality risk than boys.
Second, son preference may increase the probability that boys are remembered relative to girls.
Finally, in NFHS-1 and NFHS-2 enumerators probed only for a missed birth if the
initial reported birth interval was four calendar years or more.
But, given short durations between births, especially after the birth of a girl,
that procedure is unlikely to pick up all missed children.

Observed sex ratios by cohort provide a straightforward way to determine 
whether recall error is a problem.
Because prenatal sex determination techniques did not become widely available until the 
mid-1980s, a higher than natural sex ratio for cohorts born before that time must be 
the result of systematic recall error.
As shown in the online Appendix, the observed sex ratio by parity becomes more male 
dominated the further back births took place.
In addition, births in the same cohort tend to be more male dominated the more recent the 
survey (births in the cohort took place longer ago relative to the survey).
Hence, there is evidence of recall error and the degree of recall error increases
with length of time between survey and cohort.

Using cohort year of birth to analyze recall error and decide which observations
to keep is, however, problematic because the year of birth for a given parity is affected 
by recall error; for example, a second born child listed as first born will be 
born later than the real first born child.
Year of marriage should, however, be unaffected by recall error.
Using year of marriage the basic recall error pattern remains with women married longer 
ago more likely to report that their child was a boy for a given parity.
Similarly, comparing women married in the same period across surveys shows
that women married longer ago are more likely to report having sons.

That recall error increases the longer ago somebody was married means
that duration of marriage is a better predictor of recall error than calendar year of 
marriage.
Figure \ref{fig:sexRatioMarriage} shows the observed sex ratio for children 
reported as first born as a function of duration of marriage combining all three surveys.%
\footnote{
The graph for second births shows a similar pattern.
The graphs for the second births and the individual survey rounds are available upon request.
}
The solid line is the sex ratio of children reported 
as first born by the number of years between the survey and marriage, 
the dashed lines indicate the 95 percent confidence interval 
and the horizontal line the natural sex ratio (approximately 0.512).
To ensure sufficient cell sizes the years are grouped in twos.

% \begin{figure}[htp]%
% \centering
% \includegraphics[width=0.9\textwidth]{sex_ratio_marriage}
% \caption{Ratio of boys in ``first'' births}
% \label{fig:sexRatioMarriage}
% \end{figure}


Figure \ref{fig:sexRatioMarriage} clearly illustrates the systematic recall error
problem.
The observed sex ratio is increasingly above the expected value the
longer ago the parents were married.
The increasingly unequal sex ratio with increasing marriage duration suggests that
a solution to the recall error problem is to drop women who were married ``too far'' from 
the survey year.
The main problem is establishing what the best cut-off point should be.
The observed sex ratio is consistently significantly higher than the natural sex ratio 
from around 24 years of marriage, so one possibility is simply to drop all women married 
more than 24 years at the time of the survey.
But, as the Appendix shows, there are differences across the three surveys and between 
parities.
I therefore use different cut-off points by survey round.
For NFHS-1 women married 22 years or more were dropped, with the corresponding cut-off 
points 23 years for NFHS-2  and 26 years for NFHS-3.
The final sample consists of 146,096 women, with 332,951 parity one through four births.%
\footnote{
The online Appendix presents the results for a more restrictive definition and for the
women dropped because of recall error concerns.
}


\subsection{Spell Definition\label{sec:spell_def}}

Spell duration is measured in quarters of a year, that is 3 month periods, hereafter
referred to as quarters.
The first spell begins at the month of marriage rather than 9 months after because many 
women report giving birth less than 9 months after they were married.
For women who began living with their husbands at too young an age to conceive the 
starting point should ideally be first ovulation, when she becomes ``at risk'' for a 
pregnancy, rather than month of marriage.
Unfortunately, information on age of menarche is only available in NFHS-1.
Instead, for women who began living with their husband before age 12, I set the 
the month they turned 12 years of age as staring point for the first spell.

The second and subsequent spells begin 9 months after the previous birth 
because that is the earliest we should expect to observe a new birth.
A few women report births that occurred less than 9 months 
after the previous birth; those women are dropped.

All spells continue until either a child is born or the spell is censored.
Censoring can happen for three reasons:
the survey takes place;
the woman is sterilized;
or the number of births observed becomes too sparse for the method to work.
The timing of censoring because of too few births vary slightly by spell but is generally 
21 quarters after the beginning of the spell.

I group spells into three time periods based on spell start date:
1972--1984, 1985--1994, and 1995--2006.
The first period covers the time before sex-selective abortions became widely available.
Abortion was legalized in 1971 and amniocentesis was introduced
in India in 1975, but the first newspaper reports on the availability of prenatal sex 
determination were not until 1982--83 \citep{Sudha1999,bhat06,Grover2006}.
The number of clinics quickly increased, and knowledge about sex selection became widespread
after a senior government official's wife aborted a fetus that turned out to be male \citep[p.\ 598]{Sudha1999}.
The second period covers the time from the widespread emergence of sex-selective abortions
until the prenatal Diagnostic Techniques (PNDT) act was passed in 1994.
The final period is from the PNDT act until the last available survey.
The PNDT act made it a criminal offence to reveal the sex of the fetus and was
followed by a campaign against the use of sex selection, although
enforcement appears to be relatively lax.

By dividing spells into these periods I can examine how the use of sex selection 
has changed across the three different regimes.
Note that the periods are based on the spells' beginning year, and some spells 
will therefore cover two periods.
A couple may, for example, be married in 1984, but not have their first child until 1986.
That couple's first spell will be in the 1972--1984 period, even though most of the 
spell actually falls in the 1985--1994 period.
Some children born from spells that began in the 1972--1984 period may therefore have been
conceive when prenatal sex determination techniques were available, which could result
in evidence of sex-selective abortions even for this period.
Similarly, a spell that began in the 1985--1994 period may have been partly or mostly
under the PDNT act.
The overall effect is to bias downwards any differences between the periods.

\subsection{Explanatory Variables}

The explanatory variables are divided into two groups.
The first group consists of variables expected to affect the shape of the hazard function: 
mother's education, sex composition of previous children, and area of residence.
Increasing the number of variables interacted with the baseline hazard lowers the risk 
of bias but requires more data to precisely estimate.
I chose these variables because the prior literature shows that they affect 
spacing choices and because the prior literature on sex selection indicates 
that these are correlated with sex selection.
The second group of variables are those expected to have an approximately 
proportional effect on the hazard: age of the mother at the beginning of the spell, 
length of her first spell (for second spell and above), whether the household owns 
land, and whether the household belongs to a scheduled tribe or caste.

Increasing education of mothers is strongly associated with lower fertility, with
the negative effect of higher opportunity cost on fertility more than outweighing the 
positive effect of higher income \citep{schultz97}.
Higher education should therefore be associated with higher use of sex selection.%
\footnote{
Fathers' education has two opposite predicted effects: the associated higher income
should increase fertility and therefore lower the pressure to use sex selection, but
the higher income also makes the use of sex selection cheaper.
In practice, fathers' education had little effect on the hazards and the use of 
sex-selective abortions and is not included.
}
I divide women into three education groups:
no education, 1 to 7 years of education, and 8 and more years of education.
The models are estimated separately for each education level.%
\footnote{
A potential concern here is reverse causation, where the sex of children
affect women's education.
Although no direct information is available on when women left school it is
possible to estimate school leaving age from the highest completed grade
and the usual starting age.
As described in the online Appendix, relatively few women could potentially 
have returned to school after the birth of their first child,
and only 56 could possibly have ended up in the wrong education group. 
Hence, there is little likelihood that reverse causation is a substantial
concern here.
}

As discussed, the sex composition of previous children affects both the timing
of births and the use of sex-selective abortions.
I capture sex composition of previous children with dummy variables for the
possible combinations for the specific spell, ignoring the ordering of births.
As an example, for the third spell three groups are used: Two boys,
one girl and one boy, and two girls.

The area of residence is a dummy variable for the household living in
an urban area.%
\footnote{
NFHS uses four categories for area of residence: Large city, small city, town
and countryside.
To reach a sufficient sample size urban areas are merged into one group.
}
The cost of children is higher in urban areas than in rural and access to prenatal
sex determination is easier, and both are expected to be associated with
greater use of sex-selective abortions in urban than in rural areas.
Because of concerns about selective migration I use where the household was living
at the end of each spell.%
\footnote{
As online Appendix Table D.1 shows there is little difference across
education levels in migration patterns within each period and patterns across periods
are also relatively stable.
}

The sex composition of children, area of residence, and the interactions between
these are all interacted with the piece-wise linear baseline hazard dummies.
In other words, the baseline hazards are assumed to be different depending on
where a woman lives and the sex composition of her previous children.
As an example, for the second spell a separate regression is
run for each education level and in each regression four different baseline hazards 
are included (first child a boy in rural area, first child a boy in urban area,
first child a girl in rural area, first child a girl in urban area).
Although this approach substantially increases the number of regressions and 
estimated parameters it reduces the potential problem of including other variables 
as proportional effects.

The remaining variables are expected to affect hazards proportionally.
Although fecundity cannot be observed directly a suitable
proxy is the duration from marriage until first birth.
Most Indian women do not use contraception before the first birth
and there is pressure to show that a newly married woman can conceive 
\citep{dyson83,Sethuraman2007,Dommaraju2009}.
This is confirmed by the very short spells between marriage and first birth,
even among the most educated.
Hence, a long spell between marriage and first birth is likely due to low fecundity.
For both this variable and the age of the mother at the beginning of the spell 
the squares are also included.
The remaining variables are dummies for household ownership of land and membership
of a scheduled caste or tribe.


\subsection{Descriptive Statistics}

Appendix Table \ref{tab:des_stat1} presents descriptive statistics for
the spells by education level and when the spell began.
There is a substantial number of censored observations.
As an example, for highly educated women who had their first child in the 1995--2006
period, almost half did not have their second child by the time of the survey.
Hence, although about 13,000 women began the second spell 
there are only about 7,000 births to these women.
Censoring becomes even more important for the third and fourth
spells, where around 70 percent of the observations are censored.
Generally, censoring increases with parity and time period.
This reflects a combination of factors: timing of the surveys
relative to the periods of interest, later beginning of childbearing, 
falling fertility, and the longer spells from sex-selective abortions.
The high number of censored observations underscore the importance of controlling for
censoring when examining the relationship between fertility and sex selection.

The descriptive statistics also provide a first indication of how the
sex ratio at birth changes over time and by spell.
For the first spell the sex ratio is very close to the natural
for all education groups and all three time periods.
As an example, among the highly educated group for the 1995--2006 period,
51.3 percent of the children born were boys.%
\footnote{
There still appears to be some recall error for the group of women without
education for the 1972--1984 period, where 52.3 percent of the children born were boys.
}
For the second spell, all but the highly educated group in the last two
periods have sex ratios in line with the natural sex ratio.
Women with 8 or more years of education have 53.1 and 54.3 percent
boys in the 1985--1994 and 1995--2006 periods, respectively.
This pattern repeats itself for the third spell, except the percentage
boys is higher for the high education group (55.3 and 55.9 for the last
two time periods).
Finally, for the fourth spell the high education group 
had 60 percent boys in the last period, i.e. after the PNDT act was introduced.
Note, however, that for the fourth spell the number of births is substantially
smaller and censoring even more important than for the other spells.

India's population has become progressively more urban.
For the first period, 32 percent of the women entering the first spell lived in urban areas.
This increases to 35 percent for the second period and to 42 percent for the final period.
The population is also substantially better educated.
Women with no education constituted almost 60 percent 
in the first period, but less than thirty percent in the last period.
Correspondingly, in the first period just over twenty percent had 8 or more 
years of education, but in the last period it was almost half.
Part of the increase in education is correlated with the increase in urbanization,
but the proportion of better-educated women has increased substantially
in the rural areas as well.
Among the high education group almost 70 percent lived in urban areas
during the first period but this had fallen to less than 60 percent
in the last period.

The increase in urbanization and education is likely to exert downward pressure
on fertility and the high censoring rates for the later periods are evidence of this.
The average number of children born to women by the time they turn 35 illustrate how
strong the decline in fertility has been.%
\footnote{
Figure \ref{fig:fertility} shows the full results.
}
Women born in the early 1940s had on average close to 5 children when 
they reached 35, but women born in the early 1970s had only just over three children.
The low number of children is especially remarkable because it combines
all education levels and all areas of residence.
% Hence, fertility in cities is likely substantially lower.


\section{Results\label{sec:results}}

One of the main questions here is how fertility affects sex selection use.
I start by showing the relationship between self-reported desired fertility and sex 
selection, which also illustrates how to interpret the results.
Although self-reported desired fertility is an imperfect proxy for actual
fertility behavior, the analysis of desired fertility indicates that there is 
a relationship between fertility decisions and the use of sex selection and motivates 
examining the relationship more closely.

Because of the desired fertility measure's potential shortcomings, I next examine the 
relationship between actual fertility and use of sex selection for better-educated women.
There are two steps in this process.
First, I show how fertility has declined for better-educated women with one son---a group 
unlikely to use sex selection, which means that sex selection will not affect the 
progression rates to next birth for these women.
Second, I show how, corresponding to the fertility decline, there is an increase in
sex-selective abortions for better-educated women without a son.
I next show how women with lower cost of children appear to follow a high fertility, low
use of sex selection strategy compared to the low fertility, high use of sex selection of
the better-educated women.
The results all show clear evidence of son preference in fertility and sex selection use, 
but exactly what form son preference takes is important, so I next discuss what we can
learn about son preference from the sex selection behavior.
Finally, I predict lifetime fertility, number of abortions, and final sex ratios for
the better-educated women.

Except for the predictions, results are presented using graphs of estimated percentage boys 
born by quarter, the associated 95 percent confidence interval, and the survival curve for 
a ``representative woman'' using the method detailed in Section \ref{sec:strategy}.
The ``representative woman'' characteristics are based on means of continuous 
explanatory variables and the majority category for categorical explanatory variables.
Common to all ``representative women'' is that they do not own land and belong to neither 
a scheduled caste nor a scheduled tribe.
The graphs also show the expected natural rate of boys, approximately 51.2 percent, for
comparison.
Graphs for all groups and spells, even if not discussed here, are available in 
the online Appendix.


\subsection{One Son or Many Sons?}

The extent to which people use sex selection for different sex compositions
of previous children is of interest for two reasons.
First and foremost, it can help us understand what form son preference takes in India.
This is important in itself, but is especially important as we try to predict 
what will happen to sex selection use in the future.
[need an explanation of this based on what I have in the introduction]
Second,
if families use sex selection to reach two or more sons, progression rates for families 
with one son may overestimate the decline in fertility.

Women with 8 or more years of education is the group the most likely to use sex 
selection if they have no son as shown above.
Hence, if sex selection is also used to achieve more than one son, we should be
most likely to find evidence for this group of women as well.
Figure \ref{fig:boys_latest} therefore shows sex ratios and survival curves for 
the 1995--2006 period for urban and rural women with 8 or more years of education, 
conditional on having at least one son.
The first column shows outcomes for the second spell if the first born is a son,
the second column shows third spell when the first two children were a son and 
a daughter, and the third column shows third spell when the first two children were sons.


% \begin{figure}[htpb]
% \centering
% \caption*{Urban}
% \setcounter{subfigure}{-2}
% \subfloat[Spell 2 - 1 boy (N=3,969)]{
%     \begin{minipage}{0.30\textwidth}
%         \captionsetup[subfigure]{labelformat=empty,position=top,captionskip=-1pt,farskip=-0.5pt}
%         \subfloat[Prob. boy (\%)]{\includegraphics[width=\textwidth]{spell2_g3_high_r2_pc}}\\
%         \subfloat[Prob. no birth yet]{\includegraphics[width=\textwidth]{spell2_g3_high_r2_s}}
%         \captionsetup[subfigure]{labelformat=parens}
%     \end{minipage}
% }
% \setcounter{subfigure}{-1}
% \subfloat[Spell 3 - 1 boy/1 girl (N=2,357)]{
%     \begin{minipage}{0.30\textwidth}
%         \captionsetup[subfigure]{labelformat=empty,position=top,captionskip=-1pt,farskip=-0.5pt}
%         \subfloat[Prob. boy (\%)]{\includegraphics[width=\textwidth]{spell3_g3_high_r4_pc}}\\
%         \subfloat[Prob. no birth yet]{\includegraphics[width=\textwidth]{spell3_g3_high_r4_s}}
%         \captionsetup[subfigure]{labelformat=parens}
%     \end{minipage}
% }
% \setcounter{subfigure}{0}
% \subfloat[Spell 3 - 2 boys (N=1,098)]{
%     \begin{minipage}{0.30\textwidth}
%         \captionsetup[subfigure]{labelformat=empty,position=top,captionskip=-1pt,farskip=-0.5pt}
%         \subfloat[Prob. boy (\%)]{\includegraphics[width=\textwidth]{spell3_g3_high_r2_pc}}\\
%         \subfloat[Prob. no birth yet]{\includegraphics[width=\textwidth]{spell3_g3_high_r2_s}}
%         \captionsetup[subfigure]{labelformat=parens}
%     \end{minipage}
% }
% \caption*{Rural}
% \setcounter{subfigure}{1}
% \subfloat[Spell 2 - 1 boy (N=3,044)]{
%     \begin{minipage}{0.30\textwidth}
%         \captionsetup[subfigure]{labelformat=empty,position=top,captionskip=-1pt,farskip=-0.5pt}
%         \subfloat[Prob. boy (\%)]{\includegraphics[width=\textwidth]{spell2_g3_high_r1_pc}}\\
%         \subfloat[Prob. no birth yet]{\includegraphics[width=\textwidth]{spell2_g3_high_r1_s}}
%         \captionsetup[subfigure]{labelformat=parens}
%     \end{minipage}
% }
% \setcounter{subfigure}{2}
% \subfloat[Spell 3 - 1 boy/1 girl (N=1,819)]{
%     \begin{minipage}{0.30\textwidth}
%         \captionsetup[subfigure]{labelformat=empty,position=top,captionskip=-1pt,farskip=-0.5pt}
%         \subfloat[Prob. boy (\%)]{\includegraphics[width=\textwidth]{spell3_g3_high_r3_pc}}\\
%         \subfloat[Prob. no birth yet]{\includegraphics[width=\textwidth]{spell3_g3_high_r3_s}}
%         \captionsetup[subfigure]{labelformat=parens}
%     \end{minipage}
% }
% \setcounter{subfigure}{3}
% \subfloat[Spell 3 - 2 boys (N=790)]{
%     \begin{minipage}{0.30\textwidth}
%         \captionsetup[subfigure]{labelformat=empty,position=top,captionskip=-1pt,farskip=-0.5pt}
%         \subfloat[Prob. boy (\%)]{\includegraphics[width=\textwidth]{spell3_g3_high_r1_pc}}\\
%         \subfloat[Prob. no birth yet]{\includegraphics[width=\textwidth]{spell3_g3_high_r1_s}}
%         \captionsetup[subfigure]{labelformat=parens}
%     \end{minipage}
% }
% \caption{Predicted probability of having a boy and probability of
% no birth by quarter (3 month period) for women with 8 or more years
% of education for 1995--2006. 
% N is number of women in the relevant group in the underlying samples.
% }
% \label{fig:boys_latest}
% \end{figure}


For the second spell the predicted sex ratios closely follow the natural sex 
ratio for both urban and rural women.
Furthermore, the survival curves follow the expected pattern if no sex selection 
is taking place.
There is also no evidence of sex selection for the third spell for women with two sons;
the sex ratios follow the natural sex ratio for both urban and rural women.
Sample sizes are, however, less than \sfrac{1}{3} of the sample sizes for the second 
spell and the progression rate to a third birth is low at only 30\% for urban women and 
55\% for rural women, resulting in wide confidence intervals.
The survival curves are particularly flat for the first year, meaning that the apparent 
deviations from the natural sex ratio in the beginning of the spell are based on very few births.

% [Spell 3 - 1b/1g most interesting]

Even though there is no evidence of using sex selection to have an additional son when the 
previous children are all sons, it is possible that families want more sons than daughters 
and therefore use sex selection for the third spell when they already have one of each.
Urban women show little evidence of this being the case.
Although the sex ratio is above the natural from quarter 11 until the end of the spell,
only about \sfrac{1}{3} of the births occur in this interval.
Furthermore, the dip below the natural sex ratio prior to this point also accounts for
approximately \sfrac{1}{3} of the births.
Hence, these two parts cancel each other out.
With the first part being to the natural sex ratio the end result is that there
is little evidence for urban women using sex selection if they already have a son and
a daughter.

For rural women, however, the dip and the associated number of births is not
large enough to bring the overall sex ratio for the spell back to natural sex ratio.
The sex ratio for the entire spell is therefore likely more male dominated
than the natural rate, although it is only statistically significantly higher for two 
individual quarters of the spell.
There are two possible explanations for rural women appearing more likely than urban
women to use sex selection on the third spell if they already have a son and a daughter.
First, rural women could have a stronger son preference than urban women and want two sons 
rather than only one.
Second, mortality is higher in rural than urban areas and this higher mortality, either 
experienced or anticipated, make rural women use sex selection to secure a second son, 
essentially as an insurance.
Separating these two possible explanations is difficult, but the ``heir and a spare''
explanation appears  more likely than difference in preferences given that urban women use 
sex selection more than rural women for the second birth in the absence of a son.
An alternative explanation is that what we are observing is simply random variation.
This would explain why we get see differences in sex ratios between urban and rural
women with one son across the third and fourth spells as shown in the online Appendix.%
\footnote{
Urban women with a son and two daughters as their first 3 births show a sex ratio
that is significantly higher than the natural, but no evidence of sex
selection if they have two sons and a daughter as their first children.
Rural women show only a marginally higher sex ratio with a son and two daughters and no 
effect at all with two sons and a daughter.
These results are, however, based on very small samples.
There are 590 urban women with a son and two daughters, of which less than 40\% have a
fourth child, 344 urban women with two sons and one daughter, and only 20\% of those
have a fourth child.
For rural women the numbers are 564, with 40\% having a fourth child, and 367 with
approximately 30\% having a fourth birth.
This comes to approximately 300 urban births and only slightly more for rural across
the two sex compositions.
}

Son preference can manifest itself through other channels than sex selection.
Differential investments in education and health across boys and girls, resulting in
lower education and higher mortality of girls relatively to boys, could be thought 
of as evidence of son preference, although \cite{rosenzweig1982a} argue that both may 
represent parents' rational response to market opportunities rather than inherent 
preferences of parents.
Whatever way son preference presents itself, it appears that the dominant  
preference---as expressed through use of sex selection---is for one son, rather 
than multiple sons.


\begin{figure}[htp]%
\centering
\includegraphics[width=0.9\textwidth]{spell2_g3_high_r4_s.eps}
\caption{Survival curve example}
\label{fig:survival}
\end{figure}



\section{Conclusion\label{sec:conclusion}}


If women have preferences over both the number of children they have \emph{and} the sex 
composition of these children, they face a trade-off between the cost---both monetary 
and psychological---of sex selection and the cost of children.
This paper introduces a novel approach to understanding the relationship between
fertility and use of sex selection, and how sex selection interact with birth spacing.
The proposed method allows for joint estimation of fertility and sex-selective abortions 
using a non-proportional, competing risk hazard model.

Three results stand out.
First, lower fertility is an important factor in the decision to use sex-selective abortions.
% Women wanting more than two births did not use sex selection on the second birth, while
% those who want two or fewer use sex selection intensively.
In both urban and rural areas, only women with the lowest predicted fertility---those
with 8 or more years of education---use sex selection, and as fertility falls they use it 
for earlier parities.
Women with less education instead follow a high fertility strategy to ensure they have
a son and do not appear to use sex selection.
Previous research was unable to explain why sex selection in India only occurred 
among higher education women because it failed to tie the use of sex selection and the 
fertility decision together.

Second, the legal steps taken to combat sex-selective abortion have not been able
to reverse the practice.
The use of sex selection has been increasing over time and is higher now than before the
PNDT Act was passed.
The predicted number of sex-selective abortions per 100 women during their childbearing 
years is now around 10 for women with 8 or more years of education in both urban and
rural areas.
This is especially interesting given that the cost of prenatal sex determination 
has likely increased in response to the PNDT Act.
This also emphasizes why accounting for the possibility that parents may change
their decision to use sex selection within a spell is important. 
As the theory shows an increase in the cost of prenatal sex determination 
increases the likelihood of parents changing their decision to use sex selection.%
\footnote{
A feature of the new approach is that it allows for changes in the decision 
to use sex selection within a spell.
This is most clearly seen for better-educated urban women whose only child is a girl, 
for whom the use of sex selection declines as the second spell becomes longer.
}


Third,  there is little evidence that women with one or more sons use sex selection,
and their probability of another birth declines substantially once they have a son.
The main exception is for rural women with one son and one daughter, presumably as an 
insurance against mortality.
Hence, parents appear to have a preference for one son rather than multiple sons.

These results imply that the way we have been measuring son preference may not be useful
for understanding how people make decisions on use of sex selection.
Recent research suggests that son preference in India, when measured as ideally having 
more boys than girls, is decreasing over time and with higher education \citep{bhat03,pande07}.%
\footnote{
This measure of son preference is commonly used in the literature. 
See, for example, \citet{clark00}, \citet{Jensen2009}, and \cite{Hu2015}.
In NFHS, and Demographic and Health Surveys in general, son preference is measured as part 
of the general fertility questions.
The first question is how many children a woman would have if she could choose exactly how 
many.
This if followed by the sex preference question:
``How many of these children would you like to be boys, how many would you like to be girls 
and for how many would the sex not matter?''
}
Despite this, use of sex selection has increased and for exactly the group 
argued to show declining preference for sons.
The culprit may be that parents want \emph{one} son and will use sex selection 
intensively to get that son, but refrain from using sex selection to have more sons, 
and our standard measure of son preference cannot captured that.
A better approach may be to directly ask if there is a minimum number of each sex that 
respondents want, possibly in combination with ideal number questions.

This is of more than theoretical interest.
If son preference is expressed as a very strong desire for having at least one
son then reductions in fertility will lead to a further increase in the use of sex 
selection across all groups that hold that preference.
What is more, as shown in the theory section, it is possible to get increasingly
unequal sex ratio when fertility falls even if parents do not want more sons
than daughters as long as they only use abortion on female fetuses.
A deeper understanding of exactly what type of son preference is responsible 
for the increase use of sex selection is clearly an important question for
future data collection and research.

An often-repeated, but poorly substantiated, argument for why there is sex selection 
in India is the dowry system.
This explanation fits poorly with observed behavior.
If true, dowries increase the cost of having girls relative to boys because you have to 
pay to ensure your girls are married.
This implies that poorer households should be the most likely to use sex selection, and 
that the use of sex selection should increase the more girls they have.%
\footnote{
The exception would be if dowries increase more than proportionally with education level 
and income, but there does not appear to be evidence of this.
} 
Similarly, for all households there would be a financial incentive to achieve a higher 
number of sons than daughters.
As shown here, however, sex selection is predominantly used by better educated and 
therefore, on average, wealthier households and only to secure one son rather than multiple 
sons.
Finally, fertility's rapid decline after the introduction of prenatal sex 
determination, combined with low cost of sex selection, suggests that parents'
perceived costs of boys and girls are similar.%
\footnote{
See the discussion in Section \ref{sec:changes} for more detail.
}
Hence, banning dowries, as India has done, is unlikely to affect the use of sex 
selection. 

The results presented here lead to a number of important questions for future research.
One is whether marriage market sex ratios affect the use of sex selection.
Parents may care not only about the number and type of children they have but 
also whether they will have grandchildren.
If they care about grandchildren, then a boy could still be preferred over a girl, but a 
married girl would be preferred over an unmarried boy \citep{Bhaskar2011}.
The implication is that parents respond to changes in the expected sex ratio of their
children's marriage market, even given their preference for a son.
This hypothesis can be tested using measures of the observed distributions of
boys and girls and the method proposed here.

A second question for future research, which is especially important when designing 
policies aimed at reducing the use of sex selection, is who the agents are in the decisions 
to use sex selection.
Here, I have treated the household as a single unit, but it is possible, and indeed likely, 
that husband and wife have different preferences over both the factors that determine  
use of sex selection and sex selection itself.
Some of the popular press have assumed that the decision to use sex selection rests 
with the husband.
This, however, does not fit well with the general pattern that women, on average, prefer to 
have fewer children than men do and that declining fertility appear to be the main
driver of sex selection.
A better understanding of who holds what preferences in the family would be a good 
first step to disentangling who makes decisions about sex selection.

A third question is whether there are interactions between sex selection and child outcomes,
such a survival and education.
It has been argued that access to sex selection may benefit girls because those girls
who are born are more likely to be wanted by their parents.
For mortality it is, however, possible that improvements are the result of a purely 
mechanical effect.
As I have shown, increased use of sex selection lengthens the birth spacing after a girl 
and that by itself can improve her survival chances, even if she is no more wanted than before.
The method suggested here can predict the likelihood of a woman using prenatal sex 
determination for a given birth based on her characteristics.
This predicted probability can then be used in estimations of the determinants
of child health to directly test whether increasing use of sex selection is 
beneficial for the girls who are born.
It is unlikely, however, that a substantial mortality effect exists since sex 
selection is mainly used by well-educated women, who tend to have lower mortality, 
but there might be effects on education investments.

% [developments in the future]
In conclusion, the results provide strong clues to how sex selection use will change in the future.
Because lower fertility is responsible for the increase in sex-selective abortions, 
it is likely that we will see further increases in the practice as more families 
want fewer children, either because of urbanization or because of increases in female 
education.
With already low fertility for better-educated urban women, a substantial future increase 
in sex-selective abortions per woman is unlikely, 
but a higher proportion of women will belong to this group in the future.
For rural, better-educated women fertility is still falling.
To the extent that it falls to the same level as for urban
women, I expect a corresponding increase in sex selection.
Finally, and most importantly, we are beginning to see evidence of lower fertility 
for women with lower levels of education.
If women with less education hold the same preference for one son as the better-educated,
once the less-educated women's fertility begins to drop to three---and even two---we are 
likely to see a substantially increased use of sex selection in India.






\newpage
\onehalfspacing
\bibliographystyle{aer}
\bibliography{collection}

\addcontentsline{toc}{section}{References}



\clearpage
\newpage

\appendix
\section{Appendix}

% CHANGING NUMBERING OF FIGURES AND TABLES FOR APPENDIX
\renewcommand\thefigure{\thesection.\arabic{figure}}    
\setcounter{figure}{0}
\renewcommand\thetable{\thesection.\arabic{table}}    
\setcounter{table}{0}
  
% Descriptive statistics tables
\input{../tables/des_stat.tex}



\end{document}



