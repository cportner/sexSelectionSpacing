% Revamped version for Demography focusing on method

\documentclass[12pt,letterpaper]{article}

\usepackage{amsmath}
\usepackage{fontspec}
\setromanfont[Ligatures=TeX]{TeX Gyre Pagella}
\usepackage{unicode-math}
\setmathfont{TeX Gyre Pagella Math}
\usepackage[title]{appendix}
\usepackage[margin=1.0in]{geometry}
\usepackage[figuresleft]{rotating}
\usepackage[longnamesfirst]{natbib}
\usepackage{dcolumn}
\usepackage{booktabs}
\usepackage{multirow}
\usepackage[flushleft]{threeparttable}
\usepackage{setspace}
\usepackage[justification=centering]{caption}
\usepackage[font=scriptsize]{subfig}
\usepackage[xetex,colorlinks=true,linkcolor=black,citecolor=black,urlcolor=black]{hyperref}
\usepackage{adjustbox}
\usepackage{xfrac}
\usepackage{placeins}


% \bibpunct{(}{)}{;}{a}{}{,}
\newcommand{\mco}[1]{\multicolumn{1}{c}{#1}}
\newcommand{\mct}[1]{\multicolumn{2}{c}{#1}}
\newcommand{\mcth}[1]{\multicolumn{3}{c}{#1}}
\newcommand{\X}{$\times$ }
\newcommand{\hs}{\hspace{15pt}}

% Attempt to squeeze more floats in
\renewcommand\floatpagefraction{.9}
\renewcommand\topfraction{.9}
\renewcommand\bottomfraction{.9}
\renewcommand\textfraction{.1}
\setcounter{totalnumber}{50}
\setcounter{topnumber}{50}
\setcounter{bottomnumber}{50}


%------------------------------------------------------------------------


\title{Birth Spacing in the Presence of Son Preference and Sex-Selective Abortions:
India's Experience over Four Decades%
\protect\thanks{%
I am grateful to Andrew Foster and Darryl Holman for discussions about the method.
I owe thanks to Shelly Lundberg, Daniel Rees, David Ribar, 
Hendrik Wolff, seminar participants at University of Copenhagen, University of Michigan, 
University of Washington, University of Aarhus, the Fourth 
Annual Conference on Population, Reproductive Health, 
and Economic Development, and the Economic Demography Workshop for helpful 
suggestions and comments.
I would also like to thank Nalina Varanasi for research assistance.
Support for development of the method from the University of Washington Royalty 
Research Fund and the Development Research Group of the World Bank is gratefully 
acknowledged.
The views and findings expressed here are those of the author and
should not be attributed to the World Bank or any of its member countries.
Partial support for this research came from a Eunice Kennedy Shriver National
Institute of Child Health and Human Development research infrastructure grant,
5R24HD042828, to the Center for Studies in Demography and Ecology at the
University of Washington.
}
}

\author{Claus C P\"ortner\\
    Department of Economics\\
    Albers School of Business and Economics\\
    Seattle University, P.O. Box 222000\\
    Seattle, WA 98122\\
    \href{mailto:claus@clausportner.com}{\texttt{claus@clausportner.com}}\\
    \href{http://www.clausportner.com}{\texttt{www.clausportner.com}}\\
    \& \\
    Center for Studies in Demography and Ecology \\
    University of Washington\\ \vspace{2cm}
    }

\date{March 2018}


%------------------------------------------------------------------------


\begin{document}
\graphicspath{{../figures/}}
\DeclareGraphicsExtensions{.eps,.jpg,.pdf,.mps,.png}

\setcounter{page}{-1}
\maketitle
\thispagestyle{empty}

% \setcounter{page}{0}


\newpage
\thispagestyle{empty}
\doublespacing

\begin{abstract}

% Demography abstract
\noindent 

Strong son preference is typically associated with shorter birth spacing in the absence of 
sons, but access to sex selection has the potential to reverse this pattern because each 
abortion extends spacing by six to twelve months. 
I introduce an empirical method that simultaneously accounts for how sex selection 
increases both birth spacing and the likelihood of a son. 
Using the four rounds of India's National Family and Health Surveys, I show that birth 
intervals increased substantially over the last four decades, most substantially among 
women most likely to use sex selection.
Specifically, well-educated women with no boys now have significantly longer birth
intervals, and more male-biased sex ratios, than similar women with boys. 
Women with no education still follow the standard pattern of short average spacing when 
they have girls and little evidence of sex selection, with medium-educated women showing 
mixed results.
Furthermore, the traditional pattern of longer average spacing with sons appears to arise 
not from fewer very short birth intervals, but rather from the upper end of the distribution.
Finally, northern and western India are responsible for most of the spacing pattern 
reversal, although even the south experienced substantially increased intervals from
sex selection. 


\noindent JEL: J1, O12, I1
\noindent Keywords: India, prenatal sex determination, censoring, competing risk, non-proportional hazard
\end{abstract}

\newpage


%------------------------------------------------------------------------

\section{Introduction\label{sec:intro}}

Birth spacing has long served as a measure of son preference, with strong
son preference typically associated with shorter birth intervals the
fewer sons a family has \citep{ben-porath76b,Leung1988}.
However, the introduction of prenatal sex determination has the potential to
fundamentally change the relationship between son preference and birth spacing.
The stronger the son preference, the more likely couples are to resort to sex-selective 
abortions, and each abortion automatically increases the interval between births by six 
months to a year.%
\footnote{
The increase consists of three parts.
First, starting from the time of the abortion, the uterus needs at 
least two menstrual cycles to recover;  otherwise, the likelihood 
of spontaneous abortion increases substantially \citep{zhou00b}.
The second part is the waiting time to conception, which is between 
one and six months \citep{Wang2003}.
Finally, sex determination tests are reliable only from three months 
of gestation onwards.
}
Longer birth intervals in the absence of sons, which previously would be taken to indicate 
a lower son preference, now may arise precisely because a sufficiently strong son 
preference makes couples use sex selection.

In this paper, I examine how birth spacing in India has changed over time and across 
groups with the introduction of sex selection.
I introduce and apply an empirical method that directly incorporates the effects of 
sex-selective abortions on the duration between births 
\emph{and} 
the likelihood of a son. 
The method can be used to analyze both situations with and without prenatal
sex selection.
I apply the method to birth histories of Hindu women, using data from the four
India's National Family and Health Surveys (NFHS), covering the period 
1972 to 2016. 


% [Why India?]
India is a particularly compelling case since it has a long history of male preference, 
especially in the northern states \citep{Kishor1993,murthi95,arnold98}.
A significant proportion of Indian families exhibit differential stopping behavior, where, 
for a given number of children, they are more likely to have an additional child if they 
have not yet reached their preferred number of sons 
\citep{repetto72,Das1987,Arnold1997,arnold98,clark00,Basu2010,Barcellos2014}.%
\footnote{
Other countries show a similar pattern
\citep[see, for example,][]{larsen98,filmer09,Altindag2016}.
}
Moreover, mortality risk is higher for females than males for most ages groups, and
there has been an almost continuous increase in India's overall ratio of males to females 
over the last century \citep{dyson01,Navaneetham2011,Bongaarts2015}.

% [What do we know about use of sex selection?]
Over the last three decades, India has seen dramatic increases in the males-to-females 
ratio at birth as access to prenatal sex determination expanded 
\citep{das_gupta97,Sudha1999,Arnold2002,retherford03b,jha06,Guilmoto2012,Portner2015b}.
Simultaneously, fertility has declined substantially, to the point where it is now at, or 
even below, replacement in some areas 
\citep{Guilmoto2013,Dharmalingam2014,International-Institute-for-Population-Sciences-IIPS2017}.
This decline in fertility, combined with a continued strong preference for sons, has 
intensified the pressure to use sex selection 
\citep{das_gupta97,Guilmoto2009a,Bongaarts2013,Jayachandran2017}.%
\footnote{
In some instances, however, the fertility decline appears to have created a stronger 
aversion to daughters rather than a stronger preference for son \citep{DiamondSmith2008}.
}
The strongest predictor of the use of sex selection is the sex composition of previous 
children; for families without a son, the higher the parity, the higher is the probability
of having a son as the next birth \citep{retherford03b,jha06,abrevaya09,Portner2015b}.
The use of sex selection also increases with socioeconomic status, especially 
education, and is more prevalent in cities than in rural areas, both possibly because 
of lower desired fertility \citep{retherford03b,jha06,Portner2015b}.


% [Why spacing?] 
There are four main reasons for examining birth spacing and how it changes with 
access to sex selection.
First, researchers have made extensive use of birth spacing as a measure of son preference, 
and it is critical to understand to what extent spacing is still useful as a measure of son 
preference.
In India, the sex of the last-born child significantly impacts the duration to next birth,
with the expected birth interval about three percent longer if the last-born was a son than
if it was a daughter \citep{Bhalotra2008,Kumar2016}.
Similarly, ethnic Indians in South Africa also show a longer duration after the birth of a 
son than after a daughter \citep{Gangadharan2003}.%
\footnote{
Ethnic Indians in Malaysia do, however, show little evidence of son preference, although
the sample sizes are small \citep{Pong1994}.
}
In both Bangladesh and Vietnam the more boys a family has the longer the expected duration 
to the next birth \citep{Haughton1995,Haughton1996,Rahman1993,Soest2018}.
Outside Asia, the evidence is more mixed.
North Africa shows shorter spacing in the absence of sons, while a similar pattern does
not exist in Sub-Saharan Africa \citep{Rossi2015}.%
\footnote{
A possible exception is Senegal, where women with a higher risk of widowhood show
shorter birth spacing until they have secured a son \citep{Lambert2016}.
}

Second, birth spacing can have a substantial impact on children's outcomes.
Most of the research finds a negative impact of short spacing on child health---especially 
for very short intervals of 18 months or less---although identifying the causal mechanisms 
have proven more difficult \citep{Conde-Agudelo2006,Conde-Agudelo2012}.
There are three pathways through which birth intervals can affect health.
First, there is maternal depletion, where the mother does not have sufficient time to 
recover after her last pregnancy before she becomes pregnant again.%
\footnote{
This depletion can take the form of nutrition, folate, or cervical insufficiency.
}
Second, disease transmission either from mother to child or between children.
Third, sibling competition, where siblings close in age compete for scarce resources and
parental care.
Which mechanism dominates is of particular interest when son preference drives spacing 
decisions.
Both the maternal depletion and the transmission of disease explanations predominately
affect the subsequent child and the mother.
Only sibling competition has the potential to affect the older child negatively.

In India, there is evidence that both the prior and the subsequent child have worse health 
outcomes and a higher likelihood of dying the shorter the interval between them, although 
the effect is not symmetrical
\citep{Whitworth2002,Bhargava2003,Rutstein2005,Maitra2008,Makepeace2008,Ghosh2018}.%
\footnote{
\cite{Bhalotra2008}, however, find that birth spacing only explains a limited amount
of neonatal mortality.
}
The increased mortality risk for prior children may come from shorter breastfeeding of 
girls to ensure that the mother can conceive again sooner, and with poor water
quality this shorter breastfeeding may result both in higher risk of dying and shorter
stature if the child survives \citep{Jayachandran2011,Jayachandran2017a}.
In other countries, focus has been mostly on the subsequent birth, but there is evidence 
in countries as diverse as Bangladesh, Brazil, El Salvadore, and Pakistan that
short intervals between births have adverse health effects
\citep{Cleland1984,Curtis1993,Davanzo2008,Gribble2009,Saha2013}.

The potential effects of spacing are not restricted to early outcomes, although
there is less evidence on longer-term effects and most comes from Western countries.
In the US, for example, a longer interval between births increases the older sibling's
test scores on the Peabody Individual Achievement Test, although there is no
statistically significant effect on the younger sibling \citep{Buckles2012}.
In line with this, close spacing appears to increase the risk of dropping out of
high school and lower the probability of attending a post-secondary school in both the
US and Sweden \citep{Powell1993,Pettersson-Lidbom2009}.
These results are, however, not uniformly supported, with recent results from Sweden 
showing no relationship between birth spacing, even if very short, and outcomes such as 
years of education completed, earnings, and unemployment \citep{Barclay2017}.

Third, the mother may also be affected by the length of the birth intervals.
The effect of spacing on mothers have received less attention than the effect on children, 
and reviews of the literature show mixed results on mother's anthropometric status 
\citep{Dewey2007,Conde-Agudelo2012}.
For India, there is, however, evidence that women with first-born girls are both
more likely to have short birth spacing and more likely to have anemia, possibly as
the result of the short spacing resulting in maternal depletion \citep{Milazzo2018}.
In Western countries, having a longer interval between births appear to have positive
effects on long-term labor market outcomes, such as participation and 
income \citep{Gough2017,Karimi2014}.

Finally, we know less about what determines spacing behavior in developing countries than 
in developed countries, and with declining fertility and increasing numbers of women 
entering the labor force in developing countries, understanding how couples make timing 
decisions will be necessary for the design of suitable policies \citep{Portner2018}.%
\footnote{
It is clear that birth spacing does respond to policies in both developed and 
developing countries \citep{Pettersson-Lidbom2009,Todd2012,Meckel2015,Ghosh2018}.
}
For example, the evidence on the effects of education and socio-economic status on
birth intervals is mixed, with some research showing either a positive association, no
association, or one that is changing over time 
\citep{Tulasidhar1993,Whitworth2002,Bhalotra2008,Kim2010,Soest2018}.

Even the effects of access to contraceptive and declining fertility on spacing are unclear.
On the one hand, access to contraceptives allows women to avoid too short spacing between 
birth, which would increase birth spacing.
On the other hand, increased reliability of access and effectiveness of contraceptives can 
lead to shorter intervals between births  \citep{Keyfitz1971,Heckman1976}.
With less reliable contraception parents choose a higher level of contraception, which 
results in longer spacing, to avoid having too many children by accident.
However, as contraception becomes more effective parents can more easily avoid future 
births, allowing them to reduce the spacing between births without having to worry about 
overshooting their preferred number of children.
These two counteracting effects may explain why better-educated women have shorter spacing 
than less educated women in some instances but not in others and why finding statistically 
significant effects of contraception use on birth spacing is difficult \citep{Yeakey2009}.%
\footnote{
There is evidence in Matlab that the effect of son preference on birth spacing is
stronger in areas with better access to family planning \citep{Rahman1993}.
Other evidence, however, points to families ability to time birth, even in the absence
of modern contraceptives \citep{Jayachandran2011,Alam2018}.
}

% [Hypotheses / questions]
Despite the increasing use of sex selection, the important impacts of spacing on outcomes 
for both mother and children, and the extensive use of spacing to measure son 
preference in the literature, there has so far been no examination of what effect access 
to sex selection has had on spacing.
The overall hypothesis is that spacing patterns significantly changed as prenatal sex 
determination became available.
Specifically, I address three questions.

First, how did average birth intervals change over time for different groups and how are 
those changes related to the likely use of sex selection?
The underlying hypothesis is that in the absence of sex selection son preference leads to 
shorter spacing the fewer boys a family has for a given parity, whereas son preference 
increases the spacing the fewer boys a family has when prenatal sex selection is available.
I still expect parents to \emph{try} to conceive earlier after the birth of a girl than 
after a boy, but the abortions will increase the duration between births, and may even 
lead to longer average spacing after the birth of girls than after boys.
Families with son preference who---for one reason or another---do not use prenatal sex 
selection may continue to have shorter birth spacing.
Based on the prior literature on sex selection, I focus on how birth intervals have
changed for different education levels, whether urban or rural, and the sex composition of 
prior children.

Second, have the shapes of the survival curves changed?
Very short birth intervals are associated with worse health outcomes and increasing 
mortality risk \citep{Conde-Agudelo2012}.
Hence, understanding how the distribution of birth intervals changes help us understand to 
what extent we can expect improvements in health outcomes.
We should, for example, expect different impacts on the distribution of birth intervals
depending on whether parents change their spacing behavior based on an increasing 
concern for child health or from a strong son preference leading to continued pregnancies
until a son is conceive.

Finally, how are the changes in birth intervals with increasing access to sex selection 
distributed geographically?
Given the size and cultural differences within India, it is unlikely that the effect of
increasing access to sex selection will have had the same effect throughout 
\citep{retherford03b}.
I, therefore, examine how birth intervals have changed across four broad regions.


Direct information on the use of sex selection is not available, but it is possible
to compare different periods as a proxy since there has been substantial changes in access 
and legality of prenatal sex determination in India.
Abortion has been legal in India since 1971 and still is.
The first reports of sex determination appeared around 1982--83 
\citep{Sudha1999,bhat06,Grover2006}.
The number of clinics quickly increased, and knowledge about sex selection became 
widespread after a senior government official's wife aborted a fetus that turned out to be 
male \citep[p.\ 598]{Sudha1999}.
In 1994, the Central Government passed the Prenatal Diagnostic Techniques (PNDT) Act, 
making determining and communicating the sex of a fetus illegal.%
\footnote{
Details about the act are at \href{http://pndt.gov.in/}{http://pndt.gov.in/}.
The number of convictions has been low.
It took until January 2008 for the first state, Haryana, to reach five convictions.
Hence, private clinics apparently operate with little risk of legal action 
\citep{Sudha1999}.
Furthermore, there is little evidence that bans like this significantly
affect sex ratios \citep{Das-Gupta2016}.
}

Based on these development, I split the four decades into four periods.
The first period cover until prenatal sex determination became widely available, 
1972--1984.
The second period saw (mostly) unrestricted access to sex selection, 1985--1994.
Although, as reviewed above, there is evidence that there has been a substantial 
increase in the use of sex selection, some researchers argue that we might be close to or 
have already passed a turning-point in the use of sex selection 
\citep{Das_Gupta2009,Diamond-Smith2015}.
I, therefore, split the period from the implementation of the PNDT Act until the last 
survey into two, 1995--2004, and 2005--2016.


There are four main results.
First, there has been a general increase in the intervals between births for all education 
groups over the four decades covered by the data.
These increases are larger the higher the parity and the higher the education level.
Second, those women who are most likely to use sex selection, well-educated women with no
sons, have seen the largest increases in birth intervals.
As a result, we now have a complete reversal of the traditional spacing pattern for
well-educated women, and instead observe the longest spacing for women with no sons.
Third, women who are least likely to use sex selection, those with no education in
rural areas, still follow the standard pattern of short spacing when they have girls and 
little evidence of the use of sex selection.
In other words, these women adhere to a strategy where they achieve a son through higher 
fertility rather than the use of sex selection.
Fourth, the traditional pattern of longer average spacing when women have one or more sons
appear to arise from more very long intervals, rather than from fewer very short birth 
intervals; across education groups there appear to be little difference in the likelihood
of having a very short birth interval across sex compositions, especially for lower
parities.


%------------------------------------------------------------------------------------


\section{Estimation Strategy\label{sec:strategy}}

% Requirements for method and the intuition behind each:  
%  - Account for censoring (hazard model)
%  - Work both with and without sex selection (competing risk part)
%  - Allow for changes in use of sex selection within spell (flexible specification of baseline hazard ?)
%  - Capture that the shape of the hazard function differ across groups
%    (non-proportionality); needs a better description, maybe using example
%    of shorter spacing and/or use of sex selection. Differences in parity
%    progression likelihood. 

The standard approach in the birth spacing literature is to use proportional hazard
models with a single exit, the birth of a child.
There are two problems with the standard approach in this setting.
First, and most importantly, the introduction of sex selection means that the sex of the 
next child is no longer necessarily random and that parents' choices will impact the 
spacing to the birth of a girl or a boy differentially.
I, therefore, use a competing risk set-up, which can capture both the non-randomness of
the birth outcome and the differential spacing to the birth of a girl or a boy.
To my knowledge, this is the first application of competing risk models to birth
spacing.%
\footnote{
\cite{Merli2000} used a discrete hazard model to examine whether 
there were under-reporting of births in rural China, although they 
estimated separate waiting time regressions for boys and girls.
}

Second, it is unlikely---even in the absence of prenatal sex determination---that the 
effect of household characteristics such as the sex composition of previous births have 
the same effect throughout the entire spell as is assumed by the proportional hazard
model.
Assuming that the effect of sex composition is the same throughout a spell is especially 
problematic for higher-order spells where different sex composition of previous births can 
lead to substantial differences both in the likelihood of progressing to the next birth 
and how soon couples want their next child if they are going to have one.
The introduction of prenatal sex determination exacerbates any bias from the 
proportionality assumption since sex-selective abortion use varies across groups,
which affects birth spacing, and because a household's use of sex selection may vary 
within a spell, which means that the effects of covariates vary within the spell as well.
I, therefore, use a non-proportional hazard specification, which allows the shape of the 
hazard functions to vary across groups.
The use of a non-proportional specification also mitigates any potential effects 
of unobserved heterogeneity when used in conjunction with a flexible baseline hazard 
\citep{Dolton1995}.

The model is a discrete time, non-proportional, competing risk hazard model with two exit 
states: either a boy or a girl is born.
The unit of analysis is a spell, the period from one birth to the next.
For each woman, $i=1,\ldots,n$, the starting point for a spell is time $t=1$, and 
the spell continues until time $t_i$, when either a birth occurs or the spell 
is censored.%
\footnote{
The time of censoring is assumed independent of the hazard rate,
as is standard in the literature.
}
There are two exit states: the birth of a boy, $j=1$, or the birth of a girl, $j=2$, and 
$J_i$ is a random variable indicating which event took place.
The discrete time hazard rate $h_{ijt}$ is specified as
\begin{equation}
 h_{ijt} = \frac{\exp(D_j(t) + \alpha_{jt}'\mathbf{Z}_{it} + \beta_j'\mathbf{X}_{i})} 
 {1 + \sum_{l=1}^2 \exp(D_j(t) + \alpha_{lt}'\mathbf{Z}_{it} + \beta_l'\mathbf{X}_{i})} \: \: \; \; \;  j = 1,2
 \label{eq:hazard}
\end{equation}
where the explanatory variable vectors, $\mathbf{Z}_{it}$ and $\mathbf{X}_{i}$, capture 
individual, household, and community characteristics discussed below,
and $D_{j}(t)$ is the piece-wise linear baseline hazard for outcome $j$, captured
by dummies and the associated coefficients,
\begin{equation}
D_j(t) = \gamma_{j1} D_1 + \gamma_{j2} D_2 + \ldots + \gamma_{jT} D_T,
\end{equation}
with $D_m = 1$ if $t=m$ and zero otherwise.
This approach to modeling the baseline hazard is flexible and does not place 
overly strong restrictions on the baseline hazard.

The explanatory variables in $\mathbf{Z}$ and the interactions between them 
are the non-proportional part of the model, which means that they are
interacted with the baseline hazard:
\begin{equation}
 \mathbf{Z}_{it} = D_j(t) \times (\mathbf{Z}_1 + Z_2 + \mathbf{Z}_1 \times Z_2),
\end{equation}
where $D_j(t)$ is the piece-wise linear baseline hazard and $\mathbf{Z}_1$ captures sex 
composition of previous children, if any, and $Z_2$ captures area of residence.
The remaining explanatory variables, $\mathbf{X}$, enter proportionally,
but to further minimize any potential bias from assuming proportionality, estimations 
are done separately for different levels of mothers' education and for different 
time periods.%
\footnote{
The choice of which variables to use for non-proportionality is discussed in more detail
in the ``Explanatory Variables'' section.
}

Equation (\ref{eq:hazard}) is equivalent to the logistic hazard model and has the same 
likelihood function as the multinomial logit model \citep{allison82,jenkins95}.
Hence, transforming the data, so each observation is an interval---here equal
to three months---the model can be estimated using a standard multinomial logit model.%
\footnote{
The multinomial model assumes that alternative exit states are stochastically independent
(Independence of Irrelevant Alternatives or IIA), which rules out any individual-specific 
unmeasured or unobservable factors that affect both the hazard of having a girl and the 
hazard of having a boy.
I, therefore, include a proxy for fecundity discussed in Section \ref{sec:data}.
Also, the multivariate probit model can be used as an alternative because the IIA is not 
imposed \citep{han90}.
The results are substantially identical between these two models and
available upon request.
}
In the reorganized data the outcome variable is 0 if the woman does not have a child in a 
given interval (the base outcome), 1 if she gives birth to a son in that interval, and 2 
if she gives birth to a daughter in that interval.

The distribution of spacing is captured by the survival curve, which shows the probability 
of not having had a birth yet by spell duration, for a given set of explanatory variables.
The survival curve at time $t$ is 
\begin{equation}
\label{eq:survival}
S_{t} 
= 
\prod_{d=1}^t
\left(
\frac{ 1 }
{1 + \sum_{l=2}^2 \exp(D_j(t) + \alpha_{ld}'\mathbf{Z}_{kd} + \beta_l'\mathbf{X}_{k})}
\right).
\end{equation}

Because the probability of ever having a next birth varies across groups, a direct 
comparison of standard survival curves tells us little about how the spread of sex 
selection affects birth spacing across groups.
I, therefore, condition on the predicted likelihood of parity progression when examining 
birth spacing measures, such as the average duration to a birth.
The reliability of this approach depends on whether the spell length covered is 
sufficiently long that few women are likely to give birth after the spell cut-off.
I discuss the choice of spell length below.%
\footnote{
It is important to note that the approach is not the same as merely calculating 
the birth spacing measures for women who already have a given 
parity child in the survey because that number does not take into account
the censoring of spells that will eventually lead to a birth.
}
In addition, I present graphs of survival curves conditional on parity progression,
which therefore begin at 100\% and end at 0\%.

Interpretation of the model coefficients is challenging \citep{thomas96}.
It is, however, possible to calculate the predicted probabilities of 
having a boy, $b$, and of having a girl, $g$, in period $t$, conditional on 
a set of explanatory variables and not having had a child before that period, as
\begin{align}
P(b_{t} | \mathbf{X}_{k}, \mathbf{Z}_{kt}, t ) 
& =  
\frac{ \exp(D_j(t) + \alpha_{1t}' \mathbf{Z}_{kt} + \beta_1' \mathbf{X}_{k} )}
{1 + \sum_{l=1}^2 \exp(D_j(t) + \alpha_{lt} ' \mathbf{Z}_{kt} + \beta_l ' \mathbf{X}_{k})}
\label{eq:probability_boy} \\
P(g_{t} | \mathbf{X}_{k}, \mathbf{Z}_{kt},t ) 
& =  
\frac{ \exp(D_j(t) + \alpha_{2t}'\mathbf{Z}_{kt} + \beta_2'\mathbf{X}_{k} )}
{1 + \sum_{l=2}^2 \exp(D_j(t) + \alpha_{lt}'\mathbf{Z}_{kt} + \beta_l'\mathbf{X}_{k})}
\label{eq:probability_girl}
\end{align}
It is then straightforward to calculate the estimated percentage of children born that 
are boys, $\hat{Y}$, at each $t$:  
\begin{equation}
\hat{Y}_t 
= 
\frac{ P(b_{t} | \mathbf{X}_{k}, \mathbf{Z}_{kt},t )}
{ P(b_{t} | \mathbf{X}_{k}, \mathbf{Z}_{kt},t) + P(g_{t} | \mathbf{X}_{k}, \mathbf{Z}_{kt},t )} 
\times 100.
\label{eq:probability_son}
\end{equation}
Combining the percentage boys and the likelihood of exiting the spell 
across all $t$ gives the predicted percent boys born over the entire spell.


\section{Data\label{sec:data}}

The data come from the four rounds of the National Family Health Survey 
(NFHS-1, NFHS-2, NFHS-3, and NFHS-4),
collected in 1992--1993, 1998--1999, 2005--2006, and 2015--2016.%
\footnote{
A delay in the survey for Tripura means that NFHS-2 has some observation 
collected in 2000.
}
The surveys are large: NFHS-1 covered 89,777 ever-married women 
aged 13--49 from 88,562 households;
NFHS-2 covered 90,303 ever-married women aged 15--49 from 92,486 households;
NFHS-3 covered 124,385 never-married and ever-married women aged 
15--49 from 109,041 households;
and 
NFHS-4 covered 699,686 never-married and ever-married women aged
15--49 from 601,509 households surveyed.

I exclude visitors to the household, as well as
women who have been married more than once, divorced, or who are 
not living with their husband,
women with inconsistent age at marriage,
and those with missing information on education.
Women interviewed in NFHS-3 or NFHS-4 who were never married or where Gauna 
had not yet been performed were also dropped.
The same goes for women who had at least one multiple births,
reported having a birth before age 12, had a birth before marriage, or
duration between births of less than nine months.

Finally, I restrict the sample to Hindus,
who constitute about 80\% of India's population.
If the use of sex selection differ between Hindu and other religions, such 
as Sikhs, assuming that the baseline hazard is the same would lead to bias.
The other groups are each so small relative to Hindus that it is not
possible to estimate different baseline hazards for each group.
Furthermore, the other groups are so different in background and son 
preference that combining them into one group would not make sense.

There are four advantages to using the NFHS.
First, surveys enumerators pay careful attention to spacing between births and
probe for ``missed'' births.
Second, no other surveys cover as extended a period in the same amount of detail.
The four NFHS rounds allow me to show how spacing and 
sex ratio changed from before sex-selective abortions were available until 2016.
Third, NFHS has birth histories for a large number of women.
% %
% \footnote{
% The Special Fertility and Mortality Survey appears to cover a much large number of households
% than the three rounds of the NFHS combined, but \citet{jha06} only use the births that 
% took place in 1997 making their sample sizes by parity smaller than here.
% Their sample consists of 133,738 births of which 38,177 were first-born, 
% 36,241 second-born, and 23,756 were third-born.
% }
Finally, even if probing for missing births does not eliminate recall error,   
the overlap in cohorts covered and the large sample size makes it possible 
to establish where recall error remains a problem.

Recall error arises mainly from child mortality when respondents are 
reluctant to discuss deceased children.
Systematic recall error, where the likelihood of reporting a deceased 
child depends on the sex of the child, is especially problematic because 
it biases both spacing and sex ratios.
Probing catches many missed births, but systematic recall error is 
still a potentially substantial problem.

Three factors contribute to the problem here.
First, girls have significantly higher mortality risk than boys.
Second, son preference may increase the probability that parents recall boys 
more readily than girls.
Finally, in NFHS-1 and NFHS-2 enumerators probed only for a missed birth if the
initially reported birth interval was four calendar years or more,
but, given short durations between births, especially after the birth of a girl,
that procedure is unlikely to pick up all missed children.

Recall error is heavily dependent on how long ago a woman was married.%
\footnote{
See the discussion in the online Appendix.
}
I, therefore, drop women married 22 years or more.
The final sample consists of 
  395,695 women, with   815,360 parity one through four births.
% [number] women, with [number] parity one through four births.


\subsection{Spell Definition\label{sec:spell_def}}

I focus here on the second through fourth spells, that is, from the second birth until
at most the fourth birth.
The first spell from marriage to the first spell is excluded because of data issues.
In NFHS-1 and NFHS-2 the only information on marriage timing is age at marriage and all
spell durations are therefore imputed.
In NFHS-3 about a third of respondent did not provide complete information on the
timing of marriage and therefore had the length of their first spell imputed using only
the age at marriage.
The result of the imputation is that very large numbers of women end up with very short,
i.e. less than 9 months, or negative spells and that comparing spell duration over time
for the first spell is subject to an excessive amount of noise.

The second and subsequent spells begin nine months after the previous birth 
because that is the earliest we should expect to observe a new birth.
All spells continue until either a child is born or censoring occurs.
Censoring can happen for three reasons:
the survey takes place;
sterilization of the woman or her husband;
or too few births are observed for the method to work.
For all spells censoring is set at 96 months (eight years) after a birth can occur.
With these cut-offs, less than one percent of observed births occur after
the spell cut-off.%
\footnote{
\input{../tables/num_missed.tex}
}

I group spells into four periods based on the year of the previous birth:
1972--1984, 1985--1994, 1995--2004, and 2005--2016.
The periods follow the main changes in availability and legality of 
prenatal sex determination discussed in the Introduction.
The allocation into periods is determined by when conception and therefore 
decisions on sex selection can begin, even if we do not observe
any births until nine months later.

The allocation into periods means that some spells cover two periods.
A couple may, for example, have their first child in 1984, but not have their second 
child until 1986.
That couple's second spell will be in the 1972--1984 period, even though 
most of the interval falls in the 1985--1994 period.
Hence, prenatal sex determination was likely available when some
children from spells that began in the 1972--1984 period were conceived,
which could result in evidence of sex-selective abortions even for this
period.
Similarly, a spell that started in the 1985--1994 period may have been partly 
or mostly under the PDNT act.
The effect is a downward bias in the differences between the periods.

\subsection{Explanatory Variables}

I divide the explanatory variables into two groups.
The first group consists of variables expected to affect the shape of the hazard function
(the $\mathbf{Z}$ variables): 
mother's education, sex composition of previous children, and area of residence.
I chose these variables because the prior literature shows that they affect 
spacing choices and the likelihood of using sex selection.%
\footnote{
Increasing the number of variables interacted with the baseline hazard would
further lower the risk of bias but at the cost of requiring more data to 
precisely estimate.
}

The first variable that is likely to affect the hazard rate in a non-proportional manner
is education.
First, there is evidence, even in the absence of sex selection, that women with
different education levels have different hazard profiles, although the size and direction
of the effect vary across areas and time 
\citep{Tulasidhar1993,Whitworth2002,Bhalotra2008,Kim2010,Soest2018}.
Second, as discussed above, the use of sex selection increases with socioeconomic 
status, especially education, and the use of sex selection is unlikely to be the
same throughout a spell, which means that education cannot enter in a proportional
fashion. \citep{retherford03b,jha06,Portner2015b}.
One potential reason for the correlation between education and sex selection is that
increasing education of mothers correlates closely with lower fertility 
\citep{schultz97}.
This lower fertility leads, in turn, to a higher likelihood of using sex selection
\citep{das_gupta97,Guilmoto2009a,Bongaarts2013,Jayachandran2017}.%
\footnote{
In principle, it is possible that part of the higher likelihood is driven by a stronger
preference for son, independently of desired fertility.
There is, however, evidence that higher education women have shown a decline in
son preference \citep{bhat03,pande07}
}
Another reasons is that having more education is closely associated with higher income and 
wealth, which makes it easier to access illegal services such as prenatal sex 
determination.
It is not possible to separate these two effects here.

I divide women into three groups based on education attainment: no
education, one to seven years of education, and eight and more years of education.
The models are estimated separately for each education level to reduce
the potential problem of including other variables as proportional.%
\footnote{
Fathers' education has two opposite predicted effects: the associated higher income
should increase fertility and therefore lower the pressure to use sex selection, but
the higher income also makes the use of sex selection cheaper.
In practice, fathers' education had little effect on the hazards and the use of 
sex-selective abortions, and I, therefore, do not include it.
}

As discussed, the sex composition of previous children affects both the timing
of births and the use of sex-selective abortions 
\citep{retherford03b,jha06,Bhalotra2008,abrevaya09,Portner2015b,Kumar2016,Soest2018}.
I capture sex composition with dummy variables for the
possible combinations for the specific spell, ignoring the ordering of births.
As an example, for the third spell, three groups are used: Two boys,
one girl and one boy, and two girls.
It is, in principle, possible to estimate the model taking into account
the ordering of children, but this would further lower the power of the
test, by adding one additional group for the third spell and four
additional groups for the fourth spell. 

Whether a woman lives in an urban or rural area affect both her fertility level and
her access to sex selection \citep{retherford03b,jha06,Portner2015b}.
Area of residence is a dummy variable for the household living in
an urban area.
The cost of children is higher in urban areas than in rural areas, and 
there is easier access to prenatal sex determination in urban areas.
Both factors are likely to lead to higher use of sex-selective abortions
in urban than in rural areas.

The second group of variables consists of those expected to have an 
approximately proportional effect on the hazard.
These include
the age of the mother at the beginning of the spell, 
whether the household owns any land, 
and whether the household belongs to a scheduled tribe or caste.


\subsection{Descriptive Statistics}

Appendix Table \ref{tab:des_stat1} presents descriptive statistics for
the spells by education level and period they began.
There is a substantial number of censored observations.
As an example, for highly educated women who had their first child in the 2005--2016
period, more than 40\% did not have their second child by the time of the survey.
Censoring becomes even more important for the third and fourth
spells, with around 70\% of the observations censored.
The level of censoring also increases with parity and time,
which reflects a combination of factors: timing of the surveys
relative to the periods of interest, later beginning of childbearing, 
falling fertility, and the increase in spell lengths from 
sex-selective abortions.

The share of urban women in the sample has fallen slightly over the
periods, even though India's population has become progressively more urbanized.
The most likely explanation is that the age of marriage has increased
faster in urban areas than in rural areas.
Hence, there are relatively more urban women, but fewer show up in the
sample because they are not yet married.

The population has become substantially better educated over time.
Women with no education constituted almost 60\% 
in the first period, but less than twenty percent in the last period.
Correspondingly, in the first period just over twenty percent had eight or more 
years of education, whereas in the last period more than 60\% did.
These changes are an underestimate of the increase in female
education overall because many of the younger women with more education
have not yet married than therefore are not in the sample.%
\footnote{
Women with eight or more years of education accounted for 65.9\% of
unmarried women in NFHS-3 and 82.8\% of unmarried women in NFHS-4.
See \citet[p 56]{International-Institute-for-Population-Sciences-IIPS2007}
and \citet[p 61]{International-Institute-for-Population-Sciences-IIPS2017}
for more information.
 }

% Age at marriage increased over time across all three education groups.
% The most substantial increase was for women without any education, where
% the average went from below 16 years of age to 18.5.
% The smallest change is for the most educated women where the average age
% at marriage went up by less than a year---from 19.6 to 20.4---across the 
% four periods.


\section{Results\label{sec:results}}


% [How to read tables]
Figures \ref{fig:bootstrap_low}, \ref{fig:bootstrap_med}, and  \ref{fig:bootstrap_high} 
show predicted average birth intervals, sex ratios, and probabilities of having
a birth by decade, spell, and sex composition for the three education levels separated 
by area of residence.
For each subsample, I first estimate the model described in Section 
\ref{sec:strategy} and then use the estimated coefficients to predict 
average spell length, sex ratio, and the probability of having a birth in that spell.

The expected average duration is calculated as follows.
For each woman in a sample, I calculate her probability of giving birth for each period
in the spell and use these probabilities as weights to calculated her expected duration 
for that spell.
I then take the average of these expected durations using her probability of having a
birth by the end of the spell as weights.

The predicted sex ratio is the weighted average over the women in the sample's 
individual predicted sex ratios, using the individual probabilities of having
given birth by the end of the spell as weights.
Each woman's predicted sex ratio is the weighted average of the predicted
percentage boys over each period in the spell calculated using equation 
(\ref{eq:probability_son}) and the probability of giving birth in each period 
as weights.%
\footnote{
Imagine a spell has two periods and that the estimated percentage boys for a
woman are 54\% and 66\% and that the likelihood of having a birth is 20\% and 40\%.
The likelihood of having a birth is the change in the survival curve; 
in other words, there is 40\% chance that she will not have given birth by the end of 
this spell.
This woman's percentage boys is then $\frac{54*0.2+66*0.4}{0.2+0.4} = 62$.
}
The predicted sex ratio captures the percent boys that will have been born to women 
in the sample when childbearing for that spell is over.

For legibility none of the graphs show standard errors.
Instead, the standard errors, together with the graphed values, are shown in Appendix 
Tables TK, TK, and TK.
The standard errors are based on bootstrapping for all three measures,
where the model is repeatedly estimated using resampling with replacement.

I also show whether durations for sex composition other than only girls are statistically 
significantly different from the duration with only girls based on bootstrapped 
differences. 
The cleanest test is comparing durations after only boys with durations after
only girls, but the number of births to women with only sons becomes small 
in the later periods.
Hence, it is possible to have substantial differences in spacing that are
not statistically significant because of low power, especially for the third 
and fourth spell.

Each predicted percent boys is tested against the natural percentage 
boys using the bootstrapped standard errors.
The natural sex ratio is approximately 105 boys to 100 girls or
51.2\% \citep{ben-porath76b,jacobsen99,Portner2015b}.
The predicted percentage boys may differ from the natural rate because of 
natural variation, any remaining recall error not corrected for, or 
sex selection. 

\subsection{No Education Women}

% \begin{figure}[htpb]
% \centering
% \caption*{Urban}
% \subfloat[Second Spell]{\includegraphics[width=0.27\textwidth]{bs_spell2_low_urban_all}} 
% \subfloat[Third Spell]{\includegraphics[width=0.27\textwidth]{bs_spell3_low_urban_all}} 
% \subfloat[Fourth Spell]{\includegraphics[width=0.27\textwidth]{bs_spell4_low_urban_all}} 
% \\
% \caption*{Rural}
% \subfloat[Second Spell]{\includegraphics[width=0.27\textwidth]{bs_spell2_low_rural_all}} 
% \subfloat[Third Spell]{\includegraphics[width=0.27\textwidth]{bs_spell3_low_rural_all}} 
% \subfloat[Fourth Spell]{\includegraphics[width=0.27\textwidth]{bs_spell4_low_rural_all}} 
% \caption{low education}
% \label{fig:bootstrap_low}
% \end{figure}

% \captionsetup[subfigure]{position=top,captionskip=-1pt,farskip=-0.5pt}
\captionsetup[subfigure]{captionskip=-1pt,farskip=-0.5pt}


\begin{figure}[htpb]
\centering
\rotatebox[origin=c]{90}{\small{Urban}}
\subfloat[Second Spell]{
    \begin{minipage}{0.29\textwidth}
        \includegraphics[width=\textwidth]{bs_spell2_low_urban_all}
    \end{minipage}
} 
\subfloat[Third Spell]{
    \begin{minipage}{0.29\textwidth}
        \includegraphics[width=\textwidth]{bs_spell3_low_urban_all} 
    \end{minipage}
}
\subfloat[Fourth Spell]{
    \begin{minipage}{0.29\textwidth}
        \includegraphics[width=\textwidth]{bs_spell4_low_urban_all}
    \end{minipage}
}
\\
\rotatebox[origin=c]{90}{\small{Rural}}
\subfloat[Second Spell]{
    \begin{minipage}{0.29\textwidth}
        \includegraphics[width=\textwidth]{bs_spell2_low_rural_all}
    \end{minipage}
} 
\subfloat[Third Spell]{
    \begin{minipage}{0.29\textwidth}
        \includegraphics[width=\textwidth]{bs_spell3_low_rural_all} 
    \end{minipage}
} 
\subfloat[Fourth Spell]{
    \begin{minipage}{0.29\textwidth}
        \includegraphics[width=\textwidth]{bs_spell4_low_rural_all} 
    \end{minipage}
} 
\caption{Estimated average spell length, sex ratio, and probability of 
a next birth for women with no education}
\label{fig:bootstrap_low}
\end{figure}

The previous literature finds that women with no education are the least likely to use 
sex selection, which suggests that they would instead follow a traditional strategy to son 
preference with higher fertility and shorter birth intervals the fewer sons they have.
This pattern is supported by Figure \ref{fig:bootstrap_low}, but with important
differences across spells.
Most of the women without education live in rural areas, and I therefore focus on rural 
women.%
\footnote{
Only between ten and twenty percent of the surveyed women without education live in urban 
areas, and this proportion has been decreasing over time.
}

The initial high fertility of women with no education shows clearly in the results.
In the 1972--1984 period almost all women had at least four children, irrespectively
of the sex composition of the children.
Over time, however, the likelihood of having a third or fourth child decreased for women 
with at least one son, and more so the more sons they had.
Hence, fertility have begun to fall even among women with no education, as long as they 
have at least one son.
Two additional fertility trends stand out.
First, essentially all women still have a second birth, whether the first child was
a boy or a girl.
Second, for women with only girls there is almost no reduction in the probability of 
having a birth across time and across spells.

Birth intervals have increased in length over time, although the increases are not
uniform over spells and sex compositions of prior children.
The smallest increases have been for the second spell, with most increases between one
and two months.
Interestingly, son preference appear to have little effect on spacing for the second
spell, with very small, albeit statistically significant, differences in birth intervals 
depending on the sex of the first child.
This small difference is in line with the high likelihood of having a second child.
The second spell also shows convergence in birth intervals, especially in the urban
areas where the predicted average duration is now the same whether the first child was
a boy or a girl.

Where son preference in birth intervals shows most clearly is for the fourth spell.
Despite the increase in spacing even for those with three girls, the increases in 
spacing have been substantially larger for families with at least one son.
As for fertility, the differences are larger the more sons a family has.

Despite the general finding that use of sex selection is confined to higher education
women, the predicted sex ratios for those with only girls is statistically significantly
higher than the natural ratio for the third and fourth spell.
Hence, it is possible that at least part of the increased  birth intervals for those
with only girls comes from the use of sex selection and that the differences in
birth spacing across sex compositions would be even larger in the absense of sex selection.


\subsection{Low to Medium Education Women}


% \begin{figure}[htpb]
% \centering
% \caption*{Urban}
% \subfloat[Second Spell]{\includegraphics[width=0.27\textwidth]{bs_spell2_med_urban_all}} 
% \subfloat[Third Spell]{\includegraphics[width=0.27\textwidth]{bs_spell3_med_urban_all}} 
% \subfloat[Fourth Spell]{\includegraphics[width=0.27\textwidth]{bs_spell4_med_urban_all}} 
% \\
% \caption*{Rural}
% \subfloat[Second Spell]{\includegraphics[width=0.27\textwidth]{bs_spell2_med_rural_all}} 
% \subfloat[Third Spell]{\includegraphics[width=0.27\textwidth]{bs_spell3_med_rural_all}} 
% \subfloat[Fourth Spell]{\includegraphics[width=0.27\textwidth]{bs_spell4_med_rural_all}} 
% \caption{med education}
% \label{fig:bootstrap_med}
% \end{figure}


\begin{figure}[htpb]
\centering
\rotatebox[origin=c]{90}{\small{Urban}}
\subfloat[Second Spell]{
    \begin{minipage}{0.29\textwidth}
        \includegraphics[width=\textwidth]{bs_spell2_med_urban_all}
    \end{minipage}
} 
\subfloat[Third Spell]{
    \begin{minipage}{0.29\textwidth}
        \includegraphics[width=\textwidth]{bs_spell3_med_urban_all} 
    \end{minipage}
}
\subfloat[Fourth Spell]{
    \begin{minipage}{0.29\textwidth}
        \includegraphics[width=\textwidth]{bs_spell4_med_urban_all}
    \end{minipage}
}
\\
\rotatebox[origin=c]{90}{\small{Rural}}
\subfloat[Second Spell]{
    \begin{minipage}{0.29\textwidth}
        \includegraphics[width=\textwidth]{bs_spell2_med_rural_all}
    \end{minipage}
} 
\subfloat[Third Spell]{
    \begin{minipage}{0.29\textwidth}
        \includegraphics[width=\textwidth]{bs_spell3_med_rural_all} 
    \end{minipage}
} 
\subfloat[Fourth Spell]{
    \begin{minipage}{0.29\textwidth}
        \includegraphics[width=\textwidth]{bs_spell4_med_rural_all} 
    \end{minipage}
} 
\caption{Estimated average spell length, sex ratio, and probability of 
a next birth for women with one to seven years of education}
\label{fig:bootstrap_med}
\end{figure}

Figure \ref{fig:bootstrap_med} shows that women with one to seven years of education 
follow a fertility pattern broadly similar to those with no education.
As expected, fertility is lower than for women with no education, but still show
substantial son preference in the parity progression behavior.
Except for the second spell, the likelihood of having a birth by the end of spell is
substantially lower if there has been at least one male birth than if there are only girls.
Despite this, the probability of a birth in the third and fourth spells has also been 
falling for those with only girls.
Having a second birth is still close to universal, although in urban areas it is down to
just over 85 percent for those with a son as their first child.

The spacing between births has increased over the four decades, and the increases are 
larger than for women without any education across all spells.
Contrary to women without education, however, there has not been an increased divergence 
in birth intervals lengths across sex compositions for the third and fourth spells.
On the contrary, for the third spell the largest increases in birth intervals are for women with
two daughters and no sons, with an increase of about seven months across the four decades.
The results is that for urban women the standard spacing pattern reversed, so the longest 
predicted spacing is now for those with only girls, while for rural women the predicted 
average birth intervals are now almost identical across sex compositions of prior children.
For the fourth spell, the increases in spacing for women with only girls have kept pace
with the other sex compositions in rural areas.%
\footnote{
Because of low number of observations the pattern for the fourth spell is more noisy for 
urban women.
}

The substantial increases in spacing for women with only girls is not evidence of lower
son preference as the changes in the predicted sex ratios show.
For both the third and fourth spell the predicted sex ratio in both urban and rural
areas is statistically significantly above the natural sex ratio at between 55.5 and 57.7 
percent boys.
The elevated sex ratios strongly suggests that sex selection is driving the increases in
birth intervals for women with only girls, and that without sex selection we would 
instead have observed a diverging pattern in birth intervals across sex composition as
for women without education.


\subsection{High Education Women}


% \begin{figure}[htpb]
% \centering
% \caption*{Urban}
% \subfloat[Second Spell]{\includegraphics[width=0.27\textwidth]{bs_spell2_high_urban_all}} 
% \subfloat[Third Spell]{\includegraphics[width=0.27\textwidth]{bs_spell3_high_urban_all}} 
% \subfloat[Fourth Spell]{\includegraphics[width=0.27\textwidth]{bs_spell4_high_urban_all}} 
% \\
% \caption*{Rural}
% \subfloat[Second Spell]{\includegraphics[width=0.27\textwidth]{bs_spell2_high_rural_all}} 
% \subfloat[Third Spell]{\includegraphics[width=0.27\textwidth]{bs_spell3_high_rural_all}} 
% \subfloat[Fourth Spell]{\includegraphics[width=0.27\textwidth]{bs_spell4_high_rural_all}} 
% \caption{high education}
% \label{fig:bootstrap_high}
% \end{figure}


\begin{figure}[htpb]
\centering
\rotatebox[origin=c]{90}{\small{Urban}}
\subfloat[Second Spell]{
    \begin{minipage}{0.29\textwidth}
        \includegraphics[width=\textwidth]{bs_spell2_high_urban_all}
    \end{minipage}
} 
\subfloat[Third Spell]{
    \begin{minipage}{0.29\textwidth}
        \includegraphics[width=\textwidth]{bs_spell3_high_urban_all} 
    \end{minipage}
}
\subfloat[Fourth Spell]{
    \begin{minipage}{0.29\textwidth}
        \includegraphics[width=\textwidth]{bs_spell4_high_urban_all}
    \end{minipage}
}
\\
\rotatebox[origin=c]{90}{\small{Rural}}
\subfloat[Second Spell]{
    \begin{minipage}{0.29\textwidth}
        \includegraphics[width=\textwidth]{bs_spell2_high_rural_all}
    \end{minipage}
} 
\subfloat[Third Spell]{
    \begin{minipage}{0.29\textwidth}
        \includegraphics[width=\textwidth]{bs_spell3_high_rural_all} 
    \end{minipage}
} 
\subfloat[Fourth Spell]{
    \begin{minipage}{0.29\textwidth}
        \includegraphics[width=\textwidth]{bs_spell4_high_rural_all} 
    \end{minipage}
} 
\caption{Estimated average spell length, sex ratio, and probability of 
a next birth for women with eight or more years of education}
\label{fig:bootstrap_high}
\end{figure}


Women with eight or more years of education are those expected to have the lowest 
fertility and the highest use of sex selection based on the prior literature.
Figure \ref{fig:bootstrap_high} shows that not only do this group, indeed, have lower 
fertility than the two prior groups, it is also falling rapidly over time.
Of women with one birth, the probability of having a second birth when the first-born
is a son is only around 70 percent for urban women and around 80 percent for rural women,
and for those with a first-born daughter it is around 80 percent for urban women and 90 
percent for rural women.

The rapid decline in fertility for the most educated women is even more clear for the 
third and fourth spells when the woman has given birth to at least one son.
At the beginning of the sample, urban women with at least one boy had a close to 75
percent likelihood of having a birth, but that has fallen to only around 25 percent.
However, son preference still shows clearly in the parity progression probability for 
women without a son, with the decline for the third and fourth spell substantially
slower for both urban and rural women without a son than for those with at least one son.
The falling fertility means that relatively few women makes it to the fourth spell,
which, therefore, has more noise than the other spells.


Average birth spacing for the second spell is almost identically across sex compositions 
for both urban and rural women, and increased by around five months for rural women and
by almost ten months for urban women over the four decades.
These similarities hides, however, what would have been a standard son preference spacing 
pattern with significantly shorter spacing after a first-born girl than after a first-born 
boy if there had not been a substantial amount of sex selection.
For both of the last two periods the predicted sex ratio for women with a daughter as 
their first-born child is above 55 percent for both urban and rural areas.

The tremendous impact that sex selection can have on birth spacing is illustrated 
particularly vividly by the changes over time for the third spell.
In the 1972-1984 period the predicted average birth intervals for the three possible sex 
compositions were less than two months apart.
For the later periods, however, the highest predicted average spacing was for women
with only daughters with most of the differences to the other sex compositions 
statistically significant.
In urban areas the predicted average birth interval for a woman with only daughters 
increased by almost a full year from 26.8 to 38.3 months.
What is more, this increase is likely an underestimate.
The sex ratios for the 1972-84 period are substantially higher than the natural sex ratio 
despite sex selection not being widely available until around 1985.
If this early elevated sex ratio is because of recall error it means that the actual birth 
intervals would have been substantially shorter than the ones predicted here and the
increase over time larger.

The reversal pattern also shows for the fourth spell for urban women, although not for
rural women.
An important caveat is that because of the falling fertility relatively few women make
it to the fourth spell, and even if they do we do not observe a birth by the time of the
survey.
Of the 11,886 women across urban and rural areas observed for the fourth spell for the 
2005-2016 period more than 70 percent did not have a birth at the time of the survey.
Even though the censoring is less prevalent for the earlier periods this is countered by
those samples being substantially smaller.
Despite this, the pattern for the fourth spell is similar to that of the third spell.

The predicted sex ratios at the end of the spells show that this reversal in spacing 
patterns is not the result of a declining son preference but instead corresponds to a 
substantially more male-biased sex ratio as use of sex selection spread.
For urban women, the predicted sex ratio with only girls is consistently above 60\% boys 
for the third and fourth spells.
For rural women, the predicted sex ratios are lower than for urban women but still 
substantial and statistically significant at close to 60\%.

When trying to understand the strength of son preference, it is interesting that the sex 
ratio is also statistically significantly different from the natural rate in the case 
where women already have one son for the third and fourth spells.
Again, the sex ratio in the presence of one son for the third and fourth spells are higher 
in urban areas than rural areas, although the difference is less than for women with only 
girls.
Hence, it is possible that women are still willing to use sex selection even after giving 
birth to one, although this behavior may also be in response to either experienced
or expected mortality of the first son born.
This result is different from prior studies using NFHS data.
The NFHS-4 sample is substantially larger than the three prior surveys.
Hence, it is possible that the effect has been there all along, but we did not have the 
power to detect it.


\subsection{Distribution of Birth Spacing Within Spells}

The results above show consistent increases in average spacing across the board.
Average spacing is a convenient way to understand the overall changes in behavior but may 
hide important differences in the distribution of spacing.
Specifically, whether these increases in average spacing come from fewer births after 
very short intervals, because of even longer spacing from births with already long 
spacing, or a general increase in spacing have important implications for what effects the
increase in average spacing are likely to have and how to interpret the motivation for
the changes.
On one hand, if there still are many births that occur after very short intervals, the 
increase in average spacing is unlikely to have a substantial impact on health outcomes
since the literature on birth spacing and child health suggests that an interpregnancy 
interval of less than 18 months is associated with a higher risk of adverse health 
outcomes and mortality \citep{Conde-Agudelo2012}.
On the other hand, we may see a reduction in early observed births and more long spells if 
families with fewer sons are both more likely to have a very early pregnancy and to 
continue through pregnancies and sex-selective abortions until they conceive a son.%
\footnote{
It is unclear what the optimal response in timing of pregnancies to the availability of 
sex selection should be.
The expected birth intervals become longer with the use of sex selection, which may be 
costly for parents leading them to try to conceive even earlier than previously.
Conversely, if parents know that short birth intervals are detrimental to the next child's
health, and with sex selection the next child is likely be a boy, it might make sense to 
wait longer before conceiving.
}

This section, therefore, provides more detail on how spacing is distributed within a spell 
and across sex compositions using a graphical approach.
I show survival curves conditional on predicted parity progression rather than standard 
survival curves. 
The advantage of this approach is that it is possible to directly compare the distribution 
of spacing to next birth across groups, independently of differences in how likely the 
next birth is.
Because the conditional survival curves are independent of the likelihood of parity 
progression, they all begin at 100\% and end at 0\%.

Instead of averaging across the entire sample, I calculate the conditional survival curves 
for an average woman using the method detailed in Section \ref{sec:strategy}.
For each combination of education and spell, I use values based on the average age at the 
start of the spell.
Furthermore, I use the majority categories for the categorical explanatory  variables, 
which means no ownership of land and not in a scheduled caste or tribe.
The characteristics used do not change across the four decades to ensure that composition 
effects do not drive any changes.
To complement this approach, Appendix Tables \ref{tab:p25_p50_p75_low}, 
\ref{tab:p25_p50_p75_med}, and \ref{tab:p25_p50_p75_high} show 25th, 50th, and 75th 
percentiles durations with bootstrapped standard errors based on averages across all women 
in the each subsample as for the results above.

In the interest of brevity, I discuss only two subsamples.
Figures \ref{fig:pps_low} and \ref{fig:pps_high} show spacing across sex compositions for 
the second, third, and fourth spells for the first period, 1972--1984, and the last 
period, 2005--2016, for rural women with no education and for urban women with eight or 
more years of education.
The Appendix shows conditional survival curves for all groups and decades.

% First spell for low and high education women

% \begin{figure}[htpb]
% \centering
% \setcounter{subfigure}{-1}
% \subfloat[Rural Women with No Education]{
%     \begin{minipage}{0.48\textwidth}
%         \captionsetup[subfigure]{labelformat=empty,position=top,captionskip=-1pt,farskip=-0.5pt}
%         \subfloat[Probability of no birth yet]{\includegraphics[width=\textwidth]{spell1_low_urban_pps}} 
%         \captionsetup[subfigure]{labelformat=parens}
%     \end{minipage}
% } 
% \setcounter{subfigure}{0}
% \subfloat[Urban Women with 8 or More Years of Education]{
%     \begin{minipage}{0.48\textwidth}
%         \captionsetup[subfigure]{labelformat=empty,position=top,captionskip=-1pt,farskip=-0.5pt}
%         \subfloat[Probability of no birth yet]{\includegraphics[width=\textwidth]{spell1_high_urban_pps}} 
%         \captionsetup[subfigure]{labelformat=parens}
%     \end{minipage}
% } 
% \caption{
% Survival curves conditional on progression to first birth; start point is month of marriage
% }
% \label{fig:pps_spell1}
% \end{figure}

% The biased sex ratio for the first spell for the most educated 
% women suggests that there might be sex selection on the first birth.
% I, therefore, begin by comparing in Figure \ref{fig:pps_spell1} how the distribution 
% of spacing has changed over time for rural women with no education and urban women 
% with eight or more years of education.
% What is most striking is how similar the distribution of spacing is across the
% two groups.
% 
% Both groups show little change in median spacing, but that masks a substantial
% compression of when most of the births occur.
% In the 1972--1984 period, the middle 80\% of births for women with no education 
% are predicted to occur between approximately 6 and 64 months, whereas in the 
% 2005--2016 period it is between 12 and 54 months.
% Hence, the compression is equivalent to more than a full year.
% Women with eight or more years of education show a slightly smaller compression:
% in the 1972--1984 period, the middle 80\% of births are predicted to occur between
% approximately 6 and 48 months, whereas in the 2005--2016 period it is between
% 12 and 46 months.
% Hence, the compression is eight months.
% 
% Women became less likely to conceive before marriage over time.
% In the first two periods there was a relatively smooth decline in the number of
% women without a birth starting at the time of marriage, but in the last two
% periods, there are few women who exit early after marriage and
% instead there is a substantial dip between 9 and 12 months after
% marriage.
% 
% It is likely that the compressed spacing, beginning at nine months
% after marriage in the later periods, is associated with better health
% and higher age at marriage.
% For example, women without education have seen an increase in the average 
% age at marriage from below 16 to 18.5 from the first period to the last 
% while women with the most education increased from 19.5 to 20.4.


\begin{figure}[htpb]
\centering
\rotatebox[origin=c]{90}{\footnotesize{Second Spell}}
\setcounter{subfigure}{-1}
\subfloat[1972--1984]{
    \begin{minipage}{0.46\textwidth}
        \captionsetup[subfigure]{labelformat=empty,position=top,captionskip=-1pt,farskip=-0.5pt}
        \subfloat[Probability of no birth yet]{\includegraphics[width=\textwidth]{spell2_g1_low_rural_pps}} 
        \captionsetup[subfigure]{labelformat=parens}
    \end{minipage}
} 
\setcounter{subfigure}{0}
\subfloat[2005--2016]{
    \begin{minipage}{0.46\textwidth}
        \captionsetup[subfigure]{labelformat=empty,position=top,captionskip=-1pt,farskip=-0.5pt}
        \subfloat[Probability of no birth yet]{\includegraphics[width=\textwidth]{spell2_g4_low_rural_pps}} 
        \captionsetup[subfigure]{labelformat=parens}
    \end{minipage}
} 
\\
\rotatebox[origin=c]{90}{\footnotesize{Third Spell}}
\setcounter{subfigure}{1}
\subfloat[1972--1984]{
    \begin{minipage}{0.46\textwidth}
        \captionsetup[subfigure]{labelformat=empty,position=top,captionskip=-1pt,farskip=-0.5pt}
        \subfloat[Probability of no birth yet]{\includegraphics[width=\textwidth]{spell3_g1_low_rural_pps}} 
        \captionsetup[subfigure]{labelformat=parens}
    \end{minipage}
} 
\setcounter{subfigure}{2}
\subfloat[2005--2016]{
    \begin{minipage}{0.46\textwidth}
        \captionsetup[subfigure]{labelformat=empty,position=top,captionskip=-1pt,farskip=-0.5pt}
        \subfloat[Probability of no birth yet]{\includegraphics[width=\textwidth]{spell3_g4_low_rural_pps}} 
        \captionsetup[subfigure]{labelformat=parens}
    \end{minipage}
} 
\\
\rotatebox[origin=c]{90}{\footnotesize{Fourth Spell}}
\setcounter{subfigure}{3}
\subfloat[1972--1984]{
    \begin{minipage}{0.46\textwidth}
        \captionsetup[subfigure]{labelformat=empty,position=top,captionskip=-1pt,farskip=-0.5pt}
        \subfloat[Probability of no birth yet]{\includegraphics[width=\textwidth]{spell4_g1_low_rural_pps}} 
        \captionsetup[subfigure]{labelformat=parens}
    \end{minipage}
} 
\setcounter{subfigure}{4}
\subfloat[2005--2016]{
    \begin{minipage}{0.46\textwidth}
        \captionsetup[subfigure]{labelformat=empty,position=top,captionskip=-1pt,farskip=-0.5pt}
        \subfloat[Probability of no birth yet]{\includegraphics[width=\textwidth]{spell4_g4_low_rural_pps}} 
        \captionsetup[subfigure]{labelformat=parens}
    \end{minipage}
} 
\caption{
Survival curves conditional on progression to next birth for rural women without
education; start point for each spell is nine months after prior birth
}
\label{fig:pps_low}
\end{figure}

Figure \ref{fig:pps_low} shows the distribution of birth within a spell for women with no 
education.
For the second spell, what is most interesting is that there is almost no difference 
between those with a first-born boy and those with a first-born girl.
For the 1972--1984 period the two conditional survival curves only begin to diverge 
at around 18 months (corresponding to 27 months after the first birth took place), and
at that point close to 50 percent of the women have already had their second child.
With a nine-month pregnancy, this means that almost 50 percent of the children born are
born within the interpregnancy interval associated with a higher risk of adverse health
outcomes and mortality, and, somewhat surprisingly, that there do not appear to be any 
particular disadvantage for the child born after a girl than after a boy.%
\footnote{
Note that this does not imply that the first-born girl is not in a more adverse position
than a first-born boy when the second child arrives, just that this is not directly
related to a short birth interval.
}
The difference in spacing between a first-born girl and a first-born boy first-born 
remains almost the same across the four decades.
Furthermore, the proportion with very short spacing remains almost the same.

For the third spell the son preference driven short spacing is evident throughout the
distribution and becomes more pronounced over the four decades.
Families with two girls are especially likely to have a birth early with 25 percent
having a child within 12 months of the beginning of the spell (21 months after the prior
second birth).
That number is only slightly smaller four decades later. 
The fourth spell shows changes in the distribution that appear to be more pronounced
than for the third spell, although these results are based on fewer observed births.
For families with three girls, 75 percent have had their fourth child within 36 months 
after the beginning of the spell, but this occurs more than a year later for families
with two or more boys.


\begin{figure}[htpb]
\centering
\rotatebox[origin=c]{90}{\footnotesize{Second Spell}}
\setcounter{subfigure}{-1}
\subfloat[1972--1984]{
    \begin{minipage}{0.46\textwidth}
        \captionsetup[subfigure]{labelformat=empty,position=top,captionskip=-1pt,farskip=-0.5pt}
        \subfloat[Probability of no birth yet]{\includegraphics[width=\textwidth]{spell2_g1_high_urban_pps}} 
        \captionsetup[subfigure]{labelformat=parens}
    \end{minipage}
} 
\setcounter{subfigure}{0}
\subfloat[2005--2016]{
    \begin{minipage}{0.46\textwidth}
        \captionsetup[subfigure]{labelformat=empty,position=top,captionskip=-1pt,farskip=-0.5pt}
        \subfloat[Probability of no birth yet]{\includegraphics[width=\textwidth]{spell2_g4_high_urban_pps}} 
        \captionsetup[subfigure]{labelformat=parens}
    \end{minipage}
} 
\\
\rotatebox[origin=c]{90}{\footnotesize{Third Spell}}
\setcounter{subfigure}{1}
\subfloat[1972--1984]{
    \begin{minipage}{0.46\textwidth}
        \captionsetup[subfigure]{labelformat=empty,position=top,captionskip=-1pt,farskip=-0.5pt}
        \subfloat[Probability of no birth yet]{\includegraphics[width=\textwidth]{spell3_g1_high_urban_pps}} 
        \captionsetup[subfigure]{labelformat=parens}
    \end{minipage}
} 
\setcounter{subfigure}{2}
\subfloat[2005--2016]{
    \begin{minipage}{0.46\textwidth}
        \captionsetup[subfigure]{labelformat=empty,position=top,captionskip=-1pt,farskip=-0.5pt}
        \subfloat[Probability of no birth yet]{\includegraphics[width=\textwidth]{spell3_g4_high_urban_pps}} 
        \captionsetup[subfigure]{labelformat=parens}
    \end{minipage}
} 
\\
\rotatebox[origin=c]{90}{\footnotesize{Fourth Spell}}
\setcounter{subfigure}{3}
\subfloat[1972--1984]{
    \begin{minipage}{0.46\textwidth}
        \captionsetup[subfigure]{labelformat=empty,position=top,captionskip=-1pt,farskip=-0.5pt}
        \subfloat[Probability of no birth yet]{\includegraphics[width=\textwidth]{spell4_g1_high_urban_pps}} 
        \captionsetup[subfigure]{labelformat=parens}
    \end{minipage}
} 
\setcounter{subfigure}{4}
\subfloat[2005--2016]{
    \begin{minipage}{0.46\textwidth}
        \captionsetup[subfigure]{labelformat=empty,position=top,captionskip=-1pt,farskip=-0.5pt}
        \subfloat[Probability of no birth yet]{\includegraphics[width=\textwidth]{spell4_g4_high_urban_pps}} 
        \captionsetup[subfigure]{labelformat=parens}
    \end{minipage}
} 
\caption{
Survival curves conditional on progression to next birth for urban women with eight or
more years of education; start point is nine months after prior birth
}
\label{fig:pps_high}
\end{figure}


The distribution of spacing for the second spell is almost indistinguishable 
between urban women with eight or more years of education and rural women 
with no education for the 1972--1984 period, as shown in Figure \ref{fig:pps_high}.
The small differences point to these two very different groups of women behaving similarly
in response to son preference in a situation where there was little access to
prenatal sex determination or little incentive to use it early even for those 
who might have had access.
However, whereas the no education group shows almost no change across the four decades for 
the second spell, the high education women show an upward shift and a relatively even
distribution of births within the spell for both women with a first-born son and women
with a first-born girl.
The result is that the spacing patterns are close to identical after first-born 
girls and after first-born boys until after approximately 80 percent have had their 
second child, although as mentioned above, this is driven in large part by the significant
use of sex selection among women with a first-born girl.

For the third spell, the 1972--1984 period show a close to standard pattern throughout, 
although, as mentioned above, the sex ratios for women without girls was substantially 
higher than the natural sex ratio either because of recall error or early use of 
sex selection.
The correlation between the increased use of sex selection and the reversal of the 
standard spacing pattern show up clearly in the 2005--2016 period for the third spell with 
spacing to the third child consistently longer with two girls than with either two boys 
or one boy and one girl throughout the entire spell.
This pattern is most consistent with early and continued use of sex selection, rather 
than a general reduction in the likelihood of very short birth intervals.
It also means, that previous girls can expect a longer period without a younger sibling,
which may reduce the sibling competition for resources.

There is an even more pronounced rightward and upward shift for spacing after only
girls in the fourth spell for the first years of the spell.%
\footnote{
The caveat is that there are few births to base the results on,
especially for women with two or more sons.
} 
Furthermore, the differences are substantial across most of the distribution.
Twelve months after the beginning of the spell there is a more than ten 
percentage points difference in the conditional survival curves for only girls
and only boys.
Both of the spells therefore show changes in the distribution of births that is in line
with extensive use of sex selection among families with no or one son.

% We should expect 51.2 percent of the women to conceive a boy when they have their first 
% pregnancy in the spell, which means that they do not need to use sex selection to have
% a son.
% Hence, even with sex selection, we might expect to observe a very short interval for the 
% initial births if families with strong son preference still try to conceive sooner than 
% those with lower son preference.
% If, however, parents know that children born after short interpregnancy periods suffer 
% from worse health outcomes it makes sense to wait longer before conceiving again in
% this situation, precisely because they know that with sex selection the next child they 
% carry to term will be a boy.
% 
% Hence, there are two potential reasons why son preference combined with access to
% sex selection can lead to longer birth intervals.
% First, the ``abortion'' effect, where each abortion extend birth spacing by between six
% and twelve months.
% Second, a ``delay'' effect, where parents conceive later because they know that short
% spacing is detrimental to the next child's health, and that the next child is more likely
% to be a boy because of access to sex selection.



\subsection{Regional Differences\label{sec:regional}}


To understand how regional differences in son preference, fertility, and use of sex 
selection affect birth spacing, I divide India into four broad regions based on degree of 
son preference and initial fertility levels.
Focus is on the most educated women since they are the main users of sex selection.
The four regions are based on \citet{retherford03b}, although expanded to include all 
states in the surveys.
Table \ref{tab:regions} shows the four regions and the states that they contain.%
\footnote{
The geographical names do not perfectly reflect the geography of India, and mainly
serves as mnemonic short-hands.
}
The state names are based on the states in existence when NFHS-1 was collected.
States that are formed later are allocated as closely as possible to the original state.
The ``West'' contains states with a high degree of son preference and use of sex selection.
The ``North'' consists of states that also have a strong son preference, but initially had
higher fertility and, therefore, lower likelihood of using sex selection.
The ``East'' states generally have only a moderate level of son preference.
Finally, the ``South'' contains states with traditionally low son preference and lower
fertility than the rest of India.

\input{../tables/desc_region.tex}


% \begin{figure}[htpb]
% \centering
% \caption*{Urban}
% \subfloat[all]{\includegraphics[width=0.27\textwidth]{bs_spell2_high_urban_r1}} 
% \subfloat[all]{\includegraphics[width=0.27\textwidth]{bs_spell3_high_urban_r1}} 
% \subfloat[all]{\includegraphics[width=0.27\textwidth]{bs_spell4_high_urban_r1}} 
% \\
% \caption*{Rural}
% \subfloat[all]{\includegraphics[width=0.27\textwidth]{bs_spell2_high_rural_r1}} 
% \subfloat[all]{\includegraphics[width=0.27\textwidth]{bs_spell3_high_rural_r1}} 
% \subfloat[all]{\includegraphics[width=0.27\textwidth]{bs_spell4_high_rural_r1}} 
% \caption{high education - r1}
% \end{figure}


\begin{figure}[htpb]
\centering
\rotatebox[origin=c]{90}{\small{Urban}}
\subfloat[Second Spell]{
    \begin{minipage}{0.29\textwidth}
        \includegraphics[width=\textwidth]{bs_spell2_high_urban_r1}
    \end{minipage}
} 
\subfloat[Third Spell]{
    \begin{minipage}{0.29\textwidth}
        \includegraphics[width=\textwidth]{bs_spell3_high_urban_r1} 
    \end{minipage}
}
\subfloat[Fourth Spell]{
    \begin{minipage}{0.29\textwidth}
        \includegraphics[width=\textwidth]{bs_spell4_high_urban_r1}
    \end{minipage}
}
\\
\rotatebox[origin=c]{90}{\small{Rural}}
\subfloat[Second Spell]{
    \begin{minipage}{0.29\textwidth}
        \includegraphics[width=\textwidth]{bs_spell2_high_rural_r1}
    \end{minipage}
} 
\subfloat[Third Spell]{
    \begin{minipage}{0.29\textwidth}
        \includegraphics[width=\textwidth]{bs_spell3_high_rural_r1} 
    \end{minipage}
} 
\subfloat[Fourth Spell]{
    \begin{minipage}{0.29\textwidth}
        \includegraphics[width=\textwidth]{bs_spell4_high_rural_r1} 
    \end{minipage}
} 
\caption{Estimated average spell length, sex ratio, and probability of 
a next birth for women with eight or more years of education in the ``West''}
\label{fig:bootstrap_high_r1}
\end{figure}


Figure \ref{fig:bootstrap_high_r1} shows the predicted average birth intervals, sex 
ratios, and likelihoods of having a child by the end of the spell over time for the three 
spells.
What most stands out is how close to the national estimates the ``West'' estimates are.
The predicted second spell average birth spacing is similar across families with a girl
or a boy, and the average spacing substantially higher in the absence of boys than when
there is at least one boy for the third spell.
The main difference is that the very high predicted sex ratios reflect the expected higher 
use of sex selection.
For the second spell, the predicted sex ratios are around 60 percent for both urban and
rural women, and for the third spell it it around 70 percent for urban women and 
consistently above 60 percent for rural women.
This higher use of sex selection, in turn, translates into longer average spacing.
Furthermore, fertility has fallen faster than the national trend, especially for women
with one or more sons.


% \begin{figure}[htpb]
% \centering
% \caption*{Urban}
% \subfloat[all]{\includegraphics[width=0.27\textwidth]{bs_spell2_high_urban_r2}} 
% \subfloat[all]{\includegraphics[width=0.27\textwidth]{bs_spell3_high_urban_r2}} 
% \subfloat[all]{\includegraphics[width=0.27\textwidth]{bs_spell4_high_urban_r2}} 
% \\
% \caption*{Rural}
% \subfloat[all]{\includegraphics[width=0.27\textwidth]{bs_spell2_high_rural_r2}} 
% \subfloat[all]{\includegraphics[width=0.27\textwidth]{bs_spell3_high_rural_r2}} 
% \subfloat[all]{\includegraphics[width=0.27\textwidth]{bs_spell4_high_rural_r2}} 
% \caption{high education - r2}
% \end{figure}

\begin{figure}[htpb]
\centering
\rotatebox[origin=c]{90}{\small{Urban}}
\subfloat[Second Spell]{
    \begin{minipage}{0.29\textwidth}
        \includegraphics[width=\textwidth]{bs_spell2_high_urban_r2}
    \end{minipage}
} 
\subfloat[Third Spell]{
    \begin{minipage}{0.29\textwidth}
        \includegraphics[width=\textwidth]{bs_spell3_high_urban_r2} 
    \end{minipage}
}
\subfloat[Fourth Spell]{
    \begin{minipage}{0.29\textwidth}
        \includegraphics[width=\textwidth]{bs_spell4_high_urban_r2}
    \end{minipage}
}
\\
\rotatebox[origin=c]{90}{\small{Rural}}
\subfloat[Second Spell]{
    \begin{minipage}{0.29\textwidth}
        \includegraphics[width=\textwidth]{bs_spell2_high_rural_r2}
    \end{minipage}
} 
\subfloat[Third Spell]{
    \begin{minipage}{0.29\textwidth}
        \includegraphics[width=\textwidth]{bs_spell3_high_rural_r2} 
    \end{minipage}
} 
\subfloat[Fourth Spell]{
    \begin{minipage}{0.29\textwidth}
        \includegraphics[width=\textwidth]{bs_spell4_high_rural_r2} 
    \end{minipage}
} 
\caption{Estimated average spell length, sex ratio, and probability of 
a next birth for women with eight or more years of education in the ``North''}
\label{fig:bootstrap_high_r2}
\end{figure}

Figure \ref{fig:bootstrap_high_r2} shows the same outcomes for the ``North'' region. 
Despite the idea that the ``North'' is characterized by high fertility and lower 
likelihood of sex selection the differences to the ``West'' region is surprisingly small,
at least in the urban areas.
Fertility is, indeed, higher than in the ``West'' region, but the declines in the 
probability of parity progression has been relatively rapid and mirrors the national one.
The use of sex selection is also close to the ``West'', and there is the same reversal of
spacing pattern as there.
The rural areas of the ``North'' are a slightly different story.
Fertility is substantially higher, and the use of sex selection much lower than in the
rural areas of the ``West''. 
That said, there still is clear evidence that the sex ratio is substantially above normal
at close to 60 percent for the latest decade.



% \begin{figure}[htpb]
% \centering
% \caption*{Urban}
% \subfloat[all]{\includegraphics[width=0.27\textwidth]{bs_spell2_high_urban_r3}} 
% \subfloat[all]{\includegraphics[width=0.27\textwidth]{bs_spell3_high_urban_r3}} 
% \subfloat[all]{\includegraphics[width=0.27\textwidth]{bs_spell4_high_urban_r3}} 
% \\
% \caption*{Rural}
% \subfloat[all]{\includegraphics[width=0.27\textwidth]{bs_spell2_high_rural_r3}} 
% \subfloat[all]{\includegraphics[width=0.27\textwidth]{bs_spell3_high_rural_r3}} 
% \subfloat[all]{\includegraphics[width=0.27\textwidth]{bs_spell4_high_rural_r3}} 
% \caption{high education - r3}
% \end{figure}

\begin{figure}[htpb]
\centering
\rotatebox[origin=c]{90}{\small{Urban}}
\subfloat[Second Spell]{
    \begin{minipage}{0.29\textwidth}
        \includegraphics[width=\textwidth]{bs_spell2_high_urban_r3}
    \end{minipage}
} 
\subfloat[Third Spell]{
    \begin{minipage}{0.29\textwidth}
        \includegraphics[width=\textwidth]{bs_spell3_high_urban_r3} 
    \end{minipage}
}
\\
\rotatebox[origin=c]{90}{\small{Rural}}
\subfloat[Second Spell]{
    \begin{minipage}{0.29\textwidth}
        \includegraphics[width=\textwidth]{bs_spell2_high_rural_r3}
    \end{minipage}
} 
\subfloat[Third Spell]{
    \begin{minipage}{0.29\textwidth}
        \includegraphics[width=\textwidth]{bs_spell3_high_rural_r3} 
    \end{minipage}
} 
\caption{Estimated average spell length, sex ratio, and probability of 
a next birth for women with eight or more years of education in the ``East''}
\label{fig:bootstrap_high_r3}
\end{figure}

The results for the ``East'' region of India are shown in Figure 
\ref{fig:bootstrap_high_r3}, although the fourth spell results are not included because 
the sample size is smaller for this region and the fourth spell results very noisy.
Despite the ``South'' region generally considered to be the low fertility region, the
reduction in fertility has been even faster in the ``East''.
For the last decade only half of urban women with a first-born son are predicted to have
a second child by the end of the spell period covered, which means almost nine years after
the first birth.
The probability is only slightly higher for those in urban areas with a first-born 
daughter and still less than 70 percent.
Not surprisingly, the probabilities are higher in rural areas, but still below the 
national pattern.
Similarly, conditionally on having a second child, the probabilities of having a third
child are low, even in the absence of sons.
In urban areas a woman with two girls have only around a 40 percent likelihood of 
having a third child.

What really stands out, however, is the rapid increase in the lengths of birth intervals,
especially for the second spell.
In both urban and rural areas, the predicted average birth intervals have gone up by
more than a year across the four decade, and that is independent of whether the first-born
was a boy or a girl.
The increases in birth intervals are slightly smaller for third spell, but still 
substantial.
Although spacing pattern is consistently with some son preference with the average
spacing slightly shorter in the absence of boys than when at least one boy is present,
there is little evidence of sex-selective abortions.
The sex ratios after a first-born daughter is slightly elevated, but much lower than 
the prior two regions.
There is more evidence for the third spell, where the sex ratio is close to 60 percent
in the last period for urban women and the last two for rural women, but recall that 
there is a much smaller likelihood of even making it to the third spell for this region
than the other or the national average.


% \begin{figure}[htpb]
% \centering
% \caption*{Urban}
% \subfloat[all]{\includegraphics[width=0.27\textwidth]{bs_spell2_high_urban_r4}} 
% \subfloat[all]{\includegraphics[width=0.27\textwidth]{bs_spell3_high_urban_r4}} 
% \subfloat[all]{\includegraphics[width=0.27\textwidth]{bs_spell4_high_urban_r4}} 
% \\
% \caption*{Rural}
% \subfloat[all]{\includegraphics[width=0.27\textwidth]{bs_spell2_high_rural_r4}} 
% \subfloat[all]{\includegraphics[width=0.27\textwidth]{bs_spell3_high_rural_r4}} 
% \subfloat[all]{\includegraphics[width=0.27\textwidth]{bs_spell4_high_rural_r4}} 
% \caption{high education - r4}
% \end{figure}


\begin{figure}[htpb]
\centering
\rotatebox[origin=c]{90}{\small{Urban}}
\subfloat[Second Spell]{
    \begin{minipage}{0.29\textwidth}
        \includegraphics[width=\textwidth]{bs_spell2_high_urban_r4}
    \end{minipage}
} 
\subfloat[Third Spell]{
    \begin{minipage}{0.29\textwidth}
        \includegraphics[width=\textwidth]{bs_spell3_high_urban_r4} 
    \end{minipage}
}
\\
\rotatebox[origin=c]{90}{\small{Rural}}
\subfloat[Second Spell]{
    \begin{minipage}{0.29\textwidth}
        \includegraphics[width=\textwidth]{bs_spell2_high_rural_r4}
    \end{minipage}
} 
\subfloat[Third Spell]{
    \begin{minipage}{0.29\textwidth}
        \includegraphics[width=\textwidth]{bs_spell3_high_rural_r4} 
    \end{minipage}
} 
\caption{Estimated average spell length, sex ratio, and probability of 
a next birth for women with eight or more years of education in the ``South''}
\label{fig:bootstrap_high_r4}
\end{figure}

The final region is the ``South'', shown in Figure \ref{fig:bootstrap_high_r4}.
This region is historically thought to have the lowest degree of son preference of the
regions of India.
As for the ``East'', I do not show the fourth spell results because of the small sample 
size and the associated noisy estimates.
Judging from the second spell progression rate and the associated spacing pattern, it
does, indeed, appear that there is little son preference in this region.
The likelihood of having a second birth is relatively high at around 80 to 90 percent,
and is close to independent of the sex of the first birth in both urban and rural areas.
The predicted average spacing also show no sign of son preference, and the predicted
sex ratios are close to the natural rate.
The average birth spacing started high relatively to the other regions, and with little 
change in fertility and no significant use of sex selection, it has increased only 
slightly over time.

With the high likelihood of a second birth and lower decline in the likelihood of a third
birth, the ``South'' has a higher predicted fertility than the ``East''.
What is more, for urban areas the average birth interval have increased substantially
for those with only girls relative to those with a boy and a girl, corresponding to
a predicted sex ratio close to 60 percent with only girls. 
The signs of the use of sex selection also show up for rural women with no sons.
Furthermore, over the four decades the likelihood of having a third birth has decreased
much less than for either a boy and a girl or two boys.%
\footnote{
The relatively high parity progression for women without son does raise the question
whether ``daughter avoidance'' is behind the fertility behavior in this region as
suggested by \citet{DiamondSmith2008}.
}
The differences in parity progression and the sex ratios suggests that despite originally
having less son preference, the ``South'' now appears to have the longer average spacing 
in absence of son and the higher sex ratio associated with the use of sex selection.

\section{Conclusion\label{sec:conclusion}}


The central question addressed here is the extent to which spacing patterns significantly 
changed as prenatal sex determination became available.
The underlying idea is that in the absence of sex selection son preference leads to shorter 
spacing after the birth of a girl than after the birth of a boy, whereas when prenatal sex 
selection is available son preference increases the spacing after the birth of a girl 
relative to after the birth of a boy.
I introduce an empirical method that simultaneously accounts for spacing between births 
and the potential use of sex selection. 
I apply the method to over four decades of data from India's NFHS.

Three major, interrelated trends over the the four decades emerge from the data.
First, fertility has declined for all groups, as shown by the parity progression
probabilities, with a larger decline the more educated the woman.
Across all groups, the likelihood of having an additional child in each spell depends 
strongly on how many sons you already have, with women with none or only one son having
the highest probability of a birth.
Second, there has been a general increase in the length of average birth 
intervals for all education groups.
The increase in average birth interval is larger the higher the spell and the higher the 
education group.
Finally, as shown by the increasing sex ratios for women with no or only one son there
has been a substantial increase in the use of sex selection over time.
As for fertility and birth intervals, this increase is larger the higher the spell and
with higher education.

The results show two very different approaches to son preference with important
impacts on spacing patterns.
At one extreme, rural women without education mostly follow the standard pattern of 
shorter spacing when a woman does not have the desired number of sons.
The strength of this pattern does, however, depend critically on spell number.
For the second spell, although the average interval after a first-born girl is shorter
than the average interval after a first-born boy, the difference is small.
The differences in spacing is larger for the third spell and largest for the fourth.
For the the fourth spell, the divergence in intervals across sex compositions have 
increased over time, with the largest increases in average intervals for women with 
two or three boys.
There has also been an increase in average spacing for women with only girls, and there
is some limited evidence of biased sex ratios, which suggests that this increases in
spacing follow from the use of sex selection.

At the other extreme, urban women with eight or more years of education have experienced 
an almost complete reversal of the traditional spacing patterns as a result of falling
fertility and the associated extensive use of sex selection.
Rather than having the next birth sooner as they mostly did before sex selection became 
available, women with either no or one son now have substantially longer spacing than if 
they have two or more sons.
As shown by the increasingly significant deviations in the sex ratio from the natural sex 
ratio, these changes are likely due to increased use of sex selection rather than sudden
reversal in son preference.
Only the second spell has not seen a reversal despite a substantial increased use of
sex selection, mainly because the average spacing with a first-born boy has increased 
independently by more than nine months.

Despite the differences in average spacing across sex compositions of prior children,
the picture of how son preference affects spacing is more complicated than excepted.
First, for the second spell the differences in birth intervals for the first 25 percent of 
births are negligible across sex compositions, despite that these birth intervals are
clearly within the durations considered the most likely to generate adverse outcomes.
Furthermore, there is little difference in this spacing patterns across education.
Hence, most of the divergence in spacing patterns across sex compositions arise from
longer spells, which are likely to have less negative impacts on children and the mother.
Second, one of the side-effects of the extensive use of sex selection for women with eight
or more years of education for the third and fourth spell is that only-girls families
not only end up with longer average intervals to the birth of the next, they also are now
much less likely to have a short, i.e. less than 18 months, interpregnancy interval,
which may reduce the sibling competition for resources.


With a country as large as India, it is natural to expect variation in both fertility
behavior, spacing patterns, and the degree of son preference.
The regional results confirm this, although differences are less apparent than one would
expect given the prior research on regional differences.
The ``North'' and the ``West'' regions have relatively similar spacing patterns, likely
precipitated by rapidly declining fertility and substantial use of sex selection.
The exception is that the rural areas of the ``North'' still has higher fertility and
consequently less use of sex selection and have, therefore, not yet experienced the
same reversal of spacing patterns.
Even the ``South'', which is otherwise considered the area with the least son preference,
has has seen a substantial increase in the use of sex selection and the associated changes 
in the spacing patterns.
Even though the ``South'' is often mentioned as the low-fertility region of India, 
fertility has declined fastest in the ``East'' region, although this does not appear to 
have translated into the use of sex selection.


The results here lead to a set of broader questions that future research should address.
First, what is behind the substantial increases in spacing even for women who are unlikely 
to use sex selection, and, related, why does spacing appear to increase with declining 
parity progression probabilities?
The distributional results provide an initial indication that most of the increases in 
average birth intervals come from the upper end of the birth intervals rather an generic
increase across the distribution.
It is, however, unclear why we are seeing relatively more very long birth intervals over
time, and the longer average spacing seem to run counter to a general reduction in the 
length of women's reproductive spans found in earlier research \citep{Padmadas2004}.
Understanding whether the longer average spacing is by choice at the beginning of the 
spell, because of unplanned pregnancies, or because of changes in economic circumstances 
that allow families to have an additional child later has important implications for both 
policy design and outcomes for mother and children.

Second, to what extent are the improvements in health for girls relative to boys the result 
of selection, the longer spacing between births, or changing son preference?
We know that health outcomes for girls appear to improve in the presence of sex selection 
\citep{Lin2014,Hu2015}.
Furthermore, there is some evidence that sample selection can make it appear that girls 
are healthier, even though the underlying cause is a combination of sex selection and 
higher mortality together with recall error \citep{Portner2018a}. 
Understanding what role these factors play in better health for girls is especially 
important when evaluating policies that aim to directly limit the use of sex selection 
rather than changing preferences or incentives.
If the better health outcomes for girls are, for example, an unintended side-effect of the 
longer spacing that arises from sex-selective abortions, rather than because the smaller 
number of girls makes them more valued as is often assumed, then an effective ban on sex 
selection may, at least temporarily, worsen health outcomes for girls.

Finally, we need to understand how female labor force participation interacts with the use 
of sex selection.
Increased autonomy for women, arising, for example, from better opportunities for working 
outside the home, has been suggested as a way to increase women's status and thereby lower 
the use of sex selection \citep{Das-Gupta2016}.
This may, however, be a double-edged sword.
On the one hand, it is a clear benefit to the women who gain bargaining power, and it 
increases the cost of repeated sex-selective abortions because the increased duration 
between births would cause a stronger disruption in labor market participation.
On the other hand, it may further lower desired fertility, and that may, everything else 
equal, lead to higher use of sex selection.
Understanding the trade-off between long-term benefits from improvements in women's labor 
force participation and short-term costs from potential increases in sex selection is of 
paramount importance.


\clearpage

\onehalfspacing
\bibliographystyle{aer}
\bibliography{sex_selection_spacing}

\addcontentsline{toc}{section}{References}



\clearpage
\newpage

\appendix

% CHANGING NUMBERING OF FIGURES AND TABLES FOR APPENDIX
\renewcommand\thefigure{\thesection.\arabic{figure}}    
\renewcommand\thetable{\thesection.\arabic{table}}    

\section*{Appendices for Online Publication}

These appendices are intended for online publication.
They provide the descriptive statistics, additional
estimated duration tables, and graphs for all 
education groups and spells.


\clearpage
\newpage

\section{Recall Error and the Sex Ratio}

\setcounter{figure}{0}
\setcounter{table}{0}

The reliability of the results depends on the correctness of the birth histories
provided by the respondents.
A significant concern here is underreporting of child mortality, especially a systematic
recall error where respondents' likelihood of reporting a deceased child depends on the 
sex of that child. 
This section assesses the degree of recall error across the surveys and discusses methods
to address it.

NFHS enumerators probe for any missed births, although the method depends on the survey.
NFHS-1 probe for each calendar birth interval that is four or more years.
NFHS-2 asked for stillbirths, spontaneous and induced abortions and also probed 
for each calendar birth interval four or more years.
NFHS-3 and NFHS-4 did not directly use birth intervals, but asked whether there were any 
other live births between (name of previous birth) and (name), including any children who 
died after birth, and asked for births before the birth listed as first birth and
after the last birth listed as the last birth.

Probing catches many initially missed births, but systematic recall error based on son
preference may still be a problem.
First, son preference leads to significantly higher mortality for girls than boys.
Secondly, son preference makes it more likely that parents will remember deceased boys 
than deceased girls.
Finally, in the absence of sex-selective abortions, parents with a preference for sons may
have the next birth sooner if the last child was a girl than if it was a boy.
If this girl subsequently dies, she is more likely to be missed if probing for missed 
births is only done for long intervals as in NFHS-1 and NFHS-2.

I use two approaches to examine the degree of recall error.
The first approach is to test whether the observed sex ratio is significantly different
from the natural sex ratio.
Prenatal sex determination techniques did not become widely available until the mid-1980s, 
so any significant deviation from the natural sex ratio before that time is likely the 
result of recall error.
The second approach is to compare births that took place during the same period but
where captured in different surveys.
Recall error is likely to increase with time, so births and deaths that took place earlier 
are more likely to be subject to recall error than more recent events.

Table \ref{tab:recallBirthBO1} shows the sex ratios of children recorded as first-born by 
year of birth, together with tests for whether the observed sex ratio is significantly 
higher than the natural sex ratio and whether more recent surveys have a higher sex ratio 
for the cohort than earlier surveys for the same period births.
Births are combined into five-year cohorts to achieve sufficient power.

\input{../tables/recallBirthBO1.tex}

% \input{../tables/recallBirthBO2.tex}


The ``first-born'' sex ratios illustrate the systematic recall error problem well.
In all four surveys around 55 percent of children reported as first-born are boys
for the first cohort of births observed.
Given that these cohorts cover from 1960-1964 to 1980-1984, which is before sex selection 
techniques became available in India, the most likely explanation for the skewed sex ratio 
is that some children listed as first-borns were not, in fact, the first children born in 
their families.
Instead, for a substantial proportion of families, their first-born was a girl who died 
and went unreported when enumerators asked about birth history.

As expected, the difference between the observed sex ratio and the natural sex ratio is 
less pronounced the closer to the survey date the cohort is.
The observed sex ratio for children born just before the NFHS-1 survey and listed as 
first-born is 0.517, which is not statistically significantly different from the
natural sex ratio.
The same general pattern holds for the other three surveys, with cohorts further away
from the survey date more likely to have a sex ratio skewed male.

% Second births show a pattern very similar to that for first births.
% An interesting difference is that there is evidence of a U-shaped relationship between
% time and sex ratio for second births.
% Cohorts furthest away from the survey year show the highest sex ratios, but declines to the
% natural sex ratio in the mid-1980s, and the sex ratio is then significantly higher again 
% for more recent births.
% This is in line with the results in the main paper that show that sex selective abortions 
% take place on lower parity births as desired fertility declined.

Finally, across surveys, the same cohort tends to show a higher sex ratio the more recent 
the survey (births in the cohort took place earlier relative to the survey date).
Despite this, few cohorts show significantly different sex ratios across surveys, most 
likely because of a lack of power.
The exception is that comparisons involving NFHS-4 are mostly statistically significant
since the number of surveyed households in NFHS-4 were much larger than in prior surveys.

The problem with the above approach is that the year of birth is affected by recall error; 
a second born child listed as first-born is born later than the real first born child.
Year of marriage should, however, be affected neither by parental recall error 
nor the use of sex-selective abortions.
Tables \ref{tab:recallMarriageBO1} and \ref{tab:recallMarriageBO2}, therefore, shows sex 
ratios of children recorded as first-born and second-born by year of parents' marriage, 
together with tests for whether the observed sex ratio is significantly higher than the 
natural sex ratio and whether more recent surveys show a higher sex ratio for the cohort 
than earlier surveys.
The basic recall error pattern remains, with women married longer ago more
likely to report that their first-born is a boy.
Similarly, comparing women married in the same five-year period across surveys shows
that women married longer ago are more likely to report having a son.


\input{../tables/recallMarriageBO1.tex}

\input{../tables/recallMarriageBO2.tex}


The relationship between the length of marriage and recall error can also be seen in 
Figures \ref{fig:sex_ratio_recall_rounds_bo1} and \ref{fig:sex_ratio_recall_rounds_bo2}, 
which show the observed sex ratio for children reported as first born as a function of 
the duration of marriage at the time of the survey.
The solid line is the sex ratio of children reported as first-born by the number of years 
between the survey and marriage, while the dashed lines indicate the 95 percent confidence 
interval and the horizontal line the natural sex ratio (approximately 0.512).
To ensure sufficient cell sizes I group years into twos.
In line with the results from Tables \ref{tab:recallMarriageBO1} and 
\ref{tab:recallMarriageBO2}, the observed ratio of boys is increasingly above the expected 
value the longer ago the parents were married.

\begin{figure}
\centering
\subfloat[NFHS-1]{\includegraphics[width=.49\textwidth]{recall_sex_ratio_marriage_round_1}}
\subfloat[NFHS-2]{\includegraphics[width=.49\textwidth]{recall_sex_ratio_marriage_round_2}} \\
\subfloat[NFHS-3]{\includegraphics[width=.49\textwidth]{recall_sex_ratio_marriage_round_3}} 
\subfloat[NFHS-3]{\includegraphics[width=.49\textwidth]{recall_sex_ratio_marriage_round_4}} 
\caption{Ratio of Boys for ``First'' Births by Survey Round}
\label{fig:sex_ratio_recall_rounds_bo1}
\end{figure}

\begin{figure}
\centering
\subfloat[NFHS-1]{\includegraphics[width=.49\textwidth]{recall_sex_ratio_marriage_round_1_bo2}}
\subfloat[NFHS-2]{\includegraphics[width=.49\textwidth]{recall_sex_ratio_marriage_round_2_bo2}} \\
\subfloat[NFHS-3]{\includegraphics[width=.49\textwidth]{recall_sex_ratio_marriage_round_3_bo2}} 
\subfloat[NFHS-3]{\includegraphics[width=.49\textwidth]{recall_sex_ratio_marriage_round_4_bo2}} 
\caption{Ratio of Boys for ``Second'' Births by Survey Round}
\label{fig:sex_ratio_recall_rounds_bo2}
\end{figure}


The increasingly unequal sex ratio with increasing marriage duration suggests that
a solution to the recall error problem is to drop observations for 
women who were married ``too far'' from the survey year.
The main problem is establishing what the best cut-off point should be, with the
trade-off between retaining enough observations and the correctness of the information.
As Tables \ref{tab:recallMarriageBO1} and \ref{tab:recallMarriageBO2} show, there are 
differences in recall error across the three surveys and between the two birth
orders, although this may be the result of differences in the number of observations 
across surveys.
Furthermore, the recall error pattern is not entirely consistent across observed birth 
orders.
Since most of the surveys start showing significantly biased sex ratio from around 22
years of marriage on, I drop all observations where the marriage took place 22 years
or more.


\clearpage
\newpage

\section{Descriptive Statistics}
\setcounter{figure}{0}
\setcounter{table}{0}


% Descriptive statistics tables
\input{../tables/des_stat.tex}

\clearpage
\newpage



\section{Duration Results Tables}

\setcounter{figure}{0}
\setcounter{table}{0}

\vspace*{-0.51cm}

\input{../tables/bootstrap_duration_avg_sex_ratio_low_all.tex}

\input{../tables/bootstrap_duration_avg_sex_ratio_med_all.tex}

\input{../tables/bootstrap_duration_avg_sex_ratio_high_all.tex}


\input{../tables/bootstrap_duration_p25_p75_low_all.tex}

\input{../tables/bootstrap_duration_p25_p75_med_all.tex}

\input{../tables/bootstrap_duration_p25_p75_high_all.tex}

\clearpage


\section{Results by Region}

\setcounter{figure}{0}
\setcounter{table}{0}

\vspace*{-0.51cm}

\input{../tables/bootstrap_duration_avg_sex_ratio_high_r1.tex}

\input{../tables/bootstrap_duration_avg_sex_ratio_high_r2.tex}

\input{../tables/bootstrap_duration_avg_sex_ratio_high_r3.tex}

\input{../tables/bootstrap_duration_avg_sex_ratio_high_r4.tex}

\input{../tables/bootstrap_duration_p25_p75_high_r1.tex}

\input{../tables/bootstrap_duration_p25_p75_high_r2.tex}

\input{../tables/bootstrap_duration_p25_p75_high_r3.tex}

\input{../tables/bootstrap_duration_p25_p75_high_r4.tex}


\clearpage
\newpage

\section{Graphs for All Education and Spell Groups}

\setcounter{figure}{0}
\setcounter{table}{0}


% [FIRST SPELL]

% \subsection{First Spell}
% 
% \input{../figures/appendix_spell1_low.tex}
% 
% \input{../figures/appendix_spell1_med.tex}
% 
% \input{../figures/appendix_spell1_high.tex}
% 
% \begin{figure}[htpb]
% \centering
% \caption*{No Education}
% \subfloat[Urban]{\includegraphics[width=0.49\textwidth]{spell1_low_urban_pps}} 
% \subfloat[Rural]{\includegraphics[width=0.49\textwidth]{spell1_low_rural_pps}} \\
% \caption*{1-7 Years of Education}
% \subfloat[Urban]{\includegraphics[width=0.49\textwidth]{spell1_med_urban_pps}} 
% \subfloat[Rural]{\includegraphics[width=0.49\textwidth]{spell1_med_rural_pps}} \\
% \caption*{8 or more Years of Education}
% \subfloat[Urban]{\includegraphics[width=0.49\textwidth]{spell1_high_urban_pps}} 
% \subfloat[Rural]{\includegraphics[width=0.49\textwidth]{spell1_high_rural_pps}} 
% \caption{Survival curves conditional on progression to first birth; start point is month of marriage}
% \label{fig:results_spell1_pps}
% \end{figure}
% 
% 
% \clearpage
% \newpage
% 
\subsection{Second Spell}


% \input{../figures/appendix_spell2_low.tex}


% PARITY PROGRESSION SURVIVAL CURVES - Low

% Low education
\vspace*{-0.75cm}
\begin{figure}[hp!]
\centering
\caption*{Urban}
\setcounter{subfigure}{-1}
\subfloat[1972--1984]{
    \begin{minipage}{0.32\textwidth}
        \captionsetup[subfigure]{labelformat=empty,position=top,captionskip=-1pt,farskip=-0.5pt}
        \subfloat[Prob.\ no birth yet]{\includegraphics[width=\textwidth]{spell2_g1_low_urban_pps}} 
        \captionsetup[subfigure]{labelformat=parens}
    \end{minipage}
} 
\setcounter{subfigure}{-0}
\subfloat[1985--1994]{
    \begin{minipage}{0.32\textwidth}
        \captionsetup[subfigure]{labelformat=empty,position=top,captionskip=-1pt,farskip=-0.5pt}
        \subfloat[Prob. no birth yet]{\includegraphics[width=\textwidth]{spell2_g2_low_urban_pps}}
        \captionsetup[subfigure]{labelformat=parens}
    \end{minipage}
} \\
\setcounter{subfigure}{1}
\subfloat[1995--2005]{
    \begin{minipage}{0.32\textwidth}
        \captionsetup[subfigure]{labelformat=empty,position=top,captionskip=-1pt,farskip=-0.5pt}
        \subfloat[Prob. no birth yet]{\includegraphics[width=\textwidth]{spell2_g3_low_urban_pps}}
        \captionsetup[subfigure]{labelformat=parens}
    \end{minipage}
}
\setcounter{subfigure}{2}
\subfloat[2005--2016]{
    \begin{minipage}{0.32\textwidth}
        \captionsetup[subfigure]{labelformat=empty,position=top,captionskip=-1pt,farskip=-0.5pt}
        \subfloat[Prob. no birth yet]{\includegraphics[width=\textwidth]{spell2_g4_low_urban_pps}}
        \captionsetup[subfigure]{labelformat=parens}
    \end{minipage}
}
\caption*{Rural}
\setcounter{subfigure}{3}
\subfloat[1972--1984]{
    \begin{minipage}{0.32\textwidth}
        \captionsetup[subfigure]{labelformat=empty,position=top,captionskip=-1pt,farskip=-0.5pt}
        \subfloat[Prob. no birth yet]{\includegraphics[width=\textwidth]{spell2_g1_low_rural_pps}} 
        \captionsetup[subfigure]{labelformat=parens}
    \end{minipage}
} 
\setcounter{subfigure}{4}
\subfloat[1985--1994]{
    \begin{minipage}{0.32\textwidth}
        \captionsetup[subfigure]{labelformat=empty,position=top,captionskip=-1pt,farskip=-0.5pt}
        \subfloat[Prob. no birth yet]{\includegraphics[width=\textwidth]{spell2_g2_low_rural_pps}}
        \captionsetup[subfigure]{labelformat=parens}
    \end{minipage}
} \\
\setcounter{subfigure}{5}
\subfloat[1995--2004]{
    \begin{minipage}{0.32\textwidth}
        \captionsetup[subfigure]{labelformat=empty,position=top,captionskip=-1pt,farskip=-0.5pt}
        \subfloat[Prob. no birth yet]{\includegraphics[width=\textwidth]{spell2_g3_low_rural_pps}}
        \captionsetup[subfigure]{labelformat=parens}
    \end{minipage}
}
\setcounter{subfigure}{6}
\subfloat[2005--2016]{
    \begin{minipage}{0.32\textwidth}
        \captionsetup[subfigure]{labelformat=empty,position=top,captionskip=-1pt,farskip=-0.5pt}
        \subfloat[Prob. no birth yet]{\includegraphics[width=\textwidth]{spell2_g4_low_rural_pps}}
        \captionsetup[subfigure]{labelformat=parens}
    \end{minipage}
}
\caption{Survival curves conditional on parity progression
for women with no education by month beginning 9 months after prior birth.
}
\label{fig:results_spell2_low_pps}
\end{figure}




% SPELL 2 - URBAN - MEDIUM

% \input{../figures/appendix_spell2_med.tex}

% PARITY PROGRESSION SURVIVAL - MEDIUM


% Low education

\begin{figure}[htpb]
\centering
\caption*{Urban}
\setcounter{subfigure}{-1}
\subfloat[1972--1984]{
    \begin{minipage}{0.32\textwidth}
        \captionsetup[subfigure]{labelformat=empty,position=top,captionskip=-1pt,farskip=-0.5pt}
        \subfloat[Prob.\ no birth yet]{\includegraphics[width=\textwidth]{spell2_g1_med_urban_pps}} 
        \captionsetup[subfigure]{labelformat=parens}
    \end{minipage}
} 
\setcounter{subfigure}{-0}
\subfloat[1985--1994]{
    \begin{minipage}{0.32\textwidth}
        \captionsetup[subfigure]{labelformat=empty,position=top,captionskip=-1pt,farskip=-0.5pt}
        \subfloat[Prob. no birth yet]{\includegraphics[width=\textwidth]{spell2_g2_med_urban_pps}}
        \captionsetup[subfigure]{labelformat=parens}
    \end{minipage}
} \\
\setcounter{subfigure}{1}
\subfloat[1995--2004]{
    \begin{minipage}{0.32\textwidth}
        \captionsetup[subfigure]{labelformat=empty,position=top,captionskip=-1pt,farskip=-0.5pt}
        \subfloat[Prob. no birth yet]{\includegraphics[width=\textwidth]{spell2_g3_med_urban_pps}}
        \captionsetup[subfigure]{labelformat=parens}
    \end{minipage}
}
\setcounter{subfigure}{2}
\subfloat[2005--2016]{
    \begin{minipage}{0.32\textwidth}
        \captionsetup[subfigure]{labelformat=empty,position=top,captionskip=-1pt,farskip=-0.5pt}
        \subfloat[Prob. no birth yet]{\includegraphics[width=\textwidth]{spell2_g4_med_urban_pps}}
        \captionsetup[subfigure]{labelformat=parens}
    \end{minipage}
}
\caption*{Rural}
\setcounter{subfigure}{3}
\subfloat[1972--1984]{
    \begin{minipage}{0.32\textwidth}
        \captionsetup[subfigure]{labelformat=empty,position=top,captionskip=-1pt,farskip=-0.5pt}
        \subfloat[Prob. no birth yet]{\includegraphics[width=\textwidth]{spell2_g1_med_rural_pps}} 
        \captionsetup[subfigure]{labelformat=parens}
    \end{minipage}
} 
\setcounter{subfigure}{4}
\subfloat[1985--1994]{
    \begin{minipage}{0.32\textwidth}
        \captionsetup[subfigure]{labelformat=empty,position=top,captionskip=-1pt,farskip=-0.5pt}
        \subfloat[Prob. no birth yet]{\includegraphics[width=\textwidth]{spell2_g2_med_rural_pps}}
        \captionsetup[subfigure]{labelformat=parens}
    \end{minipage}
} \\
\setcounter{subfigure}{5}
\subfloat[1995--2004]{
    \begin{minipage}{0.32\textwidth}
        \captionsetup[subfigure]{labelformat=empty,position=top,captionskip=-1pt,farskip=-0.5pt}
        \subfloat[Prob. no birth yet]{\includegraphics[width=\textwidth]{spell2_g3_med_rural_pps}}
        \captionsetup[subfigure]{labelformat=parens}
    \end{minipage}
}
\setcounter{subfigure}{6}
\subfloat[2005--2016]{
    \begin{minipage}{0.32\textwidth}
        \captionsetup[subfigure]{labelformat=empty,position=top,captionskip=-1pt,farskip=-0.5pt}
        \subfloat[Prob. no birth yet]{\includegraphics[width=\textwidth]{spell2_g4_med_rural_pps}}
        \captionsetup[subfigure]{labelformat=parens}
    \end{minipage}
}
\caption{Survival curves conditional on parity progression
for women with 1-7 years of education by month beginning 9 months after prior birth.
}
\label{fig:results_spell2_med_pps}
\end{figure}



% High education

% \input{../figures/appendix_spell2_high.tex}

% PARITY PROGRESSION SURVIVAL - High education

\begin{figure}[htpb]
\centering
\caption*{Urban}
\setcounter{subfigure}{-1}
\subfloat[1972--1984]{
    \begin{minipage}{0.31\textwidth}
        \captionsetup[subfigure]{labelformat=empty,position=top,captionskip=-1pt,farskip=-0.5pt}
        \subfloat[Prob.\ no birth yet]{\includegraphics[width=\textwidth]{spell2_g1_high_urban_pps}} 
        \captionsetup[subfigure]{labelformat=parens}
    \end{minipage}
} 
\setcounter{subfigure}{-0}
\subfloat[1985--1994]{
    \begin{minipage}{0.31\textwidth}
        \captionsetup[subfigure]{labelformat=empty,position=top,captionskip=-1pt,farskip=-0.5pt}
        \subfloat[Prob. no birth yet]{\includegraphics[width=\textwidth]{spell2_g2_high_urban_pps}}
        \captionsetup[subfigure]{labelformat=parens}
    \end{minipage}
} \\
\setcounter{subfigure}{1}
\subfloat[1995--2004]{
    \begin{minipage}{0.31\textwidth}
        \captionsetup[subfigure]{labelformat=empty,position=top,captionskip=-1pt,farskip=-0.5pt}
        \subfloat[Prob. no birth yet]{\includegraphics[width=\textwidth]{spell2_g3_high_urban_pps}}
        \captionsetup[subfigure]{labelformat=parens}
    \end{minipage}
}
\setcounter{subfigure}{2}
\subfloat[2005--2016]{
    \begin{minipage}{0.31\textwidth}
        \captionsetup[subfigure]{labelformat=empty,position=top,captionskip=-1pt,farskip=-0.5pt}
        \subfloat[Prob. no birth yet]{\includegraphics[width=\textwidth]{spell2_g4_high_urban_pps}}
        \captionsetup[subfigure]{labelformat=parens}
    \end{minipage}
}
\caption*{Rural}
\setcounter{subfigure}{3}
\subfloat[1972--1984]{
    \begin{minipage}{0.31\textwidth}
        \captionsetup[subfigure]{labelformat=empty,position=top,captionskip=-1pt,farskip=-0.5pt}
        \subfloat[Prob. no birth yet]{\includegraphics[width=\textwidth]{spell2_g1_high_rural_pps}} 
        \captionsetup[subfigure]{labelformat=parens}
    \end{minipage}
} 
\setcounter{subfigure}{4}
\subfloat[1985--1994]{
    \begin{minipage}{0.31\textwidth}
        \captionsetup[subfigure]{labelformat=empty,position=top,captionskip=-1pt,farskip=-0.5pt}
        \subfloat[Prob. no birth yet]{\includegraphics[width=\textwidth]{spell2_g2_high_rural_pps}}
        \captionsetup[subfigure]{labelformat=parens}
    \end{minipage}
} \\
\setcounter{subfigure}{5}
\subfloat[1995--2004]{
    \begin{minipage}{0.31\textwidth}
        \captionsetup[subfigure]{labelformat=empty,position=top,captionskip=-1pt,farskip=-0.5pt}
        \subfloat[Prob. no birth yet]{\includegraphics[width=\textwidth]{spell2_g3_high_rural_pps}}
        \captionsetup[subfigure]{labelformat=parens}
    \end{minipage}
}
\setcounter{subfigure}{6}
\subfloat[2005--2016]{
    \begin{minipage}{0.31\textwidth}
        \captionsetup[subfigure]{labelformat=empty,position=top,captionskip=-1pt,farskip=-0.5pt}
        \subfloat[Prob. no birth yet]{\includegraphics[width=\textwidth]{spell2_g4_high_rural_pps}}
        \captionsetup[subfigure]{labelformat=parens}
    \end{minipage}
}
\caption{Survival curves conditional on parity progression
for women with 8 or more years of education by month beginning 9 months after prior birth.
}
\label{fig:results_spell2_high_pps}
\end{figure}


\clearpage
\newpage

\subsection{Third Spell}

% low education
% \input{../figures/appendix_spell3_low.tex}

% PARITY PROGRESSION SURVIVAL - LOW
\vspace*{-0.75cm}
\begin{figure}[hp!]
\centering
\caption*{Urban}
\setcounter{subfigure}{-1}
\subfloat[1972--1984]{
    \begin{minipage}{0.31\textwidth}
        \captionsetup[subfigure]{labelformat=empty,position=top,captionskip=-1pt,farskip=-0.5pt}
        \subfloat[Prob.\ no birth yet]{\includegraphics[width=\textwidth]{spell3_g1_low_urban_pps}} 
        \captionsetup[subfigure]{labelformat=parens}
    \end{minipage}
} 
\setcounter{subfigure}{-0}
\subfloat[1985--1994]{
    \begin{minipage}{0.31\textwidth}
        \captionsetup[subfigure]{labelformat=empty,position=top,captionskip=-1pt,farskip=-0.5pt}
        \subfloat[Prob. no birth yet]{\includegraphics[width=\textwidth]{spell3_g2_low_urban_pps}}
        \captionsetup[subfigure]{labelformat=parens}
    \end{minipage}
} \\
\setcounter{subfigure}{1}
\subfloat[1995--2004]{
    \begin{minipage}{0.31\textwidth}
        \captionsetup[subfigure]{labelformat=empty,position=top,captionskip=-1pt,farskip=-0.5pt}
        \subfloat[Prob. no birth yet]{\includegraphics[width=\textwidth]{spell3_g3_low_urban_pps}}
        \captionsetup[subfigure]{labelformat=parens}
    \end{minipage}
}
\setcounter{subfigure}{2}
\subfloat[2005--2016]{
    \begin{minipage}{0.31\textwidth}
        \captionsetup[subfigure]{labelformat=empty,position=top,captionskip=-1pt,farskip=-0.5pt}
        \subfloat[Prob. no birth yet]{\includegraphics[width=\textwidth]{spell3_g4_low_urban_pps}}
        \captionsetup[subfigure]{labelformat=parens}
    \end{minipage}
}
\caption*{Rural}
\setcounter{subfigure}{3}
\subfloat[1972--1984]{
    \begin{minipage}{0.31\textwidth}
        \captionsetup[subfigure]{labelformat=empty,position=top,captionskip=-1pt,farskip=-0.5pt}
        \subfloat[Prob. no birth yet]{\includegraphics[width=\textwidth]{spell3_g1_low_rural_pps}} 
        \captionsetup[subfigure]{labelformat=parens}
    \end{minipage}
} 
\setcounter{subfigure}{4}
\subfloat[1985--1994]{
    \begin{minipage}{0.31\textwidth}
        \captionsetup[subfigure]{labelformat=empty,position=top,captionskip=-1pt,farskip=-0.5pt}
        \subfloat[Prob. no birth yet]{\includegraphics[width=\textwidth]{spell3_g2_low_rural_pps}}
        \captionsetup[subfigure]{labelformat=parens}
    \end{minipage}
} \\
\setcounter{subfigure}{5}
\subfloat[1995--2004]{
    \begin{minipage}{0.31\textwidth}
        \captionsetup[subfigure]{labelformat=empty,position=top,captionskip=-1pt,farskip=-0.5pt}
        \subfloat[Prob. no birth yet]{\includegraphics[width=\textwidth]{spell3_g3_low_rural_pps}}
        \captionsetup[subfigure]{labelformat=parens}
    \end{minipage}
}
\setcounter{subfigure}{6}
\subfloat[2005--2016]{
    \begin{minipage}{0.31\textwidth}
        \captionsetup[subfigure]{labelformat=empty,position=top,captionskip=-1pt,farskip=-0.5pt}
        \subfloat[Prob. no birth yet]{\includegraphics[width=\textwidth]{spell3_g4_low_rural_pps}}
        \captionsetup[subfigure]{labelformat=parens}
    \end{minipage}
}
\caption{Survival curves conditional on parity progression
for women with no education by month beginning 9 months after prior birth.
}
\label{fig:results_spell3_low_pps}
\end{figure}




% Medium education

% \input{../figures/appendix_spell3_med.tex}

% PARITY PROGRESSION SURVIVAL - MEDIUM

\begin{figure}[htpb]
\centering
\caption*{Urban}
\setcounter{subfigure}{-1}
\subfloat[1972--1984]{
    \begin{minipage}{0.31\textwidth}
        \captionsetup[subfigure]{labelformat=empty,position=top,captionskip=-1pt,farskip=-0.5pt}
        \subfloat[Prob.\ no birth yet]{\includegraphics[width=\textwidth]{spell3_g1_med_urban_pps}} 
        \captionsetup[subfigure]{labelformat=parens}
    \end{minipage}
} 
\setcounter{subfigure}{-0}
\subfloat[1985--1994]{
    \begin{minipage}{0.31\textwidth}
        \captionsetup[subfigure]{labelformat=empty,position=top,captionskip=-1pt,farskip=-0.5pt}
        \subfloat[Prob. no birth yet]{\includegraphics[width=\textwidth]{spell3_g2_med_urban_pps}}
        \captionsetup[subfigure]{labelformat=parens}
    \end{minipage}
} \\
\setcounter{subfigure}{1}
\subfloat[1995--2004]{
    \begin{minipage}{0.31\textwidth}
        \captionsetup[subfigure]{labelformat=empty,position=top,captionskip=-1pt,farskip=-0.5pt}
        \subfloat[Prob. no birth yet]{\includegraphics[width=\textwidth]{spell3_g3_med_urban_pps}}
        \captionsetup[subfigure]{labelformat=parens}
    \end{minipage}
}
\setcounter{subfigure}{2}
\subfloat[2005--2016]{
    \begin{minipage}{0.31\textwidth}
        \captionsetup[subfigure]{labelformat=empty,position=top,captionskip=-1pt,farskip=-0.5pt}
        \subfloat[Prob. no birth yet]{\includegraphics[width=\textwidth]{spell3_g4_med_urban_pps}}
        \captionsetup[subfigure]{labelformat=parens}
    \end{minipage}
}
\caption*{Rural}
\setcounter{subfigure}{3}
\subfloat[1972--1984]{
    \begin{minipage}{0.31\textwidth}
        \captionsetup[subfigure]{labelformat=empty,position=top,captionskip=-1pt,farskip=-0.5pt}
        \subfloat[Prob. no birth yet]{\includegraphics[width=\textwidth]{spell3_g1_med_rural_pps}} 
        \captionsetup[subfigure]{labelformat=parens}
    \end{minipage}
} 
\setcounter{subfigure}{4}
\subfloat[1985--1994]{
    \begin{minipage}{0.31\textwidth}
        \captionsetup[subfigure]{labelformat=empty,position=top,captionskip=-1pt,farskip=-0.5pt}
        \subfloat[Prob. no birth yet]{\includegraphics[width=\textwidth]{spell3_g2_med_rural_pps}}
        \captionsetup[subfigure]{labelformat=parens}
    \end{minipage}
} \\
\setcounter{subfigure}{5}
\subfloat[1995--2004]{
    \begin{minipage}{0.31\textwidth}
        \captionsetup[subfigure]{labelformat=empty,position=top,captionskip=-1pt,farskip=-0.5pt}
        \subfloat[Prob. no birth yet]{\includegraphics[width=\textwidth]{spell3_g3_med_rural_pps}}
        \captionsetup[subfigure]{labelformat=parens}
    \end{minipage}
}
\setcounter{subfigure}{6}
\subfloat[2005--2016]{
    \begin{minipage}{0.31\textwidth}
        \captionsetup[subfigure]{labelformat=empty,position=top,captionskip=-1pt,farskip=-0.5pt}
        \subfloat[Prob. no birth yet]{\includegraphics[width=\textwidth]{spell3_g4_med_rural_pps}}
        \captionsetup[subfigure]{labelformat=parens}
    \end{minipage}
}
\caption{Survival curves conditional on parity progression
for women with 1 to 7 years of education by month beginning 9 months after prior birth.
}
\label{fig:results_spell3_med_pps}
\end{figure}




% High education

% \input{../figures/appendix_spell3_high.tex}

% PARITY PROGRESSION SURVIVAL - high

\begin{figure}[htpb]
\centering
\caption*{Urban}
\setcounter{subfigure}{-1}
\subfloat[1972--1984]{
    \begin{minipage}{0.31\textwidth}
        \captionsetup[subfigure]{labelformat=empty,position=top,captionskip=-1pt,farskip=-0.5pt}
        \subfloat[Prob.\ no birth yet]{\includegraphics[width=\textwidth]{spell3_g1_high_urban_pps}} 
        \captionsetup[subfigure]{labelformat=parens}
    \end{minipage}
} 
\setcounter{subfigure}{-0}
\subfloat[1985--1994]{
    \begin{minipage}{0.31\textwidth}
        \captionsetup[subfigure]{labelformat=empty,position=top,captionskip=-1pt,farskip=-0.5pt}
        \subfloat[Prob. no birth yet]{\includegraphics[width=\textwidth]{spell3_g2_high_urban_pps}}
        \captionsetup[subfigure]{labelformat=parens}
    \end{minipage}
} \\
\setcounter{subfigure}{1}
\subfloat[1995--2004]{
    \begin{minipage}{0.31\textwidth}
        \captionsetup[subfigure]{labelformat=empty,position=top,captionskip=-1pt,farskip=-0.5pt}
        \subfloat[Prob. no birth yet]{\includegraphics[width=\textwidth]{spell3_g3_high_urban_pps}}
        \captionsetup[subfigure]{labelformat=parens}
    \end{minipage}
}
\setcounter{subfigure}{2}
\subfloat[2005--2016]{
    \begin{minipage}{0.31\textwidth}
        \captionsetup[subfigure]{labelformat=empty,position=top,captionskip=-1pt,farskip=-0.5pt}
        \subfloat[Prob. no birth yet]{\includegraphics[width=\textwidth]{spell3_g4_high_urban_pps}}
        \captionsetup[subfigure]{labelformat=parens}
    \end{minipage}
}
\caption*{Rural}
\setcounter{subfigure}{3}
\subfloat[1972--1984]{
    \begin{minipage}{0.31\textwidth}
        \captionsetup[subfigure]{labelformat=empty,position=top,captionskip=-1pt,farskip=-0.5pt}
        \subfloat[Prob. no birth yet]{\includegraphics[width=\textwidth]{spell3_g1_high_rural_pps}} 
        \captionsetup[subfigure]{labelformat=parens}
    \end{minipage}
} 
\setcounter{subfigure}{4}
\subfloat[1985--1994]{
    \begin{minipage}{0.31\textwidth}
        \captionsetup[subfigure]{labelformat=empty,position=top,captionskip=-1pt,farskip=-0.5pt}
        \subfloat[Prob. no birth yet]{\includegraphics[width=\textwidth]{spell3_g2_high_rural_pps}}
        \captionsetup[subfigure]{labelformat=parens}
    \end{minipage}
} \\
\setcounter{subfigure}{5}
\subfloat[1995--2004]{
    \begin{minipage}{0.31\textwidth}
        \captionsetup[subfigure]{labelformat=empty,position=top,captionskip=-1pt,farskip=-0.5pt}
        \subfloat[Prob. no birth yet]{\includegraphics[width=\textwidth]{spell3_g3_high_rural_pps}}
        \captionsetup[subfigure]{labelformat=parens}
    \end{minipage}
}
\setcounter{subfigure}{6}
\subfloat[2005--2016]{
    \begin{minipage}{0.31\textwidth}
        \captionsetup[subfigure]{labelformat=empty,position=top,captionskip=-1pt,farskip=-0.5pt}
        \subfloat[Prob. no birth yet]{\includegraphics[width=\textwidth]{spell3_g4_high_rural_pps}}
        \captionsetup[subfigure]{labelformat=parens}
    \end{minipage}
}
\caption{Survival curves conditional on parity progression
for women with 8 or more years of education by month beginning 9 months after prior birth.
}
\label{fig:results_spell3_high_pps}
\end{figure}




\clearpage
\newpage

\subsection{Fourth Spell}

% low education

% \input{../figures/appendix_spell4_low.tex}

% PARITY PROGRESSION SURVIVAL - LOW
\vspace*{-0.75cm}
\begin{figure}[hp!]
\centering
\caption*{Urban}
\setcounter{subfigure}{-1}
\subfloat[1972--1984]{
    \begin{minipage}{0.31\textwidth}
        \captionsetup[subfigure]{labelformat=empty,position=top,captionskip=-1pt,farskip=-0.5pt}
        \subfloat[Prob.\ no birth yet]{\includegraphics[width=\textwidth]{spell4_g1_low_urban_pps}} 
        \captionsetup[subfigure]{labelformat=parens}
    \end{minipage}
} 
\setcounter{subfigure}{-0}
\subfloat[1985--1994]{
    \begin{minipage}{0.31\textwidth}
        \captionsetup[subfigure]{labelformat=empty,position=top,captionskip=-1pt,farskip=-0.5pt}
        \subfloat[Prob. no birth yet]{\includegraphics[width=\textwidth]{spell4_g2_low_urban_pps}}
        \captionsetup[subfigure]{labelformat=parens}
    \end{minipage}
} \\
\setcounter{subfigure}{1}
\subfloat[1995--2004]{
    \begin{minipage}{0.31\textwidth}
        \captionsetup[subfigure]{labelformat=empty,position=top,captionskip=-1pt,farskip=-0.5pt}
        \subfloat[Prob. no birth yet]{\includegraphics[width=\textwidth]{spell4_g3_low_urban_pps}}
        \captionsetup[subfigure]{labelformat=parens}
    \end{minipage}
}
\setcounter{subfigure}{2}
\subfloat[2005--2016]{
    \begin{minipage}{0.31\textwidth}
        \captionsetup[subfigure]{labelformat=empty,position=top,captionskip=-1pt,farskip=-0.5pt}
        \subfloat[Prob. no birth yet]{\includegraphics[width=\textwidth]{spell4_g4_low_urban_pps}}
        \captionsetup[subfigure]{labelformat=parens}
    \end{minipage}
}
\caption*{Rural}
\setcounter{subfigure}{3}
\subfloat[1972--1984]{
    \begin{minipage}{0.31\textwidth}
        \captionsetup[subfigure]{labelformat=empty,position=top,captionskip=-1pt,farskip=-0.5pt}
        \subfloat[Prob. no birth yet]{\includegraphics[width=\textwidth]{spell4_g1_low_rural_pps}} 
        \captionsetup[subfigure]{labelformat=parens}
    \end{minipage}
} 
\setcounter{subfigure}{4}
\subfloat[1985--1994]{
    \begin{minipage}{0.31\textwidth}
        \captionsetup[subfigure]{labelformat=empty,position=top,captionskip=-1pt,farskip=-0.5pt}
        \subfloat[Prob. no birth yet]{\includegraphics[width=\textwidth]{spell4_g2_low_rural_pps}}
        \captionsetup[subfigure]{labelformat=parens}
    \end{minipage}
} \\
\setcounter{subfigure}{5}
\subfloat[1995--2004]{
    \begin{minipage}{0.31\textwidth}
        \captionsetup[subfigure]{labelformat=empty,position=top,captionskip=-1pt,farskip=-0.5pt}
        \subfloat[Prob. no birth yet]{\includegraphics[width=\textwidth]{spell4_g3_low_rural_pps}}
        \captionsetup[subfigure]{labelformat=parens}
    \end{minipage}
}
\setcounter{subfigure}{6}
\subfloat[2005--2016]{
    \begin{minipage}{0.31\textwidth}
        \captionsetup[subfigure]{labelformat=empty,position=top,captionskip=-1pt,farskip=-0.5pt}
        \subfloat[Prob. no birth yet]{\includegraphics[width=\textwidth]{spell4_g4_low_rural_pps}}
        \captionsetup[subfigure]{labelformat=parens}
    \end{minipage}
}
\caption{Survival curves conditional on parity progression
for women with no education by month beginning 9 months after prior birth.
}
\label{fig:results_spell4_low_pps}
\end{figure}




% Medium education

% \input{../figures/appendix_spell4_med.tex}

% PARITY PROGRESSION SURVIVAL - MEDIUM


\begin{figure}[htpb]
\centering
\caption*{Urban}
\setcounter{subfigure}{-1}
\subfloat[1972--1984]{
    \begin{minipage}{0.31\textwidth}
        \captionsetup[subfigure]{labelformat=empty,position=top,captionskip=-1pt,farskip=-0.5pt}
        \subfloat[Prob.\ no birth yet]{\includegraphics[width=\textwidth]{spell4_g1_med_urban_pps}} 
        \captionsetup[subfigure]{labelformat=parens}
    \end{minipage}
} 
\setcounter{subfigure}{-0}
\subfloat[1985--1994]{
    \begin{minipage}{0.31\textwidth}
        \captionsetup[subfigure]{labelformat=empty,position=top,captionskip=-1pt,farskip=-0.5pt}
        \subfloat[Prob. no birth yet]{\includegraphics[width=\textwidth]{spell4_g2_med_urban_pps}}
        \captionsetup[subfigure]{labelformat=parens}
    \end{minipage}
} \\
\setcounter{subfigure}{1}
\subfloat[1995--2004]{
    \begin{minipage}{0.31\textwidth}
        \captionsetup[subfigure]{labelformat=empty,position=top,captionskip=-1pt,farskip=-0.5pt}
        \subfloat[Prob. no birth yet]{\includegraphics[width=\textwidth]{spell4_g3_med_urban_pps}}
        \captionsetup[subfigure]{labelformat=parens}
    \end{minipage}
}
\setcounter{subfigure}{2}
\subfloat[2005--2016]{
    \begin{minipage}{0.31\textwidth}
        \captionsetup[subfigure]{labelformat=empty,position=top,captionskip=-1pt,farskip=-0.5pt}
        \subfloat[Prob. no birth yet]{\includegraphics[width=\textwidth]{spell4_g4_med_urban_pps}}
        \captionsetup[subfigure]{labelformat=parens}
    \end{minipage}
}
\caption*{Rural}
\setcounter{subfigure}{3}
\subfloat[1972--1984]{
    \begin{minipage}{0.31\textwidth}
        \captionsetup[subfigure]{labelformat=empty,position=top,captionskip=-1pt,farskip=-0.5pt}
        \subfloat[Prob. no birth yet]{\includegraphics[width=\textwidth]{spell4_g1_med_rural_pps}} 
        \captionsetup[subfigure]{labelformat=parens}
    \end{minipage}
} 
\setcounter{subfigure}{4}
\subfloat[1985--1994]{
    \begin{minipage}{0.31\textwidth}
        \captionsetup[subfigure]{labelformat=empty,position=top,captionskip=-1pt,farskip=-0.5pt}
        \subfloat[Prob. no birth yet]{\includegraphics[width=\textwidth]{spell4_g2_med_rural_pps}}
        \captionsetup[subfigure]{labelformat=parens}
    \end{minipage}
} \\
\setcounter{subfigure}{5}
\subfloat[1995--2004]{
    \begin{minipage}{0.31\textwidth}
        \captionsetup[subfigure]{labelformat=empty,position=top,captionskip=-1pt,farskip=-0.5pt}
        \subfloat[Prob. no birth yet]{\includegraphics[width=\textwidth]{spell4_g3_med_rural_pps}}
        \captionsetup[subfigure]{labelformat=parens}
    \end{minipage}
}
\setcounter{subfigure}{6}
\subfloat[2005--2016]{
    \begin{minipage}{0.31\textwidth}
        \captionsetup[subfigure]{labelformat=empty,position=top,captionskip=-1pt,farskip=-0.5pt}
        \subfloat[Prob. no birth yet]{\includegraphics[width=\textwidth]{spell4_g4_med_rural_pps}}
        \captionsetup[subfigure]{labelformat=parens}
    \end{minipage}
}
\caption{Survival curves conditional on parity progression
for women with 1 to 7 years of education by month beginning 9 months after prior birth.
}
\label{fig:results_spell4_med_pps}
\end{figure}



% High education

% \input{../figures/appendix_spell4_high.tex}

% PARITY PROGRESSION SURVIVAL - high

\begin{figure}[htpb]
\centering
\caption*{Urban}
\setcounter{subfigure}{-1}
\subfloat[1972--1984]{
    \begin{minipage}{0.31\textwidth}
        \captionsetup[subfigure]{labelformat=empty,position=top,captionskip=-1pt,farskip=-0.5pt}
        \subfloat[Prob.\ no birth yet]{\includegraphics[width=\textwidth]{spell4_g1_high_urban_pps}} 
        \captionsetup[subfigure]{labelformat=parens}
    \end{minipage}
} 
\setcounter{subfigure}{-0}
\subfloat[1985--1994]{
    \begin{minipage}{0.31\textwidth}
        \captionsetup[subfigure]{labelformat=empty,position=top,captionskip=-1pt,farskip=-0.5pt}
        \subfloat[Prob. no birth yet]{\includegraphics[width=\textwidth]{spell4_g2_high_urban_pps}}
        \captionsetup[subfigure]{labelformat=parens}
    \end{minipage}
} \\
\setcounter{subfigure}{1}
\subfloat[1995--2004]{
    \begin{minipage}{0.31\textwidth}
        \captionsetup[subfigure]{labelformat=empty,position=top,captionskip=-1pt,farskip=-0.5pt}
        \subfloat[Prob. no birth yet]{\includegraphics[width=\textwidth]{spell4_g3_high_urban_pps}}
        \captionsetup[subfigure]{labelformat=parens}
    \end{minipage}
}
\setcounter{subfigure}{2}
\subfloat[2005--2016]{
    \begin{minipage}{0.31\textwidth}
        \captionsetup[subfigure]{labelformat=empty,position=top,captionskip=-1pt,farskip=-0.5pt}
        \subfloat[Prob. no birth yet]{\includegraphics[width=\textwidth]{spell4_g4_high_urban_pps}}
        \captionsetup[subfigure]{labelformat=parens}
    \end{minipage}
}
\caption*{Rural}
\setcounter{subfigure}{3}
\subfloat[1972--1984]{
    \begin{minipage}{0.31\textwidth}
        \captionsetup[subfigure]{labelformat=empty,position=top,captionskip=-1pt,farskip=-0.5pt}
        \subfloat[Prob. no birth yet]{\includegraphics[width=\textwidth]{spell4_g1_high_rural_pps}} 
        \captionsetup[subfigure]{labelformat=parens}
    \end{minipage}
} 
\setcounter{subfigure}{4}
\subfloat[1985--1994]{
    \begin{minipage}{0.31\textwidth}
        \captionsetup[subfigure]{labelformat=empty,position=top,captionskip=-1pt,farskip=-0.5pt}
        \subfloat[Prob. no birth yet]{\includegraphics[width=\textwidth]{spell4_g2_high_rural_pps}}
        \captionsetup[subfigure]{labelformat=parens}
    \end{minipage}
} \\
\setcounter{subfigure}{5}
\subfloat[1995--2004]{
    \begin{minipage}{0.31\textwidth}
        \captionsetup[subfigure]{labelformat=empty,position=top,captionskip=-1pt,farskip=-0.5pt}
        \subfloat[Prob. no birth yet]{\includegraphics[width=\textwidth]{spell4_g3_high_rural_pps}}
        \captionsetup[subfigure]{labelformat=parens}
    \end{minipage}
}
\setcounter{subfigure}{6}
\subfloat[2005--2016]{
    \begin{minipage}{0.31\textwidth}
        \captionsetup[subfigure]{labelformat=empty,position=top,captionskip=-1pt,farskip=-0.5pt}
        \subfloat[Prob. no birth yet]{\includegraphics[width=\textwidth]{spell4_g4_high_rural_pps}}
        \captionsetup[subfigure]{labelformat=parens}
    \end{minipage}
}
\caption{Survival curves conditional on parity progression
for women with 8 or more years of education by month beginning 9 months after prior birth.
}
\label{fig:results_spell4_high_pps}
\end{figure}




\end{document}



