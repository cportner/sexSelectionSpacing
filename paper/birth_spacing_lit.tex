
\citet{Dewey2007}

Review study

The three conceptual models demonstrate that the relationship between
interpregnancy interval and maternal or child nutritional status is by
no means simple.

The studies on child nutrition outcomes indicate that a longer birth
interval is associated with a lower risk of malnutrition in some
populations, but not all.

The studies on maternal anthropometric outcomes yielded mixed results.
Because the nutritional burden on the mother between pregnancies depends
on the extent of breastfeeding, the interpregnancy interval is not the
best measure of whether the mother has had a chance to recover from the
pregnancy, in terms of repleting her nutritional status. Therefore, some
studies examined the ‘recuperative interval’ (duration of the
non-pregnant, non-lactating interval) instead.

Lactation represents a greater nutritional burden than pregnancy,

[link with paper on shorter breastfeeding for girls than for boys]

Important methodological limitations in papers used.

Mechanisms: 
- new pregnancy can lead to reduction is breastfeeding (reduced volume
of breastmilk), which
can be hazardous both nutritionally and in terms of resistance to
infection.
(lower breastfeeding -> less contraceptive effect -> shorter birth
intervals)

- care of new infant affects prior child; might even happen before birth
of new child

\citet{Gribble2009}

El Salvadore NFHS 2002-2003

Our objectives are to explore how short birth intervals may increase the
risk of stunting and underweight, and to identify policy and
programmatic efforts that can contribute to reducing the risk of
undernutrition, including healthy timing and spacing of pregnancies.

Nevertheless, the findings do suggest that birth intervals of less than 3
years are either directly related to stunting or act as a proxy for a
risk factor that is either unobserved or not included in the analysis.

(no attempt to estimate what determines birth spacing, although do
acknowledge that
causality is difficult to establish)

\citet{Soest2018}

we use dynamic panel data techniques to analyse the causal effects of
infant mortality on birth intervals and fertility, as well as the causal
effects of birth intervals on mortality in rural Bangladesh, accounting
for unobserved heterogeneity and reverse causality


The main novelty compared to [14] is that our application concerns
infant mortality in rural Bangladesh. While Bhalotra and van Soest used
retrospective data to analyse neonatal mortality in India, we use
prospective data from the Demographic and Health Surveillance System,
Matlab, following mothers residing in the study area over time.


Many studies also found an association between a short birth interval
and infant death of the next child, particularly when the preceding
sibling survived [21–23]. An explanation for this is that the mother has
not recuperated physiologically from the previous birth [24–25].

(!this is an important point. In the literature, the mortality effect
discussed
generally comes from the subsequent child, not the prior one. 
Hence, longer spacing should be beneficial for the boy born next if 
sex selection has been used.
This could explain part of the reduction in female mortality: The
girls that would "normally" have been born relatively soon after the
prior birth, and, therefore, have higher mortality risk, now are no
longer born.
)

Sibling competition may also explain why short birth intervals and high
fertility increase death risk: sources of food and care per head
diminish as the number of dependent members of a family increases [5].

(this effect can be used to explain any reductions in mortality for
the prior girls.

A type of this would be cession of breastfeeding because of a short
birth interval - see Conde et al. paper below for discussion of 
breastfeeding. 
)

We assume gender of each birth is exogenous and thus do not incorporate
the possibility of selective abortion. Descriptive statistics confirm
that this is not an issue in these data, revealing no relation at all
between gender of a newborn child and gender composition of the older
children in the family.


keeping other covariates and unobserved mother specific factors
constant, if the previous child survived its infancy, the mortality
probability falls with the length of the birth interval as long as the
birth interval is less than 63.3 months

If the previous child died, mortality falls with birth interval length
for intervals up to 52.5 months, still much beyond the median.

In the comparison area, death at infancy of the previous child shortens
the subsequent birth interval by about 46\%.

Having at least one boy has a stronger positive effect on the birth
interval than having a girl. The same applies to each additional boy.

Birth intervals increase with the mother’s education level, in line with
the positive relation between birth intervals and socioeconomic status.
This is in line with the finding that the use of contraception is more
common among higher socioeconomic status groups [38].

A large negative correlation is observed between unobserved
heterogeneity in birth interval and fertility equations, confirming that
both equations are strongly related: mothers who desire many children
also tend to use shorter birth intervals. This is consistent with the
target fertility model of Wolpin [4] and in line with the findings in
[14].

We find evidence of causal effects in two directions: a short preceding
birth interval reduces survival chances of infants, and an infant death
increases the probability of a next birth and shortens the time until
the next birth (replacement behaviour)

Estimates of reproductive behaviour are consistent with gender
preference: having more surviving boys significantly reduces the
probability of having a next child and this effect is strongest in the
comparison area

\citet{Zimmermann2018}

Son preference is widespread in a number of developing countries despite
substantial improvements in education levels and economic development.
One potential explanation for the persistence of this phenomenon is that
individual household members like the mother derive large non-monetary
benefits from giving birth to a son and therefore prefer boys to girls.

I focus on first-born children for whom the sex ratio of girls relative
to boys is normal.

The results show little evidence of consistently large female benefits
shortly after birth, and any positive impacts of having a son disappear
after the first six months. There are also no large benefits for adult
sons. These empirical patterns do not support qualitative evidence
suggesting that women benefit from the birth of a son through larger
decision-making powers in the household because of increased respect by
other household members.

(Li and Wu (2011) is the paper on China that found positive effects on
bargaining power of a first born son)

(does not seem directly relevant for this paper.)

\citet{Conde-Agudelo2012}

The following hypothetical causal mechanisms for explaining the
association between short intervals and adverse outcomes were
identified: maternal nutritional depletion, folate depletion, cervical
insufficiency, vertical transmission of infections, suboptimal lactation
related to breastfeeding–pregnancy overlap, sibling competition,
transmission of infectious diseases among siblings, incomplete healing
of uterine scar from previous cesarean delivery, and abnormal remodeling
of endometrial blood vessels.

Interpregnancy intervals shorter than 18 months and longer than 59
months are significantly associated with increased risk of adverse
perinatal outcomes such as preterm birth, low birth-weight, and small
for gestational age (Conde-Agudelo, Rosas-Bermudez, and Kafury-Goeta
2006).

Preceding interpregnancy intervals shorter than 36 months are
significantly associated with a greater risk of child and
under-five-years mortality, and intervals shorter than 24 months
significantly increase risk of early neonatal, neonatal, and infant
mortality (Rutstein 2008).

Hypotheses generally adopt either biological or behavioral orientations,
but no one framework or hypothesis has emerged as dominant (Erickson and
Bjerkedal 1979; Winikoff 1983; Klebanoff 1999).

The current maternal nutritional depletion hypothesis states that a
close succession of pregnancies and of periods of lactation worsens the
mother’s nutritional status because of inadequate time to recover from
the physiological stresses of the preceding pregnancy before becoming
subject to the stresses of the next pregnancy (Winkvist, Rasmussen, and
Habicht 1992; King 2003).

In summary, the studies that evaluated the effects of birth spacing on
maternal anthropometric status, anemia, and micronutrient status did not
provide clear evidence to support the maternal nutritional depletion
hypothesis.

(if parents know that/believe that there are substantially negative 
effects of short spacing, why would they engage in it. It would put
they subsequent son at an disadvantage. Unless the compensatory effects
are so large that it balances out the negative effects.)

if a new pregnancy starts before complete folate restoration, the woman
will be at higher risk of maternal folate deficiency and subsequent
adverse perinatal outcomes such as low birthweight, preterm birth, and
small for gestational age.

Strong evidence exists that folate depletion occurs in women during the
first three to four months postpartum, and growing evidence supports the
hypothesis that this depletion constitutes a hypothetical causal
mechanism that explains the increased risk of adverse perinatal outcomes
in women with short interpregnancy intervals.

Haaga (1988) proposed that inadequate time to regain muscle tone in
reproductive tissues after a pregnancy might lead to increased incidence
of cervical insufficiency (formerly called cervical incompetence) toward
the end of the next pregnancy, resulting in increased incidence of
preterm birth. Cervical insufficiency is described as the inability of
the uterine cervix to retain a pregnancy in the absence of contractions
or labor.

Strong evidence suggests that cervical insufficiency is a cause of
spontaneous preterm birth (Romero et al. 2006).

Adverse perinatal outcomes associated with maternal infections can occur
because of direct infections of the fetus or neonate, or because of
infections that cause early delivery without directly involving the
fetus. For organisms that attack the fetus directly, transmission may
occur within the uterus via transplacental or ascending infection, or in
the intrapartum period secondary to fetal contact with infected genital
secretions or maternal blood (vertical transmission).

emerging evidence supports the hypothesis that the association between
short intervals and adverse perinatal outcomes could also be mediated by
vertical transmission of infections.

Breastfeeding–pregnancy overlap is defined as the continuation of
breastfeeding into the first, second, or even third trimester of
pregnancy.

short intervals could indirectly increase the risk of adverse
neonatal/infant outcomes through changes in breastfeeding patterns or
the composition and/or quantity of breast milk secondary to
breastfeeding–pregnancy overlap.

Although we did not identify any studies that specifically evaluated
suboptimal lactation related to the breastfeeding–pregnancy overlap
hypothesis for explaining the association between short intervals and
neonatal/infant health, some evidence presented above supports this
hypothetical causal mechanism.

the relationship between short interpregnancy or birth intervals and
infant and child mortality may be explained by sibling competition,
which may interact with other proposed mechanisms such as the
transmission of infectious diseases among closely spaced siblings. If
two or more young children within a family are close in age, they may
compete for resources and for parental care and attention.

although the results are conflicting, they suggest that the effects of
preceding short birth intervals on neonatal and infant mortality seem to
be stronger when the preceding sibling dies than when she/he survives.
This finding suggests that neither competition nor transmission of
infectious diseases among siblings is the main mechanism by which short
birth intervals may affect neonatal and infant mortality. The effects of
preceding short birth intervals on post-neonatal mortality seem to be
stronger when the preceding sibling survives than when she/he dies,
which is consistent with both the sibling competition hypothesis and the
hypothesis of transmission of infectious diseases among siblings.
Evidence regarding the effects of preceding short birth intervals on
child and under-five-years mortality according to survival status of the
preceding sibling is inconclusive.

Interpregnancy intervals longer than five years are associated with an
increased risk of adverse perinatal outcomes, such as preterm birth, low
birth-weight, small for gestational age (Conde-Agudelo, Rosas-Bermudez
and Kafury-Goeta 2006), and preeclampsia (Conde-Agudelo, Rosas-Bermudez,
and Kafury-Goeta 2007).

According to this hypothesis, the mother’s physiological processes are
primed for fetal growth during pregnancy and decline gradually after
delivery. The benefit gained during pregnancy declines gradually
postpartum if the mother does not become pregnant again.

\citet{Buckles2012}

US analysis.

Using the NLSY79 and NLSY79 Child and Young Adult Surveys we investigate

the effect of the age difference between between siblings (spacing) on
educational achievement. Because spacing may be endogenous, we use an
instrumental variables strategy that exploits variation in spacing
driven
by miscarriages. The IV results indicate that a one-year increases in
spacing increases test scores for older siblings by about 0.17 standard
deviations. These results are larger than the OLS estimates, suggesting
that failing to account for the endogeneity of spacing may understate
its benefits. For younger siblings, we find no causal impact of spacing
on test scores.

The identification strategy exploits variation in spacing driven by 
miscarriages that occur between two live births; there are several
caveats to 
consider when using this instrument, which will be discussed in detail
in Section V. 
We show that a miscarriage between siblings is associated with an
increase in 
spacing of about eight months, and decreases the likelihood that the
siblings 
are less than two years apart by 19 percentage points.

Among economists, Rosenzweig (1986) develops a model of optimal child
spacing
in which spacing is an input into child quality. An important feature of
the model
is that the endowments of older children affect the optimal timing of
subsequent
births.
Rosenzweig, Mark R. 1986. "Birth Spacing and Sibling Inequality:
Asymmetric Information
Within the Household. "International Economic Review 27(l):55-76. 

Heckman and Walker (1990) consider the effects of female labor market
outcomes
on fertility timing and birth spacing and found that higher female wages
led to
delayed childbearing and greater spacing between children. Troske and
Voicu (2009)
show that women who delay the birth of a second child reduce their labor
force
participation by less than women with closely spaced children, but are
more likely
to work part-time.

Estimates of average time to conception for women who conceived within
one year of 
a miscarriage range from 17.35 weeks (Goldstein, Croughan, and Robertson
2002) to 23.2
weeks (Wyss, Biedermann, and Huch 1994).

\citet{Saha2013}

Matlab, Bangladesh

this study analyzes the causal effects of birth spacing on subsequent
infant mortality and of infant mortality on the use of contraceptives
and the length of the next birth interval.

Our main finding is that complete contraceptive use could reduce infant
mortality of birth order two and higher by 7.9 percent. The net effect
of complete contraceptive use on the total infant mortality rate is
small (2.9 percent), however, because the favorable effect on higher
order births is partly offset by the rise in the proportion of high-risk
first births.

we use the estimates to perform simulations that show how the use of
contraceptives is linked to the infant mortality rate in two ways: (1)
by reducing mortality among higher order births by lengthening birth
intervals, and (2) through a composition effect: reducing the number of
higher order births leads to a larger proportion of relatively
vulnerable first order births, thereby increasing the average mortality
rate for all birth orders.

keeping other factors constant (including the decision to use
contraceptives at any time after the previous birth), birth intervals
tend to be shorter for higher birth orders, which is consistent with the
association between short birth intervals and high fertility.

\citep{Jayachandran2011}

If son preference exists, girls will be weaned earlier than boys because
parents want to try again for a son. Given the large health benefits of
breastfeeding in environments with contaminated food and water (Feachem
and Koblinsky 1984; Palloni and Millman 1986; Habicht, DaVanzo, and Butz
1988), such behavior can lead to gender disparities in child health. In contrast
to the standard explanation of gender health disparities—that parents
actively invest more in sons’ health—we show that if parents merely
prefer to have sons, then mothers will breastfeed daughters less,
causing a gender gapin health even when parents
equally value the health of all their children. Our results highlight how
health disparities can arise “passively” from fertility preferences, and
relate to earlier work (Yamaguchi 1989; Clark 2000; Jensen 2003) showing
that a “try until you have a son” fertility rule results in girls, on
average, having more siblings and thus more competition for household
resources.

Model predictions:
First, breastfeeding duration increases with birth order, as demand for
breastfeeding’s contraceptive properties should rise as couples approach
or exceed their ideal family size, and in fact should increase
discontinuously once ideal family size is met.
Second, as explained, as long as some son preference exists, girls are
breastfed less than boys.
Third, by the same logic, children with older brothers are breastfed
more.
Fourth, the gender effect is largest when a couple approaches its target
family size since, at that point, their decision to have another child
is highly marginal and thus most sensitive to sex composition; the
effect is smallest for children whose birth order is either
significantly below or above ideal family size because couples will want
to, respectively, wean or nurse such children regardless of gender.

several papers show empirically that in settings with son preference, a
couple that just had a son is more likely to stop having children (Das
1987) or wait longer to have the next child (Rindfuss et al. 1982;
Trussell et al. 1985; Arnold, Choe, and Roy 1998; Retherford and Roy
2003). Hemochandra, Singh, and Singh (2010) show that birth intervals
are shorter if parents have not reached their self-reported ideal number
of sons. Second, Nemeth and Bowling (1985) examine whether having no
sons
shortens the duration of breastfeeding.

Muhuri and Menken (1997) show that birth spacing and the sex of older
siblings affect mortality and state that shorter birth intervals after
girls are one potential cause of excess female mortality. In unreported
results, Arnold, Choe, and Roy (1998) examine whether the gender gap in
mortality in India is due to birth spacing; they donot findevidence
tosupport the hypothesis. Finally, Mutharayappa et al. (1997) show that
there is a gender gap in breastfeeding in India and conjecture that it
is related to son-biased fertility patterns.

((Bhatia 1978; Basu 1992) on lower son preference in the south)

Retherford et al. (1989) find that controlling for breastfeeding largely
eliminates the negative correlation between infant mortality and
subsequent birth spacing in Nepal.

\citep{Whitworth2002}

This paper examines the impact of the length of the preceding birth
interval on under-two mortality in India, and examines the pathways
through which short preceding birth intervals may lead to an increased
risk of mortality

1992 NFHS

Three mortality periods are examined: neonatal, early post neonatal and
late post-neonatal and toddler 

short preceding birth intervals (< 18 months) are associated with an
increased risk of mortality in all three age groups, and the effect is
particularly marked in the early post-neonatal period.

diluting effect that a higher level of maternal education has on the
relationship between short preceding birth intervals and mortality risk

There is evidence to suggest that sibling rivalry is a pathway through
which short birth intervals influence mortality, with the death of the
previous sibling removing the competition for scarce resources, and
resulting in lower risks of mortality than if the previous sibling was
still alive.


\citep{Bhalotra2008}

NFHS-2 1998-99

We use data for Uttar Pradesh (UP), the largest Indian state, which, in
the year 2000, contained 17.1\% of the country’s population
(approximately 165 million people). It has social and demographic
indicators that put it well below the Indian average (Dreze and Sen,
1997).

Birth intervals explain only a limited share of the correlation between
neonatal mortality of successive children in a family.

The main contribution of this paper is to use panel data based on
retrospective fertility histories to estimate causal effects of birth
interval length on subsequent neonatal mortality risk and of neonatal
mortality on subsequent birth interval length, controlling for
unobserved heterogeneity in both processes (referred to as frailty and
fecundity, respectively)


The gender of the last-born child is significant and its sign is
consistent with son-preference. If the last birth was a girl, the
expected birth interval is about 3\% shorter than if it was a boy.

There is a significant humpshaped trend in birth-spacing, with a maximum
in 1980. That birth intervals got shorter in the last 20 years might be
because rising nutritional standards for mothers allow them to support
shorter intervals. Alternatively, the rise in women’s labour force
participation may have encouraged clustering of fertility.

birth intervals increase until the sixth child is born.

Other things equal, birth-order exhibits a non-monotonic pattern, with
the shortest birth intervals preceding the birth of the fourth child.

Maternal education decreases both mortality and fertility, but has no
effect on birth-spacing.

Fertility decline in India seems to have set in from 1981. Despite this,
birth intervals have got shorter since about then.

\citep{Zavier2000}

NFHS 92-93 Kerala

Many couples in India never use a reversible method to delay or space
births, and in-
stead adopt sterilization as their first and only method.

higher educational attainment (of either partner) independently
increases the likelihood 
that a couple will have used a method to delay or space births, as does
middle 
socioeconomic status. 

\citep{Maitra2008}

We find that the estimates of birth spacing on child mortality are
different when we do not account for fertility selection.

Our analysis is based on the National Family Health Survey (NFHS)
1992–1993 data from the Indian province of the Punjab and the
Demographic and Health Survey (DHS) 1991–1992 data from the Pakistani
province of the Punjab.

India: There is evidence of a statistically signifiant effect of birth
spacing on child survival: an increase in the duration between child i
and child i + 1 significantly reduces the hazard of mortality of child i
(equivalently increases the survival chances of the child).
for the sample of Pakistani households. As with the Indian case, there
is evidence of maternal depletion/sibling competition effect in the
Pakistani sample: an increase in the duration between child i and child
i + 1 significantly reduces the hazard of mortality of child i.

early child death may also result in a reduction in the duration between
successive children because parents want to replace children that have
died.
Our results show that the child replacement effect is significant in the
Indian sample, but not in the Pakistani sample. In other words,
increased child survival increases birth spacing in India but not in
Pakistan. 

\citep{Makepeace2008}

NFHS 1992-1993 West Bengal

Our central result is that child mortality decreases with increases in
either prior or posterior spacing (although the size of the fall is
different).

There is evidence that uncorrected estimates under-estimate the effects
of prior and posterior spacing on child mortality.

the spacing between consecutive children, i.e. the age composition of
siblings, could capture the intensity of competition between successive
siblings better,

\citep{Bhargava2003}

NFHS 1992-93: Uttar Pradesh

indicator variables constructed on the basis of the stated preferences
for sons and daughters indicated that unwanted fertility contributes to
excess mortality of higher order births, especially of girls.

\citep{Davanzo2008}

Matlab

Shorter intervals are associated with higher mortality. Interval effects
are greater if the
interval began with a live birth than with another pregnancy outcome.

previous studies have generally considered the preceding interval
between births* the interbirth interval*as their measure of spacing

the appropriate measure to use when considering the effect of
competition or disease transmission from another young child in the
family

sometimes there is a pregnancy that resulted in an outcome other than a
live birth (NLB) between two live births, in which case the interbirth
interval will include two (or more) interpregnancy intervals.

Some of the hypotheses about why reproductive spacing may affect infant
and child health and survival are related to the interpregnancy interval

Furthermore, an intervening NLB may reflect something about the mother’s
health that may affect the health of her children

Some, but very little, of the effect is explained by the fact that
shorter interbirth and inter-outcome intervals are associated with
shorter gestations of pregnancy.

Our results give some credence to the maternal depletion hypothesis.
We see that very short inter-outcome intervals are generally more
detrimental when they follow a live birth or stillbirth than when they
follow a preceding miscarriage or induced abortion. Because of their
longer gestation, live births and stillbirths should be more depleting
than miscarriages or induced abortions.
The breastfeeding that follows a live birth leads to further maternal
depletion. Indeed we find that the effects of short inter-outcome
intervals are greatest when the preceding outcome was a live birth.

We also find significant negative effects of subsequent short
interpregnancy intervals on child survival (and have investigated this
in a way that avoids reverse causality): a child is much more likely to
die between ages 1 and 5 if the mother became pregnant again before this
index child’s first birthday (RR02.31, pB0.001).

\citep{Haughton1995}

Vietnam Living Standards Survey (VLSS) 1992-1993

This article assesses the strength of son preference in Vietnam, as
reflected in fertility behavior. It
formulates and estimates a proportional hazards model applied to birth
intervals, and a contraceptive
prevalence model, using household survey data from 2,636 ever-married
women aged 15-49 with at
least one living child who were interviewedfor the Vietnam Living
Standards Survey 1992-1993.

 Two tests are applied to the data in an attempt to discern the presence
of son preference. The hazard
model test estimates the risk (hazard) of having another
child at any point in time; if the hazard is lower for families with a
son (or sons),
the implication is that son preference is present. The contraceptive
prevalence test
determines whether families with a son (or sons) are significantly more
likely to use
contraceptives; if the answer is yer, a presumption may be made that son
preference is present.

The central finding of this study is that son preference,
as reflected in household behavior, is strong in Vietnam.
This finding is surprising in the sense that Vietnamese
women are not secluded but are major participants in
the labor force and are relatively well educated.

\citep{Milazzo2018}

This paper is the first to show that morbidity and mortality among adult
women in India can be partially explained by son preference. First, I
show suggestive evidence that women with a first-born girl have lower
survival. Second, consistent with prior literature, I find that having a
first-born girl leads to fertility behaviors medically known to hurt
women’s survival. Third, I show new evidence of effects of a first-born
girl on a mother’s likelihood and severity of anemia. These outcomes are
severely aggravated with each successive female birth

I pool the three available rounds of the India National Family Health
Survey (NFHS) conducted in 1992/93, 1998/99, 2005/06, and consider the
cross-sectional sample of women ages 15 to 39.

I show nonparametric evidence that the share of living women whose first
birth is female is a decreasing function of their age (or,
alternatively, of the number of years since first birth). This decreasing
pattern starts around age 24 and continues to decrease below the
biological range, stabilizing for women in their mid-30s

Although indirect, this evidence is suggestive of lower survival among
women with first girls due, at least partly, to son preference.

I estimate the effects of having a first-born girl on several
fertility-related outcomes, including the probability of desiring more
children, being sterilized, having undergone ultrasound, having had a
terminated pregnancy, short birth spacing, and the number of children
ever born.

I provide original evidence that women with first-born girls are more
likely to develop anemia. Specifically, a first-born daughter increases
the probability of being moderately or severely anemic by 5.4 percent.
The additional effect of a second and third-born girl (conditional on
previous female births) is equivalent to 9 and 17.6 percent,
respectively. Predetermined health or current pregnancy or breastfeeding
status do not appear to drive these results.

\citep{Gangadharan2003}

We use an accelerated hazard model to estimate the duration between
successive births and our results indicate that son preference exists
only for the Indian community in South Africa.

Indian households are observed to have a higher duration between
children following the birth of a son, irrespective of the number of
children they already have. For the rest of the population, there is
very little evidence of son preference.

For the Indian households, regardless of the number of existing sons,
the duration is higher after a son. Additionally, an increase in the
number of sons also significantly increases the duration between the
second and the third children.

\citep{Kim2010}

Indonesia

among earlier cohorts, women who are more educated tend to have shorter
birth intervals than those who are less educated and that the opposite
is true among later cohorts.

The empirical results based on the first hypothesis show that the level
of industrialization of the local economy does not have a differential
effect on birth spacing across educational groups. However, empirical
results using measures of the availability of a family planning program
show a differential effect on delaying births across educational groups.

women’s education had an impact on the second birth interval mainly
through changing the availability of contraceptives rather than through
change in the demand for children over the period 1974–90.

My findings are consistent with the hypothesis that more educated women
are better at adopting new contraceptive technology.

\citep{Teachman1989}

USA

Our findings suggest
that a preference to balance the gender of children affects the timing
of births, not a
preference for either sons or daughters. At parity 2, women with
children of the same
sex time a third birth more rapidly than women with a boy and a girl. At
parity 1,
women with a boy time second births more rapidly than women with a girl.
This
seemingly anomalous finding is explained, however, by the fact that
women with boys
are more likely than women with girls to be married at any point in time
and thus
less likely to have disrupted fertility careers.

\citep{DiamondSmith2008}

Tamil Nadu, India

The southern Indian state of Tamil Nadu has experienced a dramatic
decline in fertility, accompanied by a trend of increased son
preference.

Findings suggest that daughter aversion, fuelled primarily by the
perceived economic burden of daughters due to the proliferation of
dowry, is playing a larger role in fertility decision-making than son
preference. The desire for a son is often trumped by the worry over
having many daughters. Women use various means of controlling the sex of
their children, which in this study appear to be primarily female
infanticide. It is important to distinguish between son preference and
daughter aversion and to examine repercussions of low fertility within
this setting.

In the last decade, Tamil Nadu – along with its neighbouring southern
states – has achieved very low fertility. In the year 2005–2006, the
Total Fertility Rate (TFR, the average number of children born to a
woman over her lifetime) was estimated at 1.80, down from a TFR of 4.97
in 1974 (Dyson and Moore 1983, Guilmoto and Rajan 2001, International
Institute for Population Sciences and ORC Macro 2006).

This impressive decline in fertility has been accompanied by worrisome
evidence of the emergence of son preference in the past decade, which is
reflected in increasingly disproportional sex ratios at birth and
younger ages (Das Gupta and Bhat 1997, Basu 1999, Srinivasan 2005).

If strong daughter aversion is a prime driver of fertility limitation,
it may put downward pressure on population growth, possibly helping to
explain the recent trend toward very low fertility rates in Tamil Nadu.
Although the phenomenon of daughter aversion has received less attention
than son preference, there are a few past studies on it, primarily in
Northern India, some which suggest that it may differ by religious group
(Khanna 1997, Borooah and Iyer 2004). Our work here supports the idea
that daughter avoidance, often described inaccurately as son preference,
is increasing in rural Tamil Nadu and suggests that economic
constraints, primarily due to increased use of dowry, are the cause.

\citep{Kishor1993}

India

Gender differentials in early childhood
mortality are best modeled by simultaneously
considering kinship structures and female labor force participation.

the analysis suggests that cultural factors may contribute more to
explaining gender
differences in mortality across India than does female labor force
participation.

this analysis unequivocally suggests that excess female mortality in
India has both 
economic and cultural roots and any explanation must consider both
economic and cultural 
factors.

\citep{Curtis1993}

Brazil

In this paper random-effects logistic models are used to analyze the
effects of the
preceding birth interval on postneonatal mortality in Brazil,
controlling for the
correlation of survival outcomes between siblings. The results are
compared to those
obtained by using ordinary logistic regression. Family effects are found
to be highly
significant in the random-effects model, but the substantive conclusions
of the
ordinary logistic model are preserved. In particular, birth interval
effects remain
highly significant

Birth interval effects are highly significant in Brazil, particularly in
the less developed
Northeast region. This point is important because short birth intervals
are relatively common
in Brazil, particularly in the Northeast.


\citep{Rahman1993}

Matlab, Bangladesh

We hypothesize that mothers in Matlab have a strong preference about the
gender composition of their surviving children; in particular, they want
several surviving sons and at least one surviving daughter.

the basic premise of this paper is that gender preference will have a
greater influence on fertility where contraception is readily available
and widely used; we expect to see little or no effect of gender
preference on fertility in a population that does not use contraception.

In the earlier stages of family formation (i.e., in families with three
or fewer surviving children), the number of surviving sons is a
significant (negative) predictor of subsequent fertility. In the later
stages (i.e., in families with four or more surviving children), the
subsequent pace of reproduction is affected significantly not only by
the number of sons but also by whether there is a son and by whether
there is a daughter in the family. Our results show not only a very
strong desire for sons, which is seen early in the family formation
process, but also a desire for at least one daughter, which is seen at a
more advanced stage.

The effect of gender composition of children is significantly stronger
in the treatment area than in the comparison area. Moreover, whether the
family has children of both genders seems to be more important in the
treatment area than in the comparison area.

These differentials in the impact of gender preference on fertility as a
result of the availability and use of contraception imply strongly that
gender preference effects become more important when contraception is
readily availableand when a population is more able to control its
fertility, other things being equal.

\citep{Rafalitnanana2000} 

Sub-Saharan Africa 20 countries

Examines preferred spacing as compared to observed spacing. In 5 of 20
countries the
actual spacing is significantly shorter than the preferred spacing.


\citep{Lambert2016}

Exploiting original data from a Senegalese household survey, we provide
evidence that fertility choices are partly driven by women's needs for
widowhood insurance. We use a duration model of birth intervals to show
that women most at risk in case of widowhood intensify their fertility,
shortening birth spacing, until they get a son. Insurance through sons
might entail substantial health costs since short birth spacing raises
maternal and infant mortality rates.

\citep{Jayachandran2017}

This paper’s contribution is to directly estimate the causal
relationship between family size and the desired sex ratio and to
quantify an important cause of rising sex ratios. The challenge in
estimating this effect is to isolate exogenous variation in the
fertility level. The approach I use is to elicit sex composition
preferences at different fertility levels: a hypothetical fertility
level is specified to the survey respondent, and she is asked, given
that fertility level, what is her preferred composition of boys and
girls. By imposing the total number of children, one can characterize
the respondents’ sex ratio preferences at different exogenously
determined fertility levels.


\citep{Basu2010}

This article draws out some implications of son targeting fertility
behavior and studies its determinants. We demonstrate that such behavior
has two notable implications at the aggregate level: (a) girls have a
larger number of siblings (sibling effect), and (b) girls are born at
relatively earlier parities within families (birth-order effect).
Empirically testing for these effects, we find that both are present in
many countries in South Asia, Southeast Asia, and North Africa but are
absent in the countries of sub-Saharan Africa. Using maximum likelihood
estimation, we study the effect of covariates on son targeting fertility
behavior in India, a country that displays significant sibling and
birth-order effects. We find that income and geographic location of
families significantly affect son targeting behavior.

\citep{Kugler2017}

Using data from nationally representative household surveys, we test
whether Indian parents make trade-offs between the number of children
and investments in education. To address the endogeneity due to the
joint determination of quantity and quality of children, we instrument
family size with the gender of the first child, which is plausibly
random. Given a strong son preference in India, parents tend to have
more children if the firstborn is a girl. Our instrumental variable
results show that children from larger families have lower educational
attainment and are less likely to be enrolled in school, with larger
effects for rural, poorer, and low-caste families as well as for
families with illiterate mothers.

\citep{Hotz1997}

This hazard approach provides a natural way to model incomplete
histories, nonoccurrence of the event (a subsequent birth) and a natural
set of covariates (current period values). It, however, does not solve
several other problems. First, it provides
little insight into how to summarize the information in past and future
covariates. Second, it does not solve the dynamic selection problem. To
be in the sample of individuals on which we estimate the time between
the first and second birth, one must have had a first birth. This is a
selected sample. This simple hazard model provides no insight into the
effects of that selection. Finally, this model has the unfortunate
characteristic of mixing the parameters for the speed with which the
event occurs with the parameters for whether or not the event occurs.


