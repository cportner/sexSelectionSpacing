
To read:

In birth_spacing:


In collection.bib
\citep{Rafalitnanana2000} 
\citep{Rahman1993}
\citep{Jayachandran2017}
\citep{Jayachandran2017a}
\citep{Lambert2016}


In collection.bib but maybe not that relevant
\citep{Basu2010} 
\citep{dasgupta95} 
\citep{Kugler2017}



\citet{Dewey2007}

Review study

The three conceptual models demonstrate that the relationship between
interpregnancy interval and maternal or child nutritional status is by
no means simple.

The studies on child nutrition outcomes indicate that a longer birth
interval is associated with a lower risk of malnutrition in some
populations, but not all.

The studies on maternal anthropometric outcomes yielded mixed results.
Because the nutritional burden on the mother between pregnancies depends
on the extent of breastfeeding, the interpregnancy interval is not the
best measure of whether the mother has had a chance to recover from the
pregnancy, in terms of repleting her nutritional status. Therefore, some
studies examined the ‘recuperative interval’ (duration of the
non-pregnant, non-lactating interval) instead.

Lactation represents a greater nutritional burden than pregnancy,

[link with paper on shorter breastfeeding for girls than for boys]

Important methodological limitations in papers used.

Mechanisms: 
- new pregnancy can lead to reduction is breastfeeding (reduced volume of breastmilk), which
can be hazardous both nutritionally and in terms of resistance to infection.
(lower breastfeeding -> less contraceptive effect -> shorter birth intervals)

- care of new infant affects prior child; might even happen before birth of new child

\citet{Gribble2009}

El Salvadore NFHS 2002-2003

Our objectives are to explore how short birth intervals may increase the
risk of stunting and underweight, and to identify policy and
programmatic efforts that can contribute to reducing the risk of
undernutrition, including healthy timing and spacing of pregnancies.

Nevertheless, the findings do suggest that birth intervals of less than 3
years are either directly related to stunting or act as a proxy for a
risk factor that is either unobserved or not included in the analysis.

(no attempt to estimate what determines birth spacing, although do acknowledge that
causality is difficult to establish)

\citet{Soest2018}

The main novelty compared to [14] is that our application concerns
infant mortality in rural Bangladesh. While Bhalotra and van Soest used
retrospective data to analyse neonatal mortality in India, we use
prospective data from the Demographic and Health Surveillance System,
Matlab, following mothers residing in the study area over time.


Many studies also found an association between a short birth interval
and infant death of the next child, particularly when the preceding
sibling survived [21–23]. An explanation for this is that the mother has
not recuperated physiologically from the previous birth [24–25].

(!this is an important point. In the literature, the mortality effect discussed
generally comes from the subsequent child, not the prior one. 
Hence, longer spacing should be beneficial for the boy born next if 
sex selection has been used.
This could explain part of the reduction in female mortality: The
girls that would "normally" have been born relatively soon after the
prior birth, and, therefore, have higher mortality risk, now are no
longer born.
)

Sibling competition may also explain why short birth intervals and high
fertility increase death risk: sources of food and care per head
diminish as the number of dependent members of a family increases [5].

(this effect can be used to explain any reductions in mortality for
the prior girls.

A type of this would be cession of breastfeeding because of a short
birth interval - see Conde et al. paper below for discussion of 
breastfeeding. 
)

We assume gender of each birth is exogenous and thus do not incorporate
the possibility of selective abortion. Descriptive statistics confirm
that this is not an issue in these data, revealing no relation at all
between gender of a newborn child and gender composition of the older
children in the family.


keeping other covariates and unobserved mother specific factors
constant, if the previous child survived its infancy, the mortality
probability falls with the length of the birth interval as long as the
birth interval is less than 63.3 months

If the previous child died, mortality falls with birth interval length
for intervals up to 52.5 months, still much beyond the median.

In the comparison area, death at infancy of the previous child shortens
the subsequent birth interval by about 46\%.

Having at least one boy has a stronger positive effect on the birth
interval than having a girl. The same applies to each additional boy.

Birth intervals increase with the mother’s education level, in line with
the positive relation between birth intervals and socioeconomic status.
This is in line with the finding that the use of contraception is more
common among higher socioeconomic status groups [38].

A large negative correlation is observed between unobserved
heterogeneity in birth interval and fertility equations, confirming that
both equations are strongly related: mothers who desire many children
also tend to use shorter birth intervals. This is consistent with the
target fertility model of Wolpin [4] and in line with the findings in
[14].

\citet{Zimmermann2018}

Son preference is widespread in a number of developing countries despite
substantial improvements in education levels and economic development.
One potential explanation for the persistence of this phenomenon is that
individual household members like the mother derive large non-monetary
benefits from giving birth to a son and therefore prefer boys to girls.

I focus on first-born children for whom the sex ratio of girls relative to boys is normal.

The results show little evidence of consistently large female benefits
shortly after birth, and any positive impacts of having a son disappear
after the first six months. There are also no large benefits for adult
sons. These empirical patterns do not support qualitative evidence
suggesting that women benefit from the birth of a son through larger
decision-making powers in the household because of increased respect by
other household members.

(Li and Wu (2011) is the paper on China that found positive effects on
bargaining power of a first born son)

(does not seem directly relevant for this paper.)

\citet{Conde-Agudelo2012}

The following hypothetical causal mechanisms for explaining the
association between short intervals and adverse outcomes were
identified: maternal nutritional depletion, folate depletion, cervical
insufficiency, vertical transmission of infections, suboptimal lactation
related to breastfeeding–pregnancy overlap, sibling competition,
transmission of infectious diseases among siblings, incomplete healing
of uterine scar from previous cesarean delivery, and abnormal remodeling
of endometrial blood vessels.

Interpregnancy intervals shorter than 18 months and longer than 59
months are significantly associated with increased risk of adverse
perinatal outcomes such as preterm birth, low birth-weight, and small
for gestational age (Conde-Agudelo, Rosas-Bermudez, and Kafury-Goeta
2006).

Preceding interpregnancy intervals shorter than 36 months are
significantly associated with a greater risk of child and
under-five-years mortality, and intervals shorter than 24 months
significantly increase risk of early neonatal, neonatal, and infant
mortality (Rutstein 2008).

Hypotheses generally adopt either biological or behavioral orientations,
but no one framework or hypothesis has emerged as dominant (Erickson and
Bjerkedal 1979; Winikoff 1983; Klebanoff 1999).

The current maternal nutritional depletion hypothesis states that a
close succession of pregnancies and of periods of lactation worsens the
mother’s nutritional status because of inadequate time to recover from
the physiological stresses of the preceding pregnancy before becoming
subject to the stresses of the next pregnancy (Winkvist, Rasmussen, and
Habicht 1992; King 2003).

In summary, the studies that evaluated the effects of birth spacing on
maternal anthropometric status, anemia, and micronutrient status did not
provide clear evidence to support the maternal nutritional depletion
hypothesis.

(if parents know that/believe that there are substantially negative 
effects of short spacing, why would they engage in it. It would put
they subsequent son at an disadvantage. Unless the compensatory effects
are so large that it balances out the negative effects.)

if a new pregnancy starts before complete folate restoration, the woman
will be at higher risk of maternal folate deficiency and subsequent
adverse perinatal outcomes such as low birthweight, preterm birth, and
small for gestational age.

Strong evidence exists that folate depletion occurs in women during the
first three to four months postpartum, and growing evidence supports the
hypothesis that this depletion constitutes a hypothetical causal
mechanism that explains the increased risk of adverse perinatal outcomes
in women with short interpregnancy intervals.

Haaga (1988) proposed that inadequate time to regain muscle tone in
reproductive tissues after a pregnancy might lead to increased incidence
of cervical insufficiency (formerly called cervical incompetence) toward
the end of the next pregnancy, resulting in increased incidence of
preterm birth. Cervical insufficiency is described as the inability of
the uterine cervix to retain a pregnancy in the absence of contractions
or labor.

Strong evidence suggests that cervical insufficiency is a cause of
spontaneous preterm birth (Romero et al. 2006).

Adverse perinatal outcomes associated with maternal infections can occur
because of direct infections of the fetus or neonate, or because of
infections that cause early delivery without directly involving the
fetus. For organisms that attack the fetus directly, transmission may
occur within the uterus via transplacental or ascending infection, or in
the intrapartum period secondary to fetal contact with infected genital
secretions or maternal blood (vertical transmission).

emerging evidence supports the hypothesis that the association between
short intervals and adverse perinatal outcomes could also be mediated by
vertical transmission of infections.

Breastfeeding–pregnancy overlap is defined as the continuation of
breastfeeding into the first, second, or even third trimester of
pregnancy.

short intervals could indirectly increase the risk of adverse
neonatal/infant outcomes through changes in breastfeeding patterns or
the composition and/or quantity of breast milk secondary to
breastfeeding–pregnancy overlap.

Although we did not identify any studies that specifically evaluated
suboptimal lactation related to the breastfeeding–pregnancy overlap
hypothesis for explaining the association between short intervals and
neonatal/infant health, some evidence presented above supports this
hypothetical causal mechanism.

the relationship between short interpregnancy or birth intervals and
infant and child mortality may be explained by sibling competition,
which may interact with other proposed mechanisms such as the
transmission of infectious diseases among closely spaced siblings. If
two or more young children within a family are close in age, they may
compete for resources and for parental care and attention.

although the results are conflicting, they suggest that the effects of
preceding short birth intervals on neonatal and infant mortality seem to
be stronger when the preceding sibling dies than when she/he survives.
This finding suggests that neither competition nor transmission of
infectious diseases among siblings is the main mechanism by which short
birth intervals may affect neonatal and infant mortality. The effects of
preceding short birth intervals on post-neonatal mortality seem to be
stronger when the preceding sibling survives than when she/he dies,
which is consistent with both the sibling competition hypothesis and the
hypothesis of transmission of infectious diseases among siblings.
Evidence regarding the effects of preceding short birth intervals on
child and under-five-years mortality according to survival status of the
preceding sibling is inconclusive.

Interpregnancy intervals longer than five years are associated with an
increased risk of adverse perinatal outcomes, such as preterm birth, low
birth-weight, small for gestational age (Conde-Agudelo, Rosas-Bermudez
and Kafury-Goeta 2006), and preeclampsia (Conde-Agudelo, Rosas-Bermudez,
and Kafury-Goeta 2007).

According to this hypothesis, the mother’s physiological processes are
primed for fetal growth during pregnancy and decline gradually after
delivery. The benefit gained during pregnancy declines gradually
postpartum if the mother does not become pregnant again.

\citet{Buckles2012}

US analysis.

Using the NLSY79 and NLSY79 Child and Young Adult Surveys we investigate 
the effect of the age difference between between siblings (spacing) on
educational achievement. Because spacing may be endogenous, we use an
instrumental variables strategy that exploits variation in spacing driven
by miscarriages. The IV results indicate that a one-year increases in
spacing increases test scores for older siblings by about 0.17 standard
deviations. These results are larger than the OLS estimates, suggesting
that failing to account for the endogeneity of spacing may understate
its benefits. For younger siblings, we find no causal impact of spacing
on test scores.

The identification strategy exploits variation in spacing driven by 
miscarriages that occur between two live births; there are several caveats to 
consider when using this instrument, which will be discussed in detail in Section V. 
We show that a miscarriage between siblings is associated with an increase in 
spacing of about eight months, and decreases the likelihood that the siblings 
are less than two years apart by 19 percentage points.

Among economists, Rosenzweig (1986) develops a model of optimal child spacing
in which spacing is an input into child quality. An important feature of the model
is that the endowments of older children affect the optimal timing of subsequent
births.
Rosenzweig, Mark R. 1986. "Birth Spacing and Sibling Inequality: Asymmetric Information
Within the Household. "International Economic Review 27(l):55-76. 

Heckman and Walker (1990) consider the effects of female labor market outcomes
on fertility timing and birth spacing and found that higher female wages led to
delayed childbearing and greater spacing between children. Troske and Voicu (2009)
show that women who delay the birth of a second child reduce their labor force
participation by less than women with closely spaced children, but are more likely
to work part-time.

Estimates of average time to conception for women who conceived within one year of 
a miscarriage range from 17.35 weeks (Goldstein, Croughan, and Robertson 2002) to 23.2
weeks (Wyss, Biedermann, and Huch 1994).

\citet{Saha2013}

Matlab, Bangladesh

this study analyzes the causal effects of birth spacing on subsequent
infant mortality and of infant mortality on the use of contraceptives
and the length of the next birth interval.

Our main finding is that complete contraceptive use could reduce infant
mortality of birth order two and higher by 7.9 percent. The net effect
of complete contraceptive use on the total infant mortality rate is
small (2.9 percent), however, because the favorable effect on higher
order births is partly offset by the rise in the proportion of high-risk
first births.

we use the estimates to perform simulations that show how the use of
contraceptives is linked to the infant mortality rate in two ways: (1)
by reducing mortality among higher order births by lengthening birth
intervals, and (2) through a composition effect: reducing the number of
higher order births leads to a larger proportion of relatively
vulnerable first order births, thereby increasing the average mortality
rate for all birth orders.

keeping other factors constant (including the decision to use
contraceptives at any time after the previous birth), birth intervals
tend to be shorter for higher birth orders, which is consistent with the
association between short birth intervals and high fertility.

\citep{Jayachandran2011}

If son preference exists, girls will be weaned earlier than boys because
parents want to try again for a son. Given the large health benefits of
breastfeeding in environments with contaminated food and water (Feachem
and Koblinsky 1984; Palloni and Millman 1986; Habicht, DaVanzo, and Butz
1988), such behaviorcanleadtogenderdisparities inchildhealth. Incontrast
to the standard explanation of gender health disparities—that parents
actively invest more in sons’ health—we show that if parents merely
prefer to have sons, then mothers will breastfeed daughters less,
causing a gender gapin health even when parents
equallyvaluethehealthofall theirchildren. Ourresults highlight how
health disparities can arise “passively” from fertility preferences, and
relate to earlier work (Yamaguchi 1989; Clark 2000; Jensen 2003) showing
that a “try until you have a son” fertility rule results in girls, on
average, having more siblings and thus more competition for household
resources.

Model predictions:
First, breastfeeding duration increases with birth order, as demand for
breastfeeding’s contraceptive properties should rise as couples approach
or exceed their ideal family size, andin fact shouldincrease
discontinuously once ideal family size is met.
Second, as explained, as long as some son preference exists, girls are
breastfed less than boys.
Third, by the same logic, children with older brothers are breastfed more.
Fourth, the gender effect is largest when a couple approaches its target
family size since, at that point, their decision to have another child
is highly marginal and thus most sensitive to sex composition; the
effect is smallest for children whose birth order is either
significantly below or above ideal family size because couples will want
to, respectively, wean or nurse such children regardless of gender.

several papers show empirically that in settings with son preference, a
couple that just had a son is more likely to stop having children (Das
1987) or wait longer to have the next child (Rindfuss et al. 1982;
Trussell et al. 1985; Arnold, Choe, and Roy 1998; RetherfordandRoy
2003). Hemochandra, Singh, and Singh (2010) show that birth intervals
are shorter if parents have not reached their self-reported ideal number
of sons. Second, Nemeth and Bowling (1985) examine whether having nosons
shortens the duration of breastfeeding.

Muhuri and Menken (1997) show that birth spacing and the sex of older
siblings affect mortality and state that shorter birth intervals after
girls are one potential cause of excess female mortality. In unreported
results, Arnold, Choe, and Roy (1998) examine whether the gender gap in
mortality in India is due to birth spacing; they donot findevidence
tosupport the hypothesis. Finally, Mutharayappa et al. (1997) show that
there is a gender gap in breastfeeding in India and conjecture that it
is related to son-biased fertility patterns.

((Bhatia 1978; Basu 1992) on lower son preference in the south)

Retherford et al. (1989) find that controlling for breastfeeding largely
eliminates the negative correlation between infant mortality and
subsequent birth spacing in Nepal.

\citep{Whitworth2002}

This paper examines the impact of the length of the preceding birth
interval on under-two mortality in India, and examines the pathways
through which short preceding birth intervals may lead to an increased
risk of mortality

1992 NFHS

Three mortality periods are examined: neonatal, early post neonatal and
late post-neonatal and toddler 

short preceding birth intervals (< 18 months) are associated with an
increased risk of mortality in all three age groups, and the effect is
particularly marked in the early post-neonatal period.

diluting effect that a higher level of maternal education has on the
relationship between short preceding birth intervals and mortality risk

There is evidence to suggest that sibling rivalry is a pathway through
which short birth intervals influence mortality, with the death of the
previous sibling removing the competition for scarce resources, and
resulting in lower risks of mortality than if the previous sibling was
still alive.


\citep{Bhalotra2008}

NFHS-2 1998-99

We use data for Uttar Pradesh (UP), the largest Indian state, which, in
the year 2000, contained 17.1\% of the country’s population
(approximately 165 million people). It has social and demographic
indicators that put it well below the Indian average (Dreze and Sen,
1997).

Birth intervals explain only a limited share of the correlation between
neonatal mortality of successive children in a family.

The main contribution of this paper is to use panel data based on
retrospective fertility histories to estimate causal effects of birth
interval length on subsequent neonatal mortality risk and of neonatal
mortality on subsequent birth interval length, controlling for
unobserved heterogeneity in both processes (referred to as frailty and
fecundity, respectively)
