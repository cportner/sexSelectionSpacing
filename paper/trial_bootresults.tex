% Revamped version for Demography focusing on method

\documentclass[12pt,letterpaper]{article}

\usepackage{amsmath}
\usepackage{fontspec}
\setromanfont[Ligatures=TeX]{TeX Gyre Pagella}
\usepackage{unicode-math}
\setmathfont{TeX Gyre Pagella Math}
\usepackage[title]{appendix}
\usepackage[margin=1.0in]{geometry}
\usepackage[figuresleft]{rotating}
\usepackage[longnamesfirst]{natbib}
\usepackage{dcolumn}
\usepackage{booktabs}
\usepackage{multirow}
\usepackage[flushleft]{threeparttable}
\usepackage{setspace}
\usepackage[justification=centering]{caption}
\usepackage[font=scriptsize]{subfig}
\usepackage[xetex,colorlinks=true,linkcolor=black,citecolor=black,urlcolor=black]{hyperref}
\usepackage{adjustbox}
\usepackage{xfrac}


% \bibpunct{(}{)}{;}{a}{}{,}
\newcommand{\mco}[1]{\multicolumn{1}{c}{#1}}
\newcommand{\mct}[1]{\multicolumn{2}{c}{#1}}
\newcommand{\X}{$\times$ }
\newcommand{\hs}{\hspace{15pt}}

% Attempt to squeeze more floats in
\renewcommand\floatpagefraction{.9}
\renewcommand\topfraction{.9}
\renewcommand\bottomfraction{.9}
\renewcommand\textfraction{.1}
\setcounter{totalnumber}{50}
\setcounter{topnumber}{50}
\setcounter{bottomnumber}{50}


%------------------------------------------------------------------------


\title{Birth Spacing in the Presence of Son Preference and Sex-Selective Abortions:
India's Experience over Four Decades%
\protect\thanks{%
I am grateful to Andrew Foster and Darryl Holman for discussions about the method.
I owe thanks to Shelly Lundberg, Daniel Rees, David Ribar, 
Hendrik Wolff, seminar participants at University of Copenhagen, University of Michigan, 
University of Washington, University of Aarhus, the Fourth 
Annual Conference on Population, Reproductive Health, 
and Economic Development, and the Economic Demography Workshop for helpful 
suggestions and comments.
I would also like to thank Nalina Varanasi for research assistance.
Support for development of the method from the University of Washington Royalty 
Research Fund and the Development Research Group of the World Bank is gratefully 
acknowledged.
The views and findings expressed here are those of the author and
should not be attributed to the World Bank or any of its member countries.
Partial support for this research came from a Eunice Kennedy Shriver National
Institute of Child Health and Human Development research infrastructure grant,
5R24HD042828, to the Center for Studies in Demography and Ecology at the
University of Washington.
}
}

\author{Claus C P\"ortner\\
    Department of Economics\\
    Albers School of Business and Economics\\
    Seattle University, P.O. Box 222000\\
    Seattle, WA 98122\\
    \href{mailto:claus@clausportner.com}{\texttt{claus@clausportner.com}}\\
    \href{http://www.clausportner.com}{\texttt{www.clausportner.com}}\\
    \& \\
    Center for Studies in Demography and Ecology \\
    University of Washington\\ \vspace{2cm}
    }

\date{November 2018}


%------------------------------------------------------------------------


\begin{document}
\graphicspath{{../figures/}}
\DeclareGraphicsExtensions{.eps,.jpg,.pdf,.mps,.png}

\setcounter{page}{-1}
\maketitle
\thispagestyle{empty}

% \setcounter{page}{0}


\newpage
\thispagestyle{empty}
\doublespacing

\begin{abstract}

% Demography abstract
\noindent 

Strong son preference is typically associated with shorter birth spacing
in the absence of sons, but access to sex selection has the potential to
reverse this pattern because each abortion extends spacing by six to
twelve months. 
I introduce a statistical method that simultaneously
accounts for how sex selection increases the spacing between 
births and the likelihood of a son. 
Using four rounds of India's National
Family and Health Surveys, I show that, except for first births,
the spacing between births increased substantially over the last four
decades, with the most substantial increases among women most 
likely to use sex selection.
Specifically, well-educated women with no boys now
exhibit significantly longer spacing and more male-biased sex ratios
than similar women with boys. 
Women with no education still follow the standard
pattern of short spacing when they have girls and little evidence of sex
selection, with medium-educated women showing mixed results. 
Finally,
sex ratios are more likely to decline within spells at lower parities,
where there is less pressure to ensure a son, and more likely to
increase or remain consistently high for higher-order spells, where the
pressure to provide a son is high.


\noindent JEL: J1, O12, I1
\noindent Keywords: India, prenatal sex determination, censoring, competing risk
\end{abstract}

\section{All Results}

\input{../tables/bootstrap_duration_sex_ratio_low_all.tex}

\input{../tables/bootstrap_duration_sex_ratio_med_all.tex}

\input{../tables/bootstrap_duration_sex_ratio_high_all.tex}

\clearpage

\newpage

\section{Region Results}

\input{../tables/bootstrap_duration_sex_ratio_high_r1.tex}

\input{../tables/bootstrap_duration_sex_ratio_high_r2.tex}

\input{../tables/bootstrap_duration_sex_ratio_high_r3.tex}

\input{../tables/bootstrap_duration_sex_ratio_high_r4.tex}

\clearpage
\newpage

\section{All Results}

\begin{figure}[htpb]
\centering
\caption*{Urban}
\subfloat[all]{\includegraphics[width=0.27\textwidth]{bs_spell2_low_urban_all}} 
\subfloat[all]{\includegraphics[width=0.27\textwidth]{bs_spell3_low_urban_all}} 
\subfloat[all]{\includegraphics[width=0.27\textwidth]{bs_spell4_low_urban_all}} 
\\
\caption*{Rural}
\subfloat[all]{\includegraphics[width=0.27\textwidth]{bs_spell2_low_rural_all}} 
\subfloat[all]{\includegraphics[width=0.27\textwidth]{bs_spell3_low_rural_all}} 
\subfloat[all]{\includegraphics[width=0.27\textwidth]{bs_spell4_low_rural_all}} 
\caption{low education}
\end{figure}


\begin{figure}[htpb]
\centering
\caption*{Urban}
\subfloat[all]{\includegraphics[width=0.27\textwidth]{bs_spell2_med_urban_all}} 
\subfloat[all]{\includegraphics[width=0.27\textwidth]{bs_spell3_med_urban_all}} 
\subfloat[all]{\includegraphics[width=0.27\textwidth]{bs_spell4_med_urban_all}} 
\\
\caption*{Rural}
\subfloat[all]{\includegraphics[width=0.27\textwidth]{bs_spell2_med_rural_all}} 
\subfloat[all]{\includegraphics[width=0.27\textwidth]{bs_spell3_med_rural_all}} 
\subfloat[all]{\includegraphics[width=0.27\textwidth]{bs_spell4_med_rural_all}} 
\caption{med education}
\end{figure}


\begin{figure}[htpb]
\centering
\caption*{Urban}
\subfloat[all]{\includegraphics[width=0.27\textwidth]{bs_spell2_high_urban_all}} 
\subfloat[all]{\includegraphics[width=0.27\textwidth]{bs_spell3_high_urban_all}} 
\subfloat[all]{\includegraphics[width=0.27\textwidth]{bs_spell4_high_urban_all}} 
\\
\caption*{Rural}
\subfloat[all]{\includegraphics[width=0.27\textwidth]{bs_spell2_high_rural_all}} 
\subfloat[all]{\includegraphics[width=0.27\textwidth]{bs_spell3_high_rural_all}} 
\subfloat[all]{\includegraphics[width=0.27\textwidth]{bs_spell4_high_rural_all}} 
\caption{high education}
\end{figure}



\clearpage
\newpage

\section{Regional Results}


\begin{figure}[htpb]
\centering
\caption*{Urban}
\subfloat[R1]{\includegraphics[width=0.23\textwidth]{bs_spell2_high_urban_r1}} 
\subfloat[R2]{\includegraphics[width=0.23\textwidth]{bs_spell2_high_urban_r2}} 
\subfloat[R3]{\includegraphics[width=0.23\textwidth]{bs_spell2_high_urban_r3}} 
\subfloat[R4]{\includegraphics[width=0.23\textwidth]{bs_spell2_high_urban_r4}} 
\\
\caption*{Rural}
\subfloat[R1]{\includegraphics[width=0.23\textwidth]{bs_spell2_high_rural_r1}} 
\subfloat[R2]{\includegraphics[width=0.23\textwidth]{bs_spell2_high_rural_r2}} 
\subfloat[R3]{\includegraphics[width=0.23\textwidth]{bs_spell2_high_rural_r3}} 
\subfloat[R4]{\includegraphics[width=0.23\textwidth]{bs_spell2_high_rural_r4}} 
\caption{Spell 2 - high education}
\end{figure}


\begin{figure}[htpb]
\centering
\caption*{Urban}
\subfloat[R1]{\includegraphics[width=0.23\textwidth]{bs_spell3_high_urban_r1}} 
\subfloat[R2]{\includegraphics[width=0.23\textwidth]{bs_spell3_high_urban_r2}} 
\subfloat[R3]{\includegraphics[width=0.23\textwidth]{bs_spell3_high_urban_r3}} 
\subfloat[R4]{\includegraphics[width=0.23\textwidth]{bs_spell3_high_urban_r4}} 
\\
\caption*{Rural}
\subfloat[R1]{\includegraphics[width=0.23\textwidth]{bs_spell3_high_rural_r1}} 
\subfloat[R2]{\includegraphics[width=0.23\textwidth]{bs_spell3_high_rural_r2}} 
\subfloat[R3]{\includegraphics[width=0.23\textwidth]{bs_spell3_high_rural_r3}} 
\subfloat[R4]{\includegraphics[width=0.23\textwidth]{bs_spell3_high_rural_r4}} 
\caption{Spell 3 - high education}
\label{fig:results_spell1_pps}
\end{figure}

\begin{figure}[htpb]
\centering
\caption*{Urban}
\subfloat[R1]{\includegraphics[width=0.23\textwidth]{bs_spell4_high_urban_r1}} 
\subfloat[R2]{\includegraphics[width=0.23\textwidth]{bs_spell4_high_urban_r2}} 
\subfloat[R3]{\includegraphics[width=0.23\textwidth]{bs_spell4_high_urban_r3}} 
\subfloat[R4]{\includegraphics[width=0.23\textwidth]{bs_spell4_high_urban_r4}} 
\\
\caption*{Rural}
\subfloat[R1]{\includegraphics[width=0.23\textwidth]{bs_spell4_high_rural_r1}} 
\subfloat[R2]{\includegraphics[width=0.23\textwidth]{bs_spell4_high_rural_r2}} 
\subfloat[R3]{\includegraphics[width=0.23\textwidth]{bs_spell4_high_rural_r3}} 
\subfloat[R4]{\includegraphics[width=0.23\textwidth]{bs_spell4_high_rural_r4}} 
\caption{Spell 4 - high education}
\label{fig:results_spell1_pps}
\end{figure}


\onehalfspacing
\bibliographystyle{aer}
\bibliography{sex_selection_spacing}

\end{document}
