This section provides a conceptual framework for understanding how birth spacing changes
and guiding the subsequent analysis.
In non-Western countries, lower fertility is associated with longer birth spacing, but 
outside a ``natural fertility'' regime it is not clear that lower fertility necessarily 
leads to longer spacing \citet{Newman1988,Rutstein2011,Casterline2016}.
Thus, I first introduce three theories that link fertility and birth spacing decisions
through economic conditions, desired child outcomes, and son preference \citep{Casterline2016,Portner2018}.
Next, I argue that female education and area of residence are the most important factors 
to consider in the empirical analysis and discuss how they tie into the three theories.
Finally, I describe how female education, together with female labor force participation, 
have changed over time in India and how these changes affect the predictions for how birth 
spacing has changed.

The dramatic improvements in economic conditions in India is best illustrated by the 
doubling of real wages between 1987 and 2011 \citep{Klasen2015}.
Economic theory predicts that the effect of higher wages on fertility is determined by 
the relative strengths of the substitution and income effects.
The substitute effect captures that as a good becomes more expensive relative to other
goods, people substitute away from the now more expensive good and towards other goods.
With higher wages, the substitute effect predicts that people work more and spend less 
time on other time-intensive non-wage earning activities, such as children or leisure.
The income effect captures that even if we ignore the substitution effect and hold hours 
worked constant higher wages increase the available income. 
This positive income effect leads people to spend more time on time-intensive activities 
such as children and leisure and less time working.

Empirically, higher female wages generally lowers fertility, whereas higher male 
wages increases fertility \citet{Hotz1997,schultz97}.
This pattern is consistent with women spending substantially more time on child care 
than men, leading to the substitution effect dominating for women while the income
effect dominates for men.

In addition to lowering fertility, higher female wages will tend to shorten spacing between 
children if having children requires the mother to reduce her economic activities.
Specifically, shortening birth spacing allows parents to take advantage of economies of 
scale in childrearing---for example, looking after two children simultaneously requires 
less than double the time needed to look after one child \citep{Vijverberg1982,Hotz1997}.
The reduction in the length of birth intervals from higher female wages will be more 
pronounced the more women participate in the labor force rather than working at home as 
economic activities at home are more compatible with child rearing.

[bla, bla]
The effect of increasing male wages on birth spacing is ambiguous.
On one hand, if fertility increases one might expect spacing to become shorter, although 
unless the updated desired fertility is close to natural fertility the higher desired 
fertility could simply be accommodated by having the last child later in life.
On the other hand, higher income could ``purchase'' both higher fertility and 
better quality children.
If child quality is increasing in birth spacing we would then expect higher male wages
to lead to longer birth spacing.


Parents care not only about how many children they have but also about the outcomes
for the children, and, everything else equal, more children means less resources to invest 
per child.
Hence, with declining fertility we should see better child outcomes such as lower 
mortality and more schooling.
However, with increasing income, it is, in principle, possible to have both more
children and better outcomes for them.
Increasing returns to education might make the parents reduce fertility and invest 
more in each of the children's education \citep{Rosenzweig1982a}.

Longer spacing between births is generally expected to have positive effects on 
child outcomes.
The clearest example of birth spacing affecting child outcomes is for health,
where very short spacing---approximately 24 months or less---leads to worse health 
and mortality outcomes for children 
\citep{Whitworth2002,Conde-Agudelo2006,Conde-Agudelo2012,Molitoris2019}.
More speculative is whether the increased individual parental attention per child that 
should arise from longer spacing results in better human capital outcome for children 
\citep{Zajonc1975,Zajonc1976,Razin1980}.
The evidence on spacing's effect on child quality measures such as IQ and education is 
mixed for developed countries and nonexisting for developing countries
\citep{Powell1993,Pettersson-Lidbom2009,Buckles2012,Barclay2017}.


Finally, as mentioned above, the stronger the son preference, the more likely we are to
see short spacing, and the associated worse health outcomes, when there are no sons 
in the household 
\citep{Whitworth2002,Bhalotra2008,Maitra2008,Jayachandran2011,Jayachandran2017a}.

The introduction of sex selection allows parents to avoid giving birth to girls but 
increases the expected interval to the next birth by 6--12 months.
The increase consists of three parts. 
First, after an abortion, the uterus needs at least two menstrual cycles to recover, 
or the likelihood of spontaneous abortion increases substantially \citep{zhou00b}. 
Second, the waiting time to conception is one to six months. 
Finally, sex determination tests are reliable only from three months of gestation. 
Theory suggests that sex selection increases with lower desired fertility and with
higher parity for a given desired number of children \citep{Portner2015b}.



[move to the prediction part?]
In India, real wages for both men and women have almost doubled between 1987 and 
2011, although the mean male wage is still close to 70\% higher than the female wage 
\citep{Klasen2015,Bhargava2018}.






