Female Education and Labor Force Participation in India

There has been substantial improvements in female educational attainment over time.
Figure XX shows the distribution of schooling by birth cohort for urban and rural women 
twenty years or older whether married or not.
The education groups are no education, one to seven years, and eight years or
above.

For rural areas, the percentage of women with no education has gone from around 90 percent
for the 1930s cohorts to less than 20 percent for 1990s cohorts. 
The proportion of women with one to seven years of education has remained remarkably
constant at around ten percent.
The difference is made up by the women with eight or more years of education, who have
gone from almost zero for the 1930 cohort to more than sixty percent for the 1990s cohorts.

Female education is higher in urban areas than in rural areas.
Around sixty percent of urban women born in the 1930s had no education, 25 percent had
between one and seven years, and about 15 had eight or more years of education.
The proportion with no education has declined almost to five percent for the latest
cohort, and the proportion with one to seven years of education has shrunk to just over
ten percent.
The result is that close to 80 percent of urban women now have eight or more years of
education.

Even as the level of female education has increased the female labor force participation 
has decreased over time [should this be by level of education?].

[Discuss recent papers on labor supply in India]


To place my results in context, Figure XX shows the percent of married women who are 
working at the time of the survey by age group.%
\footnote{
This question is the only question on labor force participation consistently available 
across all four surveys.
Since the question is asked only of married women it is possible that the overall
female labor force participation might have developed differently, especially since
many young, unmarried women have begun working in, for example, the business process 
outsourcing industry [TK cite Jensen 2012 QJE].
}


The declines over the the four rounds of the NFHS are in line with what previous
research has found using other data sets [TK cite Bhargava 2018 and 2019, Chatterjee 2018].
Finding a decline in labor force participation in the beginning of development is
in line with the hypothesis of a U-shaped labor force participation as a country
develop [TK cite Goldin 1994, Chatterjee 2018].



The theories on the relationship between education and birth spacing often works 
through the changes in the opportunity cost of time. 
If women are unlikely to work, this makes the predictions that are based on
opportunity cost less relevant. 
instead, theories that emphasize income effects or effects of spacing become
more relevant. 

