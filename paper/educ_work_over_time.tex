Female Education and Labor Force Participation in India

There has been substantial improvements in female educational attainment over time.
Figure XX shows the distribution of schooling by birth cohort for urban and rural women 
twenty years or older whether married or not.
The education groups are no education, one to seven years, and eight years or
above.

For rural areas, the percentage of women with no education has gone from around 90 percent
for the 1930s cohorts to less than 20 percent for 1990s cohorts. 
The proportion of women with one to seven years of education has remained remarkably
constant at around ten percent.
The difference is made up by the women with eight or more years of education, who have
gone from almost zero for the 1930 cohort to more than sixty percent for the 1990s cohorts.

Female education is higher in urban areas than in rural areas.
Around sixty percent of urban women born in the 1930s had no education, 25 percent had
between one and seven years, and about 15 had eight or more years of education.
The proportion with no education has declined almost to five percent for the latest
cohort, and the proportion with one to seven years of education has shrunk to just over
ten percent.
The result is that close to 80 percent of urban women now have eight or more years of
education.

Despite the increasing level of female education the female labor force participation has
decreased over time.




