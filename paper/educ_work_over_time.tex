Female Education and Labor Force Participation in India

There has been substantial improvements in female educational attainment over time.
Figure XX shows the distribution of schooling by birth cohort for urban and rural women 
twenty years or older whether married or not.
The education groups are no education, one to seven years, and eight years or
above.

For rural areas, the percentage of women with no education has gone from around 90 percent
for the 1930s cohorts to less than 20 percent for 1990s cohorts. 
The proportion of women with one to seven years of education has remained remarkably
constant at around ten percent.
The difference is made up by the women with eight or more years of education, who have
gone from almost zero for the 1930 cohort to more than sixty percent for the 1990s cohorts.

Female education is higher in urban areas than in rural areas.
Around sixty percent of urban women born in the 1930s had no education, 25 percent had
between one and seven years, and about 15 had eight or more years of education.
The proportion with no education has declined almost to five percent for the latest
cohort, and the proportion with one to seven years of education has shrunk to just over
ten percent.
The result is that close to 80 percent of urban women now have eight or more years of
education.


Even as the level of female education has increased, the female labor force participation 
in both urban and rural areas has stagnated or decreased
\citep{Klasen2015,Afridi2018,Bhargava2018, Chatterjee2018, Bhargava2019}.
A decline in female labor force participation in the beginning of development is
consistent with the hypothesis of a U-shaped labor force participation as a country
develop \citep{Goldin1994}.
India's female labor force participation is, however, lower than most other
countries and more in line with countries in the Middle East and North Africa, and
does not yet show any signs of increasing.
\citep{Klasen2015,Chatterjee2018}.

Theory predicts that increasing women's wage leads to higher labor force participation, 
while both a higher male income and stronger social restrictions on working, possibly in 
specific industries or jobs, reduce female labor supply \citep{Goldin1994}.
India's economy has grown substantially since the mid-seventies with
an average annual growth rate of about 5.5 percent from 1978 to 2004 \citet{Bosworth2008}.
As a result, real wages for both men and women have almost doubled between 1987 and 2011 
\citep{Klasen2015}.
Despite the increases in wages and the higher female education, the mean male wage is
still close to 70 percent higher than the mean female wage \citep{Bhargava2018}.
Furthermore, the development in women's wage is not uniform across education levels.
The real wages for women with middle school education or below has converged, 
for women with secondary education, the change over time depends on the sample used
with some showing a relative strong decline and others a modest increase,
and women with more than secondary education experienced an increased in their real 
wage, especially in cities \citep{Klasen2015,Bhargava2018}.

After controlling for household characteristics and husband income, there is
still a U-shaped relationship between female education and labor supply with the
lowest labor force participation for women with secondary education \citep{Chatterjee2018}.
There are two possible explanations for this effect, both directly related to the strong
pro-male bias in India.
First, it becomes less socially acceptable for women to work in manual labor as their
education increases, but women are still mostly excluded from white-collar
employment despite the strong economic growth \citep{Klasen2015,Chatterjee2018}.
Second, with increasing education, women's productivity at home increases, especially
in the production of offspring human capital, and with access to sex selective abortions
leading to relatively more boys being born and an increasing return to male education 
there can be an increased demand for women with more education, even though they do not 
participate in the labor market \citep{Behrman1999}.


To place my results in context, Figure XX shows the percent of married women who are 
currently working at the time of the survey by age group and education level.
This is the only question on labor force participation consistently available 
across all four surveys, and because the question refers to currently working the 
percentages will be lower than what other studies have found.
Furthermore, since the question is asked only of married women, the overall female labor 
force participation might have developed differently, especially since many young, 
unmarried women have begun working in, for example, the business process 
outsourcing industry, which, in turn, has increased girls' schooling investments
\citep{Jensen2012}.

Women are more likely to report working if they live in rural than urban areas
and the older they are, and the U-shaped relationship between education and working
holds in most cases with the highest percent working for women with either no education
or with twelve or more years and the lowest for women with eight to eleven years 
of education.
Comparing 1992 and 2015, the percent who currently work has either remained roughly 
the same or decreased.
The cases were the numbers were higher in 1999 and 2006 than 1992 and 2015 are mostly
based on small samples, such as rural women with twelve or more years of education
in rural areas.

Almost all of the urban women working are paid either cash or a combination of
cash and in-kind for their work (see Appendix Figure XX).
Rural women have become more likely to receive cash for their labor over time and
the better educated the more likely they are to receive cash.
Despite this, women across all groups have become substantially more likely to 
work for a family member rather than for themselves or somebody else
(see Appendix Figure XX).
Hence, most of the reduction in the likelihood of working appear have come from 
a retraction from the general labor market and self-employment.

As shown above, some of the theories on the relationship between education and birth 
spacing works through the changes in the opportunity cost of time. 
For those women unlikely to work, the predictions based on opportunity cost are 
less relevant. 
Instead, theories that emphasize income effects or effects of spacing on child
outcomes become more relevant. 

With substantial increases in husband income and a declining female labor force 
participation, I expect a push toward longer birth spacing over time, independent
of education levels.
Even lower desired fertility with increasing education, I would, however, expect
the most substantial increase in birth spacing among the best educated because
of their use of sex selection.
Working in the opposite direction is that women with more education can space
their children closer together without substantially increasing child mortality risk.


[How would education levels behave relative to each other?
- longer among better educated because of sex selection
- convergence because better educated can space children closer together without 
  substantially increased child mortality risk.
- Sanskritization (?) would mean that lower caste caste women try to mimic 
  and with increasing education that becomes easier]
