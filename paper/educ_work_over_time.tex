Female Education and Labor Force Participation in India

There has been substantial improvements in female educational attainment over time.
Figure XX shows the distribution of schooling by birth cohort for urban and rural women 
twenty years or older whether married or not.
The education groups are no education, one to seven years, and eight years or
above.

For rural areas, the percentage of women with no education has gone from around 90 percent
for the 1930s cohorts to less than 20 percent for 1990s cohorts. 
The proportion of women with one to seven years of education has remained remarkably
constant at around ten percent.
The difference is made up by the women with eight or more years of education, who have
gone from almost zero for the 1930 cohort to more than sixty percent for the 1990s cohorts.

Female education is higher in urban areas than in rural areas.
Around sixty percent of urban women born in the 1930s had no education, 25 percent had
between one and seven years, and about 15 had eight or more years of education.
The proportion with no education has declined almost to five percent for the latest
cohort, and the proportion with one to seven years of education has shrunk to just over
ten percent.
The result is that close to 80 percent of urban women now have eight or more years of
education.


Even as the level of female education has increased, the female labor force participation 
in both urban and rural areas has stagnated or decreased
\citep{Klasen2015,Afridi2018,Bhargava2018, Chatterjee2018, Bhargava2019}.
A decline in female labor force participation in the beginning of development is
consistent with the hypothesis of a U-shaped labor force participation as a country
develop \citep{Goldin1994}
India's female labor force participation is, however,  lower than most other
countries and more in line with countries in the Middle East and North Africa and
does not yet show any signs of increasing.
\citep{Klasen2015,Chatterjee2018}.

India has experience substantial economic growth since the mid-seventies with
an average growth rate of about 5.5 percent from 1978 to 2004 \citet{Bosworth2008}.

[men and women's wages]

Klasen2015
In line with India’s high growth rates, earnings data form the NSS
surveys show that real wages roughly doubled between 1987 and 2011 (see
Fig. S1.1 in the supplemental appendix). In absolute terms, real wages
increased almost equally for men and women, but the ratio of male to
female average weekly earnings declined from 1.6 in 1987 to 1.3 in 2011.


[Discuss recent papers on labor supply in India]


To place my results in context, Figure XX shows the percent of married women who are 
working at the time of the survey by age group.%
\footnote{
This question is the only question on labor force participation consistently available 
across all four surveys.
For the NFHS-4, it is only asked of women selected for the state samples, so the
sample size is comparable across all four surveys.
Furthermore, because the question refers to currently the percentages will be lower than
what other studies have found.
Finally, since the question is asked only of married women it is possible that the overall
female labor force participation might have developed differently, especially since
many young, unmarried women have begun working in, for example, the business process 
outsourcing industry, which, in turn, has increased girls' schooling investments
\citep{Jensen2012}.
}


 
 





The theories on the relationship between education and birth spacing often works 
through the changes in the opportunity cost of time. 
If women are unlikely to work, this makes the predictions that are based on
opportunity cost less relevant. 
instead, theories that emphasize income effects or effects of spacing become
more relevant. 

