% Poster presentation for PAA 2018

\documentclass[final]{beamer}
  \mode<presentation>
  {
  \usetheme{CCP}
  }
  \usepackage[orientation=landscape,scale=1.2]{beamerposter}

\setlength{\paperwidth}{48in} 
\setlength{\paperheight}{36in}

\usepackage{fontspec}
\setsansfont[Ligatures=TeX]{Gandhi Sans}
% \usepackage{unicode-math}
% \setmathfont{TeX Gyre Pagella Math}
\usepackage{dcolumn}
\usepackage{booktabs}
\usepackage{multirow}
\usepackage[flushleft]{threeparttable}
\usepackage{caption}
\usepackage[font=scriptsize,labelfont=scriptsize]{subfig}


\newcommand{\mco}[1]{\multicolumn{1}{c}{#1}}
\newcommand{\mct}[1]{\multicolumn{2}{c}{#1}}
\newcommand{\X}{$\times$ }
\newcommand{\hs}{\hspace{15pt}}

% Beamer thinks no table numbers and small text
\setbeamertemplate{caption}[numbered]
\captionsetup{font=large,labelfont=large}

\newlength{\sepwid}
\newlength{\onecolwid}
\newlength{\twocolwid}
\newlength{\threecolwid}
\setlength{\sepwid}{0.024\paperwidth}
\setlength{\onecolwid}{0.24\paperwidth}
\setlength{\twocolwid}{0.4\paperwidth}
\setlength{\threecolwid}{0.19\paperwidth}


\title{Birth Spacing in the Presence of Son Preference and Sex-Selective Abortions:\\ India's Experience over Four Decades}
\author{Claus P\"ortner}
\institute{Albers School of Business and Economics, Seattle University, \& Center for Studies in Demography and Ecology, University of Washington}


%%% Put the name of conference here.
	\def\conference{2018 Annual Meeting of the Population Association of America} 
 %%% Put your e-mail address here.
 	\def\yourEmail{cportner@seattleu.edu}


\begin{document}

\graphicspath{{../figures/}}
\DeclareGraphicsExtensions{.eps,.jpg,.pdf,.mps,.png}


\begin{frame}{}
\begin{columns}[t]


%%%%%%%%%%%%%%%%%%%%%%%%%%%%%%%%%%%%%%%%%%%%%%%%%%%%%%%%%%%%%%%%%%%%%%%%%%%%%%%%%%%%

\hspace{1cm}%
\begin{column}{\onecolwid}


\begin{block}{Background}

Son preference has been associated with shorter spacing after the 
births of girls than boys.

\vskip1ex

Access to prenatal sex determination fundamentally changed the 
relationship between son preference and birth spacing because each abortion 
increases birth spacing by six months to a year.

\vskip1ex

We may therefore now observe \alert{longer} spacing after daughters than sons, 
precisely because strong son preference leads to the use of sex selection.

\vskip1ex
% 
% To complicate matters, shorter birth spacing after daughters may still 
% be a representation of son preference for 
% families who do not use prenatal sex selection.


\end{block}

\begin{alertblock}{Contributions}
\vskip1ex
\begin{description}
\setlength{\itemsep}{2ex}
\item[Main idea:]  in the absence of sex selection son preference leads to shorter 
spacing after the birth of a girl than after the birth of a boy, whereas 
son preference increases the spacing after the birth of a girl relative 
to after a boy when prenatal sex selection is available.

\item[Method:] I introduce an empirical method that directly incorporates 
the effects of sex-selective abortions on 
the duration between births
\emph{and} 
the likelihood of a son. 

\item[Central questions:] how did spacing patterns change with access
to prenatal sex determination?
Specifically, did the spacing after girls increase relative to the spacing
after boys?
\end{description}
\vskip1ex
\end{alertblock}

\vskip1ex
\begin{block}{Methods}


The model is a discrete time, non-proportional, competing risk 
hazard model with two  exit states: either a boy or a girl is born.
\vskip1ex
\begin{description}
\item[Competing risk] captures that
parents can choose the sex of children born with access to prenatal 
sex determination.
\item[Non-proportional hazard] captures that sex composition and 
the use of sex selection likely affect the shape of the hazard 
functions across groups and that use of sex selection may vary
within a spell.
\end{description}

\end{block}
\vskip1ex
\begin{block}{Data}

The data come from the four rounds of the National Family Health Survey (NFHS).
Sample is restricted to Hindu women who are married and regular residents of
the household and where all needed information are available.
Because of recall error, I drop women married 22 years or more 
for NFHS-1, with the corresponding cut-off points 23 years for NFHS-2, 
and 25 years for NFHS-3 and NFHS-4.
The final sample consists of 
  395,695 women, with   815,360 parity one through four births.
% [number] women, with [number] parity one through four births.

% \end{block}
% 
% \begin{block}{Spell Definition}

\vskip2ex


The first spell begins at the month of marriage, whereas the second and 
subsequent spells begin nine months after the previous birth.
All spells continue until either a child is born or censoring occurs.
Spells are grouped into four periods based on start of spell:
1972--1984, 1985--1994, 1995--2004, and 2005--2016.

\end{block}


\end{column}


%%%%%%%%%%%%%%%%%%%%%%%%%%%%%%%%%%%%%%%%%%%%%%%%%%%%%%%%%%%%%%%%%%%%%%%%%%%%%%%%%%%%

\begin{column}{\twocolwid}



\begin{block}{Results}

        
\begin{table}[hp!]
\centering
\begin{threeparttable}
\caption{Estimated Median Duration and Sex Ratio}
\label{tab:median_sex_ratio_low}
\begin{tabular}{@{} c l D{.}{.}{2.3} D{.}{.}{2.3}  D{.}{.}{2.3} D{.}{.}{2.3} p{2ex} D{.}{.}{2.3} D{.}{.}{2.3}  D{.}{.}{2.3} D{.}{.}{2.3} @{}}
\toprule
                   &                            & \multicolumn{4}{c}{Rural Women -- No Education}                                                            & & \multicolumn{4}{c}{Urban Women -- 8+ Years of Education} \\  \cmidrule(lr){3-6} \cmidrule(lr){8-11}
                   &                            & \mct{1972--1984}                                    & \mct{2005--2016}                                     & & \mct{1972--1984}                                    & \mct{2005--2016}                                      \\ \cmidrule(lr){3-4} \cmidrule(lr){5-6} \cmidrule(lr){8-9} \cmidrule(lr){10-11} 
                   & \mco{Composition of}       & \mco{Duration\tnote{a}}  & \mco{Percent\tnote{b}}   & \mco{Duration\tnote{a}}  & \mco{Percent\tnote{b}}    & & \mco{Duration\tnote{a}}  & \mco{Percent\tnote{b}}   & \mco{Duration\tnote{a}}  & \mco{Percent\tnote{b}}     \\ 
 \mco{Spell}       & \mco{Prior Children}       & \mco{(Months)}  & \mco{Boys}                        & \mco{(Months)}  & \mco{Boys}                         & & \mco{(Months)}  & \mco{Boys}                        & \mco{(Months)}  & \mco{Boys}                          \\ \midrule
\multirow[c]{2}{*}{1} &                            &    26.2       &    52.1^{**}  &    24.4       &    51.5        &&    19.1       &    52.8^{***} &    18.8       &    52.7^{***}  \\
                      &                            &    (0.2)      &    (0.3)      &    (0.1)      &    (0.4)       &&    (0.1)      &    (0.3)      &    (0.1)      &    (0.3)       \\
\addlinespace 
\multirow[c]{3}{*}{2} & \mco{One girl}             &    18.5       &    51.1       &    19.7       &    51.8        &&    25.9       &    58.1^{***} &    29.0       &    56.1^{***}  \\
                      &                            &    (0.2)      &    (0.6)      &    (0.2)      &    (0.6)       &&    (0.2)      &    (0.5)      &    (0.4)      &    (0.7)       \\
                      & \mco{One boy}              &    19.6^{***} &    52.7^{**}  &    20.1^{*}   &    52.2^{*}    &&    25.6       &    50.1^{**}  &    29.0       &    49.6^{***}  \\
                      &                            &    (0.1)      &    (0.7)      &    (0.2)      &    (0.5)       &&    (0.2)      &    (0.6)      &    (0.4)      &    (0.6)       \\
\addlinespace 
\multirow[c]{5}{*}{3} & \mco{Two girls}            &    17.9       &    49.4^{*}   &    20.3       &    52.9^{**}   &&    28.6       &    65.8^{***} &    33.2       &    66.3^{***}  \\
                      &                            &    (0.3)      &    (1.0)      &    (0.2)      &    (0.7)       &&    (0.7)      &    (1.1)      &    (0.9)      &    (1.5)       \\
                      & \mco{One boy / one girl}   &    19.1^{***} &    53.0^{**}  &    21.1^{***} &    51.8        &&    24.0^{***} &    53.5^{**}  &    26.7^{***} &    55.6^{***}  \\
                      &                            &    (0.2)      &    (0.7)      &    (0.2)      &    (0.6)       &&    (0.5)      &    (1.0)      &    (0.9)      &    (1.4)       \\
                      & \mco{Two boys}             &    19.1^{***} &    51.4       &    22.0^{***} &    50.9        &&    24.2^{***} &    46.6^{***} &    28.2^{***} &    48.8        \\
                      &                            &    (0.2)      &    (0.9)      &    (0.3)      &    (0.8)       &&    (0.8)      &    (1.4)      &    (1.3)      &    (2.3)       \\
\addlinespace 
\bottomrule
\end{tabular}
\begin{tablenotes} 
\footnotesize
\item \hspace*{-0.5em} \textbf{Note.}
Bootstrapped standard errors shown in parentheses.
\item[a] For spells two and higher duration for sex compositions other than all girls are 
tested against the duration for all girls, with *** indicating significantly different 
at the 1\% level, ** at the 5\% level, and * at the 10\% level, based on bootstrapped
differences.
\item[b] The predicted percent boys is tested against the natural percentage boys, 105 boys per 100 girls,
with *** indicating significantly different at the 1\% level, ** at the 5\% level, 
and * at 10\% level.
\end{tablenotes}
\end{threeparttable}
\end{table}

\end{block}    

\begin{columns}[t,totalwidth=\twocolwid]

\begin{column}{\threecolwid}

\begin{block}{Rural -- No Education}

\begin{figure}[htpb]
\centering
\rotatebox[origin=c]{90}{\footnotesize{Second Spell}}
\setcounter{subfigure}{-1}
\subfloat[1972--1984]{
    \begin{minipage}{0.45\textwidth}
        \captionsetup[subfigure]{labelformat=empty,position=top,captionskip=-1pt,farskip=-0.5pt}
        \subfloat[Probability of no birth yet]{\includegraphics[width=\textwidth]{spell2_g1_low_rural_pps}} 
        \captionsetup[subfigure]{labelformat=parens}
    \end{minipage}
} 
\setcounter{subfigure}{0}
\subfloat[2005--2016]{
    \begin{minipage}{0.45\textwidth}
        \captionsetup[subfigure]{labelformat=empty,position=top,captionskip=-1pt,farskip=-0.5pt}
        \subfloat[Probability of no birth yet]{\includegraphics[width=\textwidth]{spell2_g4_low_rural_pps}} 
        \captionsetup[subfigure]{labelformat=parens}
    \end{minipage}
} 
\\
\rotatebox[origin=c]{90}{\footnotesize{Third Spell}}
\setcounter{subfigure}{1}
\subfloat[1972--1984]{
    \begin{minipage}{0.45\textwidth}
        \captionsetup[subfigure]{labelformat=empty,position=top,captionskip=-1pt,farskip=-0.5pt}
        \subfloat[Probability of no birth yet]{\includegraphics[width=\textwidth]{spell3_g1_low_rural_pps}} 
        \captionsetup[subfigure]{labelformat=parens}
    \end{minipage}
} 
\setcounter{subfigure}{2}
\subfloat[2005--2016]{
    \begin{minipage}{0.45\textwidth}
        \captionsetup[subfigure]{labelformat=empty,position=top,captionskip=-1pt,farskip=-0.5pt}
        \subfloat[Probability of no birth yet]{\includegraphics[width=\textwidth]{spell3_g4_low_rural_pps}} 
        \captionsetup[subfigure]{labelformat=parens}
    \end{minipage}
} 
\\
\rotatebox[origin=c]{90}{\footnotesize{Fourth Spell}}
\setcounter{subfigure}{3}
\subfloat[1972--1984]{
    \begin{minipage}{0.45\textwidth}
        \captionsetup[subfigure]{labelformat=empty,position=top,captionskip=-1pt,farskip=-0.5pt}
        \subfloat[Probability of no birth yet]{\includegraphics[width=\textwidth]{spell4_g1_low_rural_pps}} 
        \captionsetup[subfigure]{labelformat=parens}
    \end{minipage}
} 
\setcounter{subfigure}{4}
\subfloat[2005--2016]{
    \begin{minipage}{0.45\textwidth}
        \captionsetup[subfigure]{labelformat=empty,position=top,captionskip=-1pt,farskip=-0.5pt}
        \subfloat[Probability of no birth yet]{\includegraphics[width=\textwidth]{spell4_g4_low_rural_pps}} 
        \captionsetup[subfigure]{labelformat=parens}
    \end{minipage}
} 
\caption{
Survival curves conditional on predicted progression to next birth
}
\label{fig:pps_low}
\end{figure}

\end{block}     
     
     
\end{column}

\begin{column}{\threecolwid}

\begin{block}{Urban -- 8+ Years of Education}

\begin{figure}[htpb]
\centering
\rotatebox[origin=c]{90}{\footnotesize{Second Spell}}
\setcounter{subfigure}{-1}
\subfloat[1972--1984]{
    \begin{minipage}{0.45\textwidth}
        \captionsetup[subfigure]{labelformat=empty,position=top,captionskip=-1pt,farskip=-0.5pt}
        \subfloat[Probability of no birth yet]{\includegraphics[width=\textwidth]{spell2_g1_high_urban_pps}} 
        \captionsetup[subfigure]{labelformat=parens}
    \end{minipage}
} 
\setcounter{subfigure}{0}
\subfloat[2005--2016]{
    \begin{minipage}{0.45\textwidth}
        \captionsetup[subfigure]{labelformat=empty,position=top,captionskip=-1pt,farskip=-0.5pt}
        \subfloat[Probability of no birth yet]{\includegraphics[width=\textwidth]{spell2_g4_high_urban_pps}} 
        \captionsetup[subfigure]{labelformat=parens}
    \end{minipage}
} 
\\
\rotatebox[origin=c]{90}{\footnotesize{Third Spell}}
\setcounter{subfigure}{1}
\subfloat[1972--1984]{
    \begin{minipage}{0.45\textwidth}
        \captionsetup[subfigure]{labelformat=empty,position=top,captionskip=-1pt,farskip=-0.5pt}
        \subfloat[Probability of no birth yet]{\includegraphics[width=\textwidth]{spell3_g1_high_urban_pps}} 
        \captionsetup[subfigure]{labelformat=parens}
    \end{minipage}
} 
\setcounter{subfigure}{2}
\subfloat[2005--2016]{
    \begin{minipage}{0.45\textwidth}
        \captionsetup[subfigure]{labelformat=empty,position=top,captionskip=-1pt,farskip=-0.5pt}
        \subfloat[Probability of no birth yet]{\includegraphics[width=\textwidth]{spell3_g4_high_urban_pps}} 
        \captionsetup[subfigure]{labelformat=parens}
    \end{minipage}
} 
\\
\rotatebox[origin=c]{90}{\footnotesize{Fourth Spell}}
\setcounter{subfigure}{3}
\subfloat[1972--1984]{
    \begin{minipage}{0.45\textwidth}
        \captionsetup[subfigure]{labelformat=empty,position=top,captionskip=-1pt,farskip=-0.5pt}
        \subfloat[Probability of no birth yet]{\includegraphics[width=\textwidth]{spell4_g1_high_urban_pps}} 
        \captionsetup[subfigure]{labelformat=parens}
    \end{minipage}
} 
\setcounter{subfigure}{4}
\subfloat[2005--2016]{
    \begin{minipage}{0.45\textwidth}
        \captionsetup[subfigure]{labelformat=empty,position=top,captionskip=-1pt,farskip=-0.5pt}
        \subfloat[Probability of no birth yet]{\includegraphics[width=\textwidth]{spell4_g4_high_urban_pps}} 
        \captionsetup[subfigure]{labelformat=parens}
    \end{minipage}
} 
\caption{
Survival curves conditional on predicted progression to next birth
}
\label{fig:pps_high}
\end{figure}


\end{block}     

     
\end{column}    

\end{columns}

\end{column}


%%%%%%%%%%%%%%%%%%%%%%%%%%%%%%%%%%%%%%%%%%%%%%%%%%%%%%%%%%%%%%%%%%%%%%%%%%%%%%%%%%%%

\begin{column}{\onecolwid}

\begin{alertblock}{Conclusions}
\vskip1ex
\begin{enumerate}
\setlength{\itemsep}{2ex}
\item There has been a general increase in the length of spacing between births
over the four decades.
The exception is for first births, where the median duration has either
remained the same or declined slightly, although this hides a
significant compression of the variation in spell length.

\item The most substantial increase in spacing is for the women who
are most likely to use sex selection.
Among the best-educated women, those with no boys now has 
significantly longer spacing---and a more male-biased sex ratio---than 
similar women with boys.

\item Women with no education still follow the standard pattern of
short spacing when they have girls and little evidence of the use of sex
selection.
In other words, they adhere to a strategy where they achieve a
son through higher fertility rather than the use of sex selection.

\item Sex ratios are the most likely to decline within spells at
lower-order spells, where the pressure to provide a son is smaller, and
are more likely to increase or remain consistently high for higher-order
spells, where the pressure to ensure a son is high.

\end{enumerate}
\vskip1ex
\end{alertblock}

\vskip1ex

\begin{block}{Future Research}

The results lead to two important, interconnected, questions:
What is the connection between falling fertility and the 
use of sex selection, and is the use of sex selection increasing 
or decreasing over time?

\bigskip

There are also two broader questions:
\begin{itemize}
\item To what extent are improvements in health for girls
relative to boys the result of selection, the longer spacing between
births, or changing son preference or higher value placed on girls
with fewer of them around?
\item What is the interaction between female labor force participation
and the use of sex selection?
\end{itemize}
\vskip1ex
\end{block}


\begin{block}{Acknowledgements}

Support from the University of Washington Royalty Research Fund and the 
Development Research Group of the World Bank
for development of the method is gratefully acknowledged.
The views and findings expressed here are those of the author and
should not be attributed to the World Bank or any of its member countries.

\vskip1ex

Partial support for this research came from a Eunice Kennedy Shriver National
Institute of Child Health and Human Development research infrastructure grant,
5R24HD042828, to the Center for Studies in Demography and Ecology at the
University of Washington.

\end{block}



\end{column}


\end{columns}

\end{frame}



\end{document}
