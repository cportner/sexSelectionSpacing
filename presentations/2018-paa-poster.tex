% Poster presentation for PAA 2018

\documentclass[final]{beamer}
  \mode<presentation>
  {
  \usetheme{CCP}
  }
  \usepackage[orientation=landscape,scale=1.4]{beamerposter}

\setlength{\paperwidth}{48in} 
\setlength{\paperheight}{36in}

\usepackage{fontspec}
\setsansfont[Ligatures=TeX]{Gandhi Sans}
% \usepackage{unicode-math}
% \setmathfont{TeX Gyre Pagella Math}


\title{Birth Spacing in the Presence of Son Preference and Sex-Selective \\ Abortions: India's Experience over Four Decades}
\author{Claus P\"ortner}
\institute{Albers School of Business and Economics, Seattle University, \& Center for Studies in Demography and Ecology, University of Washington}
\footer{\texttt{cportner@seattleu.edu}}
\begin{document}

\begin{frame}{}
\begin{columns}[t]


%%%%%%%%%%%%%%%%%%%%%%%%%%%%%%%%%%%%%%%%%%%%%%%%%%%%%%%%%%%%%%%%%%%%%%%%%%%%%%%%%%%%

\begin{column}{.3\linewidth}


\begin{block}{\LARGE Background}

The central question addressed here is the extent to which spacing patterns 
significantly changed as prenatal sex determination became available.
Specifically, did the spacing after girls increase relative to the spacing
after boys?
The underlying idea is that in the absence of sex selection son 
preference leads to shorter spacing after the birth of a girl than after the 
birth of a boy, whereas son preference increases the spacing after the birth 
of a girl relative to after the birth of a boy when prenatal sex selection is 
available.
I introduce a new method that simultaneously accounts for spacing
between births and the potential use of sex selection. 
I apply the method to over four decades of data from India's NFHS.


\end{block}

\begin{block}{Objectives}

\end{block}


\begin{block}{Methods}


The model is a discrete time, non-proportional, competing risk 
hazard model with two exit states: either a boy or a girl is born.
\begin{itemize}
\item the competing risk framework captures that access to prenatal 
sex determination means that parents can choose the sex of children born.
\item the non-proportional hazard specification captures that  
sex composition and the use of sex selection are likely to affect 
the shape of the hazard functions across groups.
\end{itemize}


The probability of ever having a next birth varies across groups.
Direct comparison of standard survival curves, therefore, tells us little 
about how the spread of sex selection affects birth spacing across groups.
I therefore condition on the predicted likelihood of parity 
progression when examining birth spacing measures.

In addition to calculating conditional birth spacing measures for different 
percentages, I also present graphs of both standard survival curves and
survival curves conditional on parity progression (which therefore
begin at 100\% and end at 0\%).

Finally, I calculate the predicted percent boys born over the entire spell.


\end{block}



\end{column}


%%%%%%%%%%%%%%%%%%%%%%%%%%%%%%%%%%%%%%%%%%%%%%%%%%%%%%%%%%%%%%%%%%%%%%%%%%%%%%%%%%%%

\begin{column}{.3\linewidth}

\begin{block}{Data}

The data come from the four rounds of the National Family Health Survey.

I exclude visitors to the household, as well as
women who have never been married or married more than once, divorced, or who are 
not living with their husband,
women with inconsistent age at marriage,
and those with missing information on education.
The same goes for women who had at least one multiple births,
reported having a birth before age 12, had a birth before marriage, or
duration between births of less than nine months.
Finally, I restrict the sample to Hindus,
who constitute about 80\% of India's population.

Recall error is heavily dependent on how long ago a woman was married.
I, therefore, drop women married 22 years or more 
for NFHS-1, with the corresponding cut-off points 23 years for NFHS-2, 
and 25 years for NFHS-3 and NFHS-4.
The final sample consists of 
  395,695 women, with   815,360 parity one through four births.
% [number] women, with [number] parity one through four births.

\end{block}

\begin{block}{Spell Definition}

The first spell begins at the month of marriage.

The second and subsequent spells begin nine months after the previous birth 
because that is the earliest we should expect to observe a new birth.

All spells continue until either a child is born or censoring occurs.
Censoring can happen for three reasons:
the survey takes place;
sterilization of the woman or her husband;
or too few births are observed for the method to work.

I group spells into four periods based on the year of marriage
or year of the previous birth:
1972--1984, 1985--1994, 1995--2004, and 2005--2016.

\end{block}

\begin{block}{Results}

\end{block}


\end{column}

%%%%%%%%%%%%%%%%%%%%%%%%%%%%%%%%%%%%%%%%%%%%%%%%%%%%%%%%%%%%%%%%%%%%%%%%%%%%%%%%%%%%

\begin{column}{.3\linewidth}

\begin{block}{Conclusion}


The results show two very different approaches to son preference.

\bigskip

At one extreme, women without education mostly follow the standard 
pattern of shorter spacing when a woman does not have the desired number of 
sons.

\bigskip

At the other extreme, women with eight or more years of education have
experienced a reversal of the traditional spacing patterns.
Women with either no or one son now have substantially longer spacing 
than if they have two or more sons.

\bigskip

Although it is not possible to directly observe whether sex selection
causes the unequal sex ratio and longer spacing the results are
consistent with a general increase in sex selection over time. 

\bigskip

Compared to previous research one result stands out: 
the use of sex selection is spreading to women with lower levels of education.



\end{block}

\begin{block}{Future Research}

Two important, interconnected, questions:
What is the connection between falling fertility and the 
use of sex selection and is the use of sex selection increasing 
or decreasing over time?

\bigskip

Broader questions:
\begin{itemize}
\item To what extent are improvements in health for girls
relative to boys the result of selection, the longer spacing between
births, or changing son preference?
\item What is the interaction between female labor force participation
and the use of sex selection?
\end{itemize}

\end{block}




\end{column}


\end{columns}

\end{frame}



\end{document}
