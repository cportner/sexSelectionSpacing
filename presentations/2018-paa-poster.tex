% Poster presentation for PAA 2018

\documentclass[final]{beamer}
  \mode<presentation>
  {
  \usetheme{CCP}
  }
  \usepackage[orientation=landscape,scale=1.2]{beamerposter}

\setlength{\paperwidth}{48in} 
\setlength{\paperheight}{36in}

\usepackage{fontspec}
\setsansfont[Ligatures=TeX]{Gandhi Sans}
% \usepackage{unicode-math}
% \setmathfont{TeX Gyre Pagella Math}

\newlength{\sepwid}
\newlength{\onecolwid}
\newlength{\twocolwid}
\newlength{\threecolwid}
\setlength{\sepwid}{0.024\paperwidth}
\setlength{\onecolwid}{0.24\paperwidth}
\setlength{\twocolwid}{0.4\paperwidth}
\setlength{\threecolwid}{0.19\paperwidth}


\title{Birth Spacing in the Presence of Son Preference and Sex-Selective \\ Abortions: India's Experience over Four Decades}
\author{Claus P\"ortner}
\institute{Albers School of Business and Economics, Seattle University, \& Center for Studies in Demography and Ecology, University of Washington}


%%% Put the name of conference here.
	\def\conference{2018 Annual Meeting of the Population Association of America} 
 %%% Put your e-mail address here.
 	\def\yourEmail{cportner@seattleu.edu}


\begin{document}

\begin{frame}{}
\begin{columns}[t]


%%%%%%%%%%%%%%%%%%%%%%%%%%%%%%%%%%%%%%%%%%%%%%%%%%%%%%%%%%%%%%%%%%%%%%%%%%%%%%%%%%%%

\begin{column}{\onecolwid}


\begin{block}{Background}

Son preference has been associated with shorter spacing after the 
births of girls than boys.

\vskip3ex

Access to prenatal sex determination fundamentally changed the 
relationship between son preference and birth spacing because each abortion 
increases birth spacing by six months to a year.

\vskip3ex

We may therefore now observe \alert{longer} spacing after daughters than sons, 
precisely because strong son preference leads to the use of sex selection.

\vskip3ex

To complicate matters, shorter birth spacing after daughters may still 
be a representation of son preference for 
families who do not use prenatal sex selection.


\end{block}

\begin{alertblock}{Contributions}

\begin{description}

\item[Main idea:]  in the absence of sex selection son preference leads to shorter 
spacing after the birth of a girl than after the birth of a boy, whereas 
son preference increases the spacing after the birth of a girl relative 
to after a boy when prenatal sex selection is available.

\item[Method:] I introduce an empirical method that directly incorporates 
the effects of sex-selective abortions on 
the duration between births
\emph{and} 
the likelihood of a son. 

\item[Central question:] how did spacing patterns changed with access
to prenatal sex determination became available?
Specifically, did the spacing after girls increase relative to the spacing
after boys?
\end{description}

\end{alertblock}


\begin{block}{Methods}


The model is a discrete time, non-proportional, competing risk 
hazard model with two exit states: either a boy or a girl is born.
\begin{description}
\item[Competing risk] captures that
parents can choose the sex of children born with access to prenatal 
sex determination.
\item[Non-proportional hazard] captures that sex composition and 
the use of sex selection likely affect the shape of the hazard 
functions across groups and that use of sex selection may vary
within a spell.
\end{description}

\end{block}

\begin{block}{Data}

The data come from the four rounds of the National Family Health Survey.
Sample is restricted to Hindu women who are married and regular residents of
the household and where all needed information are available.
Because of recall error, I drop women married 22 years or more 
for NFHS-1, with the corresponding cut-off points 23 years for NFHS-2, 
and 25 years for NFHS-3 and NFHS-4.
The final sample consists of 
  395,695 women, with   815,360 parity one through four births.
% [number] women, with [number] parity one through four births.

\end{block}



\end{column}


%%%%%%%%%%%%%%%%%%%%%%%%%%%%%%%%%%%%%%%%%%%%%%%%%%%%%%%%%%%%%%%%%%%%%%%%%%%%%%%%%%%%

\begin{column}{\twocolwid}


\begin{block}{Spell Definition}

The first spell begins at the month of marriage, whereas the second and 
subsequent spells begin nine months after the previous birth.
All spells continue until either a child is born or censoring occurs.

\vskip3ex

I group spells into four periods based on the year of marriage
or year of the previous birth:
1972--1984, 1985--1994, 1995--2004, and 2005--2016.

\end{block}

\begin{block}{Results}


The probability of ever having a next birth varies across groups.
Direct comparison of standard survival curves, therefore, tells us little 
about how the spread of sex selection affects birth spacing across groups.
I therefore condition on the predicted likelihood of parity 
progression when examining birth spacing measures.

In addition to calculating conditional birth spacing measures for different 
percentages, I also present graphs of both standard survival curves and
survival curves conditional on parity progression (which therefore
begin at 100\% and end at 0\%).

Finally, I calculate the predicted percent boys born over the entire spell.


\end{block}


\end{column}

%%%%%%%%%%%%%%%%%%%%%%%%%%%%%%%%%%%%%%%%%%%%%%%%%%%%%%%%%%%%%%%%%%%%%%%%%%%%%%%%%%%%

\begin{column}{\onecolwid}

\begin{alertblock}{Conclusion}

\begin{enumerate}
\item there has been a general increase in the length of spacing between births
over the four decades covered by the data.
The exception is for first births, where the median duration has either
remained the same or slightly declined, although this hides a
significant compression of the variation in spell length.

\item the most substantial increase in spacing is for the women who
are most likely to use sex selection.
Among the best-educated women, those with no boys now has 
significantly longer spacing---and a more male-biased sex ratio---than 
similar women with boys.

\item women with no education still follow the standard pattern of
short spacing when they have girls and little evidence of the use of sex
selection.
In other words, these women adhere to a strategy where they achieve a
son through higher fertility rather than the use of sex selection.

\item sex ratios are the most likely to decline within spells at
lower-order spells, where the pressure to provide a son is smaller, and
are more likely to increase or remain consistently high for higher-order
spells, where the pressure to ensure a son is high.

\end{enumerate}

\end{alertblock}

\begin{block}{Future Research}

The results lead to two important, interconnected, questions:
What is the connection between falling fertility and the 
use of sex selection, and is the use of sex selection increasing 
or decreasing over time?

\bigskip

There are also two broader questions:
\begin{itemize}
\item To what extent are improvements in health for girls
relative to boys the result of selection, the longer spacing between
births, or changing son preference or higher value placed on girls
with fewer of them around?
\item What is the interaction between female labor force participation
and the use of sex selection?
\end{itemize}

\end{block}

\begin{block}{Acknowledgements}

Support for development of the method from the University of Washington Royalty 
Research Fund and the Development Research Group of the World Bank is gratefully 
acknowledged.
The views and findings expressed here are those of the author and
should not be attributed to the World Bank or any of its member countries.
Partial support for this research came from a Eunice Kennedy Shriver National
Institute of Child Health and Human Development research infrastructure grant,
5R24HD042828, to the Center for Studies in Demography and Ecology at the
University of Washington.

\end{block}



\end{column}


\end{columns}

\end{frame}



\end{document}
