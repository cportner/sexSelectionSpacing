\documentclass[letterpaper,12pt]{article}

% Xetex preamble
\usepackage{fontspec}
\setromanfont[Ligatures=TeX]{TeX Gyre Pagella}
\usepackage{unicode-math}
\setmathfont{TeX Gyre Pagella Math}

\usepackage[longnamesfirst]{natbib}
%\usepackage[tabhead,nolists,tablesfirst]{endfloat}
\usepackage[flushleft]{threeparttable}
\usepackage{booktabs}
\usepackage{rotating}  
\usepackage{dcolumn} 
\usepackage{setspace}
%\usepackage{flafter} 
%\usepackage{longtable}
%\usepackage[pdftex]{graphicx}
\usepackage[xetex,colorlinks=true,linkcolor=black,citecolor=black]{hyperref} 
\usepackage[margin=1.0in]{geometry} 
\usepackage{multirow}

%opening
\title{} \author{}

\doublespacing

\begin{document}

\begin{center} \textbf{\large Birth Spacing in the Presence of Son
Preference and 
Sex-Selective Abortions: India's Experience over Four Decades}
\end{center}

\begin{center} Response to Editor and Reviewer Comments \end{center}

\noindent Please find attached the revised version of my paper,
``Birth Spacing in the Presence of Son Preference and Sex-Selective
Abortions:
India's Experience over Four Decades.''
I want to thank the three reviewers and the editor for the constructive comments and 
suggestions, which I believe substantially improved the paper.

I have substantially rewritten the paper, so before responding to the
individual comments, I provide an overview of the revised paper and the
most significant changes. 

To clarify the contribution of the paper, I now focus on three
questions. First, how has birth spacing changed over time with the
introduction of sex selection and the other significant changes in
India? Second, with substantial increases in spacing, how biased is our
standard fertility estimates? Finally, how has infant mortality changed,
and is there an effect of sex selection on infant mortality risk?

I have added four new sections: 
two analytical sections, one on fertility and one on infant mortality, 
and two background sections, one on female education and labor force 
participation and one on the conceptual framework.
I have removed the section on regional differences.

Section-by-section overview of significant changes:
\begin{enumerate}

\item I have completely rewritten the ``Introduction'' section to provide background on 
India, discuss why examining birth spacing is relevant, and highlight main results.

\item The first new section covers how female education and labor force
participation changed over time in India. Both female education and
labor force participation are factors in spacing decisions. That labor
force participation has declined even with new groups received more
schooling helps address the question of whether the expanded access to
education might change behavior. 

Because labor force participation differs by education level, I now divide the original 
eight or more years of education group into two groups: eight to eleven and twelve or
more years of education. The eight to eleven group has the lowest labor
force participation of any group, and twelve or more has one of the
highest.

\item The second new section provides a conceptual framework.
I first work through the likely general changes based on the increasing female education 
and the declining labor force participation and second through the potential effects of 
access to sex selection.
I end with a summary of the predictions.

\item I have further shortened the ``Estimation Strategy'' section and
edited it to focus on the motivation for the choice of model.

\item The Data section is also shorter, and the discussion of potential
effects of variables is now in the conceptual framework section.

\item With the new education grouping, I have completely rewritten the
section on how birth spacing has changed (the original Results section).
Instead of going through each education level in detail, I now focus
more broadly on how average birth spacing and parity progression have
changed over time and then on the effects of sex selection, drawing on
examples to illustrate the changes. The graphs are now easier to read.

\item The original subsection ``Distribution of Birth Spacing Within
Spells'' is now a new separate section ``Distribution of Birth Spacing''
and has been completely rewritten. Instead of showing the conditional
survival graphs, the section now includes figures with the 25th and 75th
percentiles for birth intervals. These figures allow me to focus on the
changes in both short and long intervals. Both to tie in with the
subsequent mortality discussion.

\item The new section, ``What has Happened to Fertility?'' looks at the
degree of tempo effect that arises from the longer spacing by comparing
a TFR-like measure with the predicted cohort fertility based on the
hazard model results.

\item The new section ``Mortality and the Changing Birth Spacing''
examines how the association between infant mortality and birth spacing
has changed over time by the education group and whether the increased
use of sex selection increases mortality risk.

\item Finally, the conclusion has been completely rewritten.
\end{enumerate}

The revised paper is just over 8,000 words per the author guidelines.



\section*{Editor}

\begin{enumerate}

\item First, greater explanation of the unique contributions of this
study is needed. As the reviewers note, the literatures in the areas of
son preference and birth spacing, for example, are extensive, and
clearer and more compelling motivations for the work should be provided.

[Response:]

To better address why and how the changes in birth spacing are relevant,
I have added two new sections. One discusses the bias that arises in
fertility estimates from the significant increases in birth spacing,
both from sex selection and from secular changes, and show that we
likely have substantially overestimated how early and fast fertility
fell in India. The other shows that mortality has fallen for all groups
no matter how long birth intervals are and that there is no apparent
adverse effect of high use of sex selection on infant mortality.


\item Second, the paper lacks a conceptual framework to organize (and
motivate) the analysis. Following on the comments of one reviewer, one
suggestion would be to focus on the role of women’s education, with a
more thorough description of the context of changing education in India
over the last half century and clearly lay out a framework for the
mechanisms through which birth spacing is affected by educational
expansion.

[Response:]

I address this in two parts.

First, I included a new descriptive section on how female education and
labor force participation has changed over time in India. I include the
discussion of labor force participation because theories and prior
results reveal it to be a crucial component in the link between female
education and birth spacing. I argue that ``Sanskritization'' is behind
the continued low labor force participation despite the expanded
schooling and the associated changes in the composition of the education
groups.
The implication is that ``Sanskritization'' likely carries over to other decisions, such 
as sex selection use.

Second, I added a conceptual framework that focuses on how birth spacing
is likely to respond in a situation where labor force participation is
low and declining, and son preference is strong. A more general theory
overview would be interesting, but that is not feasible given the word
constraint. The conceptual framework section also discusses predictions
for how access to sex selection is likely to impact birth spacing and
summarizes the expected effect of the changes in India on birth spacing.

Writing these sections also showed that my original classification of
education was too broad because labor force participation differs
markedly within the original eight-plus education group. I have,
therefore, redone all analyses in the paper with four education groups.

\item This relates to a third issue, which is space limitations and
decisions regarding what to include in the manuscript. In the last
decision letter you were asked to “reorganized to focus on the most
important issues.” Although the author has done a good job of focusing
the paper in the revision, there are still questions about lack of
detail in some places and perhaps too much in others. One suggestion is
to drop the analysis on regional differences and focus the paper more
centrally on the topic of maternal education, for example (as noted
above).

[Response:]

I agree that the regional analysis added little to the paper, and 
have deleted this section. What I consider the two most
important issues related to birth spacing are fertility and mortality.
I, therefore, now focus on how both relate to spacing, in addition to
examining how birth spacing changed.

There are other relevant questions, such as the effect on the mother
that I have ignored because of space considerations.

As part of the rewrite, I have removed any remaining discussion that
might make the paper seem more like a methodological paper. For example,
I have shortened the ``Data'' and ``Estimation Strategy'' sections and
rewritten the description of the method in the ``Introduction'' section.

\end{enumerate}

\newpage

\section*{Reviewer 1}

This paper studies birth variation in birth spacing in relation to son
preference and (access to) sex selective abortion. The data cover four
decades of birth histories in India, but have no direct information on
access to or use of sex selective abortion. Using discrete time
competing risk models, where the “risks” are the birth of a boy and the
birth of a girl, patterns of birth spacing are produced that are
plausibly in line with increasing access and use of sex selective
abortion over time and a strong positive relation between mother’s
education level and the use of sex selective abortion. While many
studies have discussed causes and consequences of the missing girls
problem due to sex selective abortion, the author(s) claim that this is
the first paper that systematically analyzes the link with birth
spacing. If this is indeed the case (I am not following this literature
close enough to be completely sure) then this is an innovative and
important study that will be of interest to readers of Demography. I
have some suggestions for changing the exposition.

\begin{enumerate}

\item The model is well explained, but the paper could be more explicit
on why a model is needed. Some of the conclusions drawn from average
probabilities, sex ratios, etc. (particularly those not conditioning on
specific values of the covariates) might also be directly visible in the
raw data. If so, the data description part could perhaps be extended.

[Response:]

Space limitations mean that I cannot add a more detailed examination of
the raw data to the data description. I have instead tried to address
this in three ways. First, I have added a discussion in the ``Estimation
Strategy'' section of why hazard models are the preferred way of dealing
with censoring. Second, I highlight that censoring of birth spells
increases with parity and time in the ``Data'' section. Third, I added a
comparison to the birth spacing numbers from the NFHS reports, which do
not attempt to deal with censoring and, therefore, have shown virtually
no change over time. 



\item Details on the estimation results are not presented. I assume
there are too many of them to be included in the paper. Still, the paper
extensively discusses model specification issues such as the
proportionality assumption and the way that is handled by splitting the
sample. I would like to see some discussion of model specification tests
informing the reader whether this is necessary and whether the resulting
flexible specifications indeed provide a good fit to the data.


[Response:]

I have tried to move away from pitching the paper as a methodological
contribution. As part of this, I shortened the empirical model
discussion and removed any mention in the ``Conclusion'' section. I
agree that a more in-depth examination of how a proportional model stack
up against the non-proportional model would be fascinating. However, I
believe this is best done separately, given the already considerable
length of the paper and the Appendix.

Furthermore, the non-proportionality assumption makes the model more
flexible than it would have been without, but cannot introduce any bias
compared to the proportional model. If the non-proportionality
assumption is not needed, the only thing affected is the efficiency of
the estimates. However, the reverse is not the case, with bias possible
if proportionality does not hold.


\item I found the analysis of geographical differences not very
exciting. It leads to several pages with not so interesting figures of
regional survival curves. I would suggest dropping the details (perhaps
make them available in an online appendix) and briefly summarize the
main findings.

[Response:]

I agree and have removed the section, as discussed above.


\item The concluding section raises some interesting points but these
are only indirectly related to the content of the paper. The development
of family planning possibilities with increasing availability and use
over time seems to be important for understanding the general decline in
fertility and lengthening of birth intervals. It is mentioned in the
introduction but not in the conclusions. The relation with labor market
participation may be of interest, but what is said about it was not
clear to me. Why would longer birth intervals disrupt labor force
participation (p. 32)

[Response:]

Unfortunately, the questions on labor force participation are not
detailed enough to allow a direct analysis using this data. I have
instead included a discussion of how female labor force participation
has changed over time in the two new background/conceptual framework
sections. Furthermore, I now directly discuss the prior literature and
theory on labor force participation and birth spacing in the ``Birth
Spacing: Mechanisms and Prior Findings'' section.

While one might expect that increasing income and female education would
also lengthen average birth spacing through better access and knowledge
of contraception, the empirical relationship between contraception use
and birth spacing is ambiguous 
\citep{Tulasidhar1993,Whitworth2002,Bhalotra2008,Yeakey2009,Kim2010,Soest2018}.
Furthermore, the ability to successfully use ``low efficacy''
contraceptive methods is increasing in education, while there is no
difference by schooling when it comes to modern contraceptives
\citep{Rosenzweig1989}.
Because there is not sufficient historical information on contraception
use in the data and that the prior literature does not provide a clear
relationship, I have not included this discussion in the paper.
\end{enumerate}

\noindent Specific comments:

\begin{enumerate}

\item p.7: eq. 2: I would call this a piecewise constant baseline hazard
rather than a piecewise linear one.

[Response:]

I have changed this throughout the paper.

\item p.7 after eq. 3: Sex composition of previous children is in
principle endogenous  if there is sex selective abortion (it is
essentially a lagged dependent variable). This is not taken into
account. This is, at least in theory, a limitation of the current model
that should be acknowledged. (I do not think it is easy to remedy
without explicitly incorporating unobserved heterogeneity.)

[Response:]

That is correct and something that would be interesting to address in a future paper.
For now, I have included a comment on this in the Estimation Strategy.



\item p.7: Why use intervals of three months? Did the author(s) perform
any sensitivity checks? 

[Response:]

I tried shorter than three months, but because the method requires
events to happen in each period, the estimations would often fail to
converge. For the same reason, I combine some three months intervals to
ensure enough information, especially when intervals are long and the
likelihood of a next birth low.



\item p.9: It might be useful to already mention that Hindus only are
analyzed in the introduction, since religion might play an important
role when choosing sex selective abortion.

[Response:]

I have included this in the rewritten ``Introduction'' section and the abstract.


\item p.10: Sterilization also seems an endogenous decision in this
context, this should be acknowledged as an additional limitation. How
often does this occur? (It might even be possible to treat this as an
additional competing risk.)

[Response:]

Across the entire sample, 36 percent of women report sterilization for
either themselves or their husbands. The decision does, however, not lend
itself easily to incorporation into the model as an additional competing
risk. First, almost nobody is sterilized after their first birth (only
around 0.5 percent), although this goes up to 19 percent after the
second birth, and 27 percent after the third. Second, a majority---often
up to 75 percent---of sterilizations take place in the same month as the
last birth or within eight months of it. Most of the sterilizations are,
therefore, never included in the model samples because those begin at
nine months after the prior birth. Hence, the absolute numbers are too
small for the competing risk model to work correctly, primarily because
the likelihood of sterilization depends strongly on the sex composition
of prior children. For a given parity, the fewer boys, the lower the
probability. 

The results presented here are conditional on a prior birth and no
sterilization within the nine months after the preceding birth. The main
concern is that the parity progression probabilities are too high
because they do not account for sterilization. There are two impacts of
this. 

First, it biases downward the differences in the parity progression
probabilities across sex compositions. I discuss this in the
``Data'' section. 

Second, the predicted cohort fertility would be too high. I, therefore,
estimate the likelihood of sterilization within the first nine months
using a Logit model (results not shown in the paper). I then predict the
probability of not getting sterilized and use that to scale down the
parity progression probability when predicting cohort fertility. In
practice, the effect of the scaling is relatively small, 0.05 to 0.25,
which is consistent with the lowest use among women who have the highest
estimated parity progression probabilities based on the hazard model.



\item p.9: why put the marriage duration threshold at 22 years?
Sensitivity check?

[Response:]

I discuss this in the Appendix section ``Recall Error and the Sex
Ratio''. Since most of the surveys start showing significantly biased
sex ratio from around 22 years of marriage on, I drop all observations
where the wedding took place 22 years or more. Changes of a couple of
years more or less do not substantially change the results. However, a
cut significantly higher than 22 does eventually make the sex ratio very
uneven in the first period, even for less-educated women. There are no
effects on the more recent periods. 



\item Results: The many small figures are hard to read and I did not
find them an attractive way to present the results. Perhaps the
author(s) can come up with an alternative?

[Response:]

I have tried to address this by increasing the font size for the legends
and turning the graphs sideways. Originally the results were presented
in table format (now available in the Appendix). Prior referees found it
hard to get an overview of the results that way, and, on balance, I
agree with that. Hence, I have kept the graphs for the spacing
discussions. For the fertility section, I present the results as a
table.



\end{enumerate}

\newpage

\section*{Reviewer 2}

This study uses four waves of the National Family and Health Surveys to
examine the interrelationship between son preference, sex ratios at
birth, and birth spacing in India, how this varies between different
regions, by parity and the sex composition of the children in the
household, by maternal educational level, and over time.

\begin{enumerate}

\item On the one hand this feels like a clear and polished piece of
research. On the other hand, I wonder how much we can learn from this
study given that there is already an extensive body of research that
examines son preference and SRBs in India, including how this varies by
parity, sex composition of the children already in the household,
maternal educational level, and over time. It seems to me that the
question is, what is the net contribution of adding birth spacing as an
additional dimension to this whole discussion.

The authors attempt to address this point by offering four motivations
for why we should care about spacing as an additional dimension. First,
they say that “researchers have made extensive use of birth spacing as a
measure of son preference, and it is critical to understand to what
extent spacing is still a useful measure of son preference” (page 2). To
my mind, the authors do not offer a particularly compelling explanation
anywhere in the paper for why it is ‘critical to understand’ whether
spacing remains a useful measure of son preference. Understanding son
preference and how that is linked to sex-selective abortions and
differential stopping behaviour (and gender dynamics more generally), is
indeed critical – but I don’t really see that the critical importance of
understanding spacing in relation to this.

[Response:]

I have reframed the paper to be about birth spacing more
generically (see also my response below) and tied it to two issues where
changes in spacing have direct relevance: fertility estimates and infant mortality.
I have removed most of the discussion of spacing as a measure of son preference,
except to note that we now have cases where son preference results in longer
spacing without boys than with because of sex selection.

% What also makes birth spacing in India interesting is that opposite of what we
% have seen elsewhere, birth spacing has increased over time with development.
% The increases is especially pronounced because of the use of sex selection.


\item The second and third motivations offered by the authors are the
potential consequences of (short) birth spacing for maternal health,
child health, and long-term child development. For example, the authors
cite literature showing that short spacing can have severe consequences
for child outcomes, particularly in low-income settings. However,
understanding the potential consequences of birth interval length is
much more confusing when long birth intervals are a consequence of
(sex-selective) abortions. One reason why short birth intervals have a
negative effect on child outcomes is because of maternal nutrient
depletion and insufficient time to recover from the previous pregnancy.
Children born after long birth intervals where those intervals have been
punctuated by multiple abortions will not see the same benefits as a
child born after a long interval that is not punctuated by abortions
since the mother would not be fully recovering in that intervening
period. Likewise the costs and benefits for mothers themselves of
short/long spacing may be overwhelmed by the effects of the abortions
and interrupted pregnancies.

[Response:]

Thank you for this fascinating point on the potential adverse effects of
the increased use of abortions. I have included this as one of the two
main reasons why the changes in spacing are worth examining and have
introduced a new section on infant mortality and how it varies by
spacing and sex composition over time. There is, however, no apparent
adverse effect of being a son born after girls and with longer spacing
in situations with high use of sex selection. Hence, if there is a
negative effect of multiple abortions, it is more than countered by the
protective effect of higher maternal education. It is still possible
that we would see more of an impact among less-educated women, but this
group is, currently, not heavy users of sex selection.

To focus the paper, I no longer discuss the potential effects on
mothers, although I still find that an exciting topic for future
analysis. The same goes for the long-term development of children.



\item The fourth and final motivation offered is that we don’t know what
determines spacing behaviour in low- and middle-income countries.
However, the arguments that the authors develop here are more about 
contraceptive use, declining fertility, and women entering the labor market. 
Those factors don’t seem to be addressed elsewhere in the paper at all.

[Response:]

As mentioned above, I now include a discussion of the changing spacing's effect on
fertility estimates.
Similarly, although I cannot directly examine the relationship between labor force
participation and spacing because of data limitations, I do discuss the low and 
falling labor force participation in India and the effects this may have on spacing.
I have removed the discussion on contraceptive use (see also my response above).


\item In the end, this study offers a careful descriptive account of how birth
spacing has changed alongside sex ratios at birth, but little more. We
should indeed care about sex ratios at birth – and people do – but there
are many studies published on that topic, and it is not clear that this
one adds something important to that particular debate. If it does, the
authors have not articulated that clearly. I think that understanding
the consequences of birth spacing for maternal and infant health is also
extremely important. But the authors do not study that link empirically.
For these reasons, I find the contribution of the study rather marginal
and confusing.

[Response:]

As described above, I have substantially rewritten the paper and
introduced new analyses to illustrate better how and why examining birth
spacing is necessary, including linking birth spacing and infant
mortality directly. I am, however, open for suggestions for alternative
ways to focus the paper and the motivations.


\end{enumerate}

\newpage

\section*{Reviewer 3}

The author(s) have considered and addressed reviewers’ comments and have
revised the paper thoughtfully. The changes to the presentation of the
results and added explanations are especially well done. There is much
to be liked in the revised version. While the revised paper is stronger
than the earlier version, there are still some issues that require
further attention.

\begin{enumerate}

\item The paper still lacks a stronger narrative on the motivation and
framework for the empirical work. Four reasons are mentioned as
motivations; but, with the exception of the first, all other reasons,
such as the impact of spacing on child and mother’s health, are not
really motivating the empirical work of this paper (they are rather
evidence for importance of spacing).

[Response:]

I have entirely rewritten the paper to focus on how and why examining
birth spacing is relevant, as described above. As part of that rewrite,
I now estimate the relationship between spacing and infant mortality,
mainly to understand whether the increased use of sex selection may
increase mortality risk. Although I still find the effects on the
mothers' health a fascinating question, I have decided not to discuss
this effect because of lack of space.
 

\item The conceptual framework is not fully developed. The paper suggests
possible explanations at various places but they don’t add up to a
coherent narrative. For instance, it is not clear whether maternal
education acts through access for sex selective technology or preference
for sons. The core idea of the paper that sex selection changes the
relationship between spacing and son preference needs to be developed
further. There is significant amount of good empirical work, but without
proper development of framework it is difficult to evaluate the
contribution and understand the findings, and also to see its
contribution beyond the Indian case.

As an illustration, the paper looks at changes over time during which
the access the sex selection has become increasing difficult AND also
almost all groups (by education, religion…) have seen an decline in
fertility. How do we consider the impact of these on spacing, sex
selection and sex ratios? How do we put the findings in the context of
changing educational attainments of women? The paper doesn’t provide us
a framework to think about the broader changes during the time period
under consideration.

[Response:]

The second of the two new background sections draw together the
different theories on the relationship between maternal education, labor
force participation, birth spacing, and use of sex selection, and use
those to guide the hypothesized changes.

To address the broader changes in India, the first of the two new
background sections show how female education has increased, while
female labor force participation has decreased. I argue that the low
female labor force participation is an indication of
``Sanskritization''. The pool of educated women has expanded, but the
behavior of the group does not change much.

I also show that although fertility fell, it is higher than generally
acknowledged, precisely because longer birth intervals biases downward
our standard fertility measures.


\item I’m not fully convinced of the main conclusions of a “substantial
increase in the use of sex selection” or of an “an almost complete
reversal of the traditional spacing patterns” for urban educated women.
They don’t seem to be supported by the findings. For urban women this
seems to apply only for third spell. The substantial increase in sex
selection is also not evident from the findings for all groups or for
all spells. The paper does acknowledge the complexity, but at places
(including in the abstract) seems to stretch the interpretation beyond
what is shown in the findings. The conclusions are presented in way that
implies that they apply to different spells or groups. But the findings
are more limited and differ clearly by spells. This should be presented
accurately in the text.

[Response:]

That was, indeed, an overstatement of the results. I now discuss the
three possible outcomes---no sex selection, sex selection with no change
in relative order by sex composition, and sex selection with a change in
the order---and provide examples of each. 

\item While I really liked the methodology and visual presentation of the
findings, in my reading, I don’t see the paper’s contribution to
methodology. Though the paper makes a claim for methodological novely, I
see it more as an adoption of a widely used technique to spacing rather
than a proposing a new technique. Despite the changes, the paper still
reads as a piece demonstrating a methodology rather than signficant
contribution to the demogrpahic literature.

[Response:]

I have tried to move away from pitching the paper as a methodological
contribution. As part of this, I shortened the empirical model
discussion and removed any mention in the ``Conclusion'' section.
I hope that the two new analytical sections and the two background sections 
make the paper more of a definite contribution to the literature.



\item Minor: It might be a disciplinary style, I would recommend cutting down
on the footnotes. 

[Response:]

I have removed or incorporated into the text a large number of footnotes.


\item I would also not go into the effect of child spacing
in countries that are very dissimilar to the context of India, unless
there is a wider point being made. 

[Response:]

I have removed this discussion.


\item Some points are not clear: for
instance, “On the other hand, increased reliability of access and
effectiveness of contraceptives can lead to shorter intervals between
births if women used to have longer spacing to avoid having too many
children by accident (Keyfitz, 1971; Heckman and Willis, 1976).”

[Response:]

This discussion is no longer in the paper because of the new conceptual
framework. If other parts are unclear, please let me know.

\end{enumerate}


\newpage
\bibliographystyle{econometrica}
\bibliography{../paper/sex_selection_spacing.bib}
\addcontentsline{toc}{section}{References}


\end{document}
