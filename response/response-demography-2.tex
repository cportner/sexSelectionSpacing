\documentclass[letterpaper,12pt]{article}

% Xetex preamble
\usepackage{fontspec}
\setromanfont[Ligatures=TeX]{TeX Gyre Pagella}
\usepackage{unicode-math}
\setmathfont{TeX Gyre Pagella Math}

\usepackage[longnamesfirst]{natbib}
%\usepackage[tabhead,nolists,tablesfirst]{endfloat}
\usepackage[flushleft]{threeparttable}
\usepackage{booktabs}
\usepackage{rotating}  
\usepackage{dcolumn} 
\usepackage{setspace}
%\usepackage{flafter} 
%\usepackage{longtable}
%\usepackage[pdftex]{graphicx}
\usepackage[xetex,colorlinks=true,linkcolor=black,citecolor=black]{
hyperref} 
\usepackage[margin=1.0in]{geometry} 
\usepackage{multirow}

%opening
\title{} \author{}

\doublespacing

\begin{document}

\begin{center} \textbf{\large Birth Spacing in the Presence of Son
Preference and 
Sex-Selective Abortions: India's Experience over Four Decades}
\end{center}

\begin{center} Response to Editor and Reviewer Comments \end{center}

\noindent Please find attached the revised version of my paper,
``Birth Spacing in the Presence of Son Preference and Sex-Selective
Abortions:
India's Experience over Four Decades.''
I want to thank the three reviewers and the editor for the very helpful comments and 
suggestions, which I believe substantially improved the paper.

\section*{Editor}

\begin{enumerate}

\item First, greater explanation of the unique contributions of this
study is needed. As the reviewers note, the literatures in the areas of
son preference and birth spacing, for example, are extensive, and
clearer and more compelling motivations for the work should be provided.

[Response:]

\item Second, the paper lacks a conceptual framework to organize (and
motivate) the analysis. Following on the comments of one reviewer, one
suggestion would be to focus on the role of women’s education, with a
more thorough description of the context of changing education in India
over the last half century and clearly lay out a framework for the
mechanisms through which birth spacing is affected by educational
expansion.

[Response:]

\item This relates to a third issue, which is space limitations and
decisions regarding what to include in the manuscript. In the last
decision letter you were asked to “reorganized to focus on the most
important issues.” Although the author has done a good job of focusing
the paper in the revision, there are still questions about lack of
detail in some places and perhaps too much in others. One suggestion is
to drop the analysis on regional differences and focus the paper more
centrally on the topic of maternal education, for example (as noted
above).

[Response:]


\end{enumerate}

\newpage

\section*{Reviewer 1}

This paper studies birth variation in birth spacing in relation to son
preference and (access to) sex selective abortion. The data cover four
decades of birth histories in India, but have no direct information on
access to or use of sex selective abortion. Using discrete time
competing risk models, where the “risks” are the birth of a boy and the
birth of a girl, patterns of birth spacing are produced that are
plausibly in line with increasing access and use of sex selective
abortion over time and a strong positive relation between mother’s
education level and the use of sex selective abortion. While many
studies have discussed causes and consequences of the missing girls
problem due to sex selective abortion, the author(s) claim that this is
the first paper that systematically analyzes the link with birth
spacing. If this is indeed the case (I am not following this literature
close enough to be completely sure) then this is an innovative and
important study that will be of interest to readers of Demography. I
have some suggestions for changing the exposition.

\begin{enumerate}

\item The model is well explained, but the paper could be more explicit
on why a model is needed. Some of the conclusions drawn from average
probabilities, sex ratios, etc. (particularly those not conditioning on
specific values of the covariates) might also be directly visible in the
raw data. If so, the data description part could perhaps be extended.

[Response:]


\item Details on the estimation results are not presented. I assume
there are too many of them to be included in the paper. Still, the paper
extensively discusses model specification issues such as the
proportionality assumption and the way that is handled by splitting the
sample. I would like to see some discussion of model specification tests
informing the reader whether this is necessary and whether the resulting
flexible specifications indeed provide a good fit to the data.


[Response:]



\item I found the analysis of geographical differences not very
exciting. It leads to several pages with not so interesting figures of
regional survival curves. I would suggest dropping the details (perhaps
make them available in an online appendix) and briefly summarize the
main findings.

[Response:]




\item The concluding section raises some interesting points but these
are only indirectly related to the content of the paper. The development
of family planning possibilities with increasing availability and use
over time seems to be important for understanding the general decline in
fertility and lengthening of birth intervals. It is mentioned in the
introduction but not in the conclusions. The relation with labor market
participation may be of interest, but what is said about it was not
clear to me. Why would longer birth intervals disrupt labor force
participation (p. 32)

[Response:]




Specific comments:


\item p.7: eq. 2: I would call this a piecewise constant baseline hazard
rather than a piecewise linear one.

[Response:]



\item p.7 after eq. 3: Sex composition of previous children is in
principle endogenous  if there is sex selective abortion (it is
essentially a lagged dependent variable). This is not taken into
account. This is, at least in theory, a limitation of the current model
that should be acknowledged. (I do not think it is easy to remedy
without explicitly incorporating unobserved heterogeneity.)

[Response:]




\item p.7: Why use intervals of three months? Did the author(s) perform
any sensitivity checks? 

[Response:]




\item p.9: It might be useful to already mention that Hindus only are
analyzed in the introduction, since religion might play an important
role when choosing sex selective abortion.

[Response:]




\item p.10: Sterilization also seems an endogenous decision in this
context, this should be acknowledged as an additional limitation. How
often does this occur? (It might even be possible to treat this as an
additional competing risk.)

[Response:]




\item p.9: why put the marriage duration threshold at 22 years?
Sensitivity check?

[Response:]



\item Results: The many small figures are hard to read and I did not
find them an attractive way to present the results. Perhaps the
author(s) can come up with an alternative?

[Response:]




\end{enumerate}

\newpage

\section*{Reviewer 2}

\begin{enumerate}

This study uses four waves of the National Family and Health Surveys to
examine the interrelationship between son preference, sex ratios at
birth, and birth spacing in India, how this varies between different
regions, by parity and the sex composition of the children in the
household, by maternal educational level, and over time.


\item On the one hand this feels like a clear and polished piece of
research. On the other hand, I wonder how much we can learn from this
study given that there is already an extensive body of research that
examines son preference and SRBs in India, including how this varies by
parity, sex composition of the children already in the household,
maternal educational level, and over time. It seems to me that the
question is, what is the net contribution of adding birth spacing as an
additional dimension to this whole discussion.

[Response:]


\item The authors attempt to address this point by offering four motivations
for why we should care about spacing as an additional dimension. First,
they say that “researchers have made extensive use of birth spacing as a
measure of son preference, and it is critical to understand to what
extent spacing is still a useful measure of son preference” (page 2). To
my mind, the authors do not offer a particularly compelling explanation
anywhere in the paper for why it is ‘critical to understand’ whether
spacing remains a useful measure of son preference. Understanding son
preference and how that is linked to sex-selective abortions and
differential stopping behaviour (and gender dynamics more generally), is
indeed critical – but I don’t really see that the critical importance of
understanding spacing in relation to this.

[Response:]


\item The second and third motivations offered by the authors are the
potential consequences of (short) birth spacing for maternal health,
child health, and long-term child development. For example, the authors
cite literature showing that short spacing can have severe consequences
for child outcomes, particularly in low-income settings. However,
understanding the potential consequences of birth interval length is
much more confusing when long birth intervals are a consequence of
(sex-selective) abortions. One reason why short birth intervals have a
negative effect on child outcomes is because of maternal nutrient
depletion and insufficient time to recover from the previous pregnancy.
Children born after long birth intervals where those intervals have been
punctuated by multiple abortions will not see the same benefits as a
child born after a long interval that is not punctuated by abortions
since the mother would not be fully recovering in that intervening
period. Likewise the costs and benefits for mothers themselves of
short/long spacing may be overwhelmed by the effects of the abortions
and interrupted pregnancies.

[Response:]


\item The fourth and final motivation offered is that we don’t know what
determines spacing behaviour in low- and middle-income countries.
However, the arguments that the authors develop here are more about 
contraceptive use, declining fertility, and women entering the labor market. 
Those factors don’t seem to be addressed elsewhere in the paper at all.

[Response:]


\item In the end, this study offers a careful descriptive account of how birth
spacing has changed alongside sex ratios at birth, but little more. We
should indeed care about sex ratios at birth – and people do – but there
are many studies published on that topic, and it is not clear that this
one adds something important to that particular debate. If it does, the
authors have not articulated that clearly. I think that understanding
the consequences of birth spacing for maternal and infant health is also
extremely important. But the authors do not study that link empirically.
For these reasons, I find the contribution of the study rather marginal
and confusing.

[Response:]


\end{enumerate}

\newpage

\section*{Reviewer 3}

\begin{enumerate}

The author(s) have considered and addressed reviewers’ comments and have
revised the paper thoughtfully. The changes to the presentation of the
results and added explanations are especially well done. There is much
to be liked in the revised version. While the revised paper is stronger
than the earlier version, there are still some issues that require
further attention.

\item The paper still lacks a stronger narrative on the motivation and
framework for the empirical work. Four reasons are mentioned as
motivations; but, with the exception of the first, all other reasons,
such as the impact of spacing on child and mother’s health, are not
really motivating the empirical work of this paper (they are rather
evidence for importance of spacing).

\item The conceptual framework is not fully developed. The paper suggests
possible explanations at various places but they don’t add up to a
coherent narrative. For instance, it is not clear whether maternal
education acts through access for sex selective technology or preference
for sons. The core idea of the paper that sex selection changes the
relationship between spacing and son preference needs to be developed
further. There is significant amount of good empirical work, but without
proper development of framework it is difficult to evaluate the
contribution and understand the findings, and also to see its
contribution beyond the Indian case.

\item As an illustration, the paper looks at changes over time during which
the access the sex selection has become increasing difficult AND also
almost all groups (by education, religion…) have seen an decline in
fertility. How do we consider the impact of these on spacing, sex
selection and sex ratios? How do we put the findings in the context of
changing educational attainments of women? The paper doesn’t provide us
a framework to think about the broader changes during the time period
under consideration.

\item I’m not fully convinced of the main conclusions of a “substantial
increase in the use of sex selection” or of an “an almost complete
reversal of the traditional spacing patterns” for urban educated women.
They don’t seem to be supported by the findings. For urban women this
seems to apply only for third spell. The substantial increase in sex
selection is also not evident from the findings for all groups or for
all spells. The paper does acknowledge the complexity, but at places
(including in the abstract) seems to stretch the interpretation beyond
what is shown in the findings. The conclusions are presented in way that
implies that they apply to different spells or groups. But the findings
are more limited and differ clearly by spells. This should be presented
accurately in the text.

\item While I really liked the methodology and visual presentation of the
findings, in my reading, I don’t see the paper’s contribution to
methodology. Though the paper makes a claim for methodological novely, I
see it more as an adoption of a widely used technique to spacing rather
than a proposing a new technique. Despite the changes, the paper still
reads as a piece demonstrating a methodology rather than signficant
contribution to the demogrpahic literature.

\item Minor: It might be a disciplinary style, I would recommend cutting down
on the footnotes. I would also not go into the effect of child spacing
in countries that are very dissimilar to the context of India, unless
there is a wider point being made. Some points are not clear: for
instance, “On the other hand, increased reliability of access and
effectiveness of contraceptives can lead to shorter intervals between
births if women used to have longer spacing to avoid having too many
children by accident (Keyfitz, 1971; Heckman and Willis, 1976).”

\end{enumerate}


\newpage
\bibliographystyle{econometrica}
\bibliography{../paper/sex_selection_spacing.bib}
\addcontentsline{toc}{section}{References}


\end{document}
