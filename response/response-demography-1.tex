\documentclass[letterpaper,12pt]{article}

% Xetex preamble
\usepackage{fontspec}
\setromanfont[Ligatures=TeX]{TeX Gyre Pagella}
\usepackage{unicode-math}
\setmathfont{TeX Gyre Pagella Math}

\usepackage[longnamesfirst]{natbib}
%\usepackage[tabhead,nolists,tablesfirst]{endfloat}
\usepackage[flushleft]{threeparttable}
\usepackage{booktabs}
\usepackage{rotating}  
\usepackage{dcolumn} 
\usepackage{setspace}
%\usepackage{flafter} 
%\usepackage{longtable}
%\usepackage[pdftex]{graphicx}
\usepackage[xetex,colorlinks=true,linkcolor=black,citecolor=black]{
hyperref} 
\usepackage[margin=1.0in]{geometry} 
\usepackage{multirow}

%opening
\title{} \author{}

\doublespacing

\begin{document}

\begin{center} \textbf{\large Birth Spacing in the Presence of Son
Preference and 
Sex-Selective Abortions: India's Experience over Four Decades}
\end{center}

\begin{center} Response to Editor and Referee Comments \end{center}

\noindent Please find attached the revised version of my paper,
``Birth Spacing in the Presence of Son Preference and Sex-Selective
Abortions:
India's Experience over Four Decades.''
I would like to thank the two referees and the editor for the very useful comments and 
suggestions, which I believe substantially improved the paper.

\section*{Editor}

First, the description of the problem and literature review are sparse.
The front end of the paper should be revised to situate the study in the
wider literature on son preference and sex selection in the region. The
contributions of the current study should be highlighted. R1 offered
particularly useful ideas in this regard. These changes would make the
paper appeal to the broad readership of Demography.

[Response:]

- Spacing as an indicator for son preference
- The relationship between spacing and child health
- The relationship between spacing and women's health
- Important for understanding how spacing in general behaves. See last point from
editor: if lower fertility, better health, and increased female labor market participation 
are all is associated with a compression of spacing, the use of sex selection would
work against this trend and make it look like India is not behaving like other countries.
- Women's role in general

Second, a conceptual framework or theoretical argument is needed to
motivate the decision to examine birth intervals and sex composition by
maternal education. Does maternal education proxy for access to sex
selective technology, preferences for sons, etc.? In addition, how does
selection into the highest/lowest educated categories change over the 45
years of the study? How could this impact the interpretation of results?

[Response:]

Third, the paper is heavy on the description of methods and results.
These sections could be shortened and reorganized to focus on the most
important issues. R1 also offered suggestions regarding the presentation
of findings.

[Response:]

Fourth, discussion of the fertility transition is neglected and only
discussed in the conclusion. At a minimum, you should consider providing
details on fertility levels and distributions across parities over time.
Is there an easy way to show the sample sizes/proportion of births in
each category of sex composition across years in tables 1-3? More could
be said about the potential impact of the ongoing fertility decline.

[Response:]

[Generically, spacing tend to decline with lower fertility (check this). 
Presumably the decline in spacing is because part of the cause is higher female labor 
market participation, which makes it more attractive to space children closer together.
Not clear whether this is the case for India - The R1 suggested paper might be a good
starting point with a tie-in to compression.
Compression can also come from improved health and/or less migration.

On the flip side, lower fertility place additional pressure on women to use
sex selection, which lead to *less* compression.
]

\newpage

\section*{Reviewer 1}


1. The key contribution of the paper is the use of birth spacing and sex
ratios to examine changes and differentials in possible sex-selective
abortions. While there is large literature using differential stopping
behaviours (DSB) and sex ratios to examine sex selection, the use of
birth spacing is new and could potentially provide new insights. The
methodology used is rigorous and suited for the purposes of the paper.

2. The paper would need to articulate its contribution clearly. The
paper seems to be focussed too much on methodology and less on
contextual, theoretical and substantive matters. As there is already
well established literature on sex-selection and son preference, the
paper must articulate what is substantially different in the paper. As
one of the key features is the focus on spacing, expanded review of the
literature on spacing is required.

[Response:]

3. The literature review must be systematic. The paper seems to move too
quickly to estimation strategy. Before presenting this, the literature
on spacing, sex ratios, sexselection should be presented. This would
help the paper to make a stronger case of its place and contribution to
the demographic literature.

[Response:]

[See response to Editor]


4. The demographic literature, for example, by works from Christophe Z
Guilmoto, doesn’t seem to be consulted. Guilmoto and colleagues many
papers including recent work in Population Studies and Lancet are
important to consult as they cover the same substantial issues of sex
ratios, sex selection and abortions as addressed in this paper. Though
these demographic papers do not necessarily use spacing, a similar
understanding of sex-selection could be obtained using examining parity
progression. Therefore, the added insights from using spacing need to
articulated much more forcefully.

[Response:]

[I should have those somewhere already. I think they were in the original version.]

5. The use of competing hazard models is not new in demographic research
and doesn’t require long exposition. This section can be cut
sustainably. I don’t think the methodological innovation is a key
contribution of the paper as the methods are established even if they
have not been used to examine spacing.

[Response:]

I have reduced the length of the section by more than 1/4 (the original section was 5
pages long, while the revised section is less than 3.5).
There are three important differences between my approach and what is standard
in the literature, and I believe further shortening would negatively impact readers'
ability to understand the results.
First, most research that use survival models employ proportional hazard models, so 
motivating and providing more information on the non-proportional set-up is important.
Second, the use of conditional survival curves is, to my knowledge, new, and, therefore,
requires a detailed discussion.
Third, the sex ratio calculations is a new way to use competing risk estimates and is 
required to understand whether the use of sex selection vary within spells.


6. The use of NFHS data including the most recent data are welcome. An
additional limitation to those addressed in the paper is the issue of
date imputation in DHS/NFHS. NFHS as with other DHS data routinely use
imputations for year and month of birth. I don’t have the reference for
India (please check data quality reports on DHS website), but the
imputation of month of birth was done for large number of cases in
earlier rounds of DHS. This could be a possible reason for seeming large
number of pre-marital conceptions which do not make sense in the context
of India.

[Response:]

[Maybe run regression only on those birth intervals that does not
involve imputations on either end of the spacing.
This could be presented in the Appendix, unless very different from
results.]

7. It is not clear why exactly 22, 23, 25 years are used a cut-off
points.

[Response:]
I have added a detailed discussion of how the cut-offs are determined in one of the
Appendices intended for online publication.


8. I understand the empirical reason for not increasing the number of
variables. However, a key variable that is essential to consider is
region. This is important because regional influences on birth spacing
and sex selection are very strong as established in the literature. Sex
selection and abortion is especially strong in certain regions but not
in others. The regional differentials in fertility transition,
contraceptive use, spacing, sex ratios, abortions are clearly documented
in the literature. So it is vital to include this.

[Response:]

[Check for broad regions to incorporate. Check how to do this for the non-proportional
hazard specification.]

9. The three tables are logically organized and contain a wealth of
information. But there are too many comparisons to make in each of the
tables and I wonder if it would make sense to split the tables or
present some of the information in a much more reader friendly way (for
instance by using charts for some of the information). 

[Response:]

10. As sex composition and sex ratios has been examined in the literature, I think
it would make sense to not present findings on that in great detail. As
the paper states, the central question addressed is spacing, and so all
other questions should not be dealt in detail

[Response:]

11. The results talk about compression of spacing. This is important and
literature on it should be presented in the intro or lit review section.
There is evidence of compression of the reproductive span at least in
some regions (see Padmadas, et. Al 2004). It would be important to
consider how this fits in with the results presented.

[Response:]

[Compression is only for first birth as far as I remember. Need to check that.
Check on paper for their analysis.]


12. As there are many results presented, it would make sense to have a
paragraph at the beginning of the conclusion highlighting the key
findings.

[Response:]

[Add discussion when other parts are done]

13. Overall, the paper is empirically strong but needs to build on the
contextual, theoretical, and prior demographic literature. As the
readers of Demography are diverse, it would be good to have a balanced
presentation of all the sections. Currently it seems the paper is too
concerned with the methodology and empirical work at the expense of
other areas.

\newpage

\section*{Reviewer 7}

The article was a careful, innovative, and thorough piece of research
picking up on a conundrum concerning how son preference and sex
selection affect birth intervals. I very much enjoyed reading this piece
and think it has useful applications outside of India.  My review is
short as I do not see much to be improved upon.  My only (minor)
suggestion concerns the use of the word "spell".  It is very clear what
you mean by a spell when the paper explains this, but I think that it is
somewhat confusing to put this in the abstract without an explanation. 
It was not until I got to the explanation in the methods that I
understood the abstract fully.  I suggest rewording this section of the
abstract so it is clear what is meant by spell.

[Response:]



\newpage
\bibliographystyle{econometrica}
\bibliography{../paper/sex_selection_spacing.bib}
\addcontentsline{toc}{section}{References}


\end{document}
