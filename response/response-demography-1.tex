\documentclass[letterpaper,12pt]{article}

% Xetex preamble
\usepackage{fontspec}
\setromanfont[Ligatures=TeX]{TeX Gyre Pagella}
\usepackage{unicode-math}
\setmathfont{TeX Gyre Pagella Math}

\usepackage[longnamesfirst]{natbib}
%\usepackage[tabhead,nolists,tablesfirst]{endfloat}
\usepackage[flushleft]{threeparttable}
\usepackage{booktabs}
\usepackage{rotating}  
\usepackage{dcolumn} 
\usepackage{setspace}
%\usepackage{flafter} 
%\usepackage{longtable}
%\usepackage[pdftex]{graphicx}
\usepackage[xetex,colorlinks=true,linkcolor=black,citecolor=black]{
hyperref} 
\usepackage[margin=1.0in]{geometry} 
\usepackage{multirow}

%opening
\title{} \author{}

\doublespacing

\begin{document}

\begin{center} \textbf{\large Birth Spacing in the Presence of Son
Preference and 
Sex-Selective Abortions: India's Experience over Four Decades}
\end{center}

\begin{center} Response to Editor and Reviewer Comments \end{center}

\noindent Please find attached the revised version of my paper,
``Birth Spacing in the Presence of Son Preference and Sex-Selective
Abortions:
India's Experience over Four Decades.''
I want to thank the two reviewers and the editor for the very helpful comments and 
suggestions, which I believe substantially improved the paper.

\section*{Editor}

\begin{enumerate}

\item First, the description of the problem and literature review are sparse.
The front end of the paper should be revised to situate the study in the
wider literature on son preference and sex selection in the region. The
contributions of the current study should be highlighted. R1 offered
particularly useful ideas in this regard. These changes would make the
paper appeal to the broad readership of Demography.

[Response:]

I have rewritten the Introduction entirely to situate the paper in the literature 
and better highlight the contributions of the paper.
I now cover the research on spacing as a measure of son preference, the 
relationship between spacing and child health (both older and younger child), the 
effect on women's health and other outcomes, and, finally, the limited and mixed results
on what determines birth spacing behavior in developing countries.
I use these to motivate why it is essential to examine how birth spacing has been affected 
by the use of sex selection, and, in addition, highlight the three main questions I address 
in the paper: the effect of sex selection on birth intervals, the distribution of births 
within spells, and, finally, differences across geographical regions.


\item Second, a conceptual framework or theoretical argument is needed to
motivate the decision to examine birth intervals and sex composition by
maternal education. Does maternal education proxy for access to sex
selective technology, preferences for sons, etc.? 
In addition, how does
selection into the highest/lowest educated categories change over the 45
years of the study? How could this impact the interpretation of results?

[Response:]

I have expanded the discussion of the choice of non-proportionality variables in the 
``Explanatory Variables'' section (and added a footnote to that effect in the ``Estimation 
Strategy'' section).
The discussion now covers education, sex composition, and area of residence in more detail,
although given that I need to keep the paper below 8,000 words there is a limit to how
detailed the discussion is.
If there is a more appropriate place for this discussion, I am happy to move it.

When it comes to selection, I see two possible explanations: one economic and one 
cultural.
In the ``economic'' explanation, well-educated women's use of sex selection is mainly 
driven by the lower fertility that comes from their higher opportunity cost of time and 
the higher income that allow them to use sex selection.
In that case, there is no selection bias since the use of sex selection is driven by 
factors that are identical across those with the same level of education.

My ``cultural'' explanation assumes that, originally, those with higher education were also 
those groups with the strongest son preference (presumably also higher caste).
The expansion of access to education brought other groups/castes into connection with
the higher castes and led them to assimilate the high caste son preference norms.
Presumably, this assimilation would not be complete, and those groups newer to higher 
education would, therefore, have lower son preference and therefore lower use of sex 
selection.
Hence, any appearance of lower use of sex selection could be a selection story,
and that the initial group's son preference is just as strong.
The cultural explanation also suggests a reason for why the stated son preference appears 
to decline over time: the ``newer'' entrants have lower initial son preference and, 
therefore, the average son preference for high education women will decline over time
\citep{bhat03,pande07}.

Since I have no way of separating these two possible explanations with the available
data and set up and because of the need to keep the paper short, I have not included this 
discussion in the paper itself. 
One possible approach would be to include this as a question for future research in the
Conclusion.
If you think it will make the paper stronger, I am, of course, open to including the
discussion there or anywhere else it might fit better.


\item Third, the paper is heavy on the description of methods and results.
These sections could be shortened and reorganized to focus on the most
important issues. R1 also offered suggestions regarding the presentation
of findings.

[Response:]

As detailed below in my response to Reviewer 1, I have substantially shortened the 
``Estimation Strategy'' section.
The results section now focuses on the spacing results, guided by the three main 
questions.
Also, I have removed the section on how the use of sex-selective abortions changes 
within a spell since it, as originally written, was more focused on sex selection than
the spacing, and because I can better address the spacing question in the section on the 
distribution of births within spells.
I still need to discuss how sex selection has changed to tie those to the 
changes in spacing, but I have tried to make these result mainly serve the overall
analysis of spacing.
In response to Reviewer 1's comments, I have added a section on regional differences,
using information on well-educated women.


\item Fourth, discussion of the fertility transition is neglected and only
discussed in the conclusion. At a minimum, you should consider providing
details on fertility levels and distributions across parities over time.
Is there an easy way to show the sample sizes/proportion of births in
each category of sex composition across years in tables 1-3? More could
be said about the potential impact of the ongoing fertility decline.

[Response:]

I address fertility changes three different ways.
First, in the Introduction, I discuss the decline in fertility when introducing why India 
is a compelling case, with a particular focus on the relationship between declining fertility 
and increased use of sex selection and, hence, by my hypothesis, on birth spacing.
Second, the suggestion by Reviewer 1 to use a graphical presentation of the results made it 
possible to include the likelihood of parity progression for each of the three spells, 
which allows me to discuss how fertility has decreased across the three education groups
and by urban and rural areas.
I have also added the parity progression likelihood to the original tables, which I now
include in the online Appendix.
Third, I discuss the changes in fertility directly in the Results section. 


% [Generically, spacing tend to decline with lower fertility (check this). 
% Presumably the decline in spacing is because part of the cause is higher female labor 
% market participation, which makes it more attractive to space children closer together.
% Not clear whether this is the case for India - The R1 suggested paper might be a good
% starting point with a tie-in to compression.
% Compression can also come from improved health and/or less migration.
% 
% On the flip side, lower fertility place additional pressure on women to use
% sex selection, which lead to *less* compression.
% ]

\end{enumerate}

\newpage

\section*{Reviewer 1}

\begin{enumerate}
\item The key contribution of the paper is the use of birth spacing and sex
ratios to examine changes and differentials in possible sex-selective
abortions. While there is large literature using differential stopping
behaviours (DSB) and sex ratios to examine sex selection, the use of
birth spacing is new and could potentially provide new insights. The
methodology used is rigorous and suited for the purposes of the paper.


\item The paper would need to articulate its contribution clearly. The
paper seems to be focussed too much on methodology and less on
contextual, theoretical and substantive matters. As there is already
well established literature on sex-selection and son preference, the
paper must articulate what is substantially different in the paper. As
one of the key features is the focus on spacing, expanded review of the
literature on spacing is required.

[Response:]

I have rewritten the paper completely, so the focus is now directly on spacing 
changes, especially how access to sex selection changes birth intervals differently across 
sex compositions.
As part of the rewrite, I now include a substantially expanded review of the literature
on sex selection, son preference, and spacing.
I use this literature review to guide my discussion of the changes in spacing throughout
the paper.


\item The literature review must be systematic. The paper seems to move too
quickly to estimation strategy. Before presenting this, the literature
on spacing, sex ratios, sexselection should be presented. This would
help the paper to make a stronger case of its place and contribution to
the demographic literature.

[Response:]

Please see my response to the Editor's first comment above.


\item The demographic literature, for example, by works from Christophe Z
Guilmoto, doesn’t seem to be consulted. Guilmoto and colleagues many
papers including recent work in Population Studies and Lancet are
important to consult as they cover the same substantial issues of sex
ratios, sex selection and abortions as addressed in this paper. Though
these demographic papers do not necessarily use spacing, a similar
understanding of sex-selection could be obtained using examining parity
progression. Therefore, the added insights from using spacing need to
articulated much more forcefully.

[Response:]

As mentioned above, I have refocused the paper to examine how birth spacing has changed
over time in India, and, especially, how the spread of prenatal sex determination can
lead to longer spacing.
As part of this rewrite, I have incorporated the suggested references and other relevant
works into the Introduction as part of the motivation for why examining the changes in 
birth  intervals that arise from the use of sex selection is essential.

\item The use of competing hazard models is not new in demographic research
and doesn’t require long exposition. This section can be cut
sustainably. I don’t think the methodological innovation is a key
contribution of the paper as the methods are established even if they
have not been used to examine spacing.

[Response:]

I have reduced the length of the section by more than 1/4 (the original section was five
pages long, while the revised section is less than 3.5).
The first two paragraphs in the section now highlight how my approach is different from
prior research, especially the non-randomness of birth outcome and non-proportionality
assumption.

There are three essential differences between my approach and what is standard
in the literature, and I believe further shortening would negatively impact readers'
ability to understand the results.
First, most research that uses survival models employ proportional hazard models, so 
motivating and providing more information on the non-proportional set-up is essential.
Second, the use of conditional survival curves is, to my knowledge, new, and, therefore,
requires a detailed discussion.
Third, the sex ratio calculations is a new way to use competing risk estimates and is 
required to understand how the use of sex selection impacts birth intervals.


\item The use of NFHS data including the most recent data are welcome. An
additional limitation to those addressed in the paper is the issue of
date imputation in DHS/NFHS. NFHS as with other DHS data routinely use
imputations for year and month of birth. I don’t have the reference for
India (please check data quality reports on DHS website), but the
imputation of month of birth was done for large number of cases in
earlier rounds of DHS. This could be a possible reason for seeming large
number of pre-marital conceptions which do not make sense in the context
of India.

[Response:]

Date imputation is, indeed, a significant concern.
For both NFHS-1 and NFHS-2, the only information collected on marriage timing is the age 
at marriage and age starting living with husband.
For NFHS-1, I, therefore, used the date of birth of the respondent in CMC and added the 
age at marriage multiplied by 12 to get the earliest date of marriage in CMC possible.
If I use this definition, 10.73 percent have a duration of eight months or fewer, 
and 0.71 percent have a negative duration.
For NFHS-2, I used the imputed duration provided in the recode data set, which leads to 
24.65 percent with a duration of eight months or fewer between marriage and first birth, 
and 0.95 percent with a negative duration.
Other people seem to have been caught by the same problem 
\citep[See, for example, ][]{Padmadas2004}.

Both NFHS-3 and NFHS-4 ask for the specific month and year of first marriage, but some
respondents did not provide complete information.
One approach to understanding the impact of imputation on spacing to first birth is to 
split spacing from marriage to first birth by completeness of date of marriage.
In NFHS-3, 2.55 percent of the 39,213 women with complete information on the date of 
marriage shows a spacing from marriage to first birth of 8 months or less and that 0.6 
percent show a negative spacing. 
For the 19,403 women with incomplete information on the date of marriage, 27.36 percent show
a spacing from marriage to first birth of eight months or fewer and that 3.37 percent show a 
negative duration.
In NFHS-4, 3.8 percent of women with complete information report a birth eight or fewer months
after marriage and 2.35 report a negative duration.
For women with imputed information, 14.99 percent report a birth eight or fewer months
after marriage with 11.02 percent reporting a negative duration.
The main difference between NFHS-3 and NFHS-4 is that only just below four percent of 
women in NFHS-4 have imputed information whereas more than 33 percent did in NFHS-3.

Given the substantial problems with the first spell data for all but the very last
survey and that the first spell analysis adds relatively little to the central question, 
I have dropped the analysis of first spells.
Removing the first spell also helps to keep the paper within the word limit.
Because of the large number of imputed observations for the first spell, I have also
remove the first spell duration and its square as explanatory variables for the second
through fourth spells.



\item It is not clear why exactly 22, 23, 25 years are used a cut-off
points.

[Response:]

I have added a detailed discussion of recall error in the Appendices intended for online 
publication.
Furthermore, I have changed the cut-off to be 22 years of marriage or more for all surveys 
to simplify.
Even with the stricter cut-off, there is, however, still some evidence of recall error 
where the sex ratio was substantially biased against females  before sex selection became 
available.


\item I understand the empirical reason for not increasing the number of
variables. However, a key variable that is essential to consider is
region. This is important because regional influences on birth spacing
and sex selection are very strong as established in the literature. Sex
selection and abortion is especially strong in certain regions but not
in others. The regional differentials in fertility transition,
contraceptive use, spacing, sex ratios, abortions are clearly documented
in the literature. So it is vital to include this.

[Response:]

I agree that the regional differences in fertility, son preference, and use 
of sex selection are essential to examine.
To try to capture these differences, I have divided India into four broad regions 
depending on the degree of son preference and initial fertility level as in 
\citet{retherford03b}, although expanded to include all states in the surveys.
In addition to the power problem, some education/region/spell/period combinations have too 
small sample sizes for the estimations to converge.
The lack of convergence is primarily an issue for the lower and middle-education groups.
I, therefore, first present the all-India results and then focus on regional differences
for only the most educated women, who are the main users of sex selection.
This new section substitutes for the removed section on how sex ratios change within 
a spell.


\item The three tables are logically organized and contain a wealth of
information. But there are too many comparisons to make in each of the
tables and I wonder if it would make sense to split the tables or
present some of the information in a much more reader friendly way (for
instance by using charts for some of the information). 

[Response:]

Using graphs is an excellent suggestion.
Instead of the tables, I now show graphs for each of the education levels.
For each spell, I show the predicted average spacing, the predicted sex ratio, and the 
probability of having a birth by the end of the spell for each decade.
Including the probability of having a birth allows me to discuss the falling fertility 
more organically as requested by the editor.

The only downside is that I had to remove the confidence intervals for legibility. 
I, therefore, still include Tables 1-3, but have moved them to the Appendix, and I have 
added the probability of having a birth by the end of the spell to the tables.
I have also combined the 75th, 50th, and 25th percentiles results tables in the Appendix.

\item As sex composition and sex ratios has been examined in the literature, I think
it would make sense to not present findings on that in great detail. As
the paper states, the central question addressed is spacing, and so all
other questions should not be dealt in detail

[Response:]

As detailed above, I have refocused the Results section to focus on birth spacing and how 
that is affected by the increased use of sex selection over time.
I now instead refer to the previous literature on sex ratios and use that as a guide for
structuring the analysis of birth spacing.
I have also removed the section on how sex ratios change within spells since that is
more relevant when examining sex selection.
I can better address birth spacing changes using the results on the distribution of 
births within a spell.


\item The results talk about compression of spacing. This is important and
literature on it should be presented in the intro or lit review section.
There is evidence of compression of the reproductive span at least in
some regions (see Padmadas, et. Al 2004). It would be important to
consider how this fits in with the results presented.

[Response:]

Compression in the original manuscript was only for the first birth interval, and I no 
longer cover that because of the data issues discussed above.
I do, however, agree that this is an important question, even if I cannot directly
address it here.
I, therefore, have added this, together with the reference, to the conclusion when I 
discuss areas of future research.


\item As there are many results presented, it would make sense to have a
paragraph at the beginning of the conclusion highlighting the key
findings.

[Response:]

With the refocus on birth intervals, there should be more coherence in the results and
how I present them.
I have, however, highlighted what I consider the main results in the Conclusion.



\item 13. Overall, the paper is empirically strong but needs to build on the
contextual, theoretical, and prior demographic literature. As the
readers of Demography are diverse, it would be good to have a balanced
presentation of all the sections. Currently it seems the paper is too
concerned with the methodology and empirical work at the expense of
other areas.

[Response:]

As mentioned, I have substantially expanded the review of the literature, and use the 
review to guide the main questions I ask, and I refer back to the main previous finding 
throughout to place my results in perspective.
With the restructuring of the Results section, including the added discussion of regional 
differences, and the rewrite of the Conclusion, I hope that the paper is now more 
balanced. 
I also trust that it is clearer how it fits into the literature.




\end{enumerate}

\newpage

\section*{Reviewer 7}

The article was a careful, innovative, and thorough piece of research
picking up on a conundrum concerning how son preference and sex
selection affect birth intervals. I very much enjoyed reading this piece
and think it has useful applications outside of India.  My review is
short as I do not see much to be improved upon.  My only (minor)
suggestion concerns the use of the word "spell".  It is very clear what
you mean by a spell when the paper explains this, but I think that it is
somewhat confusing to put this in the abstract without an explanation. 
It was not until I got to the explanation in the methods that I
understood the abstract fully.  I suggest rewording this section of the
abstract so it is clear what is meant by spell.

[Response:]

As part of the rewrite of the paper, I have redone the abstract and removed the
reference to spells.


\newpage
\bibliographystyle{econometrica}
\bibliography{../paper/sex_selection_spacing.bib}
\addcontentsline{toc}{section}{References}


\end{document}
