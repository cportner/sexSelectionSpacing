\documentclass[letterpaper,12pt]{article}

% Xetex preamble
\usepackage{fontspec}
\setromanfont[Ligatures=TeX]{TeX Gyre Pagella}
\usepackage{unicode-math}
\setmathfont{TeX Gyre Pagella Math}

\usepackage[longnamesfirst]{natbib}
%\usepackage[tabhead,nolists,tablesfirst]{endfloat}
\usepackage[flushleft]{threeparttable}
\usepackage{booktabs}
\usepackage{rotating}  
\usepackage{dcolumn} 
\usepackage{setspace}
%\usepackage{flafter} 
%\usepackage{longtable}
%\usepackage[pdftex]{graphicx}
\usepackage[xetex,colorlinks=true,linkcolor=black,citecolor=black]{hyperref} 
\usepackage[margin=1.0in]{geometry} 
\usepackage{multirow}

%opening
\title{} \author{}

\doublespacing

\begin{document}

\begin{center} \textbf{\large Birth Spacing in the Presence of Son
Preference and 
Sex-Selective Abortions: India's Experience over Four Decades}
\end{center}

\begin{center} Response to Editor and Reviewer Comments \end{center}

\noindent Please find attached the revised version of my paper,
``Birth Spacing in the Presence of Son Preference and Sex-Selective
Abortions:
India's Experience over Four Decades.''

[Other changes]
\begin{enumerate}
	\item Rewritten Estimation Strategy section
	\begin{enumerate}
		\item Clarify proportionality assumption
		\item Better explain set-up of model and transformation of data needed
		\item Remove discussion of explanatory variables so they are
		only covered once (in the ``Data'' section)
		\item Better explanation of how I calculate measures used for the spacing
		analysis
	\end{enumerate}
	\item Changes in Data section
	\begin{enumerate}
		\item Lead with what birth intervals I cover right after introducing
		data
		\item Move restriction to only Hindus to be the first restriction
		I discuss since that is the most important
	\end{enumerate}
\end{enumerate}


\section*{Editors}

\begin{enumerate}

\item We note that you kept to the table/figure limit; however, some of
the figures are several pages long and relatively small. These will have
to be evaluated at the copy edit stage as there is no obvious quick fix
- the figures reflect the analyses carried out (and some additional ones
are in an online appendix). Please make sure to submit as
high-resolution image as possible in the format required by the journal.
We are somewhat concerned about figure/table size within the print form
of the journal.

[Response:]

TK review titles of figures and table to be better reflect the
analyses undertaken.


\end{enumerate}


\section*{Deputy Editor}

\begin{enumerate}

\item The overall topic and research question is not clear in the
introduction. The main topic relates to changes in birth spacing over
time, which are influenced by sex selection/sex composition and maternal
education, and how this affects fertility and infant mortality. This is
a long list of interesting questions and analyses but this is confusing
for the reader. It would be useful to have a succinct statement about
what the study is about (what are key independent variables/concepts and
key outcomes) and place that in the introduction and before the data and
analyses section. This way, the reader can understand why the analyses
will proceed in a particular order and what is being tested. In
addition, the titles of the figures and tables should be reviewed to see
if they could better reflect the analyses undertaken. Overall, the
reader needs to be reminded at the beginning and in each section what
you aim to do.

[Response:] 

I have rewritten the Introduction to emphasize the main question I address
(``how did birth spacing respond to these changes, especially the introduction and spread 
of sex selection?'') and the two questions addressing the effects of changes in 
birth spacing (``to what extent do changes in birth spacing bias our standard fertility 
estimates?'' and `` how do the changes in birth spacing and the increased use of abortions 
impact children's chances of survival?'').

Following the three questions, I now outline how the study progresses focusing on 
how I examine each of the three questions, including the key outcomes and key explanatory 
variables.
To prevent exceeding the word count I have also edited the entire introduction.
Although sligthly longer than original, it is still less than 900 words 
(not counting references).

I have edited the introduction of each section to highlight what I question
I am addressing.
Furthermore, I have removed some of the repetition in, for example, the
discussion of tempo effects and mortality, so that the motivation is now only
in the Introduction rather than being spread over the Introduction and the
relevant sections.

Finally, I have made the discussion of the distribution of spacing a subsection
of the spacing section and combined the discussion of the results to another new 
subsection as well.
I hope this will make it easier for the reader to see the answer to the main
question of the paper.

TK I address the issue of the figure and table title above.



\item Please define sex selection early on.

[Response:] I now define ``sex selection'' as ``the selective abortion of female fetuses 
based on prenatal sex determination'' in the second paragraph of the paper.


\item Please provide a motivation for using maternal education as a key
stratification measure and justify education cutoffs (which have meaning
in India).

[Response:] 

To address this, and the questions about theories of female labor supply below, I have
combined and reorganized the background and conceptual framework sections.
I begin this combined section with a discussion of the three main reasons for using 
maternal education as a key stratification measure.
First, higher female education is associated with lower fertility, and, thereby, with
potential increased use of sex selection.
Second, female labor force participation in India, as well as other developing countries, 
tend to first decrease and then increase with education.
Finally, child morbidity and mortality decreases the better educated the mother.

After the discussion of why female education is a key variable, I explain the choice
of cut-offs, which is based on the prior literature on fertility and son preference in 
India and the NFHS reports.
The exception to the the cut-offs in the NFHS reports is that I combine
the <5 years and 5--7 years of schooling completed and 
the 8--9 and 10--11 years of schooling completed to reach sufficient sample sizes.

This grouping also allows me to capture the U-shape in female labor force participation.
The upper level of education could be split into having graduated secondary and having 
graduated with a university degree since women with a completed university degree have 
a higher labor force participation, but the sample of university graduates is too small 
to allow for estimations, especially for the first three rounds of the surveys.
Furthermore, with four groups there is already a substantially number of results to 
present.


\item The word ``spell'' is introduced with no definition. Is spell the
same as birth interval? Is it measured in months? This information would
be useful for the reader.

[Response:]

I now define spell as the unit of analysis---the period from one parity birth to 
the following birth or censoring---when I first use the word in the ``Estimation 
Strategy'' section.
Furthermore, I clarify that for estimation purposes spells begin nine months 
after the previous birth since this is the earliest we should expect to observe 
a new birth.
This duration is equivalent to the interpregnancy interval, which is often used in 
the literature on spacing and child mortality.

To make comparison with both the NFHS reports and the prior literature on
birth spacing easier and avoid confusion, I have rewritten the discussions of 
spacing and redone the graphs and table, so all results now reflect birth 
intervals in months.
I still retain the use of ``spell'' to indicate a generic period from one parity 
birth to another, i.e. second spell is from first birth to the second birth or
censoring, in the graphs/tables.


In the Data section, I have also changed the discussion of the imposed censoring 
clearer by specifying the cut-offs for both the spell duration and the birth interval.
In the ``How Birth Spacing Changed'' section, I explain that I add nine month
to the estimated spell length to get to birth intervals.
In the ``Mortality and the Changing Birth Spacing'' section, I retained the
definition of the periods but updated the numbers and the discussion of
results so they reflect birth intervals rather than pregnancy intervals.



\item Please include a brief discussion or footnote about the quality of
birth date data in the DHS/NSFH and how this could bias your results or
not.

[Response:]

I have expanded the discussion of the birth history data in the ``Data'' section and
why it is important.
As part of this expanded discussion, I included more information about systematic
recall error, why it is likely the main concern here, what the effect is on the
estimates, and how I address this problem.
I have also included a reference to \citet{Schoumaker2014} that finds that the first
rounds of the NFHS are of ``moderate quality,'' although that is without the correction
for recall error that I use.


\item Perhaps the conclusion could start with a brief overview of your
topic and research question again to remind the reader of the overall
motivation for the study.

[Response:]

I have added a brief synopsis of the changes in India in the beginning of
the Conclusion and recap the three questions after that.
I have also slightly edited the Conclusion to tie better with the 
questions.

\item I see the point to reviewer’s comment #4 about the discussion on
women’s labor supply. It’s a complex topic to introduce in a few brief
sentences. Similarly, referring to Goldin’s U-shaped theory without a
more detailed explanation could be confusing for those not well versed
in this topic. Perhaps you could also mention that decreasing FLFP is
unexpected for developing countries and some potential reasons for this
in India.

[Response:]

I agree, and to help address this, and the question about maternal education cut-offs
above, I have combined and reorganized the conceptual framework and background sections.

The main relevant theory on female labor force participation and education is the income 
and substitution effects discussed in \citet{Hotz1997} and \citet{schultz97}.
With the reorganization of the sections, the introduction of this theory now more naturally 
follows the discussion of increasing education in India.
I then discuss the declining female labor force participation and that it may come
about because of the rapid income increases.

Since the Goldin theory is not central to this argument, I have removed the sentence 
on the theory.
It is worth pointing out, however, that a decline in female labor force participation  
is in line with the hypothesis of a U-shaped labor force participation as a country 
develops \citep{Goldin1994}.
What is ``surprising" for India is that we have not yet seen any evidence of the upward 
sloping part.
I do think that the combination of rapidly rising male income, son preference, and 
relatively meager job opportunities for women with, say, high school education have a lot 
to do with this.



\item The Klasen and Pieters article is now published.

[Response:]

Yes, I already had it listed as World Bank Economic Review 2015.
Was there maybe another paper that I should cite?

\item Please have the paper carefully reviewed/edited, as there are
numerous grammatical and syntax mistakes.

[Response:]


\end{enumerate}

\newpage

\section*{Reviewer 1}

I already liked the previous version of this paper. I think the author
has done a good job incorporating most of my comments and those of the
other reviewers. Even though I would sometimes have made different
choices (for example, including fewer figures), the choices the author
has made are well understandable and appropriately motivated. I think
the paper is a useful contribution to Demography and my advice is to
accept it.

[Response:]

\begin{enumerate}

\item 

\end{enumerate}


\newpage

\section*{Reviewer 2}

[With respect to the reviewer’s comments, please address comments #1, #2, and #4.]

I think that the author has done a good job in addressing the comments
made by the three reviewers. However, on reading the draft, I felt a few
things were not clear.

\begin{enumerate}

\item It is not clear why the author restricts the analysis to Hindu
women. Given that data across four rounds of the NFHS data is being
pooled, I think there is enough data to examine this question for the
non-Hindu (and in particular Muslim) women. Given that patterns of son
preference varies across religious groups, I think it is worth examining
the question for non-Hindus. At the very lease include a discussion of
why the non-Hindus have not been included.

[Response:]

I focus on the Hindu population because they are the majority population group 
and because the prior literature argue that this group has a strong son preference and 
consequently uses sex selection.%
\footnote{
Sikhs and Jains likely have even higher use of sex selection than Hindus but Sikhs 
constitutes less than two percent of the population and Jains less than half a percent.
}
Since there is little evidence that Muslim use sex selection, combining Hindu and Muslim 
would lead to biased results.

How birth spacing of Muslim women changed is certainly of interest and is worthy
of a separate analysis.
However, there are two problems with expanding the analysis.
First, although the lastest round of NFHS provides enough data for the analyses, 
it would be hard to provide much in the way of comparison with earlier periods.
This problem becomes especially acute once the Muslim population is split by education 
levels to make the results comparable to the Hindu population.
Second, the paper is already very long and adding even more tables/graphs and the
necessary discussions of the results does not seem feasible.

I have edited the discussion in the ``Data'' section to make the reason for focusing
on Hindus clearer, and hope to write a separate paper/note looking at how spacing
has changed for Muslims later.





\item There is actually no data on sex selection and in a sense the data
is backed out using evidence on birth spacing, institutional information
on the availability of technology that enables sex selection (ultra
sound techniques) and government regulations. To do this the author
subdivides the time into four segments: 1972--1984, 1985--1994, 1995--2004
and 2005--2016. I understand the logic of the first two breaks 1984 and
1994, but what is the logic of the third break in 2004? This needs to be
made clearer.

[Response:]

The 2004 break is an attempt to understand whether we are observing the beginning of
a reversal in son preference and the use of sex selection as hypothesized by part of
the prior literature.
In the ``Data'' section, I have split the paragraph on the periods into two and expanded 
the discussion of the motivation for the breaks, including adding two additional
references arguing that we might see the reversal.
Clearly, the 2004/05 break is somewhat arbitrary, but does have the advantage of making 
the periods close to the same length (there are relatively few birth intervals that begin 
in 2015 and 2016).

\item While I like results being presented in figures, what I don’t like
is that no standard errors or confidence intervals are presented.
Therefore, it is difficult to figure out whether the differences between
categories (composition of existing children) are statistically
significant or not.

[Response:]

\item In the context of the paper, I don’t see the point of the
discussion relating to labour supply.

[Response:]

The discussion of labor supply was disjointed and unclear because I had split the 
background and conceptual framework sections, making it difficult to see the connection.
As explained above, I have combined and reorganized those two sections into one and
removed the Goldin theory discussion since it is not directly relevant to my main
argument of why labor supply is relevant.


\item Overall, I felt that the paper is still quite densely written and
it is not easy to follow what is going on. Considerable time needs to be
spent re-writing the paper and making it easier to read.

[Response:]

\end{enumerate}



\newpage
\bibliographystyle{econometrica}
\bibliography{../paper/sex_selection_spacing.bib}
\addcontentsline{toc}{section}{References}


\end{document}
