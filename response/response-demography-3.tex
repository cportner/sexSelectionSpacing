\documentclass[letterpaper,12pt]{article}

% Xetex preamble
\usepackage{fontspec}
\setromanfont[Ligatures=TeX]{TeX Gyre Pagella}
\usepackage{unicode-math}
\setmathfont{TeX Gyre Pagella Math}

\usepackage[longnamesfirst]{natbib}
%\usepackage[tabhead,nolists,tablesfirst]{endfloat}
\usepackage[flushleft]{threeparttable}
\usepackage{booktabs}
\usepackage{rotating}  
\usepackage{dcolumn} 
\usepackage{setspace}
%\usepackage{flafter} 
%\usepackage{longtable}
%\usepackage[pdftex]{graphicx}
\usepackage[xetex,colorlinks=true,linkcolor=black,citecolor=black]{hyperref} 
\usepackage[margin=1.0in]{geometry} 
\usepackage{multirow}

%opening
\title{} \author{}

\doublespacing

\begin{document}

\begin{center} \textbf{\large Birth Spacing in the Presence of Son
Preference and 
Sex-Selective Abortions: India's Experience over Four Decades}
\end{center}

\begin{center} Response to Editor and Reviewer Comments \end{center}

\noindent I have attached the revised version of my paper,
``Birth Spacing in the Presence of Son Preference and Sex-Selective
Abortions: India's Experience Over Four Decades.''
Thank you for the comments and suggestions. 
Below I outline my response to each issue raised.


In addition to responding to the comments, I have rewritten substantial parts of the paper to ensure 
that it is easier to follow. 
The following are the more significant changes not covered below in my responses:
\begin{enumerate}
    \item I substantially rewrote the Abstract to directly discuss the three main 
    questions I address in the paper
	\item ``Estimation Strategy'' section changes:
	\begin{enumerate}
		\item Clarify proportionality assumption
		\item Better explain the set-up of the model and transformation of data needed
		\item Remove discussion of explanatory variables, so they are
		only covered once (in the ``Data'' section)
		\item Better explain how I calculate the outcomes used for the spacing analysis. 
		All explanations are now gathered in this section
		\item Remove the formula for percent boys since it added little. 
		I have added a more formal version of the model to the online appendix in the
		``Empirical Model Details'' section.
	\end{enumerate}
	\item ``Data'' section changes:
	\begin{enumerate}
		\item I now lead with a discussion of which birth spells I analyze right after 
		introducing the data sets
		\item Move discussion of restriction to only Hindus to be the first restriction I 
		discuss because that is the most important one. 
		See also my response to Reviewer 2 below.
	\end{enumerate}
	\item Consistent presentation of results:
	\begin{enumerate}
		\item Update table notes to describe calculation methods more clearly
	\end{enumerate}
\end{enumerate}

Finally, I have amended the paper title slightly to convey more clearly the main questions 
addressed.
The new title is ``Birth Spacing and Fertility in the Presence of Son Preference and 
Sex-Selective Abortions: India's Experience Over Four Decades.'' 
If this becomes too long, removing ``Son Preference and'' would be a possibility.
Please let me know if you have any strong opinions on this change.


\section*{Editors}

\begin{enumerate}

\item We note that you kept to the table/figure limit; however, some of
the figures are several pages long and relatively small. These will have
to be evaluated at the copy edit stage as there is no obvious quick fix
- the figures reflect the analyses carried out (and some additional ones
are in an online appendix). Please make sure to submit as
high-resolution image as possible in the format required by the journal.
We are somewhat concerned about figure/table size within the print form
of the journal.

[Response:]

After trying a substantial number of different approaches, I believe the best way to solve 
this problem is a complete rewrite of the birth spacing analysis section.
Therefore, I now present three percentile birth intervals---25th, median, and 75th---together
with the sex ratio and the parity progression probability.
Furthermore, to keep within the figure/table limit and not have multiple-page figures,
I moved the results for urban women without education and rural women with 12 or more 
years of education to the online appendix.

The main advantage of this new approach---besides making the figures less challenging to 
read---is that it more clearly ties the section to the main question I am asking.
Furthermore, the new set-up makes it easier to understand similarities and differences 
across groups and compare the results to the prior literature.
It also ties the birth spacing discussion directly to the fertility section that follows.
The basic gist of the results, of course, remains unchanged.

The rewritten section has three parts.
First, I present the broad trends for parity progression and sex ratio to set the
stage for the analyses.

Next, I discuss how birth spacing has changed in situations where there is the least use
of sex selection (less educated women no matter the sex composition and among 
better-educated women with one or more sons).
To this part, I have added a discussion of how my results compare with the prior literature 
\citep{Rutstein2011,Casterline2016}.
To tie the results to the female labor force participation, I also show how average spacing
has increased the most for the women least likely to work.
Finally, I show how the longest spacing has changed with declining fertility, and discuss 
why the behavior of the best-educated women appear to be different from other groups in 
India and elsewhere.

The last part of the section examines how birth spacing changed with the use of sex selection.
To illustrate, I use the third spell for the best-educated women, especially the very 
rapid increase right after sex selection became available.
I then show where the reversal of the traditional spacing pattern occurred.
As part of my response to reviewer 2's question about the period grouping, I now include a 
discussion of whether the predicted decline in the use of sex selection came about.
I end with a summary of results to tie the section to the fertility section.

Finally, I can produce figures in whatever format is best for printing.
I can also easily change things like line width and labeling if that helps with the
printing process.

\end{enumerate}


\section*{Deputy Editor}

\begin{enumerate}

\item The overall topic and research question is not clear in the
introduction. The main topic relates to changes in birth spacing over
time, which are influenced by sex selection/sex composition and maternal
education, and how this affects fertility and infant mortality. This is
a long list of interesting questions and analyses but this is confusing
for the reader. It would be useful to have a succinct statement about
what the study is about (what are key independent variables/concepts and
key outcomes) and place that in the introduction and before the data and
analyses section. This way, the reader can understand why the analyses
will proceed in a particular order and what is being tested. In
addition, the titles of the figures and tables should be reviewed to see
if they could better reflect the analyses undertaken. Overall, the
reader needs to be reminded at the beginning and in each section what
you aim to do.

[Response:] 

I have rewritten the Introduction to emphasize the main question I address (``how birth 
spacing responded to these changes, especially the spread of sex selection'').
I follow this with the two questions addressing the effects of changes in 
birth spacing (``to what extent do changes in birth spacing bias our standard fertility 
estimates?'' and `` how do the changes in birth spacing and the increased use of abortions 
impact children's chances of survival?'').

I next outline how the study progresses, focusing on how I examine each of the three 
questions, including the key outcomes and key explanatory variables. 
To prevent exceeding the word count, I have edited the entire Introduction. 

The beginning of each section now highlights what I question I am addressing.

Furthermore, I have tried to removed repetitions throughout the paper. 
One example is the discussion of tempo effects, where the motivation is now 
only in the Introduction rather than being spread over multiple sections.

As described above, I have completely redone the discussion of how spacing changed.
I hope this will make it easier for the reader to see the answers to the paper's central 
question.

Finally, I have edited the figure and table titles so that they better reflect the analyses.



\item Please define sex selection early on.

[Response:] 

I now define ``sex selection'' as ``the selective abortion of female fetuses 
based on prenatal sex determination'' in the second paragraph of the paper.


\item Please provide a motivation for using maternal education as a key
stratification measure and justify education cutoffs (which have meaning
in India).

[Response:] 

To address this, and the questions about theories of female labor supply below, I have
combined and reorganized the background and conceptual framework sections.
I begin this combined section with a discussion of the three main reasons for using 
maternal education as a key stratification measure.
First, higher female education is associated with lower fertility, and, thereby, the 
potential increased use of sex selection. 
Second, female labor force participation in India---and other developing countries---tend 
first to decrease and then increase with education. 
Finally, child morbidity and mortality decrease the better educated the mother.


After discussing why female education is a crucial variable, I explain the choice of 
cut-offs, which I based on the prior literature on fertility and son preference in India 
and the NFHS reports.
The exception to the cut-offs in the NFHS reports is that I combine
the <5 years and 5--7 years groups and 
the 8--9 and 10--11 years groups to reach sufficient sample sizes.

This grouping also allows me to capture the U-shape in female labor force participation,
which I refer to in my discussion of the changes in birth spacing when sex selection
is less used.
In principle, I could split the upper level of education into having graduated secondary 
and having graduated with a university degree since women with a completed university 
degree have a higher labor force participation. 
The issue is that the sample of university graduates is too small to allow for estimations 
for the first three rounds of the surveys. 
Furthermore, with four groups, there is already a substantial number of results to present.


\item The word ``spell'' is introduced with no definition. Is spell the
same as birth interval? Is it measured in months? This information would
be useful for the reader.

[Response:]

I now define spell as the unit of analysis---the period from one parity birth to 
the following birth or censoring---when I first use the word in the ``Estimation 
Strategy'' section.
Furthermore, I clarify that for estimation purposes, spells begin nine months after the 
previous birth, which is the earliest we should expect to observe a new birth.
% This duration is equivalent to the interpregnancy interval, which is often used in 
% the literature on spacing and child mortality.

To make comparisons with both the NFHS reports and the prior literature on
birth spacing easier and avoid confusion, I have rewritten the discussions of 
spacing and redone the graphs and tables, so all results now reflect birth 
intervals in months.
I still retain the use of ``spell'' to indicate a generic period from one parity 
birth to another---i.e., the second spell is from first birth to the second birth or
censoring---in the graphs/tables.


In the Data section, I have also clarified the discussion of the imposed censoring by 
specifying the cut-offs for the spell duration and the birth interval.
In the ``Estimation Strategy'' section, I explain that I add nine months
to the estimated spell length to get to birth intervals.
In the "Mortality and the Changing Birth Spacing" section, I retained the periods but 
updated the numbers to reflect birth intervals rather than pregnancy intervals.


\item Please include a brief discussion or footnote about the quality of
birth date data in the DHS/NSFH and how this could bias your results or
not.

[Response:]

I have expanded the discussion of the quality of the birth history data and why it is
important in the ``Data'' section. 
As part of this expanded discussion, I included more information about systematic recall 
error, why it is likely the main concern here, its effect on the estimates, and how I 
address this problem.
I have also included a reference to \citet{Schoumaker2014}, who finds that the first
rounds of the NFHS are of ``moderate quality,'' although that is without the correction
for recall error that I use.


\item Perhaps the conclusion could start with a brief overview of your
topic and research question again to remind the reader of the overall
motivation for the study.

[Response:]

I have added a brief synopsis of the changes in India at the beginning of the Conclusion. 
I recap the three questions after that. 
I have also edited the Conclusion to tie better with the questions.

\item I see the point to reviewer’s comment #4 about the discussion on
women’s labor supply. It’s a complex topic to introduce in a few brief
sentences. Similarly, referring to Goldin’s U-shaped theory without a
more detailed explanation could be confusing for those not well versed
in this topic. Perhaps you could also mention that decreasing FLFP is
unexpected for developing countries and some potential reasons for this
in India.

[Response:]

I agree, and to help address this, and the question about maternal education cut-offs
above, I have combined and reorganized the conceptual framework and background sections.

One framework for understanding the relationship between female education, labor force 
participation, and birth spacing is the income and substitution effects, as  discussed 
in \citet{Hotz1997} and \citet{schultz97}.
With the reorganization, the introduction of this now more naturally follows the discussion 
of increasing education in India. 
I then discuss the declining female labor force participation and that it may be the 
result of the rapid income increases.

Since the Goldin theory is not central to this argument, I have removed it.
However, it is worth pointing out that a decline in female labor force participation  
is in line with the hypothesis of a U-shaped labor force participation as a country 
develops \citep{Goldin1994}.
What is peculiar about India is that we have not yet seen evidence of the upward 
sloping part.
My prior is that the combination of rapidly rising male income, son preference, and relatively 
meager job opportunities for women with, say, high school education has a lot to do with this.



\item The Klasen and Pieters article is now published.

[Response:]

Yes, I already had it listed as World Bank Economic Review 2015.
Was there maybe another paper that I should cite?

\item Please have the paper carefully reviewed/edited, as there are
numerous grammatical and syntax mistakes.

[Response:]

As part of the rewrite, I have endeavored to weed any remaining mistakes. 
I have also had an editor work through the paper.
There are some format changes that I am unclear about. 
One of those is the best way to refer to numbers for the education groups. 
The style guide says to write out numbers ten or less, but this becomes 
unwieldy for ranges of years of education.


\end{enumerate}

\newpage

\section*{Reviewer 1}

I already liked the previous version of this paper. I think the author
has done a good job incorporating most of my comments and those of the
other reviewers. Even though I would sometimes have made different
choices (for example, including fewer figures), the choices the author
has made are well understandable and appropriately motivated. I think
the paper is a useful contribution to Demography and my advice is to
accept it.

[Response:]

I have tried to limit the number of figures as detailed above in my response to
the Editors.



\newpage

\section*{Reviewer 2}

% [With respect to the reviewer’s comments, please address comments #1, #2, and #4.]

I think that the author has done a good job in addressing the comments
made by the three reviewers. However, on reading the draft, I felt a few
things were not clear.

\begin{enumerate}

\item It is not clear why the author restricts the analysis to Hindu
women. Given that data across four rounds of the NFHS data is being
pooled, I think there is enough data to examine this question for the
non-Hindu (and in particular Muslim) women. Given that patterns of son
preference varies across religious groups, I think it is worth examining
the question for non-Hindus. At the very lease include a discussion of
why the non-Hindus have not been included.

[Response:]

I focus on the Hindu population for two reasons. 
First, they are the majority population group. 
Second, the prior literature argues that this group has a strong son preference and 
consequently uses sex selection. 
With little evidence that Muslims use sex selection combining Hindu and Muslim 
would lead to biased results.
Sikhs and Jains likely have even higher use of sex selection than Hindus, but Sikhs 
constitute less than two percent of the population and Jains less than half a percent. 

How the birth spacing of Muslim women changed is undoubtedly an interesting question and 
worthy of a separate analysis. 
However, there are two problems with expanding the analysis here. 
First, although the latest round of NFHS provides enough data for the analyses, it would 
be hard to provide much in the way of comparison with earlier periods. 
This problem becomes especially acute once the Muslim population is split by education 
levels to make the results comparable to the Hindu population. 
Second, the paper is already very long, and adding even more tables/graphs and the 
necessary discussion of the results does not seem feasible.

I have edited the discussion in the ``Data'' section to make my reasons for focusing
on Hindus clearer, and hope to write a separate paper/note looking at how spacing
has changed for Muslims later.





\item There is actually no data on sex selection and in a sense the data
is backed out using evidence on birth spacing, institutional information
on the availability of technology that enables sex selection (ultra
sound techniques) and government regulations. To do this the author
subdivides the time into four segments: 1972--1984, 1985--1994, 1995--2004
and 2005--2016. I understand the logic of the first two breaks 1984 and
1994, but what is the logic of the third break in 2004? This needs to be
made clearer.

[Response:]

The 2004 break is an attempt to understand whether we observe the beginning of
a reversal in son preference and the use of sex selection as hypothesized by part of
the prior literature.
I have made two changes to make this clearer.

First, in the ``Data'' section, I have split the paragraph on the periods into two and expanded 
the discussion of the motivation for the breaks, including adding two additional
references arguing that we might see a decline in use.
The 2004/2005 break is somewhat arbitrary but does have the advantage of making 
the periods close to the same length (there are relatively few birth intervals that begin 
in 2015 and 2016).

Second, I now discuss the changes between the last two periods in more detail in the 
subsection on birth spacing with sex selection and in the Conclusion.

\item While I like results being presented in figures, what I don’t like
is that no standard errors or confidence intervals are presented.
Therefore, it is difficult to figure out whether the differences between
categories (composition of existing children) are statistically
significant or not.

[Response:]

The main reason for not showing confidence intervals is that the figures would become close 
to impossible to read. 
The original tables are available in the online appendix and have both standard 
errors and tests of whether differences are statistically significant. 
I have clarified this in the text.

\item In the context of the paper, I don’t see the point of the
discussion relating to labour supply.

[Response:]

I believe the female labor supply discussion was disjointed and unclear because I had split the 
background and conceptual framework sections, making it difficult to see the connection. 
As explained above, I have combined and reorganized those two sections into one and removed 
the Goldin theory discussion since it is not directly relevant to my main argument.

\item Overall, I felt that the paper is still quite densely written and
it is not easy to follow what is going on. Considerable time needs to be
spent re-writing the paper and making it easier to read.

[Response:]

As I have outlined above, I have reorganized and significantly rewritten large parts of 
the paper to make it easier to follow. 
I have also had an editor go through the paper to help with grammar and clarity.

\end{enumerate}



\newpage
\bibliographystyle{econometrica}
\bibliography{../paper/sex_selection_spacing.bib}
\addcontentsline{toc}{section}{References}


\end{document}
