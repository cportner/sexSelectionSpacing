\documentclass[letterpaper,12pt]{article}

% Xetex preamble
\usepackage{fontspec}
\setromanfont[Ligatures=TeX]{TeX Gyre Pagella}
\usepackage{unicode-math}
\setmathfont{TeX Gyre Pagella Math}

\usepackage[longnamesfirst]{natbib}
%\usepackage[tabhead,nolists,tablesfirst]{endfloat}
\usepackage[flushleft]{threeparttable}
\usepackage{booktabs}
\usepackage{rotating}  
\usepackage{dcolumn} 
\usepackage{setspace}
%\usepackage{flafter} 
%\usepackage{longtable}
%\usepackage[pdftex]{graphicx}
\usepackage[xetex,colorlinks=true,linkcolor=black,citecolor=black]{hyperref} 
\usepackage[margin=1.0in]{geometry} 
\usepackage{multirow}

%opening
\title{} \author{}

\doublespacing

\begin{document}

\begin{center} \textbf{\large Birth Spacing in the Presence of Son
Preference and 
Sex-Selective Abortions: India's Experience over Four Decades}
\end{center}

\begin{center} Response to Editor and Reviewer Comments \end{center}

\noindent Please find attached the revised version of my paper,
``Birth Spacing in the Presence of Son Preference and Sex-Selective
Abortions:
India's Experience over Four Decades.''


\section*{Editors}

\begin{enumerate}

\item We note that you kept to the table/figure limit; however, some of
the figures are several pages long and relatively small. These will have
to be evaluated at the copy edit stage as there is no obvious quick fix
- the figures reflect the analyses carried out (and some additional ones
are in an online appendix). Please make sure to submit as
high-resolution image as possible in the format required by the journal.
We are somewhat concerned about figure/table size within the print form
of the journal.

[Response:]


\end{enumerate}


\section*{Deputy Editor}

\begin{enumerate}

\item The overall topic and research question is not clear in the
introduction. The main topic relates to changes in birth spacing over
time, which are influenced by sex selection/sex composition and maternal
education, and how this affects fertility and infant mortality. This is
a long list of interesting questions and analyses but this is confusing
for the reader. It would be useful to have a succinct statement about
what the study is about (what are key independent variables/concepts and
key outcomes) and place that in the introduction and before the data and
analyses section. This way, the reader can understand why the analyses
will proceed in a particular order and what is being tested. In
addition, the titles of the figures and tables should be reviewed to see
if they could better reflect the analyses undertaken. Overall, the
reader needs to be reminded at the beginning and in each section what
you aim to do.

[Response:]

\item Please define sex selection early on.

[Response:]

\item Please provide a motivation for using maternal education as a key
stratification measure and justify education cutoffs (which have meaning
in India).

[Response:]

\item The word “spell” is introduced with no definition. Is spell the
same as birth interval? Is it measured in months? This information would
be useful for the reader.

[Response:]

\item Please include a brief discussion or footnote about the quality of
birth date data in the DHS/NSFH and how this could bias your results or
not.

[Response:]

\item Perhaps the conclusion could start with a brief overview of your
topic and research question again to remind the reader of the overall
motivation for the study.

[Response:]

\item I see the point to reviewer’s comment #4 about the discussion on
women’s labor supply. It’s a complex topic to introduce in a few brief
sentences. Similarly, referring to Goldin’s U-shaped theory without a
more detailed explanation could be confusing for those not well versed
in this topic. Perhaps you could also mention that decreasing FLFP is
unexpected for developing countries and some potential reasons for this
in India.

[Response:]

\item The Klasen and Pieters article is now published.

[Response:]

\item Please have the paper carefully reviewed/edited, as there are
numerous grammatical and syntax mistakes.

[Response:]


\end{enumerate}

\newpage

\section*{Reviewer 1}

I already liked the previous version of this paper. I think the author
has done a good job incorporating most of my comments and those of the
other reviewers. Even though I would sometimes have made different
choices (for example, including fewer figures), the choices the author
has made are well understandable and appropriately motivated. I think
the paper is a useful contribution to Demography and my advice is to
accept it.

[Response:]

\begin{enumerate}

\item 

\end{enumerate}


\newpage

\section*{Reviewer 2}

I think that the author has done a good job in addressing the comments
made by the three reviewers. However, on reading the draft, I felt a few
things were not clear.

\begin{enumerate}

\item It is not clear why the author restricts the analysis to Hindu
women. Given that data across four rounds of the NFHS data is being
pooled, I think there is enough data to examine this question for the
non-Hindu (and in particular Muslim) women. Given that patterns of son
preference varies across religious groups, I think it is worth examining
the question for non-Hindus. At the very lease include a discussion of
why the nonHindus have not been included.

[Response:]

\item There is actually no data on sex selection and in a sense the data
is backed out using evidence on birth spacing, institutional information
on the availability of technology that enables sex selection (ultra
sound techniques) and government regulations. To do this the author
subdivides the time into four segments: 1972–1984, 1985–1994, 19952004
and 2005–2016. I understand the logic of the first two breaks 1984 and
1994, but what is the logic of the third break in 2004? This needs to be
made clearer.

[Response:]

\item While I like results being presented in figures, what I don’t like
is that no standard errors or confidence intervals are presented.
Therefore, it is difficult to figure out whether the differences between
categories (composition of existing children) are statistically
significant or not.

[Response:]

\item In the context of the paper, I don’t see the point of the
discussion relating to labour supply.

[Response:]

\item Overall, I felt that the paper is still quite densely written and
it is not easy to follow what is going on. Considerable time needs to be
spent re-writing the paper and making it easier to read.

[Response:]

\end{enumerate}



\newpage
\bibliographystyle{econometrica}
\bibliography{../paper/sex_selection_spacing.bib}
\addcontentsline{toc}{section}{References}


\end{document}
