\documentclass[letterpaper,12pt]{article}

% Xetex preamble
\usepackage{fontspec}
\setromanfont[Ligatures=TeX]{TeX Gyre Pagella}
\usepackage{unicode-math}
\setmathfont{TeX Gyre Pagella Math}

\usepackage[longnamesfirst]{natbib}
%\usepackage[tabhead,nolists,tablesfirst]{endfloat}
\usepackage[flushleft]{threeparttable}
\usepackage{booktabs}
\usepackage{rotating}  
\usepackage{dcolumn} 
\usepackage{setspace}
%\usepackage{flafter} 
%\usepackage{longtable}
%\usepackage[pdftex]{graphicx}
\usepackage[xetex,colorlinks=true,linkcolor=black,citecolor=black]{hyperref} 
\usepackage[margin=1.0in]{geometry} 
\usepackage{multirow}
\usepackage{enumitem}

%opening
\title{} \author{}

\doublespacing

\begin{document}

\begin{center} \textbf{\large Birth Spacing in the Presence of Son
Preference and 
Sex-Selective Abortions: India's Experience over Four Decades}
\end{center}

\begin{center} Response to Editors' Comments \end{center}

\noindent I have attached the revised version of my paper,
``Birth Spacing and Fertility in the Presence of Son Preference and Sex-Selective
Abortions: India's Experience Over Four Decades.''
Thank you for the comments and suggestions. 
Below I outline my response to each issue raised.




\subsection*{Comments and My Responses}


The author has undertaken much reorganization and has made very useful clarifications to
benefit the reader in the text and tables. Nevertheless, we still find three issues with
the writing style:

\begin{itemize}
\item[(1)] Explanations of key concepts or theories need to be spelled out more clearly,
particularly for the non-economist. Much has been done in this regard, but further work is
still needed.

RESPONSE: With the reorganization of the Conceptual Framework section described below, I now 
have an early paragraph that explains income and substitution effects in detail and how they 
tie into fertility decisions with increasing wages for both males and females.
I then rely on this detailed explanation for my later discussions of female education and
labor force participation.
Furthermore, I provide a more in-depth explanation of fertility and investment decisions in
children based on the expected return to investment in children's education.
Finally, I have reorganized my discussion of how demand for better-educated mothers can
increase even if female labor force participation is declining to make it easier to follow
for non-economists.


\item[(2)] The arguments do not flow from section to section and paragraph to paragraph. Each
section and argument need to be introduced so the reader understands why this information
is being presented in this place. Many times, the author jumps into new arguments without
an introduction or transition or explanation.

RESPONSE: As detailed below, I have completely reorganized the Conceptual Framework section
to create a better flow and make it easier for the reader to follow.
Furthermore, I have added introductory paragraphs to the sections that did not already 
have these or an equivalent transition ("Conceptual Framework," "Estimation Strategy," "Data," 
and "How Birth Spacing Changed").
Finally, I have rewritten paragraphs in multiple places in the paper to make it easier 
to follow the thread of the arguments.


\item[(3)] There is no clear outline of the arguments presented in the Background section. Each
argument is presented separately rather than tying into the overall framework and
subsequent analyses.

RESPONSE: I have completely reorganized the Conceptual Framework section to address this
and  Issues (1) and (2) and the examples below.
The framework section now begins with an overview of the section, followed by three 
potential explanations that link fertility and birth spacing decisions: economic conditions, 
investment in children, and son preference.
I then introduce the sex composition of previous children, female education, and area of
residence as the primary explanatory variables, which proxy for the explanations and are
available in the data.
Finally, I tie each of the three primary explanatory variables to the explanations and
discuss their predicted effects.

\end{itemize}

\noindent The following are some examples of (1) and (2); however, the author should check the entire
paper for similar issues and address (3) as well.

\begin{enumerate}


\item  In general, each argument in the background section is not clear and fully described for
the reader.

\begin{enumerate}


\item  P. 4 first sentence: The author states that female education is a crucial explanatory
variable and it matters for fertility, sex ratios, mortality, etc. in a list. It would be
useful for the reader to briefly spell out how female education fits into the entire
framework looking across all these outcomes. In addition, the second paragraph on
measurement of education seems like it belongs in the data section, unless there is some
conceptual reason why this information is included here.

RESPONSE: I now introduce female education as a primary explanatory variable after the 
potential explanations linking fertility and birth spacing.
This change allows me to show how female education fits into the explanations and 
the potential effects of education on birth spacing.
I have moved the definitions of education groups to the Data section.

\item  The 2 paragraphs on female labor force participation on pp. 5-6: The two findings that
support the argument that changes in birth spacing are not due to economic reasons are
merely listed, leaving the reader to make the connection as to why these facts support the
argument herself. The income effect dominating the substitution effect and greater negative
elasticity of women’s wages need to be explained to readers unfamiliar with these concepts
or how they fit together.

RESPONSE: As detailed above, I have reorganized the whole section and now introduce income
and substitution effects early and apply these to fertility decisions.
This change should make it easier to understand the new explanation for why we are unlikely
to see a shortening of birth intervals in India, even with increasing female wages.
I have removed the discussion of elasticity (but kept the reference) since this is simply
another way of saying that the income effect dominates the substitution effect.

\item  P. 6 regarding sex selection: “The introduction of sex selection allows parents to avoid
giving birth to girls but increases the expected interval to the next birth.” The reader
needs an explanation here of why sex selection increases birth interval length. The author
mentions waiting time to conception and other components in the introduction (and
footnote). A clear explanation of this argument needs to be reiterated (or moved) here.

RESPONSE: I have moved the original footnote into the text in the Conceptual Framework and
expanded the discussion to make it easier to understand.
The Introduction section now only says that it takes 6 months or more after an abortion to
reach the same point in the next pregnancy and refers to the Conceptual Framework section's
detailed discussion.


\item  P. 6: “As women’s education increases, their productivity in the production of offspring
human capital also increases. With relatively more boys born because of increased access to
sexselective abortions and the increasing income potential for (male) offspring, demand for
bettereducated women can increase, even if they do not participate in the labor market.” I
am unclear as to why this argument is included. It needs explanation. Is this related to
length of birth intervals here? Or is this mentioned because women’s education and
increasing demand for better-education women will actually lower the sex ratio?

RESPONSE: That was, indeed, a terrible explanation. The argument is precisely the opposite
of lowering the sex ratio: the increased demand for better-educated women is associated 
with a more male-biased sex ratio. 
With declining female labor force participation, it is clear that educated women are
not desired because they bring income to the household. 
However, suppose better-educated women produce better-educated sons. 
In that case, this may increase the demand for better-educated women given the strong son 
preference and the rapidly rising return to male education. 
The result is that higher female education may be associated with lower fertility, longer 
spacing, and increased use of sex selection even if the standard economic explanation based 
on wages does not hold. 
The implication is that the usual policy recommendation of increased female education may not
change sex selection unless there is a concurrent increase in the relative return to female
education in the labor market. 
I have entirely rewritten the explanation and refer to it when discussing better job 
opportunities in the Conclusion.


\item  Pp. 6-7: I do not understand the importance of this argument and how it fits into the
authors’ story: “If more and “better” parental attention per child results in higher child
“quality,” we should expect longer birth intervals. However, the evidence on spacing’s
effect on child quality measures such as IQ and education is mixed for developed countries
and nonexisting for developing countries. The exception is health and mortality, where
longer spacing does lead to better outcomes, although this relationship weakens with
maternal education.”

RESPONSE:The idea was to show that longer spacing could be considered an investment in
children that may coincide with lower fertility. 
I have rewritten the paragraph to make this clearer.

\item  P. 7: The final summary paragraph lists predictions based on the earlier arguments. Yet
it is hard to connect these predictions to the paragraphs above. Perhaps predictions could
be noted with the earlier arguments.

RESPONSE: With the Conceptual Framework section's new structure, 
the summary of predictions now directly follows the predictions for the three primary
explanatory variables. Hence, it should connect better to the individual predictions. 

\end{enumerate}


\item  Estimation strategy section.

In the first paragraph, it would be useful to clearly outline the analyses that will be
undertaken (such as first, I document changes in birth intervals over time and how
influenced by sex selection, second how birth intervals affected fertility, and third how
birth intervals affected mortality). As it stands, this section jumps into a discussion of
the hazard model and the reader does not know in which analyses it will be used.

RESPONSE: I have added an outline of the three parts of the empirical analysis at the 
beginning of the section and highlighted that the Estimation Strategy section focuses
on the empirical model for birth spacing.

\end{enumerate}

\noindent Some additional issue as examples where more clarification is needed:


\begin{enumerate}[resume]

\item  Abstract. Please reorganize to start with a statement of the problem and main aims of
the study. As it stands, the abstract is a list of findings, with no context as to the
issues/questions the author investigates. Please be cognizant of using terms such as “women
most likely to use sexselective abortion.” The reader does not understand how this is
measured yet, and may get the false impression that sex-selection abortion is measured
directly.

RESPONSE: I now begin the abstract with the changing sex ratio after the introduction of
prenatal sex-determination technologies. I then state the main issue: abortions increase
birth spacing, but we know little about how birth spacing has changed or the effects of
these changes. After this, I highlight the main birth spacing results. Next, I discuss how
the rapid lengthening of birth spacing led the period fertility measure to overestimate the
decline in cohort fertility substantially. Finally, I cover whether repeated abortions can
counteract the other positive effects of longer birth spacing on infant mortality. 

Throughout the abstract, I avoid ascribing use of sex-selection to anybody.
Instead, I refer to the introduction of prenatal sex-determination technologies and the 
presumed use of sex-selective abortions.

\item  P. 1: The second motivation is unclear. ``the combined changes in birth spacing may
outpace what we have observed in other countries.'' Outpaced means what? decreasing faster,
or increasing faster, or sex selection plays more of a role?

RESPONSE: The goal of the sentence was to capture the combined effect from more sex
selection and the other changes.
I have rewritten the paragraph to emphasize that we do not know how much impact
the combined changes will have on birth spacing and remove the imprecise" outpace." The new
sentence is: "The combination of the apparent increasing use of sex selection and the other
societal changes can significantly impact birth spacing, but we know little about how
much."


\item  Throughout the paper, the author compares the total fertility rate to cohort fertility
rate. For the former, is the period fertility rate a more accurate definition?

RESPONSE: I would argue that the total fertility rate is our most commonly used period 
fertility measures, especially in ``public-facing'' discussions of fertility behavior 
changes \citep[see the discussion in][]{Hotz1997,Bongaarts1999,Ni-Bhrolchain2011}.
I have clarified whether I am referring to period fertility or cohort fertility 
throughout the paper. 


\item  P. 2: the main point of this sentence is unclear: “Counteracting effect is possible if
longer birth intervals arise from multiple abortions because the short duration between
pregnancies could increase mortality.”

RESPONSE: The idea is that even though the spacing between births becomes longer with
sex-selective abortions, the spacing between \emph{pregnancies} may still be very short, as
suggested by a prior reviewer. If short pregnancy spacing negatively affects child
outcomes, the short pregnancy spacing may dampen or eliminate any positive effects from
longer spacing. 
I have expanded the discussion of this possible counteracting impact in the Introduction.

The new paragraph is:
\begin{quote}
Second, what is the relationship between infant mortality and the changes in birth spacing
and sex selection? 
In India, birth intervals have traditionally been shorter with fewer sons, contributing to 
girls' higher mortality risk
\citep{Whitworth2002,Bhalotra2008,Maitra2008,Jayachandran2011,Jayachandran2017a}. 
Therefore, longer birth spacing may reduce mortality through, for example, diminished 
sibling competition \citep{Conde-Agudelo2012,Molitoris2019}. 
However, if the spacing between births lengthens because of sex-selective abortions, the 
spacing between pregnancies may still be very short. 
Short pregnancy spacing may lead to worse child outcomes because of maternal nutritional 
depletion and insufficient time to recover from the previous pregnancy. 
Hence, children born after long birth intervals where multiple abortions have punctuated 
those intervals may not see the same benefits as children born after a long interval that 
is not punctuated by abortions.
\end{quote}

\item  Sometimes the use of the term “birth interval” actually refers to “birth interval
length.” Please check the uses of these terms and correct accordingly.

RESPONSE: Thank you for pointing this out. 
I have corrected this issue in 28 different places in the paper.


\item  P. 2: “The key variables are maternal education, the sex of previous children, and the
area of residence.” Does the author mean key independent or predictor variables?

RESPONSE: Yes, those are the key explanatory variables. 
I have added that to the Introduction.
If you think that "independent" or "predictor" variables would be better, I would
be happy to change that throughout.


\item  The author should clearly note that sex selection is not observed and tone down some
conclusions, such as “Sex selection, however, is behind the most substantial increases in
birth spacing. The best-educated women with two girls had the most biased sex ratio and the
most significant increase in birth intervals.” The author could say that sex selection
appears to be behind these findings; or evidence suggests that sex selection is behind
these findings, etc. In addition, clarify that those most likely to practice sex selection
(women with highest education and two girls) is based on your analysis, not settled in the
literature.

RESPONSE:  To address this, I have made the following changes:
\begin{itemize}
\item Rewritten the Abstract as detailed above
\item In the Introduction's summary of results, I lead with the changes in birth spacing and 
sex ratios and then say that these are likely arising from sex-selective abortions.
\item Specify that the grouping into more and less sex selection when analyzing birth spacing 
is based on sex ratios and clarify that the use of sex selection is inferred rather than 
established when analyzing birth spacing.
\item In the Conclusion section, I now focus on the lengthening of birth intervals and how
these likely arose from sex selection as indicated by the increasingly male-biased sex
ratio.
\end{itemize}

10. While it might be messy to show the confidence intervals in all the figures, is it
possible to note some significant differences in the text?

RESPONSE: I tried again to incorporate confidence intervals in the figures, but the only 
way I could find tripled the number of figures.
Instead, I have expanded the discussion of results to include where there are statistically
significant differences:
\begin{itemize}
\item I now highlighted that tests for statistically significant differences across sex compositions 
are available in the online appendix.
\item In the section on how birth spacing changed, I have added discussions of statistical significance 
when discussing comparisons across sex compositions.
\item In the Conclusion, I have also added that when there are no sons, we now see statistically 
significant longer---rather than statistically significant shorter---birth intervals than 
if there is at least one son, rather than merely discussing the reversal.
\end{itemize}

\end{enumerate}


\newpage
\bibliographystyle{econometrica}
\bibliography{../paper/sex_selection_spacing.bib}
\addcontentsline{toc}{section}{References}


\end{document}
