\documentclass[letterpaper,12pt]{article}

% Xetex preamble
\usepackage{fontspec}
\setromanfont[Ligatures=TeX]{TeX Gyre Pagella}
\usepackage{unicode-math}
\setmathfont{TeX Gyre Pagella Math}

\usepackage[longnamesfirst]{natbib}
%\usepackage[tabhead,nolists,tablesfirst]{endfloat}
\usepackage[flushleft]{threeparttable}
\usepackage{booktabs}
\usepackage{rotating}  
\usepackage{dcolumn} 
\usepackage{setspace}
%\usepackage{flafter} 
%\usepackage{longtable}
%\usepackage[pdftex]{graphicx}
\usepackage[xetex,colorlinks=true,linkcolor=black,citecolor=black]{hyperref} 
\usepackage[margin=1.0in]{geometry} 
\usepackage{multirow}

%opening
\title{} \author{}

\doublespacing

\begin{document}

\begin{center} \textbf{\large Birth Spacing in the Presence of Son
Preference and 
Sex-Selective Abortions: India's Experience over Four Decades}
\end{center}

\begin{center} Response to Editor Comments \end{center}

\noindent I have attached the revised version of my paper,
``Birth Spacing and Fertility in the Presence of Son Preference and Sex-Selective
Abortions: India's Experience Over Four Decades.''
Thank you for the comments and suggestions. 
Below I outline my response to each issue raised.


The author has undertaken much reorganization and has made very useful clarifications to
benefit the reader in the text and tables. Nevertheless, we still find three issues with
the writing style:

(1) Explanations of key concepts or theories need to be spelled out more clearly,
particularly for the non-economist. Much has been done in this regard, but further work is
still needed.

(2) The arguments do not flow from section to section and paragraph to paragraph. Each
section and argument need to be introduced so the reader understands why this information
is being presented in this place. Many times, the author jumps into new arguments without
an introduction or transition or explanation.

(3) There is no clear outline of the arguments presented in the Background section. Each
argument is presented separately rather than tying into the overall framework and
subsequent analyses.

The following are some examples of (1) and (2); however, the author should check the entire
paper for similar issues and address (3) as well.

1. In general, each argument in the background section is not clear and fully described for
the reader.

a. P. 4 first sentence: The author states that female education is a crucial explanatory
variable and it matters for fertility, sex ratios, mortality, etc. in a list. It would be
useful for the reader to briefly spell out how female education fits into the entire
framework looking across all these outcomes. In addition, the second paragraph on
measurement of education seems like it belongs in the data section, unless there is some
conceptual reason why this information is included here.

b. The 2 paragraphs on female labor force participation on pp. 5-6: The two findings that
support the argument that changes in birth spacing are not due to economic reasons are
merely listed, leaving the reader to make the connection as to why these facts support the
argument herself. The income effect dominating the substitution effect and greater negative
elasticity of women’s wages need to be explained to readers unfamiliar with these concepts
or how they fit together.

c. P. 6 regarding sex selection: “The introduction of sex selection allows parents to avoid
giving birth to girls but increases the expected interval to the next birth.” The reader
needs an explanation here of why sex selection increases birth interval length. The author
mentions waiting time to conception and other components in the introduction (and
footnote). A clear explanation of this argument needs to be reiterated (or moved) here.

d. P. 6: “As women’s education increases, their productivity in the production of offspring
human capital also increases. With relatively more boys born because of increased access to
sexselective abortions and the increasing income potential for (male) offspring, demand for
bettereducated women can increase, even if they do not participate in the labor market.” I
am unclear as to why this argument is included. It needs explanation. Is this related to
length of birth intervals here? Or is this mentioned because women’s education and
increasing demand for better-education women will actually lower the sex ratio?

e. Pp. 6-7: I do not understand the importance of this argument and how it fits into the
authors’ story: “If more and “better” parental attention per child results in higher child
“quality,” we should expect longer birth intervals. However, the evidence on spacing’s
effect on child quality measures such as IQ and education is mixed for developed countries
and nonexisting for developing countries. The exception is health and mortality, where
longer spacing does lead to better outcomes, although this relationship weakens with
maternal education.”

f. P. 7: The final summary paragraph lists predictions based on the earlier arguments. Yet
it is hard to connect these predictions to the paragraphs above. Perhaps predictions could
be noted with the earlier arguments.

2. Estimation strategy section.

In the first paragraph, it would be useful to clearly outline the analyses that will be
undertaken (such as first, I document changes in birth intervals over time and how
influenced by sex selection, second how birth intervals affected fertility, and third how
birth intervals affected mortality). As it stands, this section jumps into a discussion of
the hazard model and the reader does not know in which analyses it will be used.

Some additional issue as examples where more clarification is needed:

3. Abstract. Please reorganize to start with a statement of the problem and main aims of
the study. As it stands, the abstract is a list of findings, with no context as to the
issues/questions the author investigates. Please be cognizant of using terms such as “women
most likely to use sexselective abortion.” The reader does not understand how this is
measured yet, and may get the false impression that sex-selection abortion is measured
directly.

4. P. 1: The second motivation is unclear. “the combined changes in birth spacing may
outpace what we have observed in other countries.” Outpaced means what? decreasing faster,
or increasing faster, or sex selection plays more of a role?

5. Throughout the paper, the author compares the total fertility rate to cohort fertility
rate. For the former, is the period fertility rate a more accurate definition?

6. P. 2: the main point of this sentence is unclear: “Counteracting effect is possible if
longer birth intervals arise from multiple abortions because the short duration between
pregnancies could increase mortality.”

7. Sometimes the use of the term “birth interval” actually refers to “birth interval
length.” Please check the uses of these terms and correct accordingly.

8. P. 2: “The key variables are maternal education, the sex of previous children, and the
area of residence.” Does the author mean key independent or predictor variables?

9. The author should clearly note that sex selection is not observed and tone down some
conclusions, such as “Sex selection, however, is behind the most substantial increases in
birth spacing. The best-educated women with two girls had the most biased sex ratio and the
most significant increase in birth intervals.” The author could say that sex selection
appears to be behind these findings; or evidence suggests that sex selection is behind
these findings, etc. In addition, clarify that those most likely to practice sex selection
(women with highest education and two girls) is based on your analysis, not settled in the
literature.

10. While it might be messy to show the confidence intervals in all the figures, is it
possible to note some significant differences in the text?


\newpage
\bibliographystyle{econometrica}
\bibliography{../paper/sex_selection_spacing.bib}
\addcontentsline{toc}{section}{References}


\end{document}
